\documentclass{beamer}
  \usetheme{metropolis}
  \usepackage[normalem]{ulem}
  
  \newcommand{\Turnsi}[2]
	{ {#1}\vdash  {#2}}

  \usepackage{outlines}
  
         % Use metropolis theme
         \usepackage{mathpartir}
         \usepackage{mathtools}
         \usepackage{appendixnumberbeamer}
         
         \usepackage{booktabs}
         \usepackage[scale=2]{ccicons}
         
         \usepackage{pgfplots}
         \usepgfplotslibrary{dateplot}
         
         \usepackage{xspace}
         \newcommand{\themename}{\textbf{\textsc{metropolis}}\xspace}
  \title{A reading of justification logic modality in Curry-Howard fashion}
  \date{\today}
  \author{Konstantinos Pouliasis}
  \institute{CUNY, Graduate Center}
  \begin{document}
    \maketitle
    \section{Introduction}
    \begin{frame}{Curry--Howard Isomorphism}
      \begin{itemize}
\item sparked from the investigation of the connection of 
explicit proofs in intuitionistic logic and 
programs of a simple programming language (\emph{STLC})

\item central topic of study in the field of type theory

\item standard theoretical tool for studying and designing programming languages
      \end{itemize}
\end{frame}
\begin{frame}{\emph{CHI} in logic}
  \begin{itemize}

  \item a  paradigm in logic studies that emphasizes proof relevance

  \item scales : the isomorphism holds for more complex logics and correspondingly captures complex programming language constructs.

  \item We use proofs-as-programs(-as-morphisms) and CHI interchangeably.
  \end{itemize}
\end{frame}
\section{Intuitionistic Logic and STLC}
\begin{frame}{Intuitionism and constructive mathematics}
  \begin{outline}
    \1 {Continuation of Brouwer's program}
    
    \2{Notion of truth collides with the notion of proof}
    
  
  \1 Mathematical reasoning and logic is a human faculty
  
  \2 The creative subjects use mathematical language 
  as a means to express their reasoning (proof) constructs
  \2 
  There are open problems
  \2 
  Propositions do not come with a pre-existing truth value
\end{outline}
\end{frame}

\begin{frame}{Logic within constructive mathematics}
\begin{outline} 

\1[] The object of logic is the study of ``sane'' proof constructs independently of formal systems

\1[] The traditional chasm between ``syntax'' and ``semantics'' is no longer that very fruitful at least when “engineering” new logics.
\end{outline}
\begin{columns}[T,onlytextwidth]
  \column{0.5\textwidth}

    \begin{alertblock}{Verificationist}
    \begin{quote}The meaning of a connective is the ways that one can prove it\end{quote} (Martin Lof - Verificationist approach)
    \end{alertblock}


  \column{0.5\textwidth}

  \begin{exampleblock}{Pragmatist}
  \begin{quote} The meaning of a connective is the way one can use its proofs\end{quote}
  \end{exampleblock}
\end{columns}
\end{frame}
\begin{frame}{Example: rules $\wedge$}
  \begin{mathpar}
 \inferrule*[right=Intro] { {\begin{array}{c} \mathcal{D} \\ {A} \end{array} } \\{\begin{array}{c} \mathcal{E} \\ {B} \end{array} }} {A\wedge B} 
  \end{mathpar}
\[ \begin{array}{c c} \inferrule*[right=Elim1] { {\begin{array}{c} \\ \mathcal{D} \\ {A\wedge B} \end{array} }} {A} & \inferrule*[right=Elim2] { {\begin{array}{c} \\ \mathcal{D} \\ {A\wedge B} \end{array} }} {B} \end{array} \]
\end{frame}

\begin{frame}{Proof equality and harmony}
  \begin{itemize}
  \item[] Proof trees are objects of logic and - as mathematical objects - are equipped with equality

  \item[]\alert{Logic is “algebrized”}
  
  \item[]Equalities should provide for the “harmony” of the Introduction – Elimination (constructors– distructors)
  
 \end{itemize}
\end{frame}
\begin{frame}{Harmony}
      \[ \begin{array}{l c r} {\inferrule*[right=Elim1]{\inferrule*[right=Intro] { {\begin{array}{c} \mathcal{D}\\ {A} \end{array} } \\{\begin{array}{c} \mathcal{E} \\ {B} \end{array} }} {A\wedge B}}{A}} & = & {\begin{array}{c} \mathcal{D}\\ {A} \end{array} } \end{array} \]
      \[ \begin{array}{l c r} {\inferrule*[right=Elim2]{\inferrule*[right=Intro] { {\begin{array}{c} \mathcal{D}\\ {A} \end{array} } \\{\begin{array}{c} \mathcal{E} \\ {B} \end{array} }} {A\wedge B}}{B}} & = & {\begin{array}{c} \mathcal{E}\\ {B} \end{array} } \end{array} \]
\end{frame}
\begin{frame}{Harmony}
  \begin{columns}[T,onlytextwidth]
    \column{0.7\textwidth}
        \begin{exampleblock}{Elim rules are not too weak}
        \[\begin{array}{l c r} {\begin{array}{c} \mathcal{D}\\ {A\wedge B} \end{array} } 
        & = & {\inferrule*[right=Intro]{
          {\inferrule*[right=Elim1] { {
          \begin{array}{c} \mathcal{D}\\ {A\wedge B} \end{array} 
        } } {A}}\\ {\inferrule*[right=Elim2] { {\begin{array}{c} \mathcal{D}\\ {A\wedge B} \end{array} } } {B}}}
        {A\wedge B}
      } \end{array} \]
        \end{exampleblock}
  \end{columns}
\end{frame}
\begin{frame}{Harmony}
  \begin{columns}[T,onlytextwidth]
    \column{0.5\textwidth}
    
   \[\begin{array}{c c c}
    {\inferrule{
    {\inferrule[]{
      {\begin{array}[b]{c} \overline{x:A}\\ {\mathcal{D}} \\ B  \end{array}}}{A \supset B }}\\
  {\begin{array}{c}\mathcal{E}\\{A }\end{array}}
  }
  { B }} & = &
    
        {\begin{array}[b]{c} \mathcal{E} \\ A  \\ {\mathcal{D}} \\ B  \end{array}} 
  
  \end{array}\]
  \column{0.5\textwidth}
  \[\begin{array}{c c c}
	{\begin{array}[b]{c}  {\mathcal{D}} \\ A\supset B  \end{array}}
			& = &
		
			{\inferrule{
				{\inferrule[]{
						{\begin{array}[b]{c}  {\mathcal{D}} \\ A\supset B  \end{array}}
							\\{\overline{x:A}}} 
							{ B }}}
				{A \supset B}
		}
	\end{array}\]
\end{columns}
\end{frame}
\begin{frame}{Calculi for free}
  \begin{itemize} 
  \item[] trees $\mapsto$ \emph{terms} (inhabitation judgments)
  \item[] local soundness $\mapsto$ $\beta$ reduction (equality judgments), 
  \item [] local completeness $\mapsto$ $\eta$ expansion (equality judgments)
  \end{itemize}
\end{frame}
\begin{frame}
\begin{mathpar}
    \inferrule*[right=$\top$I] { } {\Turnsi {\Gamma} { \langle \rangle:\top }}
    \and
    \inferrule*[right=Tup] {\Turnsi {\Gamma} {M:\phi_1}\\{\Turnsi {\Gamma} { M^{'}:\phi_2}}} {\Turnsi {\Gamma} {  
      \langle M,M^{'}\rangle:\phi_1\times \phi_2 }}
      \and
      \inferrule*[right=LPrj] {\Turnsi {\Gamma} {M:\phi_1\times\phi_2 }} {\Turnsi {\Gamma} {\operatorname{fst}(M):  \phi_1}}
      \and
      \inferrule*[right=RPrj] {\Turnsi {\Gamma} {M:\phi_1\times\phi_2}} {\Turnsi {\Gamma} {\operatorname{snd}(M):  \phi_2}}
      \and
  \inferrule*[right=$\lambda-$Abs] {\Turnsi {\Gamma, x:\phi_1 } 
  {M:\phi_2 }} {\Turnsi {\Gamma} { \lambda x. M:\phi_1\rightarrow \phi_2 }}
  \inferrule*[right=App] {\Turnsi {\Gamma} {M:\phi_1\rightarrow\phi_2 }\\{\Turnsi {\Gamma} {M^{'}:\phi_1}}} {\Turnsi {\Gamma} {  (M M^{'}):\phi_2 }}
\end{mathpar}
\alert{Plus equality judgments}
e.g. \[\inferrule*[right=$\beta\supset$] 
{{\Gamma,x:A\vdash M:B}\\ {\Gamma\vdash N:A}}
{\Gamma\vdash(\lambda x.M)(N)\equiv [N/x]M:B}\]
\alert{plus congruence axiomatization}
\alert{plus context judgments}
\end{frame}
\section{JCalc: Natural deduction}

\begin{frame}{Motivation}
  \[\inferrule*[right=\sout{Nec}] 
  {\vdash A} {\vdash \Box A}\]
  \[\inferrule*[right=$\Box_I$] 
  {{\vdash A}\\ {\vdash  {\sf Just\quad A}}} {\vdash \Box A}\]
In JL the motivation is that ${\vdash A}$ is ${\sf int}$ and
${\sf J A}$ is in ${\sf cl}$  so $Just$ can be seen as an embedding.
\end{frame}
\begin{frame}{Motivation}
  \[\inferrule*[right=\sout{Nec}] 
  {\vdash A} {\vdash \Box A}\]
  \[\inferrule*[right=$\Box_I$] 
  {{\vdash A}\\ {\vdash  {\sf Just\quad A}}} {\vdash \Box A}\]
In JL the motivation is that ${\vdash A}$ is ${\sf int}$ and
${\sf J A}$ is in ${\sf cl}$  so $Just$ can be seen as an embedding.
\end{frame}
\end{document}
