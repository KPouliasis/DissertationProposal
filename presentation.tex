\documentclass{beamer}
  \usetheme{metropolis}
  \usepackage[normalem]{ulem}
  \usepackage{stmaryrd}
  
  
  \newcommand{\Turnsi}[2]
	{ {#1}\vdash  {#2}}

  \usepackage{outlines}
  
         % Use metropolis theme
         \usepackage{mathpartir}
         \usepackage{mathtools}
         \usepackage{appendixnumberbeamer}
         
         \usepackage{booktabs}
         \usepackage[scale=2]{ccicons}
         
         \usepackage{pgfplots}
         \usepgfplotslibrary{dateplot}
         
         \usepackage{xspace}
         \newcommand{\themename}{\textbf{\textsc{metropolis}}\xspace}
  \title{A reading of justification logic modality in Curry-Howard fashion}
  \date{\today}
  \author{Konstantinos Pouliasis}
  \institute{CUNY, Graduate Center}
  \begin{document}
    \maketitle
    \section{Introduction}
    \begin{frame}{Curry--Howard Isomorphism}
      \begin{itemize}
\item sparked from the investigation of the connection of 
explicit proofs in intuitionistic logic and 
programs of a simple programming language (\emph{STLC})

\item central topic of study in the field of type theory

\item standard theoretical tool for studying and designing programming languages
      \end{itemize}
\end{frame}
\begin{frame}{\emph{CHI} in logic}
  \begin{itemize}

  \item a  paradigm in logic studies that emphasizes proof relevance

  \item scales : the isomorphism holds for more complex logics and correspondingly captures complex programming language constructs.

  \item We use proofs-as-programs(-as-morphisms) and CHI interchangeably.
  \end{itemize}
\end{frame}
\section{Intuitionistic Logic and STLC}
\begin{frame}{Intuitionism and constructive mathematics}
  \begin{outline}
    \1 {Continuation of Brouwer's program}
    
    \2{Notion of truth collides with the notion of proof}
    
  
  \1 Mathematical reasoning and logic is a human faculty
  
  \2 The creative subjects use mathematical language 
  as a means to express their reasoning (proof) constructs
  \2 
  There are open problems
  \2 
  Propositions do not come with a pre-existing truth value
\end{outline}
\end{frame}

\begin{frame}{Logic within constructive mathematics}
\begin{outline} 

\1[] The object of logic is the study of ``sane'' proof constructs independently of formal systems

\1[] The traditional chasm between ``syntax'' and ``semantics'' is no longer that very fruitful at least when “engineering” new logics.
\end{outline}
\begin{columns}[T,onlytextwidth]
  \column{0.5\textwidth}

    \begin{alertblock}{Verificationist}
    \begin{quote}The meaning of a connective is the ways that one can prove it\end{quote} (Martin Lof - Verificationist approach)
    \end{alertblock}


  \column{0.5\textwidth}

  \begin{exampleblock}{Pragmatist}
  \begin{quote} The meaning of a connective is the way one can use its proofs\end{quote}
  \end{exampleblock}
\end{columns}
\end{frame}
\begin{frame}{Example: rules $\wedge$}
  \begin{mathpar}
 \inferrule*[right=Intro] { {\begin{array}{c} \mathcal{D} \\ {A} \end{array} } \\{\begin{array}{c} \mathcal{E} \\ {B} \end{array} }} {A\wedge B} 
  \end{mathpar}
\[ \begin{array}{c c} \inferrule*[right=Elim1] { {\begin{array}{c} \\ \mathcal{D} \\ {A\wedge B} \end{array} }} {A} & \inferrule*[right=Elim2] { {\begin{array}{c} \\ \mathcal{D} \\ {A\wedge B} \end{array} }} {B} \end{array} \]
\end{frame}

\begin{frame}{Proof equality and harmony}
  \begin{itemize}
  \item[] Proof trees are objects of logic and - as mathematical objects - are equipped with equality

  \item[]\alert{Logic is “algebrized”}
  
  \item[]Equalities should provide for the “harmony” of the Introduction – Elimination (constructors– destructors)
  
 \end{itemize}
\end{frame}
\begin{frame}{Harmony}
      \[ \begin{array}{l c r} {\inferrule*[right=Elim1]{\inferrule*[right=Intro] { {\begin{array}{c} \mathcal{D}\\ {A} \end{array} } \\{\begin{array}{c} \mathcal{E} \\ {B} \end{array} }} {A\wedge B}}{A}} & = & {\begin{array}{c} \mathcal{D}\\ {A} \end{array} } \end{array} \]
      \[ \begin{array}{l c r} {\inferrule*[right=Elim2]{\inferrule*[right=Intro] { {\begin{array}{c} \mathcal{D}\\ {A} \end{array} } \\{\begin{array}{c} \mathcal{E} \\ {B} \end{array} }} {A\wedge B}}{B}} & = & {\begin{array}{c} \mathcal{E}\\ {B} \end{array} } \end{array} \]
\end{frame}
\begin{frame}{Harmony}
  \begin{columns}[T,onlytextwidth]
    \column{0.7\textwidth}
        \begin{exampleblock}{Elim rules are not too weak}
        \[\begin{array}{l c r} {\begin{array}{c} \mathcal{D}\\ {A\wedge B} \end{array} } 
        & = & {\inferrule*[right=Intro]{
          {\inferrule*[right=Elim1] { {
          \begin{array}{c} \mathcal{D}\\ {A\wedge B} \end{array} 
        } } {A}}\\ {\inferrule*[right=Elim2] { {\begin{array}{c} \mathcal{D}\\ {A\wedge B} \end{array} } } {B}}}
        {A\wedge B}
      } \end{array} \]
        \end{exampleblock}
  \end{columns}
\end{frame}
\begin{frame}{Harmony}
  \begin{columns}[T,onlytextwidth]
    \column{0.5\textwidth}
    
   \[\begin{array}{c c c}
    {\inferrule{
    {\inferrule[]{
      {\begin{array}[b]{c} \overline{x:A}\\ {\mathcal{D}} \\ B  \end{array}}}{A \supset B }}\\
  {\begin{array}{c}\mathcal{E}\\{A }\end{array}}
  }
  { B }} & = &
    
        {\begin{array}[b]{c} \mathcal{E} \\ A  \\ {\mathcal{D}} \\ B  \end{array}} 
  
  \end{array}\]
  \column{0.5\textwidth}
  \[\begin{array}{c c c}
	{\begin{array}[b]{c}  {\mathcal{D}} \\ A\supset B  \end{array}}
			& = &
		
			{\inferrule{
				{\inferrule[]{
						{\begin{array}[b]{c}  {\mathcal{D}} \\ A\supset B  \end{array}}
							\\{\overline{x:A}}} 
							{ B }}}
				{A \supset B}
		}
	\end{array}\]
\end{columns}
\end{frame}
\begin{frame}{Calculi for free}
  \begin{itemize} 
  \item[] trees $\mapsto$ \emph{terms} (inhabitation judgments)
  \item[] local soundness $\mapsto$ $\beta$ reduction (equality judgments), 
  \item [] local completeness $\mapsto$ $\eta$ expansion (equality judgments)
  \end{itemize}
\end{frame}
\begin{frame}
\begin{mathpar}
    \inferrule*[right=$\top$I] { } {\Turnsi {\Gamma} { \langle \rangle:\top }}
    \and
    \inferrule*[right=Tup] {\Turnsi {\Gamma} {M:\phi_1}\\{\Turnsi {\Gamma} { M^{'}:\phi_2}}} {\Turnsi {\Gamma} {  
      \langle M,M^{'}\rangle:\phi_1\times \phi_2 }}
      \and
      \inferrule*[right=LPrj] {\Turnsi {\Gamma} {M:\phi_1\times\phi_2 }} {\Turnsi {\Gamma} {\operatorname{fst}(M):  \phi_1}}
      \and
      \inferrule*[right=RPrj] {\Turnsi {\Gamma} {M:\phi_1\times\phi_2}} {\Turnsi {\Gamma} {\operatorname{snd}(M):  \phi_2}}
      \and
  \inferrule*[right=$\lambda-$Abs] {\Turnsi {\Gamma, x:\phi_1 } 
  {M:\phi_2 }} {\Turnsi {\Gamma} { \lambda x. M:\phi_1\rightarrow \phi_2 }}
  \inferrule*[right=App] {\Turnsi {\Gamma} {M:\phi_1\rightarrow\phi_2 }\\{\Turnsi {\Gamma} {M^{'}:\phi_1}}} {\Turnsi {\Gamma} {  (M M^{'}):\phi_2 }}
\end{mathpar}
\alert{Plus equality judgments}
e.g. \[\inferrule*[right=$\beta\supset$] 
{{\Gamma,x:A\vdash M:B}\\ {\Gamma\vdash N:A}}
{\Gamma\vdash(\lambda x.M)(N)\equiv [N/x]M:B}\]
\alert{plus congruence axiomatization}
\alert{plus context judgments}
\end{frame}
\section{JCalc: Motivation}

\begin{frame}{Motivation}
  \[\inferrule*[right=\sout{Nec}] 
  {\vdash A} {\vdash \Box A}\]
  \[\inferrule*[right=$\Box_I$] 
  {{\vdash A}\\ {\vdash  {\sf J A}}} {\vdash \Box A}\]
In JL the motivation is that ${\vdash A}$ is ${\sf int}$ and
${\sf J A}$ is in ${\sf cl}$  so $J$ can be seen as an embedding.
In general though the connective can be treated indepenedently of the details
of the two systems
\end{frame}
\begin{frame}{Relating Deductive systems}  
  \begin{outline}
  \1 Assume deductive systems 
  \2 $I$:with propositions in $U_I$
  a possibly non-empty signature of axioms $\Sigma_I$
  an entailment relation $\Sigma_I;\Gamma\vdash_{I}A$
  \2 $J$ with: $U_J$, $\Sigma_J$, and $\Sigma_J;\Delta\vdash_J \phi$ respectively.
  
  \1 Deductive System Requirements
  
  \2 Reflexivity :
  
  $A \in \Gamma \Longrightarrow \Gamma\vdash_{I}A$\\
  $\phi \in \Delta \Longrightarrow \Delta\vdash_{J}\phi$
  \2 Compositionality:
  
  $\Gamma\vdash_I A$ and $\Gamma^{\prime},A\vdash_{I} B \Longrightarrow \Gamma,\Gamma^{'}\vdash_I B $\\
  $\Delta\vdash_J\phi$ and $\Delta^{\prime},\phi\vdash_{J} \psi \Longrightarrow \Delta,\Delta^{'}\vdash_J \psi$
  \1 Top elements :
  
  $\Gamma\vdash_{I}\top_I $ and $\Delta\vdash_{J}\top_J $
  \end{outline}
\end{frame}

\begin{frame}{Interpetations and Validations}
  
  \begin{outline}
  \1An interpretation for $I$, ($\llbracket\bullet \rrbracket_J$) is a pair $(J,\llbracket\bullet\rrbracket)$ of a deductive system $J$ together with a (functional) mapping $\llbracket \bullet \rrbracket: U_I\rightarrow U_J$ on propositions of $I$ into propositions of $J$ extended to multisets of formulae of $U_I$ s.t.
  \2 $\llbracket\top_I \rrbracket = \top_J$
  
  \2 For $\Gamma$ of the form $A_1,\ldots, A_n$: $\llbracket\Gamma \rrbracket=\llbracket A_1 \rrbracket,\ldots, \llbracket A_n\rrbracket$
  
  \1 Given a deductive system $I$ and an interpretation function $\llbracket\bullet\rrbracket_J:U_I\rightarrow U_J$ of $I$ into $J$ we define:
  
  \2 A validation of a deduction $\Sigma_I;\Gamma\vdash_I A$ to be a deduction $\Sigma_J;\Delta\vdash_{J} \phi$ in $J$ such that $\llbracket A \rrbracket=\phi$ and $\Delta=\llbracket \Gamma \rrbracket $ . We will be writing $ \llbracket \Sigma_I;\Gamma\vdash_I A\rrbracket_J$
  \end{outline}
\end{frame}
\begin{frame}{Interpetations and Validations}
  
  \begin{outline}
  \1An interpretation for $I$, ($\llbracket\bullet \rrbracket_J$) is a pair $(J,\llbracket\bullet\rrbracket)$ of a deductive system $J$ together with a (functional) mapping $\llbracket \bullet \rrbracket: U_I\rightarrow U_J$ on propositions of $I$ into propositions of $J$ extended to multisets of formulae of $U_I$ s.t.
  \2 $\llbracket\top_I \rrbracket = \top_J$
  
  \2 For $\Gamma$ of the form $A_1,\ldots, A_n$: $\llbracket\Gamma \rrbracket=\llbracket A_1 \rrbracket,\ldots, \llbracket A_n\rrbracket$
  
  \1 Given a deductive system $I$ and an interpretation function $\llbracket\bullet\rrbracket_J:U_I\rightarrow U_J$ of $I$ into $J$ we define:
  
  \2 A validation of a deduction $\Sigma_I;\Gamma\vdash_I A$ to be a deduction $\Sigma_J;\Delta\vdash_{J} \phi$ in $J$ such that $\llbracket A \rrbracket=\phi$ and $\Delta=\llbracket \Gamma \rrbracket $ . We will be writing $ \llbracket \Sigma_I;\Gamma\vdash_I A\rrbracket_J$
  \end{outline}
\end{frame}
\begin{frame}{Logical Completeness}
  \begin{outline}
  
  \1 Given a deductive system $I$, we say that $J$ is logically complete under $\llbracket\bullet \rrbracket_J$ when for all purely logical deductions $\mathcal{D}$ in $I$ there exists a (purely logical) validation $\mathcal{E}$ in $J$. i.e: 
  $\forall \mathcal{D}. \ \mathcal{D}:\Gamma\vdash_I A \Longrightarrow \exists\mathcal{E}: \llbracket \Gamma\vdash A\rrbracket_J$
 \1 Spoiler: our system axiomatizes (extensions) of logical validations and treats 
 boxes as points of such validations  
 \1 Without loss of generality we focus on the case where $I$ is
 based on minimal intuitinistic logic
\end{outline}
\end{frame}
\begin{frame}{Roadmad}
  \begin{itemize}
 \item Create a system with two kinds of judgments 
 $A \ {\sf true}$ (as in $\Gamma\sf{true}\vdash_I \sf {true}$) and 
 $A\ {\sf valid}:=\llbracket A \rrbracket  
 {\sf true}$ as in $\Delta {\sf true}\vdash_J\llbracket A \rrbracket$ 
 and treat necessity as “double theoremhood”
\item obtain a system of basic constructive modality investigating the minimal relation between these two systems
\item Use justification logic to axiomatize validity $\llbracket \sf {judgments}\rrbracket$.
\end{itemize}
\end{frame}
\begin{frame}{Roadmad}
  \begin{itemize}
 \item Create a system with two kinds of judgments 
 $A \ {\sf true}$ (as in $\Gamma\sf{true}\vdash_I \sf {true}$) and 
 $A\ {\sf valid}:=\llbracket A \rrbracket  
 {\sf true}$ as in $\Delta {\sf true}\vdash_J\llbracket A \rrbracket$ 
 and treat necessity as “double theoremhood”
\item obtain a system of basic constructive modality investigating the minimal relation between these two systems
\item Use justification logic to axiomatize validity $\llbracket \sf {judgments}\rrbracket$.
\item focus on a minimal system based on $K$ modal logic and intuitionistic implication
\item \item add proof terms and show that such type systems are useful p

\item show that this approach scales with stronger modality
\end{itemize}
\end{frame}


\begin{frame}{$K$-$\Box$ rule}
  One modal rule for elimination and introduction to relate judgments $A\ {\sf valid}$, $ A\ {\sf true}$, $ \Box A {\sf true}$:
  
  \[ \inferrule* {{\begin{array}{c}\mathcal{D}\\ \Box A\end{array}} \\ {\inferrule* {}{{\begin{array}{c} \overline{x:A} \\ \mathcal{E} \\ {B} \end{array} } \\ {\begin{array}{c} \overline{s:\llbracket A \rrbracket} \\ \mathcal{F} \\ \llbracket B \rrbracket \end{array}} }}} {\Box B} \]
  
\end{frame}

\begin{frame}  {Modal Rule (Generically)}
  
  Or, generalizing for assumptions
  
  $\Gamma:=x_1:A_1, \ldots, x_i: A_i\ \text{and } \Delta:= s_1:\llbracket A_i \rrbracket, \ldots s_i:\llbracket A_i\rrbracket$:

  \[ \inferrule* {{\Box A_1, \ldots, \Box A_i} \\ {\inferrule* {}{{\begin{array}{c} \overline{\Gamma} \\ \vdots \\ {B} \end{array} } \\ {\begin{array}{c} \overline{\Delta} \\ \vdots \\ \llbracket B \rrbracket \end{array}} }}} {\Box B} \]
  \end{frame}
\begin{frame} {Modal Rule ($\Box$ Introduction)}
  
  We defined the connective negatively but we obtain $\Box$ constuctors by the very same rule for $\Gamma,\Delta$ empty and derivations $\mathcal{\overline{D},\overline{E}}$ closed for substitutions.
  
  \[ \inferrule* {{\inferrule* {}{{\begin{array}{c} \mathcal{\overline{D}} \\ {B} \end{array} } \\ {\begin{array}{c} \mathcal{\overline{E}} \\ \llbracket B \rrbracket \end{array}} }}} {\Box B} \]
  
\end{frame}



\section{Jcalc: provability}
\begin{frame}{Type Universe}
  \begin{mathpar}
    \inferrule*[right= Atom] { } {P_k \in {\sf Prop_0}}
    \and
    \inferrule*[right=Top] { } {\top \in {\sf Prop_0}}
    \and
    \inferrule*[right=Conj] {{ A \in {\sf Prop_i }}\\ { B \in {\sf Prop_i}}} {  A \wedge B \in {\sf Prop_i} } 
    %\inferrule*[right=Bot] { } {\bot \in {\sf Prop_0}} 
    %\and
    \and
    \inferrule*[right=Box] { A \in{\sf Prop_{0} }} {\Box  A\in{\sf Prop_{1}} }
    %\and
    %\inferrule*[right= Arr] {{ A \in {\sf Prop_{i} }}\\ { B \in {\sf Prop_{j}}}} { A\supset  B\in {\sf Prop_{max(i,j)}}}
    \and
    \inferrule*[right= Arr] {{ A \in {\sf Prop_i }}\\ { B \in {\sf Prop_j}}} { A\supset  B\in {\sf Prop_{max(i,j)}}}
    \and
    \inferrule*[right=Brc] { A \in {\sf Prop_0 }} {\llbracket  A\rrbracket \in {\sf \llbracket Prop_{0}\rrbracket}}
    %\and
    %\inferrule*[right=Brc $\supset$ Eq] {\llbracket A\supset\psi\rrbracket \in {\sf \llbracket Prop_{0}\rrbracket }} {\llbracket A\supset \psi\rrbracket=\llbracket A\rrbracket\supset \llbracket\psi\rrbracket :{\llbracket\sf Prop_{0}\rrbracket} }
  \end{mathpar}

\end{frame}
\begin{frame} {Natural deduction for ${\sf Prop_0}$}
  \begin{mathpar}
    \inferrule*[right=$\Gamma_0$-Refl] {x: A \in \Gamma}{\Turnsi {\Gamma} { A}}
    \and
    \inferrule*[right=$\top_0$I] { }{\Turnsi {\Gamma} { \top}}
  \end{mathpar}
  \begin{mathpar}
    \inferrule*[right=$\wedge_0$I] {{\Turnsi {\Gamma} { A}}\\{\Turnsi {\Gamma} { B}}} {\Turnsi {\Gamma} {   A\wedge B}}
    \and 
    \inferrule*[right=$\wedge_0$E1] {{\Turnsi {\Gamma} {A\wedge B}}} {\Turnsi {\Gamma} {   A}}
    \and 
    \inferrule*[right=$\wedge_0$E2] {\Turnsi {\Gamma} {A\wedge B}} {\Turnsi {\Gamma} {   B}}
  \end{mathpar}
  \begin{mathpar}
    \inferrule*[right=$\supset_0$I] {{\Turnsi {\Gamma, x: A} { B}}} {\Turnsi {\Gamma} {   A\supset  B}}
    \and
    \inferrule*[right=$\supset_0$E] {{\Turnsi {\Gamma} { A\supset  B}}\\{\Turnsi {\Gamma} { A}}} {\Turnsi {\Gamma} {   B}}
    
    %\inferrule*[right=$\bot$E] {{\Turn {\Gamma} {\bot}}}{\Turn {\Gamma} {   A}}
  \end{mathpar}
\end{frame}
\begin{frame}{Natural Deduction ( ${\sf Prop_1}$ )}

\[\begin{array}{c} \inferrule*{x^{\prime}: \Box A \in \Gamma}{ {\Gamma}\vdash {\Box A}} \\ \\ \inferrule*{{(\forall A_i \in \Gamma^{\prime}. \ {\Gamma}\vdash{\Box A_i})}\\{{\Gamma^{\prime}}\vdash { B}}\\ { {\llbracket \Gamma^{\prime} \rrbracket}\vdash {\llbracket B\rrbracket} }} { {\Gamma}\vdash\Box B} \\ \\ \inferrule* {{ {\Gamma, x^{\prime}: \Box A} \vdash { \Box B}}} { {\Gamma}\vdash { \Box A\supset \Box B}} \\ \\ \inferrule* {{ {\Gamma} \vdash { \Box A\supset \Box B}}\\{ {\Gamma} \vdash { \Box A}}} { {\Gamma}\vdash { \Box B}} \end{array} \]
\end{frame}

\begin{frame}{Basic facts}
  \begin{itemize}
\item Logical completeness of $\llbracket\bullet \rrbracket$:\\
 $\forall \Gamma, A \in {\sf Prop_0}$ \ $\Gamma\vdash A \Longrightarrow \llbracket \Gamma\rrbracket\vdash\llbracket A\rrbracket $,
 \item Admissibility of (standard) $K$ rule:
 $\forall \Gamma, A \in {\sf Prop_0}$ $\Gamma\vdash A \Longrightarrow \Box\Gamma\vdash\Box A$,
 \item Admissibility of Nec: 
 $\vdash A  \Longrightarrow \vdash \Box A$
  \end{itemize}
\end{frame}
\section{Jcalc: Order Theory}
\begin{frame}{Semi HAs}
  \begin{outline}
  \1 A \textit{(meet) semi-lattice} is a non-empty \emph{partial order} 
  (i.e. reflexive, antisymmetric and transitive) 
  with finite meets.
  
 \1
  A \textit{bounded (meet) semi-lattice} $(L,\le)$ is a (meet) 
  semi-lattice that additionally has 
  \2 \emph a {greatest element} (we name it $1$), which satisfies\\
  $x \le 1$ for every $x$ in $L$  
 \1 A \textit{semi-HA} is a bounded (meet) semi-lattice $(L,\le, 1)$ 
 s.t. for every $a,b\in L$ there exists an \textit{exponential} 
 (we name it $a\rightarrow b$) 
 with the properties: 
 \begin{enumerate}
 \item $a\rightarrow b\times a\le b $
 \item $a\rightarrow b$ is the greatest such element : \\ $c\times a\le b \Longrightarrow c\le a\rightarrow b $
 \end{enumerate}
  \end{outline}
\end{frame}
\begin{frame}{"Good" order preserving functions}
  \textbf{Definition}
  A function $F$ between two (semi)-HAs ($HA_1$, $HA_2$)
   is order preserving
  and commutes with top, products and exponentials \emph{iff} for every 
  $\phi,\psi \in HA_1$
    \begin{enumerate}
    \item $\phi\le_{HA_1}\psi\Rightarrow F\phi\le_{HA_2}F\psi$
    \item $F\top_{HA_1} = \top_{HA_2}$ 
    \item{$F(\phi \times\psi) = F(\phi)\times(F(\psi)$} 
    \item $F(\phi\rightarrow \psi) = F(\psi)\rightarrow F(\phi)$
    \end{enumerate}
    A \emph{$Jcalc$-triplet} is 
    
  \begin{enumerate}
  \item A semi-Hayting algebra $HA$
  \item A partial order $J$
  \item An order preserving function $F$ from $HA$ to $J$ s.t.
  \begin{enumerate}
    \item The image $F(HA)$ forms a semi-Heyting Algebra
    \item $F$ preserves top, products and exponentials
  \end{enumerate}
  \end{enumerate}
\end{frame}
\begin{frame}{Local Soundness}
  For $\mathcal{\overline{D},\overline{E}}$ closed for substitutions, the first proof tree is reducible to the second:
  
  \[\begin{array}{l r} \inferrule*{ \inferrule* {{\inferrule* {}{{\begin{array}{c} \mathcal{\overline{D}} \\ {A} \end{array} } \\ {\begin{array}{c} \mathcal{\overline{E}} \\ \llbracket A \rrbracket \end{array}} }}} {\Box A}\\{{\inferrule* {}{{\begin{array}{c} \overline{x:A} \\ \vdots \\ {B} \end{array} } \\ {\begin{array}{c} \overline{s:\llbracket A \rrbracket} \\ \vdots \\ \llbracket B \rrbracket \end{array}} }}}}{\Box B} & \Longrightarrow \end{array} \]
  \[ \inferrule* {{\inferrule* {}{{\begin{array}{c} \mathcal{\overline{D}}\\ \overline{A} \\ \vdots \\ {B} \end{array} } \\ {\begin{array}{c} \mathcal{\overline{E}}\\ \overline{\llbracket A \rrbracket} \\ \vdots \\ \llbracket B \rrbracket \end{array}} }}}{\Box B} \]
  \end{frame}
  \begin{frame}{Local Completeness}
   Any deduction $\mathcal{D}: \Box A$ can be expanded as follows:
  
  \[\begin{array} {c r} \begin{array}{c} \mathcal{D}\\ \Box A \end{array} & \Longrightarrow \\ \\ \inferrule* {{\begin{array}{c} \mathcal{D}\\ \Box A \end{array} } \\ {\begin{array}{c} \\ \overline{x:A} \end{array}} \\ {{\begin{array}{c} \\ \overline{s:\llbracket A\rrbracket} \end{array}}} }{\Box A} \end{array} \]
    \end{frame}
\end{document}
