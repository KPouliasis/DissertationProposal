\documentclass[12pt]{report}
\usepackage[toc,page]{appendix}
\usepackage{lmodern}
\usepackage{dirtytalk}
%\usepackage{quotchap}
\usepackage{hyperref}
\usepackage{xparse,nameref}
\usepackage{cleveref}
\usepackage{microtype}

\usepackage{fancyvrb}
\DefineVerbatimEnvironment{code}{Verbatim}{fontsize=\small}
% Package for customizing page layout

% \usepackage{fullpage}


\usepackage{amsmath,amsthm,amssymb}
\usepackage{mathtools}
\usepackage{etoolbox}
\usepackage{fancyhdr}
\usepackage{mathpartir}
\usepackage{xcolor}
\usepackage{hyperref}
\usepackage{xspace}
\usepackage{comment}
\usepackage{graphicx}
\usepackage{tikz-cd}
\usepackage{centernot}
\usepackage{listings}
\usetikzlibrary{shapes.geometric}
\usepackage[framemethod=TikZ]{mdframed}
\mdfdefinestyle{MyFrame}{%
    linecolor=blue,
    outerlinewidth=2pt,
    roundcorner=20pt,
    innertopmargin=\baselineskip,
    innerbottommargin=\baselineskip,
    innerrightmargin=20pt,
    innerleftmargin=20pt,
    backgroundcolor=gray!50!white}

\usepackage{url} % for url in bib entries
% for theorems, lemmas, etc
\newenvironment{theorem}[1][Theorem.]{\begin{trivlist}\item[\hskip \labelsep {\bfseries #1}]}{\end{trivlist}}
\newenvironment{lemma}[1][Lemma.]{\begin{trivlist}\item[\hskip \labelsep {\bfseries #1}]}{\end{trivlist}}
\newenvironment{corollary}[1][Corollary.]{\begin{trivlist}\item[\hskip \labelsep {\bfseries #1}]}{\end{trivlist}}
\newenvironment{definition}[1][Definition.]{\begin{trivlist}\item[\hskip \labelsep {\bfseries #1}]}{\end{trivlist}}
%%%%%%%%%%%%%%%%%%%%%%%%%%%%%%%%%%%%%%%%
% Acronyms
%%%%%%%%%%%%%%%%%%%%%%%%%%%%%%%%%%%%%%%%
\usepackage[acronym, shortcuts]{glossaries}

\newacronym{HoTT}{HoTT}{\emph{Homotopy Type Theory}}
\newacronym{IPL}{\sf{IPL}}{\emph{Intuitionistic Propositional Logic}}
\newacronym{LP}{\sf{LP}}{\emph{The Logic of Proofs}}
\newacronym{JL}{\sf{JL}}{\emph{Justification Logic}}
\newacronym{TT}{TT}{intuitionistic type theory}
\newacronym{LEM}{\sf{LEM}}{law of the excluded middle}
\newacronym{ITT}{ITT}{intensional type theory}
\newacronym{ETT}{ETT}{extensional type theory}
\newacronym{NNO}{NNO}{natural numbers object}
\newacronym{BHK}{\sf{BHK}}{\emph{Brower-Heyting-Kolmogorov}}
\newacronym{CHI}{\sf {CHI}}{\emph{Curry--Howard Isomorphism}}


% Make \ac robust.
\robustify{\ac}

%%%%%%%%%%%%%%%%%%%%%%%%%%%%%%%%%%%%%%%%
% Fancy page style
%%%%%%%%%%%%%%%%%%%%%%%%%%%%%%%%%%%%%%%%
\pagestyle{fancy}
\newcommand{\metadata}[2]{
  \lhead{}
  \chead{}
  \rhead{\bfseries Homotopy Type Theory}
  \lfoot{#1}
  \cfoot{#2}
  \rfoot{\thepage}
}
\renewcommand{\headrulewidth}{0.4pt}
\renewcommand{\footrulewidth}{0.4pt}


\newrobustcmd*{\vocab}[1]{\emph{#1}}
\newrobustcmd*{\latin}[1]{\textit{#1}}

%%%%%%%%%%%%%%%%%%%%%%%%%%%%%%%%%%%%%%%%
% Customize list enviroonments
%%%%%%%%%%%%%%%%%%%%%%%%%%%%%%%%%%%%%%%%
% package to customize three basic list environments: enumerate, itemize and description.
\usepackage{enumitem}
\setitemize{noitemsep, topsep=0pt, leftmargin=*}
\setenumerate{noitemsep, topsep=0pt, leftmargin=*}
\setdescription{noitemsep, topsep=0pt, leftmargin=*}

%%%%%%%%%%%%%%%%%%%%%%%%%%%%%%%%%%%%%%%%
% Some really basic macros.
% (Lots of them were stolen from HoTT/Book.)
% See macros.tex in HoTT/book.
%
% This is a mess.  Needs clean-ups.
%%%%%%%%%%%%%%%%%%%%%%%%%%%%%%%%%%%%%%%%
\newcommand{\Turnsi}[2]
	{ {#1}\vdash  {#2}}

\newcommand{\Turn}[2]
	{ {#1}\vdash_{\textbf{\sf IPC}}  {#2}}
\newcommand{\TurnTwo}[2]
	{ {#1}\vdash_{\textbf{\sf J}}  {#2}}
\newcommand{\TurnT}[2]
	{ \Delta_0;{#1}\vdash  {#2}}
\newcommand{\TurnTT}[2]
	{ \Delta_0;{#1}\vdash_{\sf JC_1}  {#2}}
\newcommand{\Turnj}[1]
	{ \Delta_0\vdash_{\sf J_0}  {#1}}
\newcommand{\Turnjc}[3]
    { {#1};{#2}\vdash_{\textbf{\sf JC}}  {#3}}
\newrobustcmd*{\ctx}{\Gamma}
\newrobustcmd*{\entails}{\vdash}

\newrobustcmd*{\judgmentfont}[1]{{\normalfont\sffamily #1}}
\newrobustcmd*{\postfixjudgment}[1]{%
  \relax\ifnum\lastnodetype>0\mskip\medmuskip\fi
  \text{\judgmentfont{#1}}%
}
\newrobustcmd*{\valid}{\postfixjudgment{Valid}}
\newrobustcmd*{\prop}{\postfixjudgment{Prop}}
\newrobustcmd*{\true}{\postfixjudgment{true}}
\newrobustcmd*{\type}{\postfixjudgment{type}}
\newrobustcmd*{\context}{\postfixjudgment{ctx}}
\newrobustcmd*{\nil}{\postfixjudgment{nil}}
\newrobustcmd*{\truth}{\top}
\newrobustcmd*{\conj}{\wedge}
\newrobustcmd*{\disj}{\vee}
\newrobustcmd*{\falsehood}{\bot}
\newrobustcmd*{\imp}{\supset}
\newrobustcmd*{\zero}{0}

%%% Judgmental equality
\newrobustcmd*{\jdeq}{\equiv}
%%% Definition
\newrobustcmd*{\defeq}{\vcentcolon\equiv}
%%% Binary sums
\newrobustcmd*{\inlsym}{{\mathsf{inl}}}
\newrobustcmd*{\inrsym}{{\mathsf{inr}}}
\newrobustcmd*{\inl}{\ensuremath\inlsym\xspace}
\newrobustcmd*{\inr}{\ensuremath\inrsym\xspace}
%%% Booleans
\newrobustcmd*{\ttsym}{{\mathsf{tt}}}
\newrobustcmd*{\ffsym}{{\mathsf{ff}}}
%%% Pairs
\newrobustcmd*{\pair}{\ensuremath{\mathsf{pair}}\xspace}
\newrobustcmd*{\tuple}[2]{(#1,#2)}
\newrobustcmd*{\proj}[1]{\ensuremath{\mathsf{pr}_{#1}}\xspace}
%% Empty type
\newrobustcmd*{\abort}[1]{\ensuremath{\mathsf{abort}_{#1}}}
%%% Path concatenation
\newrobustcmd*{\concat}{%
  \mathchoice{\mathbin{\raisebox{0.5ex}{$\displaystyle\centerdot$}}}%
  {\mathbin{\raisebox{0.5ex}{$\centerdot$}}}%
  {\mathbin{\raisebox{0.25ex}{$\scriptstyle\,\centerdot\,$}}}%
  {\mathbin{\raisebox{0.1ex}{$\scriptscriptstyle\,\centerdot\,$}}}
}
%%% Transport (covariant)
\newrobustcmd*{\trans}[2]{\ensuremath{{#1}_{*}\mathopen{}\left({#2}\right)\mathclose{}}\xspace}
% Natural numbers objects
\newrobustcmd*{\Nat}{\mathsf{Nat}}
\newrobustcmd*{\rec}{\ensuremath{\mathsf{rec}}\xspace}
% Sequence
\newrobustcmd*{\Seq}{\ensuremath{\mathsf{Seq}}\xspace}
% Identity type
\newrobustcmd*{\Id}[1]{\ensuremath{\mathsf{Id}_{#1}}\xspace}
% Reflection
\newrobustcmd*{\refl}[1]{\ensuremath{\mathsf{refl}_{#1}}\xspace}
\newrobustcmd*{\J}{\ensuremath{\mathsf{J}}\xspace}

% fst,snd,case,id
\newrobustcmd*{\fst}{\textsf{fst}}
\newrobustcmd*{\snd}{\textsf{snd}}
\DeclareMathOperator{\case}{\textsf{case}}
\DeclareMathOperator{\caseif}{\textsf{if}}
\DeclareMathOperator{\casesplit}{\textsf{split}}
\DeclareMathOperator{\ttrue}{\textsf{tt}\xspace}
\DeclareMathOperator{\ffalse}{\textsf{ff}\xspace}
\newrobustcmd*{\id}{\textsf{id}}

\newrobustcmd*{\op}[1]{\operatorname{#1}}

\newrobustcmd*{\universe}{\mathcal{U}}

%inductive types
\newrobustcmd*{\ind}{\ensuremath{\mathsf{ind}}\xspace}
%higher inductive types
%interval
\newrobustcmd*{\interval}{\ensuremath{I}\xspace}
\newrobustcmd*{\seg}{\ensuremath{\mathsf{seg}}\xspace}
%circle
\newrobustcmd*{\Sn}{\mathbb{S}}
\newrobustcmd*{\base}{\ensuremath{\mathsf{base}}\xspace}
\newrobustcmd*{\lloop}{\ensuremath{\mathsf{loop}}\xspace}
\tikzset{
	buffer/.style={
		draw,
		shape border rotate=180,
		regular polygon,
		regular polygon sides=3,
		fill=gray,
		node distance=2cm,
		minimum height=4em
	}
}
\tikzset{
	buffer2/.style={
		draw,
		shape border rotate=180,
		regular polygon,
		regular polygon sides=3,
		fill=gray,
		node distance=1cm,
		minimum height=4em
	}
}
\newcommand*{\PrTri}[1]{\begin{tikzpicture}
	\node[buffer]{#1};
	\end{tikzpicture}}
\tikzset{
	buffer2/.style={
		draw,
		shape border rotate=180,
		regular polygon,
		regular polygon sides=3,
		fill=gray,
		node distance=0.6cm
	}
}
\newcommand*{\PrTriSm}[1]{\begin{tikzpicture}
	\node[buffer2]{#1};
	\end{tikzpicture}}
\usepackage{stmaryrd}
\usepackage{proof-dashed}
\usepackage{tikz}
\usepackage{tikz-cd}
\usepackage[utf8]{inputenc} 
\usepackage{syntax}
\usepackage{amsfonts}
\usepackage{amssymb} 
\usepackage{amsmath}
%\usepackage{amsmath, amsthm, amssymb}
\usepackage{amsthm}
\usepackage{mathpartir}
\usepackage{appendix}
\usepackage{setspace}
\doublespacing
\pagestyle{myheadings}
%opening
\title{Relating Justification Logic Modality and Type Theory in Curry--Howard fashion}
\author{Konstantinos Pouliasis}

\begin{document}

\maketitle


\begin{abstract}
This thesis is a work in the intersection of \ac{JL} and \ac{CHI}. Justification logic is an umbrella of modal logics of knowledge with explicit evidence. 
Justification logics have been used to tackle traditional problems in proof theory (in relation to Godel's provability) and philosophy (Gettier examples, Russel's barn paradox). 
The Curry--Howard Isomorphism or \emph{proofs--as--programs} is an understanding of logic that places logical studies in conjunction with type theory and -- in current developments -- 
category theory. The point being that  understanding a system as a logic, a typed calculus and,  a language of a class of categories constitutes a useful discovery 
that can have many applications.
The applications  we will be mainly concerned with  are type systems 
for useful programming language constructs. 
This work  is structured in three parts: 
The first part (\cref{intui}, \cref{lambda}, \cref{jllogic}) is a  
a bird's eye view into my 
research topics:  
\emph{intuitionistic logic, justified modality} and \emph{type theory}. 
The relevant systems are introduced syntactically together with main 
 metatheoretic proof techniques which will be useful 
in the rest of the thesis.
The second part (\cref{proposal}, \cref{jcalcsem}, \cref{jcalccom}) 
features my main contributions.
I will propose  
a modal type system that extends simple type theory
 (or, isomorphically, intuitionistic propositional logic) with elements of
 justification logic and will argue about its computational significance. 
 More specifically, I will show  
that the obtained calculus characterizes  certain 
computational phenomena related to linking (e.g. module mechanisms, foreign function interfaces) 
that abound in semantics of modern programming languages. 
I will present full metatheoretic
results obtained for this logic/ calculus utilizing techniques from the first part 
and will provide proofs in the Appendix. The Appendix contains also information about
an implementation of our calculus in the metaprogramming framework {\sf Makam}.
Finally, I conclude this work with a small ``outro'', Chapter~\ref{ext},
where I informally show that the ideas underlying my contributions 
can be extended in interesting ways.

This work is my thesis requirement for my PhD candidacy under the supervision of Distinguished Professor Sergei Artemov at the Department of 
Computer Science of the Graduate Center, CUNY.
\end{abstract}
\section*{Acknowledgements}
This thesis would have not been possible without my supervisor Prof. Sergei Artemov.
He introduced me to the beauty of epistemic logic and its deep relations to proof theory
back at my first years as a graduate student, inspired and helped shape my topic of research. 
He has not stopped being a source of inspiration ever
since and has provided guidance, support and encouragement through our numerous meetings. 
I want to thank him deeply.

In addition, I want to thank my committee members: Prof. Fitting and Prof. Shankar. Prof. Fitting's classes
in at the graduate center have been a tremendous experience and source of knowledge wheareas,
the clarity of his both his writings and presentations helped my understand what is 
level of academic quality a young student should be striving and aiming for. Prof. Shankar has
been a great teacher and an excellent advisor. We spent hours 
discussing topics related to my thesis or to the formal methods area at large. He was one
of the people that invigorated my interest in programming language theory that has
proven very valuable for my research work and my post-doctoral career developments.

Prof. Zachos, my external committee member, has been one of the most important
chapters of academic life. It is thanks to him that I attended the PhD program at CUNY
and his guidance in all walks of my academic and non-academic life has contributed 
the most  to the successful completion of my PhD. He has been an inspiring
scientist, a great mentor and a great friend and also part time roommate through the last fifteen years that I have 
known him. 

During my studies at the Graduate Center I had the chance to meet 
prominent scientists both from CUNY and worldwide. 
These people helped me expand my knowledge and improve
the quality of my work. I would like to thank explicitly Rohit Parikh, Thomas Wies,  
 Mark Tuttle and the late Kristofer Rose for inspiring classes, discussions and collaborations.
In particular, I want to give special thanks to Giuseppe Primiero who believed in this work
from early on and co-authored the first related publication with me.

In addition, I want to thank my wonderful friends both in NYC and back home as well as 
amazing fellow graduate students. I want to thank Danai Dragonea, Dafni Anesti, Antonis Bouchagiar 
and Michalis Agathos for taking care of me every summer after my teaching obligations
had turned me to a jaded new yorker. All our beautiful excursions into under the gorgeous
greek sun have helped me stay happy and healthy throughout intense academic years.
I want to thank Krystal Raydo and George Rabanca for being a family and showing me immense support
through the thick and thin in New York. 
Cagil Tasdemir has been the best person I have ever met and a great source of both knowledge and wisdom.
I want to thank fellow graduate students: Marios Georgiou,
Nikos Melissaris and Ivo Vigan for our great time inside and outside school.

Finally, I want to thank my kinfolk. Wilson Sherwin's love and understanding has been one of my
most important motivations for completing this work and looking forward to our future together.
My family Aliki, Doria, Vaso and Spyros deserve most of the credit for all my achievements including
this work. I had the luck to have two amazing sisters and two loving parents and that's all the wealth 
a man could ask for. 

This dissertation is dedicated to my late grandparents and grandaunt: 
Aliki Giannioti, a tuberculosis survivor, Theodoros (Lolos) Gianniotis, a fisherman, and
Labro Mitsiali, a factory  worker, for growing me up at their home in Anemomylos, Corfu and showing me that can build beauty with as little as
a small house, a well-kept garden painted fresh white in the spring and a tiny fish 
boat.



\tableofcontents
\chapter{Introduction}\label{intro}
The \acrfull{CHI}~\cite{curry1934functionality,howard1995formulae} was first established as a deep connection between explicit proofs in intuitionistic logic and programs of a simple programming 
language that includes pairs, functions and union types ~\cite{Pierce:2002:TPL:509043,Srensen98lectureson}. This relation has been a central topic of study in the field of type theory and has turned into the standard
foundational approach to studying and designing programming languages especially of the functional paradigm. Since  this relation has been  established, the 
isomorphism has been extended to more complex logics and correspondingly to more complex programming language constructs.  
In the following I will be using \acrfull{CHI} and \emph{proofs-as-programs} interchangeably.

There are great benefits both for a logician
and the programming language designer in viewing things through the lenses of such a relation. From the programming language perspective certain linguistic phenomena are given categorical characterizations that do not 
depend on implementation specifics. 
For example the designer of the next hot programming language  
knows that adding pairs would have to adhere to the corresponding constructs of 
logical conjunction. In addition, adding more complex design features 
(e.g. state, concurrency, exceptions etc) can be done in a structured, orthogonal, 
and modular way by enriching the underlying logic  and, 
correspondingly, the type system(see e.g. ~\cite{Harper:2012:PFP:2431407,CERVESATO20091044,Ong:1997:CFF:263699.263722,DBLP:conf/popl/Griffin90}).

It should not be a surprise that languages of the typed functional paradigm  
have been gaining traction and more functional design principles are being added 
to languages of the object oriented paradigm. There are two main reasons for this, 
an old and a new. The older reason is mathematical correctness which is strongly 
related to the fact that reasoning about programs (of the lambda calculus and its extensions) 
can be done in an \textit{equational way}, a property that is heavily connected 
to their underlying foundational principles as we will see. Features such as side effects 
or concurrency in the language are reflected via typing. For example,  
a program that changes global state has a type that  says so explicitly. Similarly,  
a program that uses goto mechanisms  would also say register its behavior in its type. 
Alas, even ``unpure", non-functional constructs (state, mutable references) are added 
in a mathematical/ algebraic fashion under the \ac{CHI} disciple. 
As a result, reasoning about properties of such programs is significantly simpler. 
Moreover, under a strongly typed doctrine, important properties of programs are checked 
statically by the type-checker and prior to their execution. 
Henceforth, the need for testing is reduced to verify only non-trivial properties. 


The renewed interest in functional programming owes a lot to the
 difficulties of scaling concurrent programs in traditional programming paradigms. 
 It is very hard  to scale programs that make unlimited use of side effects 
 (such as state change) in an implicit way (i.e. without leaving any trace in their typing)
  from sequential to  multithreaded style of computation. Programming freedoms in traditional 
  languages (paired with the easiness and textbook familiarity of the Von Neumann model) come 
  at a large cost if one takes into account the need for proving 
  correctness. The ``purity" of
   programs in the lambda calculus -- and the delimited ``impurity" in its extensions -- makes 
   writing high-quality concurrent code an easier task. 
   It is exciting to see that important metatheoretic results in the area of combinatory logic as e.g. the Church--Rosser 
   property are the backbone of models for concurrent computation in modern functional languages. 



On the other hand, the logician has good reasons to study logics as rules of program formation and reduction. First of all, 
such designs make logics implementable ``for free"
in modern theorem provers using the programmistic side of the correspondence. 
Secondly the study of logic in such a way has put upfront a  Gentzen-style treatment of logical 
connectives where emphasis is given to the notion of proof, proof structure and proof reduction. 
This has sparked studies for more refined versions of proof relevant deduction than the ones 
discovered under the standard ``axiomatic" approach (e.g. linear logics, substructural logics etc). 
As we will see, the Gentzen-- Brouwer initiative in  logic  does not merely call for change of 
axiomatization but for a ``proof relevant" interpretation of connectives that comes with a
computational taste. Metatheory is also standardized once one studies logic this way;  
scalable techniques have been developed within the area of ``proof-theoretic" semantics 
that make the passing from  natural deduction of a logic to its cut free calculus pretty standard ~\cite{Sieg1998,pfenning2000structural}. 
In other words, treating logics within a Curry -- Howard environment 
enriches logic as a discipline with good organizing principles. 
Finally, proof relevant treatments of logic -- pushed further by ideas of 
\emph{Martin-L\"{o}f Type Theory}~\cite{martin1984intuitionistic} 
have sparked  a renewed interest in  a foundation of mathematics that stems from 
a  treatment of proofs as \textit{the} primitive objects of mathematics. 


In this work, we are interested in the study of  extending  \acrfull{CHI} with basic 
constructive  necessity of justification logic. 
There is a good reason to believe that this should be doable. Justification logic
 is a logic that relates logical necessity with the existence of a proof construct
  and that is exactly what working in  realm of proof relevancy and \ac{CHI} calls for. 
  There are challenges to this task, both syntactical and  semantical. First of all, there
   is a resemblance of the justification logic syntax with that of simple type theory 
   (e.g. the use of the semicolon $a:A$) that  initially might call for an antagonistic relation
    between the two systems. Of course, this is not a substantial issue since the two
     typing relations can be ``colored" in a syntactical way. 
     But resolving the syntactical overload would still leave a  ``meaning" question open;
      namely, how can one read  the need of having two
       proofs of the ``same thing" in a system. 
       The main contribution of this work  
       ~\cite{Pouliasis2016} it that it  shows how such a relation of binding two kinds of
        proof systems is quite natural and gives a basic reading of validity and necessity 
        on first, proof-theoretic principles.  
        We will treat justification logic as a logic of \textit{proof relevant validity}.
         by tracing justification logic back to its 
         origin as an explicit, classical semantics to \ac{BHK} proof constructs. 
         We will present a modal logic that is based on this relation and we will argue
          that such phenomena of binding two kinds of constructions abound both
           in the realm  of mathematical proofs 
           but also in the realm of programming languages  with constructs 
           such as modules, foreign function calls and dynamic linkers.

           This work  is structured in three parts: 
           The first part (\cref{intui,lambda,jllogic}) is a revision of my second examination paper and constitutes 
           a bird's eye view into my 
           research topics:  
           \emph{intuitionistic logic, justified modality} and \emph{type theory}. 
           The relevant systems are introduced syntactically together with main 
            metatheoretic proof techniques which will be useful 
           in the rest of the thesis.
           The second part (\cref{proposal, jcalccom, jcalcsem}) 
           constitutes my main contributions.
           I will propose  
           a modal type system that extends simple type theory
            (or, speaking from the logical side of \ac{CHI}, 
           intuitionistic propositional logic) with elements of
            justification logic and will argue about its computational significance. 
            More specifically, I will show  
           that the obtained calculus characterizes  certain 
           computational phenomena that abound in modern programming language semantics. 
           I will present full metatheoretic
           results obtained for this logic/ calculus utilizing techniques from the first part 
           and will provide proofs in the Appendix. 
           In the Appendix, the interested reader can find links to the active Github repo 
           that contains an implementation of  this calculus 
           (its term and type systems)
           in the metaprogramming framework \say{Makam} 
           as an additional  ``proof of concept'' result,  
           Finally, I conclude this work with a small ``outro'', Chapter~\ref{ext},
           where I informally show that the ideas underlying my contributions 
           can be extended in interesting ways.




\chapter{Intuitionistic Logic}\label{intui}
\section{Intuitionism}\label{sec:intrui}
In this and the subsequent chapter, I will be presenting foundational work in the intersection of \emph{Intuitionistic Logic} and \emph{Type Theory}. 
The presentation is scaffolding following Robert Harper's lecture videos in \emph{Homotopy Type Theory}~\cite{HarperHOTT} and the accompanying notes by students of the class~\cite{HOTTNotes1}. I will often deviate to standard textbooks in the field ~\cite{Barendregt1984-BARTLC},~\cite{citeulike:993095},~\cite{Pierce:2002:TPL:509043} to present further important results. 
\subsection{A bird's eye view}  
In a nutshell, \emph{Intuitionistic mathematics}  is a program in foundations 
of mathematics  that extends \emph{Brouwer's program}~\cite{brouwer1975collected}.
Brouwer, in an almost Kantian fashion, viewed mathematical reasoning as a human faculty 
and mathematics as a language of the ``creative subject"
aiming to communicate mathematical concepts. 
The concept of \emph{algorithm} as a step--by--step constructive process is brought in the 
foreground in Brouwer's program. As a result, intuitionistic theories are amenable to 
computational interpretations. In the following I will be using the terms intutionistic 
and \emph{constructive} interchangeably.  

For the purposes of this paper, the main diverging point of Brouwer's program, 
later explicated by Heyting~\cite{heyting1966intuitionism} and Kolmogorov~\cite{kolmogorov1925principe}~\cite{artemov2004kolmogorov}, lies in the treatment of proofs. In contrast to classical approaches to foundations 
that treat proof objects as external to theories, the constructive approach treats proofs 
as the fundamental forms of construction and hence, as first class citizens. 
As a result, the constructive view of logic draws heavily from proof theory
and Gentzen's developments~\cite{gentzen1970collected}. 
For the reader interested also in the philosophical implications  
of constructive foundations and \emph{antirealism}, 
Dummet's treatment is a classic in the field~\cite{dummett2000elements}.    
 


It has to be emphasized that proofs in the intuitionistic approach 
 are treated as stand--alone and are not bound to formal systems 
(i.e the notion of proof \textit{precedes} that of a formal system). 
It is necessary, hence, to draw a distinction between the notion of 
 \emph{proof as construction} and the typical notion of \emph{proof in a formal system} 
 ~\cite{Harper2013,Harper2012}.

A \emph{formal proof}
is a proof given in a fixed formal system, such as Peano Arithmetic, and arises
from the application of the inductively defined rules in that system. Formal proofs can, thus, be viewed as strings or g\"{o}delizations of textual derivation in some fixed system. 

Although every formal proof (in a specific system)
is also a proof (assuming soundness of the system) the converse is not true.  
This conforms with G\"{o}del's Incompleteness Theorem, which precisely states that there
exist true propositions (with a proof in \emph{some} formal system), but for which 
there cannot be given a formal proof within the formal system in question. 
This \emph{openness} of the nature of proofs is necessary for a foundational 
treatment of proofs that respects  G\"{o}delian phenomena.

Following the same line of thought, and adopting the doctrine of \emph{proof relevance} 
for obtaining true judgments, leads to another main difference of the constructive 
approach and the classical one i.e the (default) absence of the 
\emph{law of excluded middle}. 


\section{IPL}
\ac{IPL} can be viewed as ``the logic of \emph{proof relevance}" conforming with the intuitionistic view described in \ref{sec:intrui}. To judge a fact as \emph{true} one may provide a \emph{proof}
appropriate of the fact. \emph{Proofs} can be synthesized to obtain 
proofs for more complex facts (\emph{introduction rules}) and consumed
 to provide proofs relevant for other  facts (\emph{elimination rules}). The importance of the interplay between introduction and elimination rules was developed by Gentzen. 
 A discussion on the meaning of the logical connectives that is prevalent in \emph{MLTT} can be found in \cite{martin1996meanings} 
Following the presentation style  by   Martin-L\"{o}f we split the notions of \emph{judgment} and \emph{proposition}. We have two main kinds of judgments:
\begin{itemize}
\item  \emph{Judgments} that are logical arguments about the truth(or, equivalently, proof) of a \emph{proposition}. They might, optionally, involve assumptions on the truth (or, equivalently, proof) of other propositions. We might call these \emph{logical judgments}. 

\item Judgments on \emph{propositionality} or typeability. \emph{Propositions} are the \emph{subjects}  of \emph{logical judgments}. If something is judged to be a proposition then it belongs to the universe of discourse and can be mentioned in \emph{logical judgments}. 
\end{itemize} 
In addition, since a \emph{logical judgment} might involve a set $\Gamma$ of assumptions (or a \emph{context}), it is convenient to add a third kind of judgment of the form $\Gamma \context$ 
Thus, in \ac{IPL}, we get the judgments $\phi\ \in \prop$, $\phi \true$ and $\Gamma \context$:
\begin{alignat*}{2}
  \phi \in \prop &&\quad& \text{$
  \phi$ is a (well-formed) proposition} \\
  \phi\  \true &&& \text{\begin{tabular}[t]{@{}l@{}}
                Proposition $\phi$ is true \\
                i.e., has a proof.
              \end{tabular}}\\
  \Gamma \context &&\quad& \text{$\Gamma$ is a (well-formed) context of assumptions} \\
\end{alignat*}

The natural deduction system of \ac{IPL} is given below:


\begin{mdframed}
\textbf{Prop Formation}
\begin{mathpar}
\inferrule*[right=Atom] { } {P_i \in {\sf Prop}}
\and
\inferrule*[right=Top] { } {\top \in {\sf Prop}}
\and
\inferrule*[right=Bottom] { } {\bot \in {\sf Prop}}
\and
\inferrule*[right=Arr] {{\phi_1 \in {\sf Prop }}\\ {\phi_2 \in {\sf Prop}}} {\phi_1\supset\phi_2\in {\sf Prop}}
\and
\inferrule*[right=Conj] {{\phi_1 \in {\sf Prop }}\\ {\phi_2 \in {\sf Prop}}} {\phi_1\wedge\phi_2\in {\sf Prop}}
\and
\inferrule*[right=Disj] {{\phi_1 \in {\sf Prop }}\\ {\phi_2 \in {\sf Prop}}} {\phi_1\vee\phi_2\in {\sf Prop}}

\end{mathpar}
\end{mdframed}

\begin{mdframed}
\textbf{Context Formation}
\begin{mathpar}
\inferrule*[right=Nil] { } {{\sf \circ}\  \context}
\and
\inferrule*[right=$\Gamma$-Ext] {{\Gamma\ } {\sf \context}  \\ {\phi \in {\sf Prop}}} {{\Gamma, \phi \true} \ \context}
\end{mathpar}
\end{mdframed}

\begin{mdframed}
\textbf{Context Reflection}
\begin{mathpar}
\inferrule*[right=$\Gamma$-Refl] { {\Gamma}\  {\sf \context}\\ {\phi \true \in \Gamma}}{\Turnsi {\Gamma} {\phi \true}}
\end{mathpar}
\end{mdframed}

\begin{mdframed}
\textbf{Top Introduction -- Bottom Elimination}
\begin{mathpar}
\inferrule*[right=$\top$I] { } {\Turnsi {\Gamma} { \top \true}}
\and
\inferrule*[right=$\bot$E] {\Turnsi {\Gamma} {\bot \true} } {\Turnsi {\Gamma} {  \phi \true}}
\end{mathpar}
\end{mdframed}

\begin{mdframed}
\textbf{Implication Introduction and Elimination}
\begin{mathpar}
\inferrule*[right=$\supset$I] {\Turnsi {\Gamma, \phi_1 \true} {\phi_2 \true}} {\Turnsi {\Gamma} { \phi_1\supset \phi_2 \true}}
\and
\inferrule*[right=$\supset$E] {\Turnsi {\Gamma} {\phi_1\supset\phi_2 \true}\\{\Turnsi {\Gamma} {\phi_1 \true}}} {\Turnsi {\Gamma} {  \phi_2 \true}}
\end{mathpar}
\end{mdframed}
\begin{mdframed}
\textbf{Conjunction Introduction and Elimination}
\begin{mathpar}
\inferrule*[right=$\wedge$I] {\Turnsi {\Gamma} {\phi_1\true}\\{\Turnsi {\Gamma} {\phi_2 \true}}} {\Turnsi {\Gamma} {  \phi_1 \wedge\phi_2 \true}}
\end{mathpar}
\begin{mathpar}
\inferrule*[right=$\wedge$El] {\Turnsi {\Gamma} {\phi_1\wedge\phi_2 \true}} {\Turnsi {\Gamma} {  \phi_1\true}}
\and
\inferrule*[right=$\wedge$Er] {\Turnsi {\Gamma} {\phi_1\wedge\phi_2 \true}} {\Turnsi {\Gamma} {  \phi_2\true}}
\end{mathpar}
\end{mdframed}
\begin{mdframed}
\textbf{Disjunction Introduction and Elimination}
\begin{mathpar}
\inferrule*[right=$\vee$Il] {\Turnsi {\Gamma} {\phi_1 \true}} {\Turnsi {\Gamma} {  \phi_1\vee\phi_2\true}}
\and
\inferrule*[right=$\vee$Ir] {\Turnsi {\Gamma} {\phi_2 \true}} {\Turnsi {\Gamma} {  \phi_1\vee\phi_2\true}}
\end{mathpar}


\begin{mathpar}
\inferrule*[right=$\vee$E] 
{ {\Turnsi {\Gamma} {  \phi_1\vee\phi_2\true}}\\
{\Turnsi {\Gamma,\phi_1 \true} {\phi \true}}\\
{\Turnsi {\Gamma,\phi_2 \true} {\phi \true}}
}
 {\Turnsi {\Gamma} {\phi \true}}
\end{mathpar}

\end{mdframed}


\subsection{Basic Properties of Intuitionistic Entailment}
\label{ssec:entail}
			\begin{mdframed}
			\textbf{Reflexivity}
			    
				\begin{mathpar}
			   \inferrule*[] 
			    { }
			    {\Turnsi {\Gamma,\phi\true} {\phi \true}} 
				\end{mathpar}
		  \end{mdframed}

		\begin{mdframed}
		\textbf{Transitivity}
			\begin{mathpar}
					   \inferrule*[] 
					    {\Turnsi {\Gamma} {\psi \true}\\ {\Turnsi {\Gamma,\psi\true}{\phi\true}}}
					    {\Turnsi {\Gamma,\phi\true} {\phi \true}} 
						\end{mathpar}
				\end{mdframed}
   			
   			\begin{mdframed}
   			\textbf{Contraction}
   						\begin{mathpar}
   								   \inferrule*[] 
   								    {\Turnsi {\Gamma,\phi\true,\phi \true
   								   } {\psi \true}} {\Turnsi {\Gamma,\phi \true}{\psi\true}}
   								    
   									\end{mathpar}
   			\end{mdframed}
		\begin{mdframed}{Exchange}
					\begin{mathpar}
							   \inferrule*[] 
							    {\Turnsi {\Gamma
							   } {\phi \true}} {\Turnsi {\operatorname{\pi}(\Gamma)}{\phi\true}}
							    \end{mathpar}
                  Where $\pi(\Gamma)$ is a meta-symbol standing for any permutation of $\Gamma$.
                \end{mdframed}

\section{Order Theoretic Semantics: \vocab{Heyting Algebras}}\label{ha:ax}
\vocab{IPL} viewed order theoretically gives rise to a \vocab{Heyting  Algebra(HA)}. 
To define \vocab{HA} we need the notion of a \emph{lattice}.
 For our purposes we define it as follows\footnote{One can take a lattice being a partial order. The same results hold with slight modifications.}: 
  

\begin{mdframed}
\textbf{Definition:}
A \textit{lattice} is a non-empty \emph{pre--order} with finite meets and joins.
\end{mdframed}
In addition, we define \emph{bounded lattice} as follows: 
\begin{mdframed}
\textbf{Definition:}
A \textit{bounded lattice} $(L,\le)$ is a lattice that additionally has a greatest element 1 and a least element 0, which satisfy

$0\le x \le 1$ for every $x$ in $L$
\end{mdframed}
Finally, we can define \emph{HA}:

\begin{mdframed}
\textbf{Definition:}
A \textit{HA} is a bounded lattice $(L,\le,0,1)$ s.t. for every $a,b\in L$ there exists an $x$ (we name it $a\rightarrow b$) with the properties: 
\begin{enumerate}
\item $a\wedge x\le b $
\item $x$ is the greatest such element
\end{enumerate}
\end{mdframed}
\subsubsection{Axiomatization of HAs}
We can axiomatize the meet (i.e. greatest lower bound)($\wedge$) of $\phi,\psi$ for any  lower bound $\chi$.
\begin{mdframed}
\begin{mathpar}
  \infer{\phi \conj \psi \leq \phi}{
    }
  \and
  \infer{\phi \conj \psi \leq \psi}{
    } 
\end{mathpar}
\begin{equation*}
  \infer{\chi \leq \phi \conj \psi}{
    \chi \leq \phi & \chi \leq \psi} 
\end{equation*}
\end{mdframed}

We can axiomatize the join ($\vee$)(i.e. the least upper bound) of $\phi,\psi$ for any upper bound $\chi$ as follows .
\begin{mdframed}
\begin{mathpar}
  \infer{\phi  \leq \phi\vee \psi}{
    }
  \and
  \infer{\psi \leq \phi \vee \psi}{
    } 
\end{mathpar}
\begin{equation*}
  \infer{\phi \vee \psi \leq \chi}{
    \phi \leq \chi & \psi \leq \chi} 
\end{equation*}
\end{mdframed}
We can axiomatize the existence of a greatest element as follows:
\begin{mdframed}
\begin{equation*}
  \infer{\chi \leq 1}{
    } 
\end{equation*}
which says that $1$ is the greatest element.
\end{mdframed}

We can axiomatize the existence of a least element as follows:
\begin{mdframed}
\begin{equation*}
  \infer{0 \leq \chi}{
    } 
\end{equation*}
which says that $0$ is the least element.
\end{mdframed}
Finally, to axiomatize \emph{HAs} we require the existence of exponentials for every $\phi$, $\psi$ as follows:

\begin{mdframed}
\begin{mathpar}
 

  \infer{\phi \wedge  (\phi\supset \psi)\leq\psi}{
    } 
    \and
    \infer{\chi\leq\phi\supset\psi}{\phi\wedge\chi\leq\psi}
\end{mathpar}
\end{mdframed}

\subsubsection{Soundness and Completeness}

\begin{mdframed}
\begin{theorem}\label{thm:cmpha}
$\Gamma\vdash_{IPL} \phi \true$ iff for any \vocab{Heyting Algebra} $H$ we have $\Gamma^+\leq\phi^{*}$ where $*$ is  defined as the lifting of any map of $\prop$s to elements of $H$ and $(+)$ is defined inductively on the length of $\Gamma$ as follows
\begin{alignat*}{2}
  {\sf \circ}^+  &&\quad = & \quad\top\\
  (\Gamma,\phi)^+&&\quad = &\quad
  \Gamma^+\wedge\phi* \
\end{alignat*}
\end{theorem}
\end{mdframed}

\chapter{Typed lambda calculus}\label{lambda}
\section{From intuitionistic provability to  proof trees}
\vocab{IPL} can be viewed as a declarative axiomatization of proof constructs. Take the introduction rule for conjunction as an example: 
\begin{mathpar}
	\inferrule*[right=$\wedge$I] {\Turnsi {\Gamma} {\phi_1\true}\\
	{\Turnsi {\Gamma} {\phi_2 \true}}} {\Turnsi {\Gamma} 
	{  \phi_1 \wedge\phi_2 \true}}
\end{mathpar}

The rule says, ``given the existence a proof of $\phi$ and a proof of $\psi$ from assumptions $\Gamma$, there exists a proof of $\phi\wedge\psi$ from assumptions $\Gamma$ at hand ".

We used the description ``declarative" because in this format \vocab{IPL} sequents $\Gamma\vdash \true $ do not describe how such existentials are realized. It is in essence a logic of ``proof relevant truth" but it does not involve the proofs themselves as first class objects. 

An alternative presentation is to explicate proof constructs by directly providing a system of ``proof trees". Such an approach was actually championed in Gentzen's natural deduction systems and is the necessary move to obtain proof calculi. 
Once we have  explicit  proof objects (either as trees, or as we will see, as terms) the system is enriched with equality principles involving such objects. Such rules give computational value (``proof dynamics") to the constructs  and are the driver idea in the  ``Curry--Howard Isomorphism" and its extensions.

Here we provide such a  formulation in proof trees of judgments together with the equality rules on trees, essentially following Gentzen. Proof trees of judgments have the following shape:

\begin{mathpar}
\inferrule*[vdots=1.5em]{}{ J_1,\ldots,  J_i}
\end{mathpar}
\begin{mathpar}
\inferrule*[]{}{J}
\end{mathpar}
We focus one judgments $J$ of the form ${A \true}$.
Here are the the rules for constructing proof trees with labeled assumptions\footnote{Essentially the constructs are directed acyclic graphs since assumptions with the same label are ``bind" and substitutable together but we will be cavalier with such a details}.
\begin{mathpar}
\begin{array}[b]{c}{ x_1:A_1 \true,\ldots,  x_i:A_i\true}\\{\vdots}\\{A_{j\in 1\dots i} \true}
\end{array}
\and
\begin{array}[b]{c}{ x_1:A_1 \true, \ldots,   x_i:A_i\true}\\{\vdots}\\{\top\true}
\end{array}
\end{mathpar}

\begin{mathpar}
	\inferrule[]{{\begin{array}[b]{c} {\mathcal{D}} \\ A\true  \end{array}}\\ {\begin{array}[b]{c} {\mathcal{E}} \\ B\true  \end{array}}}{A \wedge B \true}
	
\end{mathpar}
\begin{mathpar}
	\inferrule[]{{\begin{array}[b]{c} {\mathcal{D}} \\ A\wedge B\true  \end{array}} }{A \true}
	\and
	\inferrule[]{{\begin{array}[b]{c} {\mathcal{D}} \\ A\wedge B\true  \end{array}} }{B \true}
\end{mathpar}
\begin{mathpar}
	
\end{mathpar}


\begin{mathpar}
	\inferrule[]{{\begin{array}[b]{c} {\mathcal{D}} \\ A \true \end{array}} }{A\vee B \true}
	\and
	\inferrule[]{{\begin{array}[b]{c} {\mathcal{D}} \\ B\true  \end{array}} }{A\vee B \true}
\end{mathpar}
\begin{mathpar}
	
\end{mathpar}
\begin{mathpar}
	\inferrule[]{{\begin{array}[b]{c}  {\mathcal{D}} \\ A\vee B \true \end{array}} \\ {\begin{array}[b]{c} {A\true}\\ {\mathcal{E}} \\ C \true \end{array}}\\
	{\begin{array}[b]{c} {B\true}\\ {\mathcal{F}} \\ C \true \end{array}		 }}{C\true}
\end{mathpar}

\begin{mathpar}
	\inferrule[]{
	{\begin{array}[b]{c}  {\mathcal{D}} \\ \bot \true \end{array}}}{C \true}
\end{mathpar}
%\begin{mathpar}
%\inferrule*[]{\PrTri{D}} {\PrTri{E} }
		
%\end{mathpar}
%\begin{mathpar}
%{\inferrule*[]{}{ \inferrule*[]	{\inferrule*[]{}{{D} \\ {E}}\\\\
			%\inferrule*[]{}{\    A\\ \ \qquad %\llbracket   A\rrbracket}}{\Box   A }}}
%\end{mathpar}
\subsection{Properties of Intuitionistic Entailment Redux}
Proof trees by their nature satisfy the properties of entailment in \ref{ssec:entail}. We will not bother with reflection and contraction. The first is trivial and the second can be shown by simple induction on the structure of trees with the proof highlighting  that reflection on hypothesis is order-irrelevant. Transitivity is established by \textit{compositionality} of proof trees and reflects the essence of hypothetical reasoning: proof trees of the appropriate proposition can be ``plugged in" for assumptions to create new valid trees.  
\begin{mdframed}
	\begin{theorem}\label{thm:cmpha}
		If {$\begin{array}{c}{ x:A }\\{\mathcal{D}}
			\\{B \true}
			\end{array}$}
		and {$\begin{array}{c}\PrTri{$\mathcal{E}$}\\{A \true}\end{array}$} are valid proof trees their composition denoted as {$\begin{array}{c}\PrTri{$\mathcal{E}$}\\{A \true}\\{\mathcal{D}}\\{B\true}\end{array}$},  defined by substituting all occurrences of $x:A$ for $E$ in $\mathcal{D}$, is a valid proof tree for $B \true$. 
	\end{theorem}
\end{mdframed}
\subsection{Equating Proof Trees}
Having proof objects as first class citizens, permits for developing logics, essentially, 
as theories of (typed) equality among such objects. 
This idea stemmed from Gentzen's work on natural deduction and 
cut elimination and it is what gives to proofs  computational content. 
Here are the proposed  equalities for the proof relevant \vocab{IPL} introduced initially 
by Gentzen as the driver of the proof  cut elimination. 
We will be revisiting these very same equalities and reframe them as equalities among
 proof terms in the next section. Nevertheless, they can be expressed in proof tree form.  
 We show indicatively the equalities regarding the $\supset,\wedge$ connectives reserving the
  rest for the more concise notation.


\begin{mathpar}
	\begin{array}{c c c}
	{\inferrule{
	{\inferrule[]{
		{\begin{array}[b]{c} \overline{x:A}\\ {\mathcal{D}} \\ B \true \end{array}}}{A \supset B \true}}\\
{\begin{array}{c}\mathcal{E}\\{A \true}\end{array}}
}
{ B \true}} & =_{\beta} &
	
			{\begin{array}[b]{c} \mathcal{E} \\ A \true \\ {\mathcal{D}} \\ B \true \end{array}} 

\end{array}
\end{mathpar}
%{\begin{array}[b]{c}  {\mathcal{D}} \\ A\supset B \true \end{array}}
%
%
\begin{mathpar}
	\begin{array}{c c c}
	{\begin{array}[b]{c}  {\mathcal{D}} \\ A\supset B \true \end{array}}
			& =_{\eta} &
		
			{\inferrule{
				{\inferrule[]{
						{\begin{array}[b]{c}  {\mathcal{D}} \\ A\supset B \true \end{array}}
							\\{\overline{x:A}}} 
							{ B \true}}}
				{A \supset B}
		}
		
	\end{array}
\end{mathpar}


\begin{mathpar}
	\begin{array}{c c c}
		{\inferrule{
				{\inferrule[]{
						{\begin{array}[b]{c}  {\mathcal{D}} \\ A \true \end{array}}
							\\{\begin{array}[b]{c}  {\mathcal{E}} \\  B \true \end{array} }
						} 
						{ A\wedge B \true}}}
				{A \true}
		}
			& =_{\beta} &
		
		{\begin{array}[b]{c}  {\mathcal{D}} \\ A\true \end{array}}
		
		
	\end{array}
\end{mathpar}

\begin{mathpar}
	\begin{array}{c c c}
		{\inferrule{
				{\inferrule[]{
						{\begin{array}[b]{c}  {\mathcal{D}} \\ A \true \end{array}}
							\\{\begin{array}[b]{c}  {\mathcal{E}} \\  B \true \end{array} }
						} 
						{ A\wedge B \true}}}
				{B \true}
		}
			& =_{\beta} &
		
		{\begin{array}[b]{c}  {\mathcal{E}} \\ B\true \end{array}}
		
		
	\end{array}
\end{mathpar}

\begin{mathpar}
	\begin{array}{c c c}
		{\begin{array}[b]{c}  {\mathcal{D}} \\ A\wedge B \true \end{array}}
		& =_{\eta} &
		\inferrule{
		\inferrule[]{\inferrule {\mathcal{D}}{ A\wedge B \true}}
		{A \true}\\\inferrule[]{\inferrule {\mathcal{D}}{ A\wedge B \true}}
		{A \true}}{A\wedge B}
	\end{array}
\end{mathpar}


\section{Linear representation of trees  with proof terms: $\lambda$ calculus}
Proof terms provide an alternative linear representation for proof trees. The simply typed lambda calculus and its equational system can, thus, be viewed as a calculus for proof trees and proof reductions for  intuitionistic logic. What's more, following the doctrine of proof relevance and of characterizing connectives by their proof reductions, i.e. working in the realm of Curry -- Howard Isomorphism, we hit two birds with one stone: we both develop proof relevant logics and we get typed programming languages that reflect their computational content. The ``simplest" language obtained within this program is the simply typed lambda calculus, but we will see that the same doctrine  extends to different logics with different judgmental constructs.
  
\subsubsection{Simply typed lambda calculus}
\begin{mdframed}
\textbf{Type Formation}
\begin{mathpar}
\inferrule*[right=Atom] { } {P_i \in {\sf Type}}
\and
\inferrule*[right=Top] { } {\top \in {\sf Type}}
\and
\inferrule*[right=Bottom] { } {\bot \in {\sf Type}}
\and
\inferrule*[right=Arr] {{\phi_1 \in {\sf Type }}\\ {\phi_2 \in {\sf Type}}} {\phi_1\rightarrow\phi_2\in {\sf Type}}
\and
\inferrule*[right=Prod] {{\phi_1 \in {\sf Type }}\\ {\phi_2 \in {\sf Type}}} {\phi_1\times \phi_2\in {\sf Type}}
\and
\inferrule*[right=Union] {{\phi_1 \in {\sf Type }}\\ {\phi_2 \in {\sf Type}}} {\phi_1 + \phi_2\in {\sf Type}}
\end{mathpar}
\end{mdframed}

\begin{mdframed}
\textbf{Context Formation}
\begin{mathpar}
\inferrule*[right=Nil] { } {{\sf nil}\  \context}
\and
\inferrule*[right=$\Gamma$-Add] {{\Gamma\ } {\sf \context}  \\ {\phi \in {\sf Type}}\\ x\text{ fresh  in }\Gamma} {{\Gamma,\  x:\phi} \ \context}
\end{mathpar}
\end{mdframed}

\begin{mdframed}
\textbf{Context Reflection}
\begin{mathpar}
\inferrule*[right=$\Gamma$-Refl] { {\Gamma}\ {\sf \context}\\ {x:\phi  \in \Gamma}}
{\Turnsi {\Gamma} {x:\phi}}
\end{mathpar}
\end{mdframed}

\begin{mdframed}
\textbf{Top Introduction -- Bottom Elimination}
\begin{mathpar}
\inferrule*[right=$\top$I] { } {\Turnsi {\Gamma} { \langle \rangle:\top }}
\and
\inferrule*[right=$\bot$E] {\Turnsi {\Gamma} {M:\bot } } {\Turnsi {\Gamma} { abort[\phi](M) :\phi}}
\end{mathpar}

\end{mdframed}

\begin{mdframed}
\textbf{Function Construction and Application}
\begin{mathpar}
\inferrule*[right=$\lambda-$Abs] {\Turnsi {\Gamma, x:\phi_1 } {M:\phi_2 }} {\Turnsi {\Gamma} { \lambda x. M:\phi_1\rightarrow \phi_2 }}
\and
\inferrule*[right=App] {\Turnsi {\Gamma} {M:\phi_1\rightarrow\phi_2 }\\{\Turnsi {\Gamma} {M^{'}:\phi_1}}} {\Turnsi {\Gamma} {  (M M^{'}):\phi_2 }}
\end{mathpar}
\end{mdframed}
\begin{mdframed}
\textbf{Tuple Construction and Projections}
\begin{mathpar}
\inferrule*[right=Tup] {\Turnsi {\Gamma} {M:\phi_1}\\{\Turnsi {\Gamma} { M^{'}:\phi_2}}} {\Turnsi {\Gamma} {  
\langle M,M^{'}\rangle:\phi_1\times \phi_2 }}
\end{mathpar}
\begin{mathpar}
\inferrule*[right=LPrj] {\Turnsi {\Gamma} {M:\phi_1\times\phi_2 }} {\Turnsi {\Gamma} {\operatorname{fst}(M):  \phi_1}}
\and
\inferrule*[right=RPrj] {\Turnsi {\Gamma} {M:\phi_1\times\phi_2}} {\Turnsi {\Gamma} {\operatorname{snd}(M):  \phi_2}}
\end{mathpar}
\end{mdframed}
\begin{mdframed}
\textbf{Union Construction and Elimination}
\begin{mathpar}
\inferrule*[right=InjL] {\Turnsi {\Gamma} {M:\phi_1}} {\Turnsi {\Gamma} {  inj_l[\phi_2](M):\phi_1+\phi_2}}
\and
\inferrule*[right=InjR] {\Turnsi {\Gamma} {M:\phi_2 }} {\Turnsi {\Gamma} {inj_r[\phi_1](M) : \phi_1+\phi_2}}
\end{mathpar}


\begin{mathpar}
\inferrule*[right=$\vee$E] 
{ {\Turnsi {\Gamma} {  M:\phi_1 + \phi_2}}\\
{\Turnsi {\Gamma,x:\phi_1 } {N:\phi }}\\
{\Turnsi {\Gamma,y:\phi_2} {O:\phi }}
}
 {\Turnsi {\Gamma} {\text{case }M \text{ of } inj_l(  x)\longmapsto N  |\  inj_r (y)\longmapsto O : \phi}}
\end{mathpar}

\end{mdframed}
\subsection{Definitional Equality: Proof tree equalities as term equalities}

Gentzen's principles transliterate to an equational system for terms. In the following we are defining a congruence relation on proof terms  which is usually coined as \emph{definitional equality} and denoted $M\equiv M':A$. We want definitional equality $\equiv$ to be the least congruence closed under the $\beta, \eta$ rules that directly reflect Gentzen's principles in term form.


\begin{mdframed}
	\textbf{Definition}
     A \emph{congruence} is 
     \begin{itemize} 
     \item	an equivalence relation  (i.e. reflexive, symmetric and transitive) 
     \item	that commutes with  operators  E.g. 
     \begin{mathpar}
     	\inferrule*[]
     	{\Gamma \entails M\equiv M':A\times B}
     	{\Gamma\entails \operatorname{fst}(M)\equiv\operatorname{fst}(M'):A}
     	\end{mathpar} 
     \end{itemize}
\end{mdframed}
  
Informally , we should be able to replace ``equals with equals" everywhere in a term.


\subsubsection{Inversion Principle}\label{ge:in}

Gentzen's Inversion Principle captures the idea that ``elim is post-inverse to intro," or ``local soundness", which is the informal notion that the elimination rules should cancel the introduction rules. The so called  $\beta$ equality rules are as follows:


\begin{mdframed}
	\begin{mathpar}
		
		\inferrule*[right=$\beta\wedge_1$] 
		{{\Turnsi {\Gamma} {  M:\phi_1}}\\
			{\Turnsi {\Gamma} {N :\phi_2 }}  } {{\Turnsi {\Gamma} {\operatorname{fst}(  \langle M,N\rangle)\equiv M:\phi_1} }}
		
		\and
		\inferrule*[right=$\beta\wedge_2$] 
		{{\Turnsi {\Gamma} {  M:\phi_1}}\\
			{\Turnsi {\Gamma} {N :\phi_2 }}} { {\Turnsi {\Gamma} {\operatorname{snd}(  \langle M,N\rangle)\equiv N:\phi_2} }}
		\and
		\inferrule*[right=$\beta\supset$] 
		{{\Gamma,x:A\vdash M:B}\\ {\Gamma\vdash N:A}}
		{\Gamma\vdash(\lambda x.M)(N)\equiv [N/x]M:B}
		\and
		\inferrule*[right=$\beta\vee_1$]
		{ {\Gamma,x
				:\phi_1\vdash N:\psi}\\{\Gamma,y:\phi_2 \vdash O:\psi}\\{\Gamma\vdash P:\phi_1} }
		{\Gamma\vdash  ({\text{case } inj_l(P) \text{ of } inj_l(x) \longmapsto N  |\  inj_r (y)\longmapsto O) \equiv[P/x]N: \psi}}
		\and
		\inferrule*[right=$\beta\vee_2$]
		{ {\Gamma,x
				:\phi_1\vdash N:\psi}\\{\Gamma,y:\phi_2 \vdash O:\psi}\\{\Gamma\vdash Q:\phi_2} }
		{\Gamma\vdash  ({\text{case } inr_r(Q) \text{ of } inj_l(x)  \longmapsto N  |\  inj_r(y) \longmapsto O) \equiv[Q/y]O: \psi}}
	\end{mathpar}
\end{mdframed}

\subsubsection{Uniqueness of Forms}

Gentzen's Uniqueness Principles on the other hand capture the idea that ``elim is inverse to intro" (a.k.a ``local completeness"). There should be only one way -- modulo definitional equivalence -- to prove something. The ``$\beta$" rules give rise to computational dynamics via reduction. The so called ``$\eta$" equality rules impose properties that the computational model should satisfy.

The $\eta$ rules (a.k.a. \textit{identity expansion})are given below:

\begin{mdframed}
	\begin{mathpar}
		\inferrule*[right=$\eta\top$]{\Gamma\vdash M:\top}{\Gamma\vdash M\equiv \langle \ \rangle: \top} 
		\and
		
		\inferrule*[right=$\eta\times$]{\Gamma\vdash M:A\times B}{\Gamma\vdash M\equiv \langle \operatorname{fst}(M),\operatorname{snd}(M)\rangle :A\times B }
		\and
		\inferrule*[right=$\eta\rightarrow$]{\Gamma\vdash M:\phi\rightarrow \psi}{\Gamma\vdash M\equiv \lambda x. M x: \phi\rightarrow\psi} 
		\and
		\inferrule*[right=$\eta\vee$]
		{\Gamma\vdash M:\phi_1+\phi_2}
		{{
				\begin{tabular}[b]{ rl}
					$\Gamma\vdash M \equiv \text{case }  M \text{ of }$  & $| inj_l(x) \mapsto inj_l(x)$
					\\
					& $ inj_r(y)\mapsto inj_r(y)) : \psi$
					\\
				\end{tabular}
			}
		}
		
	\end{mathpar}
\end{mdframed}

\section{Operational (a.k.a ``term") Semantics} 
It is obvious that the system is consistent in terms of provability. It's forgetful projection is exactly \ac{IPL} for which we have provided order-theoretic models. We would like to show consistency for the proof relevant model. One way is operational semantic.

The first step toward operational semantics is to break the symmetry of the definitional equivalence and construct a one-way reduction 
relation  on lambda terms. 
Towards this definition we first define the notion of a  \textit{redex}\footnote{we only present it for the $\times\rightarrow$ part of the calculus}: 
\begin{mdframed}
\begin{definition}	
\begin{itemize}
\item A $\beta$-redex is every term of the form:  $$(\lambda x:\phi. N) M |  \operatorname{fst}\langle M, N \rangle| \operatorname{snd}\langle M, N \rangle $$
\item  An $\eta$-redex is every term of the form:
$$\lambda x: M x | \langle \operatorname{fst} , M\operatorname{snd} M \rangle  $$    
\end{itemize}
\end{definition}	
\end{mdframed}
A \textit{normal form} is a term where no redex occurs. To make the
term surrounding the redex explicit, we can use a \textit{term context}, i.e. a term with a single term hole, such as $\lambda x:[]$, $(e[\bullet])$, $[\bullet]e$, where  a hole can be substituted for a term to give a larger term. To be strict all single hole terms have the following diagram:
$$H:= [\bullet]|\ (M[\bullet])|\  ([\bullet]M)|\ \lambda x:A. H|\  \langle H, M \rangle |\  \langle  M, H \rangle $$

Now we have enough tools to define the \textit{(one-step) $\beta\eta$-reduction} between two terms can be defined on terms that include redexes as subterms as follows :
\begin{mdframed}
	\textbf{One-step $\mapsto_{\beta\eta}$  reduction}
	\begin{mathpar}
		\inferrule*[right=($\beta$)]{} {(\lambda x:\phi. N) M \mapsto [M/x]N }
		\and
		\inferrule*[right=($\beta$)] {}{\operatorname{fst}\langle M, N \rangle\mapsto M}
		\and
		\inferrule*[right=($\beta$)] {}{\operatorname{snd}\langle M, N \rangle\mapsto N}
		\and
		\inferrule*[right=($\eta$)]{}{\lambda x: M x\mapsto M}
		\and
		\inferrule*[right=($\eta$)]{} { \langle \operatorname{fst} M\operatorname{snd} M\rangle\mapsto M }  
		\and
		\inferrule*[right=(subterm)]{M\mapsto M^{'}}{H[M]\mapsto H[M^{'}]}
	\end{mathpar}
\end{mdframed}

Now we can define the  reflexive, transitive closure of the previous relation as $\mapsto_{\beta\eta}^{*}$ to denote zero or more reduction steps. The following facts -- leading to a computational proof of consistency -- hold:

\begin{mdframed}
	\begin{theorem}	\textbf{Church -- Rosser for $\mapsto_{\beta\eta}$}
		For every term $M$, if $M\mapsto_{\beta\eta} N_1$ and $M \mapsto_{\beta\eta} N_2$ then there exists $N^{'}$ s.t. $N_1 \mapsto N^{'}$ and $N_2\mapsto N^{'}$
		
	\end{theorem}	
	\begin{theorem}	\textbf{Church -- Rosser for $\mapsto_{\beta\eta}^{*}$}
		For every term $M$, if $M\mapsto_{\beta\eta}^{*} N_1$ and $M \mapsto_{\beta\eta}^{*} N_2$ then there exists $N^{'}$ s.t. $N_1 \mapsto_{\beta\eta}^{*} N^{'}$ and $N_2\mapsto_{\beta\eta}^{*} N^{'}$
		
	\end{theorem}	
\end{mdframed}
 The first consistency result for the equational system comes straight from the Church--Rosser properties. Since, it is  easy to show that for any terms $M$, $N$ s.t where $\Gamma\vdash M\equiv N: A$ based on the $\equiv$ axiomatization there exists a finite sequence of terms $N_0,\ldots, N_i$ such that $M\mapsto_{\beta\eta}^{*}N_0\mapsfrom_{\beta\eta}^{*} N_1\mapsto_{\beta\eta}^{*} N_2\mapsfrom_{\beta\eta}^{*}\ldots \mapsfrom_{\beta\eta} N_i$. Now we can obtain:
 \begin{mdframed}
 \begin{theorem}	\textbf{Definitional equality implies common contractum}
 	For any terms, $M$,$N$ if $\Gamma\vdash M\equiv N:A$ then there exists tern $L$ s.t. $M,N\mapsto_{\beta\eta}^{*} L$
 	
 \end{theorem}
\end{mdframed}
 And as a result:
 \begin{mdframed}	
 \begin{theorem}	\textbf{Consistency of definitional equality of terms}
 	The definitional equality $\equiv$ is not trivial i.e. it won't equate any two terms.
 \end{theorem}	
\end{mdframed}
Moving toward consistency of the whole system (i.e. there is not term of type $\bot$), we prove a theorem for the existence of normal forms.
\begin{mdframed}	
\begin{theorem}	\textbf{Weak normalization theorem}
For any term $M$, there exists a finite sequence of terms s.t. $M\mapsto_{\beta}N_0\mapsto_{\beta} N_1\mapsto_{\beta} N_2\mapsto_{\beta}\ldots \mapsto_{\beta} N_i$ where $N_i$ is a $\beta$ normal form.
\end{theorem}	
\end{mdframed}
It is common place in metatheoretic proofs for such systems that induction on the structure of the term does not ``go through''. Intuitively, a reduction can be ``enlarging" the term but, yet, it is doing progress based on a different kind of metric. We build this metric based on the following definitions and facts. The idea is that we can choose a reduction strategy such that the number of \textit{redexes of a specific type} (to be defined soon) reduce. Here are the steps towards the proof. We omit redexes related to disjunction but the proof extends to such cases pretty easily.


\begin{mdframed}
	\textbf{Definition}
	The \textit{degree of a type $A$} is defined as follows:
	\begin{itemize}
		\item $\theta(P_i)=1$ if $P_i$ is atomic
		\item $\theta(A\times B)=\theta(A \rightarrow B)= \theta(A)+\theta(B)+1$
	\end{itemize}
\end{mdframed}
\begin{mdframed}
	\textbf{Definition}
	The \textit{degree of a redex} is defined as follows:
	\begin{itemize}
		\item  Given that the type of $\lambda x. M$ is of type $A\rightarrow B$ then  $d((\lambda x. M)N)=\theta(A\rightarrow B)$
		\item Similarly, $d(\operatorname{fst}\langle M, N\rangle)=\theta(A\times B)$
		where $A\times B$ is the type of $\langle M, N\rangle$
		
		\item Similarly for the other kinds of redexes.
	\end{itemize}
\end{mdframed}
\begin{mdframed}
	\textbf{Definition}
	The \textit{degree of a term} $d(t)$ is defined as the supremum of the degrees of its redexes.
\end{mdframed}
Now we can prove the following facts:
\begin{theorem}
	\begin{mdframed}
\begin{enumerate}
	\item The degree of redex $r$ is strictly larger than the degree of its type $A$: 
	$\theta(A)<d(r)$
	\item The degree of a redex ($r$) seen as term ($t$)  can be smaller than its redex degree since it might include other redexes: $d(r)\le d(t)$.
	\item The term resulting from a substitution $M[N/x]$  has degree : $d(M[N/x])\le max(d(M),d(N),\theta(A))$ where $A$ is the declared type of $x$ in the type context.
\end{enumerate}
	\end{mdframed}
\end{theorem}
Which give us the following fact that suggests the induction principle that succeeds toward the proof.
\begin{mdframed}
	\begin{theorem}
If $M\mapsto M^{\prime}$ then $d(M)<d(N)$ and hence, if $M\mapsto^{+} N$ then $d(M)<d(N)$.
\end{theorem}
\end{mdframed}
As a result we get a weak  normalization theorem by induction on pairs $(d(M),k)$ where $k$ is the number of redexes with degree $d(M)$.
\begin{mdframed}
\begin{theorem}
	\textbf{Weak Normalization Theorem}
For every term $\Gamma\vdash M:A$ there exists a normalization strategy such that $M\mapsto_{\beta}^{*}N$ and $N$ is a normal form.
\end{theorem}
\end{mdframed} 

Combining with previous results we get:
\begin{theorem}
\begin{mdframed}
\textbf{Consistency}
There is no (closed) term $M$ for which $\vdash M:\bot$ 
\end{mdframed}
\end{theorem}
 Suppose the opposite and obtain a contradiction using the previous theorem: there is no way to obtain a normal form of a bottom type from the rules.  
 
 A stronger result is the strong normalization theorem that says that \textit{every} strategy 
 in normalizing. This result is important in concurrent implementations of reduction since 
 it implies that the order in which redexes are consumed does 
 not matter during the evaluation of  expression. 
 
 The important idea behind the technique -- that generalizes to proof of 
 strong normalization for more complex calculi-- is the concept of  reducibility predicates. 
 Reducibility predicates give a stronger induction principle that delivers the desired 
 result as a lemma(see, e.g. ~\cite{prawitz1971ideas}). 
 

\subsection{The essence of proofs--as--programs }
The proofs of normalization above are essentially of the same "proof strength" (induction principles) as the logical proof of cut elimination. In a nutshell, eliminating cuts is the same as normalizing proof terms (and the corresponding proof trees).

In reality, the slogan of the Curry-Howard isomorphism and, in general, of a type theoretic treatment to logic should be "Normalization as Cut Elimination". This aspect of the isomorphism can be shown explicitly follow Sieg's extraction method ~\cite{Sieg1998}, that  showcases how a construction of the Cut-free sequent calculus comes naturally from an analysis of normal proofs in the natural deduction. 

 
\subsection{Propositions as Types}



There is a correspondence between propositions and types:
\begin{center}
  \begin{tabular}{@{} cc @{}}

    Propositions & Types \\

    $\top$ & $1$ \\
    $A\wedge B$ & $A\times B$ \\
    $A\supset B$ & function $A\Longrightarrow B$ or $B^A$ \\
    $\bot$ & $0$ \\
    $A\vee B$ & $A+B$\\

  \end{tabular}
\end{center}



\section{Categories for proof relevant \vocab{IPL}}

In a Heyting Algebra, we have a preorder (or, partial order in the ``textbook" definition) $\phi\leq \psi$ when $\phi$ implies $\psi$. \emph{HAs} are insufficient, however, for the treatment of proof objects (there can be at most one instance of $\phi\leq\psi$ for specific $\phi$,$\psi$). We can  keep track of proofs, so if $M$ is a proof from  $\Gamma$ to $\psi$, we want to think of it as a map $M:\Gamma+\Longrightarrow \psi+$. In category theory ~\cite{awodey2010category}, the analog of a Heyting Algebra is that of a Bi--Cartesian Closed Category (\emph{BiCCC}). That is a category with all finite products, co--products and exponentials. For an exposition of BiCCCs and their relation with intuitionistic logic ~\cite{Lambek1985}. The axiomatization of a category (in general), finite (and nullary) products and co-products and exponentials is given in this section. 

\subsection{Definitions and Axioms of a Category}
A category has \emph{objects} $\phi,\psi, \ldots$ and \emph{arrows} $f,g,h\ldots$ Each arrow goes from an object to an object. To say that $g$ goes from $\phi$ to $\psi$ we write $g:\phi\Rightarrow\psi$, or say that $\phi$ is the domain of $g$, and $\psi$ the \emph{co--domain}. We write $Dom(g)=\phi$ and $Cod(g)=\psi$.
We say that two arrows $f$ and $g$ are \emph{composable} with $Dom(f)=Cod(g)$. If $f$ and $g$ are composable, they have a \emph{composite}, an arrow called $f\circ g$. There is an \emph{identity} for every object $\phi$.
\begin{mdframed}
\begin{mathpar}
\inferrule*[right=$\text{ID}_{ex}$]{ }
{\operatorname{id}:\phi\Longrightarrow \phi}
\and
\inferrule*[right=Comp]{{f:\phi\Longrightarrow\psi } \\{g:\psi\Longrightarrow\chi}}
{g\circ f:\phi\Longrightarrow \chi}
\and
\inferrule*[right=$\text{ID}_l$]{f:\phi\Longrightarrow\psi }
{id_{\psi}\circ f=f:\phi\Longrightarrow \psi}
\and
\inferrule*[right=idr]{f:\phi\Longrightarrow\psi }
{f\circ id_{\phi} =f:\phi\Longrightarrow \psi}
\and
\inferrule*[right=idr]{{f:\phi\Longrightarrow\psi }\\{g:\psi\Longrightarrow\chi}\\{h:\chi\Longrightarrow\omega}}
{h\circ (g\circ f) =(h\circ g)\circ f:\phi\Longrightarrow \omega}
\end{mathpar}
\end{mdframed}



\subsection{Terminal, Co-Terminal objects, Products and Co-Products}
Now we can think about objects in the category that correspond to propositions given in the correspondence.

\paragraph{Terminal Object}
$1$ is the terminal object, also called the final object, which corresponds to $\top$. For any object $\Gamma$ there is a unique map $\Gamma\Longrightarrow 1$. 
\begin{mdframed}
\begin{mathpar}

\inferrule*[right=Existence]
{ } {\langle \ \rangle:\phi\Longrightarrow 1}
\and
\inferrule*[right=Unicity($\eta$)]
{M:\Gamma\Longrightarrow 1}{M=\langle \ \rangle:\Gamma\Longrightarrow 1}
\end{mathpar}
\end{mdframed}



\paragraph{Product} 
For any objects $\phi$ and $\psi$ there is an object $\chi=\phi\times \psi$ equipped with arrows $\operatorname{fst}:\phi\times\psi\Longrightarrow\phi$ and $\operatorname{snd}:\phi\times\psi\Longrightarrow\psi$ that is the \emph{product} of $\phi$ and $\psi$, which corresponds to the join $\phi\wedge \psi$. For any other object $\Gamma$ with arrows $M:\Gamma\Longrightarrow\phi$ and $\Gamma\Longrightarrow\psi$ there  exists \emph{unique} arrow, $\langle M,N \rangle$ s.t. $\operatorname{fst}\circ\langle M, N\rangle=M(\beta\times_1)$ and
$\operatorname{snd}\circ\langle M, N\rangle=N(\beta\times_2)$.
\begin{mdframed}
\begin{mathpar}

\inferrule*[right=Exist$_1$]{{M:\Gamma\Longrightarrow \phi} \\{ N:\Gamma\Longrightarrow \psi}} {\langle M,N\rangle : \Gamma\Longrightarrow \phi\times \psi} 
\and
\inferrule*[right=Exist$_2$($\beta_1$)]{{M:\Gamma\Longrightarrow \phi} \\{ N:\Gamma\Longrightarrow \psi}} {\operatorname{fst}\circ\langle M,N\rangle : \Gamma\Longrightarrow \phi}
\and 
\inferrule*[right=Exist$_3$($\beta_2$)]{{M:\Gamma\Longrightarrow \phi} \\{ N:\Gamma\Longrightarrow \psi}} {\operatorname{snd}\circ\langle M,N\rangle : \Gamma\Longrightarrow \phi} 
\and
\inferrule*[right=Un($\eta$)]{{P:\Gamma\Longrightarrow \phi\times \psi} \\{ \operatorname{fst}\circ P=M:\Gamma\Longrightarrow \phi}\\ {\operatorname{snd}\circ P=N:\Gamma\Longrightarrow \psi}} {P=\langle M,N\rangle : \Gamma\Longrightarrow \phi\times \psi}
\end{mathpar}
Diagrammatically:
\begin{equation*}
  \begin{tikzcd}
    {} & \Gamma \arrow[bend right]{ddl}[swap]{M} \arrow[bend left]{ddr}{N}\dar[dashed]{\langle M,N\rangle} & {} \\
    {} & \phi \times \psi \dlar{\text{fst}}\drar[swap]{\text{snd}} & {} \\
    \phi & {} & \psi
  \end{tikzcd}
\end{equation*}
\end{mdframed}




\paragraph{Exponentials}
Given objects $A$ and $B$, an exponential $B^A$ (which corresponds to $A\supset B$) is an object with the following universal property:
\[
\begin{tikzcd}
C \arrow[dashed]{dd}[swap]{\lambda(h)}&{C\times A}\arrow{ddrr}{h} \arrow[dashed]{dd}[swap]{\lambda(h)\times \operatorname{id}_A} &&\\
&& {}	&{}\\
B^A&{B^A\times A}\arrow{rr}[swap]{\operatorname{ap}} &{}&{B}
\end{tikzcd}
\]
such that the diagram commutes.

This means that there exists a map $\operatorname{ap}:B^A\times A\Longrightarrow B$ (application map) that corresponds to implication elimination.

The universal property is that for all objects $C$ that have a map $h:C\times A\Longrightarrow B$, there exists a unique map $\lambda(h):C\Longrightarrow B^A$ such that
\[
\operatorname{ap}\circ (\lambda(h)\times \operatorname{id}_A )=h:C\times A\Longrightarrow B
\]
This means that the diagram commutes. Another way to express the induced map is $\lambda(h)\times \operatorname{id}_A = \langle \lambda(h)\circ\operatorname{fst},\operatorname{snd}\rangle$.

The map $\lambda(h):C\Longrightarrow B^A$ is unique, meaning that
\begin{mathpar}
\inferrule*{\operatorname{ap}\circ(g\times \operatorname{id_A})=h:C\times A\Longrightarrow B}{g=\lambda(h):C\Longrightarrow B^A}
\end{mathpar}

\paragraph{Co--Products}
For any objects $\phi$ and $\psi$ there is an object $\chi=\phi + \psi$ equipped with arrows $\operatorname{inl}:\phi\Longrightarrow\phi+\psi$ and $\operatorname{inr}:\psi\Longrightarrow\phi+\psi$ that is the \emph{co-product} of $\phi$ and $\psi$, which corresponds to the meet $\phi\wedge \psi$. For any other object $\omega$ with arrows $M:\omega\Longrightarrow\phi\vee\psi$ and $N:\omega\Longrightarrow\phi\vee\psi$ there  exists \emph{unique} arrow, ${ M,N }$ s.t. $\{M,N\}\circ\operatorname{inl}=M$ and
$\{M,N\}\circ\operatorname{inr}=N$.

\begin{mdframed}
\begin{mathpar}

\inferrule*[right=Exist$_1$]{O:\Gamma\Longrightarrow \phi}{ \operatorname{inl}\circ O:\Gamma\Longrightarrow\phi + \psi}  
\and

\inferrule*[right=Exist$_2$]{P:\Gamma\Longrightarrow \psi}{ \operatorname{inr}\circ P:\Gamma\Longrightarrow\phi + \psi}
\and
\inferrule*[right=Exist$_3$($\beta_1$)]{{O:\Gamma\Longrightarrow \phi} \\{M:\phi\Longrightarrow\omega}\\{N:\psi\Longrightarrow\omega}} {\{M,N\}\circ\operatorname{inl}\circ O=M\circ O : \Gamma\Longrightarrow \omega}
\and 
\inferrule*[right=Exist$_3$($\beta_2$)]{{P:\Gamma\Longrightarrow \psi} \\{M:\phi\Longrightarrow\omega}\\{N:\psi\Longrightarrow\omega}} {\{M,N\}\circ\operatorname{inr}\circ P=N\circ P : \Gamma\Longrightarrow \omega}
\and
\inferrule*[right=Exist$_3$($\beta_2$)]{{M:\Gamma\Longrightarrow \phi} \\{ N:\Gamma\Longrightarrow \psi}} {\operatorname{snd}\circ\langle M,N\rangle : \Gamma\Longrightarrow \phi} 
\and
\inferrule*[right=Un($\eta$)]{ { O :\Gamma\Longrightarrow \phi} \\ {P:\Gamma\Longrightarrow \psi}\\{U:\phi + \psi\Longrightarrow\omega}\\{M:\phi\Longrightarrow\omega}\\{N:\psi\Longrightarrow\omega} \\{U\circ \operatorname{inl}\circ O=M} \\
 \\{U\circ \operatorname{inr}\circ N =M}} {U=\{M,N\} }  
\end{mathpar}
Diagrammatically:
\begin{tikzcd}
\phi\drar{\inl}\arrow[bend right]{ddr}{(\beta)}[swap]{M}  & {} & \psi\dlar[swap] {\inr}\arrow[bend left]{ddl}{N}[swap]{(\beta)}	\\
{} & \phi+\psi\dar[bend right]{(\eta)}[swap]{M}\dar[bend left]{\{M,N\}} & {}	\\
{} & C & {}
\end{tikzcd}
\end{mdframed}

\chapter{Justification Logic}\label{jllogic}
In this chapter, I will give an overview of \ac {JL}. 
I will emphasize LP, 
the very first logic of justification, and its deep relation with 
\ac{IPL}. My scaffolding will be based upon \cite{Art01BSL},~\cite{Art95TR}  
that reflect this relation. Beforehand, I will allow for a more general discussion on 
\ac{JL} following \cite{sep-logic-justification} and other 
relevant papers.

%It is well known that the provability predicate can be axiomatized using a modality \cite{citeulike:214701}, \cite{ArtBek05HPL}. The Logic of Proofs {\sf LP} \cite{Art94APAL} goes further and provides explicit proof terms (\textit{proof polynomials}) to inhabit judgments on validity. By translating reasoning in Intuitionistic Propositional Calculus ({\sf IPC}) to classical proofs, {\sf LP} obtains classical semantics for {\sf IPC} through a modality (inducing a {\sf BHK} semantics).

\section{A bird's eye view}
According to \cite{sep-logic-justification}\say {Justification logics are epistemic logics which allow knowledge and belief modalities to be ``unfolded'' into justification terms.}
 More specifically, in \ac{JL} the modality in question is witnessed by a reason and propositions of the kind $\Box\phi$ become $t:\phi$ that reads ``$\phi$ is justified by reason t". Witnesses in \ac{JL} have structure and operations. Different choices of operators result in logics that explicate different modalities 
 ({\sf {$K$, $T$, $S4$, $S5$}}). 
 In general, there is an infinite family of justification logics.
 For our purposes, and in addition to type theoretic approaches to logic, \ac{JL} reveals a computational content for \emph{validity} in classical terms. As we will see following~\cite{artemov97un}, \ac{JL} and especially its {\sf $S4$} counterpart \ac{LP}, can provide a unified classical \emph{semantics} for type theoretic 
 formulations of intuitionistic logic. In addition, following \cite{Artemov2007a, DBLP:journals/entcs/PouliasisP14}, JL mechanics can be viewed type
 theoretically to provide for modal typed systems that enrich computational
 type theories with \say{semantical} notions such as explicit reflection and modular
 binding.
 

\section{Minimal Justification Logic $J_0$}~\label{min:jo}
To permit for an account of reasons, the logic is enriched with an extra sort ($j$) for justifications. The sort of propositions is then enriched with propositions of the kind $j:\phi$ with $\phi$ being a proposition. 
The abstract syntax is as follows:

\begin{mdframed}
\begin{align*}
j := &s_i|\ C_i| j_1*j_2| j_2 + j_2\\
 \phi:=& P_i|\ \bot|\ \phi_1\wedge\phi_2|\ \phi_1\vee\phi_2| \ \phi_2\supset\phi_2|\ \neg\phi|\ j:\phi   
\end{align*}
\end{mdframed}
Constants $C_i$ are symbols that can be assigned to logic axioms that are assumed to be necessary. Weaker justification logics exist without any assignment of constants (empty \emph{constant specifications}) or with partial constant specifications. Nevertheless, in order for the  \emph{rule of necessitation} to be admissible each axiom instance of the underlying propositional logic has to be assigned a constant. We will be coming back to this topic in later sections. Symbols $s_i$ stand for variables.

A Hilbert--style axiomatization of $J_0$ is given below. Its components are Hilbert's axioms for propositional logic together with two basic rules for justification: \emph{applicativity} and \emph{concatenation}. 
Concatenation internalizes weakening of proofs.
\begin{mdframed}
\sf{Propositional Axioms}
\begin{align*}
&\sf{P1}.  \vdash \phi\supset(\psi\supset\phi)\\
& \sf{P2}. \vdash (\phi\supset(\psi\supset\chi))\supset((\phi\supset\psi)\supset(\phi\supset\chi))\\
& \sf{P3}. \vdash \phi\supset\psi\supset\phi\wedge\psi\\
&\sf{P4}. \vdash \phi\supset\psi\supset\psi\wedge\phi\\
&\sf{P5}.  \vdash \phi\supset\phi\vee\psi\\
&\sf{P6}. \vdash \psi\supset\phi\vee\psi\\
&\sf{P7}. \vdash (\phi\supset\psi)\supset(\neg\psi\supset\neg\phi)\\
\end{align*}
\end{mdframed}


\begin{mdframed}
\sf{Justification Axioms}
\begin{align*}
& \sf{Times}. \vdash j:(\phi\supset\psi)\supset(j':\phi\supset j*j':\psi)\\
& \sf{PlusL}. \vdash j:\phi\supset(j+j':\phi)\\
& \sf{PlusR}. \vdash j:\phi\supset(j'+j:\phi)\\
\end{align*}
\end{mdframed}
The rule of the system is \emph{Modus Ponens}. 
\begin{mdframed}
\sf{Modus Ponens}
\begin{mathpar}
\inferrule*[right=\sf{MP}]{{\phi\supset\psi}\\{\phi}}{\psi}
\end{mathpar}
\end{mdframed}
For the rule of necessitation to be admissible, we need necessitation of axioms to be admissible. 
For that reason a constant specification is required. 
We focus here on axiomatically appropriate constant specification $\sf{CS}$ because of its relation to combinatorial calculi. 
An axiomatization of axiomatically appropriate $\sf{CS}$ given below. 
Elements of $\sf{CS}$ are pairs $(C,\phi)$ of polymorphic 
(i.e. \textit{parametrized} over propositions) constants and propositions. The $!$ operator relates to
a concept of internalization of justified statements, i.e. witnessing the existence of a justified statement with
a (higher order) justification. We demand that all justified axiomatic schemes can be internalized.
\begin{mdframed} 
\sf{Axiomatic CS}
\begin{mathpar}
\inferrule*[right=$\sf{C_1}$]  { }   {\vdash({{\sf C_1}[\phi,\psi],\  \phi\rightarrow(\psi\rightarrow\phi))\in \sf{CS}}}
\and
\inferrule*[right=$\sf{C_2}$]  { }   {\vdash({{\sf C_2}[\phi,\psi,\chi],\ (\phi\supset(\psi\supset\chi))\supset((\phi\supset\psi)\supset(\phi\supset\chi))  ) )\in \sf{CS}}}
\and
\inferrule*[right=$\sf{C_3}$]  { }   {\vdash(  {{\sf C_3}[\phi,\psi],\  \phi\supset\psi\supset\phi\wedge\psi  )\in \sf{CS}}}
\and
\inferrule*[right=$\sf{C_4}$]  { }   {\vdash(  {{\sf C_4}[\phi,\psi],\  \phi\supset\psi\supset\psi\wedge\phi  )\in \sf{CS}}}
\and
\inferrule*[right=$\sf{C_5}$]  { }   {\vdash(  {{\sf C_5}[\phi,\psi],\  \phi\supset\phi\vee\psi  )\in \sf{CS}}}
\and
\inferrule*[right=$\sf{C_6}$]  { }   {\vdash(  {{\sf C_6}[\phi,\psi],\  \psi\supset\phi\vee\psi  )\in \sf{CS}}}
\and
\inferrule*[right=$\sf{C_7}$]  { }   {\vdash(  {{\sf C_7}[\phi,\psi],\  (\phi\supset\psi)\supset(\neg\psi\supset\neg\phi)\in \sf{CS}}}
\and
\inferrule*[right=$\sf{C_8}$] { } {\vdash(  {{\sf C_8}[\phi,\psi,j,j'],\  j:(\phi\supset\psi) \supset (j':\phi \supset j*j':\psi  ))\in \sf{CS}}}
\and
\inferrule*[right=$C!$]
{\vdash ( {\sf C},\phi)\in \sf{CS}} {\vdash(\sf{C!} ,\ \sf{C}:\phi )\in \sf{CS}}
\end{mathpar}
\end{mdframed}

Finally we require reflection on $\sf{CS}$: 
\begin{mdframed}
\sf{Specification Reflection}
\begin{mathpar}
\inferrule*[right=CSR]
{\vdash ( {\sf C},\phi)\in \sf{CS}} {\vdash\sf{C}:\phi}
\end{mathpar}

\end{mdframed}

The system can be given a Natural Deduction formulation \textit{\`a la} \ac{IPL} since the following theorem holds:
\begin{mdframed}
\textbf{Deduction Theorem}
For any set of propositional assumptions $\Gamma$, \\ $\Gamma,\phi\vdash\psi$ implies $\Gamma\vdash\phi\supset\psi$ 
\end{mdframed}
\section{Epistemic motivation} 
 \ac{JL} as an epistemic logic departs from previous traditions of logic of knowledge based on  universality judgments. From \cite{sep-logic-justification}
\begin{quotation}
The modal approach to the logic of knowledge is, in a sense, built around the universal quantifier: X is known in a situation if X is true in all situations indistinguishable from that one. Justifications, on the other hand, bring an existential quantifier into the picture: X is known in a situation if there exists a justification for X in that situation
\end{quotation}

This fresh approach to the epistemic tradition has been utilized to solve many problems in formal epistemology (see \cite{Artemov2014-ARTLOA}). We sketch 
here the solution to the famous \textit{Red barn problem} 
that, also, provides a pedagogical example 
on how deduction in the system works.

The red barn problem can be stated as follows:
\begin{quote}
Suppose I am driving through a neighborhood in which, unbeknownst to me, papier-mâché barns are scattered, and I see that the object in front of me is a barn. Because I have barn-before-me percepts, I believe that the object in front of me is a barn. Our intuitions suggest that I fail to know barn. But now suppose that the neighborhood has no fake red barns, and I also notice that the object in front of me is red, so I know a red barn is there. This juxtaposition, being a red barn, which I know, entails there being a barn, which I do not, “is an embarrassment”
\end{quote}

The red barn example can be represented in a system of modal logic where $\Box \phi$ represents knowledge of $\phi$ that, in contrast to the the justified approach, is forgetful with respect to reasons. The formalization and the accompanying problem go as follows:

\begin{enumerate}
    \item $\Box B$, ‘I believe that the object in front of me is a red barn’.
    \item  $\Box(B \wedge R),$ ‘I believe that the object in front of me is a red barn’. 

At the metalevel, 2 is actually knowledge, whereas by the problem description, 1 is not knowledge.

   \item $\Box(B\wedge R\supset B)$, a knowledge assertion of a logical axiom.
	\end{enumerate}
\begin{quote}	
Within this formalization, it appears that epistemic closure in its modal form (2) is violated:line 2, $\Box(B \wedge R )$, and line 3, $(B \wedge R \supset B)$ are cases of knowledge whereas $\Box B$ (line 1) is not knowledge. The modal language here does not seem to help resolving this issue.
\end{quote}
Of course, one can resolve this by introducing a second modality(e.g. for \say{I believe that}). But then similar problems can occur (e.g. by adding a third modality read as `it should be'). Indexing of modalities with reasons solves this problem in its generality: by permitting the applicative closure only on reasons of the same sort one can overcome this defect.
\begin{enumerate}
   \item $u:B$, ‘$u$ is a reason to believe that the object in front of me is a barn’;
   \item $v:(B \wedge R)$, ‘$v$ is a reason to believe that the object in front of me is a red barn’;
    \item $a:(B \wedge R \supset B)$, because of logical awareness.


 On the metalevel, the problem description states that 2 and 3 are cases of knowledge, and not merely belief, whereas 1 is belief which is not knowledge. 
 The formal reasoning goes as follows:

    \item $a:(B \wedge R \supset B)\supset(v:(B \wedge R) \supset  a*v:B)$, by {\sf Times} 
    \item $v:(B \wedge R) \supset a*v:B$, from 3 and 4, by propositional logic;
    $a*v:B$, from 2 and 5, by propositional logic.

\end{enumerate}
\section{Proof theoretic view}
	In  Chapter ~\ref{intui} we gave an analytic account of the \ac{BHK} principles of  constructive proofs. In the paper ``Eine Interpretation des
	intuitionistischen Aussagenkalküls ", G\"oedel gave a classical provability interpretation of \ac{BHK} using the modal system ${\sf S4}$.
	
	The standard axiomatization of ${\sf S4}$ is given below: 
	\begin{mdframed}
	The system $\sf{S4}$ 
	\begin{align*}
	& \sf{P1-P7}&\\
	 \sf{K}.& \vdash \Box(\phi\supset\psi)\supset(\Box\phi\supset\Box\psi)\\
	 \sf{T}.& \vdash \Box\phi\supset\phi\\
	\sf{4}. &\vdash \Box\phi\supset\Box\Box\phi\\
	\end{align*}


	\sf{Modus Ponens}
	\begin{mathpar}
	\inferrule*[right=\sf{MP}]{{\phi\supset\psi}\\{\phi}}{\psi}
	\end{mathpar}
	\end{mdframed}
	
	G\"oedel's result can be summarized in the following theorem:

	\begin{mdframed}
		\textbf{G\"odel-Tarski Translation of Intuitionistic Logic}
	$$\Gamma\vdash_{\sf IPL}\phi \rightharpoondown \Gamma\vdash_{\sf S4}\operatorname{tr}(\phi)$$
		where $\operatorname{tr}(\phi)$ is obtained by $\phi$ by $\Box$-ing its subformulas. 
	\end{mdframed}

After this result the state of the project of a classical interpretation of \ac{BHK} semantics was as follows:
     IPC $\hookrightarrow$ S4 $\hookrightarrow$ ? $\hookrightarrow$ CLASSICAL PROOFS. Filling the missing part was 
     the motivation behind \ac{LP}, the first Justification Logic.

	
\section{The Logic of Proofs}
An axiomatization of \ac{LP} with axiomatically appropriate constant specification as defined in \ref{min:jo} can be given
as follows:
	\begin{mdframed}
	The system $\sf{LP}$ 
	\begin{align*}
	& \sf{P1-P7}\\
	 \sf{Times}.& \vdash j:(\phi\supset\psi)\supset(j':\phi\supset j*j':\psi)&\\
	  \sf{PlusL}.& \vdash j:\phi\supset(j+j':\phi)&\\
	  \sf{PlusR}.& \vdash j:\phi\supset(j'+j:\phi)&\\
	 \sf{T}.& \vdash j:\phi\supset\phi\\
	\sf{4}. &\vdash j:\phi\supset (j!:j:\phi)
		\end{align*}
	
   \end{mdframed}
\section{Metatheoretic Results}
The \emph{Deduction Theorem} holds for \ac{LP}
\begin{mdframed}
\textbf{Deduction Theorem}
Any deduction of the kind $\Gamma,\phi\vdash\psi$ implies $\Gamma\vdash \phi\supset\psi$.
\end{mdframed}
Also, the lifting property can be obtained:
\begin{mdframed}
\textbf{Lifting Lemma}

Any deduction of the kind $\vec{j}:\Gamma,\Delta\vdash\phi$ implies $\vec{j}:\Gamma,\vec{s}:\Delta\vdash j'(\vec{j},\vec{s}):\phi$ where $\vec{j}$ is a vector metavariables to be substituted for arbitrary polynomials and $\vec{s}$ is a vector of (object) variables. 

\end{mdframed}
In addition, \ac{LP} is the forgetful projection of {\sf $S4$}. 
More specifically, consider a formula of \ac{LP} $\phi$ and the transformation $F_\Box(\phi)$ that replaces all subformulae of $\phi$ of the kind $j:\phi'$ with $\Box\phi'$. The following theorem holds:
\begin{mdframed}
\textbf{Forgetful Projection Property}

$\Gamma\vdash_{\sf LP}\phi$ implies $\Gamma\vdash_{\sf{ S4}}F_{\Box}(\phi)$ 

\end{mdframed}
  The inverse also holds as the realization theorem says. Before introducing the realization procedure we give a motivating example.
  
  \begin{mdframed}
  \textbf{Example}: Realization of $\vdash_{\sf S4}\Box\phi\vee\Box\psi\supset\Box(\phi\vee\psi)$
  \begin{enumerate}
     \item $\phi\supset\phi\vee\psi$, $\psi\supset\phi\vee\psi$ Prop. Axioms;
     \item $C:(\phi\supset \phi\vee \psi)$, $C':(\psi\supset \phi\vee \psi)$ From {\sf CS} rules.
      \item $s:\phi\supset C*s:\phi\vee\psi$, From 1,2 and{\sf Times} and {\sf MP}
      \item $t:\psi\supset C'*t:\phi\vee\psi$, Similarly 
      \item $C*s:\phi\vee\psi\supset(C*s+C'*t):\phi\vee\psi$ and $C'*t:\phi\vee\psi\supset(C*s+C'*t):\phi\vee\psi$, From {\sf Rplus, Lplus} 
      \item $s:\phi\supset(C*s+C'*t):\phi\vee\psi$, 
      From 3,5 by Propositional Logic.
      \item $t:\psi\supset(C*s+C'*t):\phi\vee\psi$,
      From 4,5 by Propositional Logic.
      \item  $s:\phi\vee t:\psi\supset(C*s+C'*t):\phi\vee\psi$,
      From 6,7 and Propositional Logic.
      
  \end{enumerate}
  
  \end{mdframed}

\subsection{Realization}							
The realization theorem gives an algorithmic process for transforming cut-free deductions in{\sf S4} to {\sf LP}.
By an {\sf LP}-realization of a modal formula $\phi$ we mean an assignment of proof polynomials to
all occurrences of the modality in$ \phi$. Let $\phi^{r}$  be the image of $\phi$ under a realization $r$. 

The polarity of $\Box$s in a formula is relevant in realizations.  
We define positive
and negative occurrences of modality in a formula and a sequent.
\begin{mdframed}

\textbf{$\Box$ Polarities}
\begin{enumerate}
\item The indicated occurrence of $\Box$ in $\Box\phi$ is of positive polarity; 
\item any occurrence of $\Box$ in the subformula $\phi$ of $\psi\supset\phi$,$\psi\wedge\phi$, $\phi\wedge\psi$, $\psi\vee\phi$, $\phi\vee\psi$, $\Box\phi$,$\Gamma\Rightarrow\Delta,\phi$ -- we will be defining $\Rightarrow$ momentarily -- has the same polarity as the same occurrence of $\Box$ in $\phi$.
\item any occurrence of $\Box$ in the subformula $\phi$ of $\neg\phi$,  $\phi\supset\psi$, $\Gamma,\phi\Rightarrow\Delta$, has polarity opposite to the polarity of the very same  occurrence of $\Box$ in $\phi$.

\end{enumerate}

\end{mdframed}


Next we give a a cut-free sequent formulation of ${\sf S4}$ (reference) with sequents $\Gamma\vdash\Delta$, where $\Gamma$ and $\Delta$ are finite multisets of modal formulas. The left hand multisets are to be read conjunctively and the right hand ones disjunctively. The rules are the rules given below together with the typical structural ones.
\begin{mdframed}

\begin{mathpar}

\inferrule*[right=Refl]  { }   {\Gamma,\phi\vdash\phi,\Delta}
\and
\inferrule*[right=$\neg$L]  {\Gamma\vdash\phi,\Delta}   {\Gamma,\neg\phi\vdash\Delta}
\and
\inferrule*[right=$\neg$R]  {\phi,\Gamma\vdash\Delta}   {\Gamma\vdash\neg\phi,\Delta}
\and
\inferrule*[right=$\wedge$L]  {\Gamma,\phi,\psi\vdash\Delta}    {\Gamma,\phi\wedge\psi\vdash\Delta}
\and
\inferrule*[right=$\wedge$L]  {{\Gamma\vdash\phi,\Delta}\\   {\Gamma\vdash\psi,\Delta}}{\Gamma\vdash\phi\wedge\psi,\Delta}
\and
\inferrule*[right=$\vee$L]  {{\Gamma,\phi\vdash\Delta}\\   {\Gamma,\psi\vdash\Delta}}{\Gamma,\phi\vee\psi\vdash\Delta}
\and
\inferrule*[right=$\vee$R]  {\Gamma\vdash\phi,\psi,\Delta}   {\Gamma\vdash\phi\vee\psi, \Delta}
\and
\inferrule*[right=$\supset$L]  {{\Gamma\vdash\phi,\Delta}\\   {\Gamma, \psi\vdash\Delta}}{\Gamma,\phi\supset\psi\vdash\Delta}
\and
\inferrule*[right=$\supset$R]  {{\Gamma,\phi\vdash\psi,\Delta}}{\Gamma\vdash\phi\supset\psi, \Delta}
\end{mathpar}
\begin{mathpar}
\inferrule*[right=$\Box$L]  {\phi,\Gamma\vdash\Delta} {\Box\phi,\Gamma\vdash\Delta}
\and
\inferrule*[right=$\Box$R]  {\Box\Gamma\vdash\phi}  {\Box\Gamma\vdash\Box\phi}
\end{mathpar}

\end{mdframed}

Relevant in the realization proof is the sequent formulation of {\sf LP}, the system {\sf LPG} which enjoys the cut-elimination property resulting in the system ${\sf LPG^{-}}$. The rules relevant to justifications are given below.  

\begin{mdframed}

\begin{mathpar}

\inferrule*[right=$:$L]  {\Gamma,\phi\vdash\phi,\Delta}   {\Gamma,t:\phi\vdash\phi,\Delta}
\and
\inferrule*[right=$!$R]  {\Gamma\vdash t:\phi,\Delta}   {\Gamma\vdash!t:t:\phi,\Delta}
\and
\inferrule*[right=$+$L]  {\Gamma\vdash t:\phi,\Delta}   {\Gamma\vdash(t+s):\phi,\Delta}
\and
\inferrule*[right=$+$R]  {\Gamma\vdash t:\phi,\Delta}    {\Gamma\vdash (s+t):\phi,\Delta}
\and
\inferrule*[right=$*$R]  {{\Gamma\vdash s:\phi\supset\psi,\Delta}\\   {\Gamma\vdash t:\phi,\Delta}}{\Gamma\vdash s*t:\psi,\Delta}
\and
\inferrule*[right=$c$R]  {\Gamma\vdash\phi,\Delta}   {\Gamma\vdash c:\phi,\Delta}

\end{mathpar}

\end{mdframed}
Utilizing the previous systems the realization theorem shows:
\begin{mdframed}
\textbf{Realization Theorem}
If $\Gamma\vdash_{\sf S4} \phi$ then there is a \emph{normal} realization s.t.  $\Gamma\vdash_{\sf LP} \phi^{r}$. By normal we mean a realization for which all occurrences of $\Box$ are realized by proof variables and the corresponding constant specification is injective.
   
\end{mdframed}
\subsection{Kripke -  Fitting Semantics}
In this section I will be discussing Kripke -- Fitting Semantics\cite{fitting2005logic} for Justification Logic $\sf{J_0} + CS$ very briefly.


  A  possible world justification logic model for the system ${\sf J_0 + CS}$  is a structure $M=\langle G, R, E, V\rangle$. $\langle G,R\rangle$ is a standard $K$ frame, where $G$ is a set of possible worlds and $R$ is a binary relation on it. $V$ is a mapping from propositional variables to subsets of $G$, specifying atomic truth at possible worlds.
$E$ is an evidence function that maps pairs of justification terms and formulas to sets of worlds.  

Given such a model, we define the $\models$ relation as follows:
\begin{mdframed}
$\forall \Gamma\in G$
\begin{itemize}
    

    \item[] $M, \Gamma \models P$ iff $\Gamma \in V(P)$ for $P$ a propositional letter
    \item  It is not the case that $M, \Gamma \models \bot$
     \item   $M, \Gamma \models \phi \supset \psi$ iff it is not the case that $M, \Gamma \models \phi$ or$ M, \Gamma\models Y$
     \item  $M,\Gamma \models (j:\phi)$ if and only if $\Gamma \in E(j,\phi)$ and, $\forall \Delta\in G$ with $\Gamma R \Delta$, we have that $M,\Delta\models \phi$.
     
\end{itemize}
\end{mdframed}


 The following conditions on evidence functions are assumed:
\begin{itemize}
\item[] $E(j,\phi\supset\psi)\cap E(j',\phi) \subseteq E(j*j',\psi)$
\item[]  $E(j,\phi) \cup E(j',\phi)\subseteq E(j + j',\phi) $
\end{itemize}
   


Finally, the Constant Specification CS should be taken into account. Recall that constants are intended to represent reasons for basic assumptions that are accepted outright. A model $M = \langle G,R,E,V\rangle$ meets Constant Specification CS provided: if $(C,\phi) \in CS$ then $E(c,\phi) = G$.

Typical, soundness and completeness results can be shown for such models. They can also be extended for all other justification logics.
\chapter{Curry -- Howard view of justification logic}
\label{proposal}
In this and the following chapter
we suggest reading a constructive necessity  
of a formula ($\Box A$) as  internalizing a notion 
of constructive truth of $A$ 
(a proof within a deductive system $I$) 
and validity of $A$
(a proof under an interpretation  $\llbracket A \rrbracket_J$ within some system $J$).  An example of such a relation is provided by the simply typed lambda calculus
(as $I$) 
and its implementation in $SK$ combinators (as $J$). 
We utilize justification logic to axiomatize the notion of 
validity-under-interpretation and, hence, 
treat  a  ``semantical'' notion in a purely
 proof-theoretic manner. We present the system  in 
Gentzen-style  natural deduction formulation  and provide reduction and expansion rules for the $\Box$ connective. Finally, we add proof-terms and proof-term equalities
to obtain a corresponding calculus ({\sf{Jcalc$^{-}$}}) in the next chapter.
The obtained system can be viewed as an extension of the Curry--Howard 
isomorphism with justifications.
We provide standard metatheoretic results  and suggest a 
programming language  interpretation in  languages with foreign function interfaces (\textit{FFI}s).

\section{Introduction: Necessity and Constructive Semantics}
In his seminal ``Explicit Provability and Constructive Semantics'' \cite{Artemov2001} Artemov developed a constructive, proof-theoretic semantics for 
\acs{BHK} proofs ~\cite{Troelstra1988} 
in what turned out to be the first development of a family of logics that we now call justification logic.
The general idea, upon which we build our calculus, is that semantics of a deductive system $I$ can be viewed in a solely proof-theoretic manner 
as mappings of proof constructs of $I$ into another proof system $J$ (which we call justifications).
As an example one could think  $I$  being  Heyting arithmetic and $J$ some  ``stronger'' system 
(e.g. a classical axiomatization of Peano arithmetic, a classical or intuitionistic set theory etc). 
 In Artemov's work $I$ is assumed to be
based on intuitionistic logic and $J$  on classical logic. 
We, initially,  mute such assumptions to focus exclusively 
on the mechanics of necessity in this framework.
We recover them later and study  their relation  to  
the Rule of Necessitation for our system.
What's more,  such a semantic relation can be treated 
logically giving  rise to a modality of explicit necessity. 
Different sorts of necessity
($K$, $D$, $S4$, $S5$) have been offered  an explicit counterpart under the umbrella of justification logic. Some of them have been studied within a
Curry--Howard setting ~\cite{ArtBon07LFCS}. Our work
focuses on  $K$ modality and  should be viewed as the  counterpart of ~\cite{Bellin2001} with justifications as we explain in \ref{relat}.
\subsection{Deductive Systems, Validity and Necessity}
Following a framework championed by Lambek \cite{Lambek1968,Lambek1969}, let us  assume two deductive systems $I$ 
(with propositional universe $U_I$, 
a possibly non-empty signature of axioms $\Sigma_I$ and an entailment relation $\Sigma_I;\Gamma\vdash_{I}A$) and $J$ 
(resp. with  $U_J$, $\Sigma_J$ and $\Sigma_J;\Delta\vdash_J \phi$). We will be using Latin letters for the formulae of $I$ and Greek letters for the formulae of $J$.
We will be omitting the $\Sigma$ signatures when they are not relevant.

For the  entailment relations of the two systems we require the following elementary principles\footnote{We are not excluding other connectives but by imposing
such minimal requirements we show that ``necessity'' ($\Box$) connective can be treated generically and orthogonally of the presence of other connectives}:
\begin{enumerate}
	\item \textit{Reflexivity}. In both relations $\Gamma$ and $\Delta$ are multisets of formulas (contexts) that enjoy reflexivity:
	$$A \in \Gamma \Longrightarrow \Gamma\vdash_{I}A$$ $$\phi \in \Delta \Longrightarrow \Delta\vdash_{J}\phi$$
	\item \textit{Compositionality}.  Both relations are closed under deduction composition:  
	$$\Gamma\vdash_I A \text{\ and\ } \Gamma^{'},A\vdash_{I} B \Longrightarrow \Gamma,\Gamma^{'}\vdash_I B $$  
	$$\Delta\vdash_J\phi \text{\ and\ } \Delta^{'},\phi\vdash_{J} \psi \Longrightarrow \Delta,\Delta^{'}\vdash_J \psi$$ 
	\item \textit{Top}. Both systems have a distinguished top formula $\top$ for which under any $\Gamma$, $\Delta$: $$\Gamma\vdash_{I}\top_I \text{\ and \ }
	\Delta \vdash_J\top_J$$
\end{enumerate}

Now we can define: 
\begin{definition}Given a deductive system $I$, an   \textit{interpretation for $I$}, noted by $\llbracket\bullet \rrbracket_J$, is a pair
	$(J,\llbracket\bullet\rrbracket)$ of a deductive  system $J$ together
	with a (functional) mapping $\llbracket \bullet \rrbracket: U_I\rightarrow U_J$ on propositions of $I$ into propositions of $J$ extended to multisets of 
	formulae of $U_I$ with the following properties:
	\begin{enumerate}
		\item \textit{Top preservation}. $\llbracket\top_I \rrbracket = \top_J$
		% \item\textit{structural interpretation of connectives}. The interpretation of compound formulas built upon primitive connectives of $I$ is structural.
		% E.g. Given $Con$ is a binary primitive connective of $I$ then in prefix notation we want:  
		% $\llbracket  Con(A, B) \rrbracket_J = \llbracket\ Con\rrbracket (\llbracket A \rrbracket , \llbracket \ B \rrbracket$ where $\llbracket Con\rrbracket$ is a connective in $J$.
		\item \textit{Structural interpretation of contexts}. For  $\Gamma$ contexts of the form $A_1,\ldots, A_n$:
		$$\llbracket\Gamma \rrbracket=\llbracket A_1  \rrbracket,\ldots,  \llbracket A_n\rrbracket$$ (trivially empty contexts map to empty contexts. 
		As in \cite{Lambek1968} they can be treated as 
		the $\top$ element).
	\end{enumerate}
\end{definition}
\begin{definition}Given a deductive system $I$  and an  interpretation $\llbracket\bullet\rrbracket_J$ for $I$ we define
	a \textit{corresponding validation of a deduction $\Sigma_I;\Gamma\vdash_I A$}  
	as a deduction $\Sigma_J;\Delta\vdash_{J} \phi$ in $J$ such that $\llbracket A \rrbracket=\phi$ and $\Delta=\llbracket \Gamma \rrbracket $ . We will be writing
	$ \llbracket \Sigma_I;\Gamma\vdash_I A\rrbracket_J$ to denote such a validation.
\end{definition}
\begin{definition}
	Given a deductive system $I$, we say that an interpretation  $\llbracket\bullet \rrbracket_J$  is \textit{logically complete} when  for
	all purely logical deductions $\mathcal{D}$ (i.e. deductions that make no use of $\Sigma_I$) in $I$ 
	there exists a corresponding (purely logical) validation $\llbracket\mathcal{D}\rrbracket$ in $J$.
	i.e. $$\forall \mathcal{D}. \ \mathcal{D}:\Gamma\vdash_I A \Longrightarrow \exists \llbracket\mathcal{D}\rrbracket: \llbracket \Gamma\vdash A\rrbracket_J$$
\end{definition}
\footnote{Note, that we require existence but not uniqueness. Nevertheless, if we treat deductive systems  in a proof irrelevant manner as preorders 
the above definition gives uniqueness vacuously. In a more refined approach where $I$ and $J$ are viewed as  categories of proofs  the above ``logical completeness''  
translates to the requirement that if the set of (purely logical) arrows $Hom_I(\Gamma,A)$ is non empty  then $Hom_J(\llbracket\Gamma\rrbracket,\llbracket A\rrbracket_J)$ 
cannot be empty (i.e. that $\llbracket\bullet \rrbracket_J$ can be extended to a functor). We leave a complete categorical semantics of our logic 
for future work but we expect   a generalization of the 
endofunctorial interpretations of $K$ modality appearing in \cite{Bellin2001,kavvos2016system}.} 


Examples of triplets ($I$, $J$, $\llbracket\bullet \rrbracket_J$) of logical systems that fall under the definition above are: any intuitionistic system mapped 
to a classical one under the embedding $\llbracket A \supset B\rrbracket= \tilde{\neg} A \tilde{\vee} B$ where  $\tilde{\neg}$
and $\tilde{\vee}$ are classical connectives, the opposite direction under double negation translation, 
an  intuitionistic system  mapped to another intuitionistic system (i.e. a mapping of atomic formulas of $I$ to atomic formulas of $J$ extended naturally to the intuitionistic connectives 
or, simply, the identity mapping) etc.  A vacuous validation  (when $\llbracket\bullet  \rrbracket_J$ maps everything to $\top$) gives another example. 

The main thesis that is exposed in this chapter is that this notion of ``double proof'' (reasoning about proofs that exists in two related systems) 
provides for an understanding of necessity in proof theoretic terms. In addition, we argue, 
that this is the driver of (at least) the simplest
form of necessity ($K$) that appears in justification logic (\textit{necessity as internalization}).
We will focus on the case where $I$ (the propositional part of our logic) is based on the implicative fragment of intuitionistic logic and show how justification logic
provides for an axiomatization of such logically complete interpretations $\llbracket\bullet\rrbracket_J$ of  implicative intuitionistic logic.
In what follows we provide  a natural deduction for
an intuitionistic system $I$ (truth), an axiomatization/specification of   $\llbracket \bullet \rrbracket_J$ (treated abstractly as a function symbol on types) and a treatment 
of basic necessity that relates the two deductions by internalizing  a  notion of ``double truth'' (proof in $I$  and  existence of corresponding validation in  $J$).
\section{Judgments of Jcalc$^{-}$}
\label{lsjcalc}

We aim for a reading of necessity that internalizes a notion of ``double proof''  in two deductive systems.
Motivated by the discussion and definitions in the previous section we will treat the notion of interpretation abstractly -- as a function symbol on types -- 
and axiomatize in accordance. Intuitively we want:
$$\Box  A\  {\sf true} :=  A\ {\sf true}\   \& \ A\ {\sf valid} = A{\sf\  true\  in\  I}\  \& \  \llbracket  A \rrbracket \ {\sf true\  in\  J}$$
We will be dropping indexes $I$, $J$ since they can be inferred by the different kinds of assumption  contexts. In addition, we  omit signatures 
$\Sigma$ since they do not offer anything from a logical perspective.

Logical entailment for the proposed  $\Box$ connective can be  summarized  easily given our previous discussion.
Given a deduction $\mathcal{D}:A \vdash B$ and  the existence of validation $\llbracket\mathcal{D}\rrbracket:\llbracket A \rrbracket\vdash \llbracket B\rrbracket$ then 
given $\Box A$ (i.e.  a proof of a  $\vdash A$ and a validation $\vdash \llbracket A \rrbracket$)
we obtain a double proof of $B$ (and hence, $\Box B$) by \textit{compositionality} of the underlying systems.  
Using standard, proof tree notation with labeled assumptions we formulate our rule of the connective in natural deduction:

\mbox{\small
	\begin{mathpar}
		\inferrule* [right=$I_{\Box B} E^{x,s}_{\Box A}$]{{\infer*{\Box  A}{}}
			\\{\inferrule*{}
				{\inferrule*[vdots=1.5em, right=$x$]{ }{ A}\\\\
					\inferrule*[]{}{ B}}}\\
			{\inferrule*{}
				{\inferrule*[vdots=1.5em, right=${s}$]{ }{\llbracket   A\rrbracket}\\\\
					\inferrule*[]{}{\llbracket  	B\rrbracket}}}
		}{\Box  B} 
		
	\end{mathpar}
}
We can, easily, generalize to $\Box$ed contexts (of the form $\Box A_1, \ldots,  \Box A_i$) of arbitrary length:
\\
\mbox{\small
	\begin{mathpar}
		\inferrule*[right=$I_{\Box B}E^{\vec{x},\vec{s}}_{\Box A_1\ldots \Box A_i} $] {{\infer*{\Box  A_1}{}}\\{\ldots}\\
			{\infer*{\Box  A_i}{}}\\	{\inferrule*{}
				{\inferrule*[vdots=1.5em, right=$\vec{x}$]{ }{\Gamma': A_1,\ldots  A_i}\\\\
					\inferrule*[]{}{ B}}}\\
			{\inferrule*{}
				{\inferrule*[vdots=1.5em, right=$\vec{s}$]{ }{\llbracket \Gamma'\rrbracket:\llbracket  A_1\rrbracket,\ldots \llbracket  A_i\rrbracket}\\\\
					\inferrule*[]{}{\llbracket  B\rrbracket}}}
		}{\Box  B}
		
	\end{mathpar}
}
\\
We read as ``Introducing $\Box B$ after eliminating $\Box A_1 \ldots \Box A_i $ crossing out (vectors of) labels $\vec{x}, \vec{s}$ ". 
Interestingly, the same rule eliminates boxes  and introduces new ones. 
This  is not surprising for $K$ modality (it is a left-right rule as we will see (\ref{seqcalc}).
See also discussion in \cite{Bellin2001,Bierman2000}). We will be referring to this rule  as ``$\Box$ Intro--After--Elim'' or, simply $\Box_{IE}$, from now on.
%Moreover, it has a very clear computational reading in our system which we will be explaining. 
%In a nutshell, theorems of $K$ correspond to programs that consume (eliminate) congruences on subterms to construct construct congruences on larger terms. In that sense, the elim-intro mix is quite natural.

Note that we define the $\Box$ connective negatively, yet (pure) introduction rules for the $\Box$ 
connective are  derivable. 
Such are instances of the previous Intro--After--Elim rule when 
$\Gamma'$ is empty which conforms exactly with the idea  of necessity internalizing double theoremhood.

\mbox{
	\begin{mathpar}	
		\inferrule*[right=$I_{\Box  B}$]{{\vdash B }\\{\vdash \llbracket B \rrbracket}}{\Box B}
	\end{mathpar}}
	
	In the next section, we provide the whole calculus in natural deduction format. As expected
	we will extend the implicational fragment of intuitionistic logic with 
	\begin{itemize}
		\item Judgments about validity (justification logic).
		\item Judgments that relate truth and validity (modal judgments).
	\end{itemize}
	%The former judgments could be viewed as a "Hilbert-style" encoding of the underlying intuitionistic one. Justification logic uses a combinatory calculus (extended with classical combinators) to represent logical principles (constants) in this second level. We follow the same route, obtaining a Curry Howard Correspondance for Justification logic under this new ``necessity as double proof" doctrine. 
	\subsection{Natural Deduction for Jcalc$^{-}$}
	The treatment of necessity in the previous section is completely orthogonal to the underlying systems \
	(it just assumes the basic requirements stated for the behavior $\llbracket\dot\rrbracket$). 
	In this section we will provide a full calculus and in congruence with justification logic we will
	assume that the underlying system ($I$) is a fragment of intuitionistic logic (the `negative' to be precise).
	The host theory $J$ can still remain unspecified, but the choice of $I$ informs for some specifications (to preserve
	completeness of logical deductions).   

	Following type theory conventions,  we first provide rules underlying type construction, then  rules for  well-formedness of (labeled) assumption contexts and rules  introducing and eliminating connectives. 
	The rules below should be obvious except for small caveat.
	 On the one hand, the type universe of $U_I$ and the proof trees of $I$ are 
	inductively defined as usual; on the other 
	hand, the host theory $J$ (its corresponding universe, connectives and  proof trees) is  ``black boxed''. What we actually axiomatize are the properties that all
	(logic preserving) interpretations of $I$ should conform to, independently of the specifics of the host theory. 
	Validity judgments should thus be read  as specifications
	of provability (existence of proofs) of any candidate $J$. 
	
	When we write $\llbracket \Gamma\rrbracket\vdash\llbracket\phi\rrbracket$ it reads as there exists derivation
	$\mathcal{D}$ : $\Delta\rrbracket\vdash_{J}\psi\rrbracket$ s.t. $\Delta=\llbracket\Gamma\rrbracket$ and $\psi = \llbracket \phi\rrbracket$ )
	
	We use ${\sf Prop_0}$  to denote  the type universe of $I$ and $\llbracket \sf Prop_0\rrbracket $ to denote its image under an interpretation, ${\sf Prop_1}$ denotes   modal (``boxed'') types
	and ${\sf Prop}$  the union of ${\sf Prop_0, Prop_1}$. We write $P_k$ with $k$ ranging in some subset of natural numbers to denote atomic propositions in $I$. 
	
	
	
	\begin{mdframed}[nobreak=true,frametitle={\footnotesize Judgments on Type Universe(s)}]
		\mbox{\small
			\begin{mathpar}
				\inferrule*[right= Atom] { } {P_k \in {\sf Prop_0}}
				\and
				\inferrule*[right=Top] { } {\top \in {\sf Prop_0}}
				\and
				\inferrule*[right=Conj] {{ A \in {\sf Prop_i }}\\ { B \in {\sf Prop_j}}} {  A \wedge B \in {\sf Prop_{max(i,j)}} } 
				%\inferrule*[right=Bot] { } {\bot \in {\sf Prop_0}} 
				%\and
				\and
				\inferrule*[right=Box] { A \in{\sf Prop_{0} }} {\Box  A\in{\sf Prop_{1}} }
				%\and
				%\inferrule*[right= Arr] {{ A \in {\sf Prop_{i} }}\\ { B \in {\sf Prop_{j}}}} { A\supset  B\in {\sf Prop_{max(i,j)}}}
				\and
				\inferrule*[right= Arr] {{ A \in {\sf Prop_i }}\\ { B \in {\sf Prop_j}}} { A\supset  B\in {\sf Prop_{max(i,j)}}}
				\and
				\inferrule*[right=Brc] { A \in {\sf Prop_0 }} {\llbracket  A\rrbracket \in {\sf \llbracket Prop_{0}\rrbracket}}
				%\and
				%\inferrule*[right=Brc $\supset$ Eq] {\llbracket A\supset\psi\rrbracket \in {\sf \llbracket Prop_{0}\rrbracket }} {\llbracket A\supset \psi\rrbracket=\llbracket A\rrbracket\supset \llbracket\psi\rrbracket :{\llbracket\sf Prop_{0}\rrbracket} }
			\end{mathpar}
		}
	\end{mdframed}
	
	%From now on we will be omitting the type construction derivations and write, informally, $ A\in U$( or, for multisets, $\Gamma\in U$) denoting the unique (\ref{bftu}) such derivation(s). 
	
	
	
	
	For labeled contexts of assumptions we require standard wellformedness conditions (i.e. uniqueness of labels).
	We use letters $x_i$, or simply $x$, for labels of  contexts with assumptions in $\sf Prop_0$, $x_i'$ or simply $x'$ for contexts with assumptions in   $\sf Prop_1$ and $s_i$, or simply $s$,  
	for $\sf \llbracket Prop_0 \rrbracket$ contexts. 
	We use $\circ$ for the empty context of ${\sf Prop_0}$ and ${\sf Prop_1}$ and  $\dagger$ for the empty context of ${ \llbracket {\sf Prop_0}\rrbracket}$.
	%We ``overload" the use of $\in$ symbol and write $x\not\in\Gamma$ to denote that ``the label $x$ is not present in the domain of $\Gamma$".
	We abuse notation and write $x:A\in\Gamma$ (or, similarly, $s:\llbracket A\rrbracket\in\Delta$) to denote that the label $x$ is assigned type $A$ in $\Gamma$; or $\Gamma\in {\sf Prop_0}$ 
	(resp. $\Gamma \in {\sf Prop_1}$, $\Delta\in {\sf \llbracket Prop_0\rrbracket}$)  
	to denote that  $\Gamma$ is a wellformed context  with  co--domain of elements in ${\sf Prop_0}$ (resp. in ${\sf Prop_1}$, $\llbracket {\sf Prop_0}\rrbracket$).
	For $\Gamma \in {\sf Prop_0}$ we define  $\llbracket \Gamma\rrbracket$ as  the lifting of the context $\Gamma$ through the $\llbracket \bullet \rrbracket$  symbol 
	(with appropriate renaming of variables -- e.g. $x_i\rightsquigarrow s_i$). For the vacuous case  when $\Gamma$  is  empty 
	we require $\llbracket\circ \rrbracket = \dagger$ to be well formed.
	
	\begin{mdframed}[nobreak=true,frametitle={\footnotesize Judgments on Context Wellformedness}]
	\mbox{\footnotesize
	\begin{mathpar}
	\inferrule*[right=\small{Nil}] { }{\Turn {\circ} {\sf wf}}
	\and
	\inferrule*[right=$\Gamma$-Ext] { {\Gamma\in {\sf Prop_0}}\\ { A \in {\sf Prop_0}} \\{x\not\in \Gamma}}{{\Gamma, x: A}\in {\sf Prop_0}}
	\and
	%\and
	%\inferrule*[right=\small{Nil}] { } {\Turn {\circ} {\sf\llbracket wf\rrbracket}}
	\inferrule*[right=$\llbracket$\small{ Nil} $\rrbracket$]{ }{\Turn {\dagger= \llbracket \circ \rrbracket} {\sf\llbracket wf\rrbracket}}
	\and
	\inferrule*[right=$\llbracket\Gamma\rrbracket$] { {  \Gamma  \in {\sf  Prop_0}}} {\llbracket \Gamma \rrbracket \in { \llbracket \sf Prop_0 \rrbracket}}
	\end{mathpar}}
	\end{mdframed}
	%Certain basic facts about well formed contexts are shown in the appendix \ref{bfcw}. 
	In the following entry we define proof trees (in turnstile representation) of the intuitionistic source theory $I$.  For all following rules we assume $\Gamma, A,B \in {\sf Prop_0}$:
	\begin{mdframed}[nobreak=true,frametitle={\footnotesize Judgments on Truth  $\Gamma, A,B \in {\sf Prop_0}$ }]
		\label{jots}
		\mbox{\small
			\begin{mathpar}
				\inferrule*[right=$\Gamma_0$-Refl] {x: A \in \Gamma}{\Turn {\Gamma} { A}}
				\and
				\inferrule*[right=$\top_0$I] { }{\Turn {\Gamma} { \top}}
				\and
				\inferrule*[right=$\supset_0$I] {{\Turn {\Gamma, x: A} { B}}} {\Turn {\Gamma} {   A\supset  B}}
				\and
				\inferrule*[right=$\supset_0$E] {{\Turn {\Gamma} { A\supset  B}}\\{\Turn {\Gamma} { A}}} {\Turn {\Gamma} {   B}}
				
				%\inferrule*[right=$\bot$E] {{\Turn {\Gamma} {\bot}}}{\Turn {\Gamma} {   A}}
			\end{mathpar}}
		\end{mdframed}
		
		For the calculus of interpretation (validity) we demand context reflexivity, compositionality and logical completeness with respect to  intuitionistic implication.
		Logical completeness is specified axiomatically, since the host theory is ``black boxed''. 
		Following justification logic, we use an axiomatic characterization of combinatory logic (for $\supset$) together 
		with the requirement that the interpretation preserves modus ponens:
		\begin{mdframed}[nobreak=true,frametitle={\footnotesize Judgments on Validity with {$\Delta\in \llbracket {\sf Prop_0} \rrbracket$}}]
			\label{jov}
			\mbox{\footnotesize
				\begin{mathpar}
					\inferrule*[right=$\Delta$-Refl] {s:\llbracket  A\rrbracket \in \Delta}{\Turn {\Delta} {\llbracket  A\rrbracket}}
					\and
					\inferrule*[right=Ax$_1$] { } {\Turn {\Delta} {\llbracket  \top\rrbracket}}
					\and
					\inferrule*[right=  Ax$_2$]{  A, B \in {\sf Prop_0} }   {\Delta\vdash \llbracket   A \supset (B \supset   A)\rrbracket }
					\and
					\inferrule*[right=  Ax$_3$]{ {  A,B, C \in {\sf Prop_0}}}{\Delta\vdash\llbracket   A\supset (B \supset C) \supset ((  A\supset B) \supset (  A \supset C))\rrbracket}
					\and
					\inferrule*[right=MP] {{\Turn {\Delta} { \llbracket  A \supset  B \rrbracket}}\\ {\Turn{\Delta} {\llbracket  A \rrbracket}}}{\Turn {\Delta} {\llbracket  B\rrbracket}}
				\end{mathpar}}
			\end{mdframed}
			
			Finally, we have judgments in the $\Box$ed universe (${\sf Prop_1}$). These are context reflection, the $\Box$ Intro-After-Elim rule, and the rules for intuitionistic implication between $\Box$ed types
			\footnote{The implication and elimination rules in ${\sf Prop_1}$ actually coincide with the ones
				in ${\sf Prop_0}$ since we are focusing on the case where $I$ is intuitionistic. This need not necessarily be the case as we have explained. Intuitionistic  implication among $\Box$ types should be read as
				``double proof of $A$ implies double
				proof of $B$'' and would still be defined even if we did not observe any kind of  implication in $I$. Similarly, one could provide intuitionistic conjunction or disjunction between $\Box$ types independently of 
				$I$ and, vice versa, one could add connectives in $I$ that are not observed between $\Box$ed types.}. 
			\begin{mdframed}[nobreak=true, frametitle={\footnotesize Judgments on Necessity with $\Gamma\in {\sf Prop_1} \text{,{\ \sf length}}(\Gamma)=i\text{,\ }
					\ 1\le k\le i  \text{\ and, }\Gamma^{\prime},A, A_k,  B\in {\sf Prop_0}$ }]
				\mbox{\footnotesize
					\begin{mathpar}
						\inferrule*[right=$\Gamma_1$-Refl] {x^{\prime}: \Box A \in \Gamma}{\Turn {\Gamma} {\Box A}}
						\and
						\inferrule*[right=$I_{\Box B}E^{\vec{x},\vec{s}}_{\Box A_1\ldots \Box A_i}$]{{(\forall  A_i \in \Gamma'. \ \Turn {\Gamma}{\Box  A_i})}\\{\Turn {\Gamma'} { B}}\\{\Turn {\llbracket \Gamma' \rrbracket} {\llbracket  B\rrbracket} }} {\Turn {\Gamma}\Box  B}
						\and
						\inferrule*[right=$\supset_1$I] {{\Turn {\Gamma, x^{\prime}: \Box A} { \Box B}}} {\Turn {\Gamma} {   \Box A\supset  \Box B}}
						\and
						\inferrule*[right=$\supset_1$E] {{\Turn {\Gamma} { \Box A\supset  \Box B}}\\{\Turn {\Gamma} { 
									\Box A}}} {\Turn {\Gamma} {  \Box B}}
					\end{mathpar}}
				\end{mdframed}
				
				%Observe, that as a consequence of our restrictions in the universe of propositions, all instances of the previous rule have $\Gamma'\in {\sf Prop_0}$. 
				%This is not a premise of the rule since it is can be proven, easily, meta--theoretically.
				\subsubsection{(Pure) $\Box I$ as derivable rule}
				We stress here that $\Box$ can be introduced positively with the previous rule with $\Gamma^{'}=\circ$. The first premise reduces to a simple requirement that $\Gamma\in{\sf Prop_1}$.
				
				\mbox{\small
					\begin{mathpar}
						\inferrule*[right=$I_{\Box A}$]{{\Turn {\circ} { A}}\\{\Turn {\dagger} {\llbracket  A\rrbracket} }} {\Turn {\Gamma}\Box  A}
					\end{mathpar}}
					\subsubsection{A simple derivation}
					\label{smpdv}
					We show here that the $K$ axiom of modal logic is a theorem (omitting some obvious steps). In the following {\small $$\Gamma := x_1^{\prime}:\Box (A\supset B),x_2^{\prime}:\Box A, \ \Gamma^{\prime}=x_1:A\supset B, x_2:A,
						\ \llbracket \Gamma^\prime\rrbracket= s_1:\llbracket A \supset B\rrbracket, s_2:\llbracket A\rrbracket$$}
					\mbox{\small
						\begin{mathpar}
							\inferrule*[right=$\supset_1$I]{\inferrule*[right=$\supset_1$I]{
									\inferrule*[right= $I_{\Box A}E^{x_1,x_2,s_1,s_2}_{\Box A\supset B, \Box A}$]{{\inferrule*[]{}{\Gamma \vdash \Box ( A \supset B)}}\\
										{\inferrule*[]{}{ \Gamma\vdash \Box  A }}\\{\inferrule*[]{}{\Gamma^\prime\vdash B}}\\{\inferrule*[]{}{\llbracket\Gamma^\prime\rrbracket\vdash \llbracket B\rrbracket}}} {\Box ( A\supset B), \Box  A \vdash \Box B }} {\Box ( A\supset B) \vdash \Box  A \supset \Box B}}{\circ\vdash \Box ( A\supset B)\supset \Box  A \supset \Box B}
						\end{mathpar}
					}
\subsection{Logical Completeness, Admissibility of Necessitation and Completeness with respect to Hilbert Axiomatization}
\label{completness}
Here we give a Hilbert  axiomatization  of the $\supset$ fragment of intuitionistic $K$ logic in order to compare it with our system. Here  $\vdash^{\mathcal{H}}$ captures the textbook (metatheoretic)
notion of ``deduction from assumptions'' in a Hilbert style axiomatization. We assume the restriction of the system to formulas up to modal degree $1$.
\begin{mdframed}[nobreak=true,frametitle={\footnotesize Hilbert Style Formulation}]
	\mbox{\footnotesize
		\begin{mathpar}
			\inferrule*[left=ax1.]{}{ A\supset (B \supset  A)}
			\and
			\inferrule*[left=ax2.]{}{ (A\supset (B \supset C) )\supset (( A\supset B) \supset ( A \supset C))}
			\and
			\inferrule*[left=K.]{}{\Box ( A\supset B)\supset\Box  A \supset \Box B}
			\and
			\inferrule*[left=MP]{ { A \supset B}\\ { A}}{B}
			\and
			\inferrule*[left=Nec]{\vdash^{\mathcal{H}}  A}{\Box  A}
		\end{mathpar}}
	\end{mdframed}
	
	It is easy to verify that axioms $1$, $2$ are derived theorems of {\sf Jcalc$^{-}$} in ${\sf Prop_0}$. The rule Modus Ponens is also admissible trivially, whereas axiom $K$  was  
	shown to be a theorem in the previous section (\ref{smpdv}). The rule of Necessitation is not obviously admissible though. In our reading of necessity the
	admissibility of this rule is directly related to the requirement of ``logical completeness of the interpretation'' i.e. preservation of logical theoremhood.  
	In general, adding more connectives in $I$ would require additional specifications for the host theory to obtain necessitation.
	
	The steps of the proof are given in the Appendix, but this is essentially  the ``lifting lemma'' in justification logic \cite{Artemov2001}. 
	The proof fully depends on
	the provability requirements imposed in the $\llbracket{\sf Prop_0}\rrbracket$ fragment.
	\begin{theorem}[$\Box$Lifting Lemma]
		In {\sf Jcalc$^{-}$}, for every  $\Gamma,   A \in {\sf Prop_0}$ if  $\Gamma\vdash A$ then  $\llbracket \Gamma\rrbracket\vdash\llbracket A\rrbracket $ and, hence, $\Box\Gamma \vdash \Box   A$.
	\end{theorem}
	We get admissibility of necessitation as a lemma for $\Gamma$ empty:
%	\begin{theorem}[Admissibility of Necessitation]{lemma}
		
%		For $  A \in {\sf Prop_0}$, if $\circ\vdash   A$ then $\circ\vdash \Box A$. 
%	\end{threorem}
	As a result:
	\begin{theorem}[Completeness]
		{\sf Jcalc$^{-}$} is complete with respect to the Hilbert style formulation of degree-$1$ intuitionistic $K$ modal logic. 
	\end{theorem}
	
	\subsection{Harmony: Local Soundness and Local Completeness}
	\label{gprinc}
	Before we move on to show (Global) Soundness we provide evidence for the so called ``local soundness" and ``local completeness'' of the $\Box$ connective
	following Gentzen's dictum. The local soundness and completeness for the $\supset$ connective 
	is given elsewhere (e.g. \cite{prawitz10natural}) and in Gentzen's original \cite{gentzen1935untersuchungen}. Gentzen's program can be described with the following two slogans:\begin{itemize} \item[a.] Elim is left-inverse to Intro \item[b.] Intro is right-inverse to Elim\end{itemize}   
	Applied to the $\Box$ connective, the first principle says that introducing a $\Box   A$ (resp. many $\Box A_1, \ldots, \Box A_i$) only to eliminate it (resp. them) directly is redundant. 
	In other words, the elimination rule cannot give you more data than what were inserted in the introduction rule(s)  (``elimination rules are not \textit{too} strong").
	We show first the ``Elim-After-Singleton-Intro" sub-case.
	
	\mbox{\footnotesize
		\begin{mathpar}
			\\
			\inferrule*[right=\large{$\quad\Longrightarrow_{R}\quad$}]{
				\inferrule*{}
				{\inferrule*[]{}{ \inferrule*[]	{\inferrule*[]{}{\PrTri{D} \\ \PrTri{E}}\\\\
							\inferrule*[]{}{\    A\\ \ \qquad \llbracket   A\rrbracket}}{\Box   A }}} \quad
				\inferrule*{}
				{\inferrule*[vdots=1.0em, right=$x$]{ }{  A}\\\\
					\inferrule*[]{}{B}} 	   \qquad 
				\inferrule*{}
				{\inferrule*[vdots=1.0em, right=$s$]{ }{\llbracket   A \rrbracket}\\\\
					\inferrule*[]{}{\llbracket B\rrbracket}} 	
			}
			{\Box B}  \inferrule*[]{
				\inferrule*{}
				{\inferrule*[vdots=1.0em]{}{\PrTri{D} \\\\  A}\\\\
					\inferrule*[]{}{B}}
				\qquad 
				\inferrule*{}
				{\inferrule*[vdots=1.0em]
					{}{\PrTri{E}\\\\\llbracket   A \rrbracket}\\\\
					\inferrule*[]{}{\llbracket B \rrbracket}} 
			}{\Box B}
			
		\end{mathpar}
	}
	The exact same principle applies in the ``Elim-after-Intro''  of   multiple $\Box$s:
	
		\mbox{\footnotesize
			\begin{mathpar}
				\inferrule*[right=$  I_{\Box B} E_{\Box   A}^{x,s}$]{
					\inferrule*{}
					{\inferrule*[]{}{ \inferrule*[]	{\inferrule*[]{}{\PrTri{$D_1$} \\ \PrTri{$E_1$}}\\\\
								\inferrule*[]{}{ \ A_1\\ \ \qquad\ \ \  \llbracket   A_1\rrbracket}}{\Box   A_1 }}}\ldots 
					{\inferrule*[]{}{ \inferrule*[]	{\inferrule*[]{}{\PrTri{$D_i$} \\ \PrTri{$E_1$}}\\\\
								\inferrule*[]{}{\    A_i\\ \ \qquad \llbracket   A_i\rrbracket}}{\Box   A_i }}}
					\quad
					\inferrule*{}
					{\inferrule*[vdots=1.0em, right=$\vec{x}$]{ }{ A_1\dots A_i}\\\\
						\inferrule*[]{}{B}} 	   \qquad 
					\inferrule*{}
					{\inferrule*[vdots=1.0em, right=$\vec{s}$]{ }{\llbracket   A_1 \ldots A_i \rrbracket}\\\\
						\inferrule*[]{}{\llbracket B\rrbracket}} 	
				}
				{\Box B} 
			\end{mathpar}
		}
		$$\Longrightarrow_{R}$$
		\mbox{\footnotesize
			\begin{mathpar}
				\inferrule*[right=$I_{\Box B}$]{
					\inferrule*{}
					{\inferrule*[vdots=1.0em]{}{\PrTri{$D_1$}\ \PrTri{$D_i$} \\\\ A_1\ldots\ldots  \ \  A_i}\\\\
						\inferrule*[]{}{B}}
					\qquad 
					\inferrule*{}
					{\inferrule*[vdots=1.0em]
						{}{\PrTri{$E_1$} \PrTri{$E_i$}\\\\\llbracket   A_1\ldots\ldots A_i\rrbracket}\\\\
						\inferrule*[]{}{\llbracket B \rrbracket}} 
				}{\Box B}
				
			\end{mathpar}
		}
	These equalities are of importance since they dictate (together with the corresponding principles for the $\supset$, $\wedge$ connectives) the proof dynamics of the calculus. 
	The proof term assignment and the corresponding computational ($\beta$-)rules  are directly instructed by these reduction principles. We see that eliminating (using) an introduced $\Box$ 
	corresponds to double substitution in the corresponding judgments. 
	
	Dually, the second principle says eliminating a $\Box A$ , should give enough information to directly reintroduce it (``elimination rules are not \textit{too weak}"). This is an expansion principle.
	
	\mbox{\small
		\begin{mathpar}
			\inferrule*[]{}{ \inferrule*[lab=$\ \mathcal{D}$]{}{\Box   A }}\quad  \Longrightarrow_{E}\quad
			\inferrule*[Right=$I_{\Box A}E^{x,s}_{\Box  A}$]{
				\inferrule*{}
				{\inferrule*[]{}{  }\\\\ \quad\inferrule*[]{}{ \inferrule*[lab=$\ \mathcal{D}$]{}{\Box   A }}\\
				} \qquad
				\inferrule*{}
				{\inferrule*[right=$x$]{ }{  A}}\\
				\inferrule*{}
				{\inferrule*[right=$s$]{ }{\llbracket A\rrbracket}
				} 	
			}
			{\Box   A} 
		\end{mathpar}}
		\subsection{(Global) Soundness}
		\label{seqcalc}
		Soundness is shown by proof theoretic techniques. Standardly, we add the bottom type ($\bot$) to {\sf Jcalc$^{-}$} together with its elimination rule and show  that the system is consistent ($\not\vdash \bot$) by   devising a sequent calculus  and showing admissibility of cut. We 
		only present the calculus here and collect the theorems towards consistency  in the Appendix. 
		
		In the following we use $\Gamma\Rightarrow   A$ (where $\Gamma,  A \in {\sf Prop_0}\cup{\sf Prop_1})$  to denote 
		sequents modulo $\Gamma$ permutations where $\Gamma$ is a multiset of ${\sf Prop}$ (no labels)  and $\Delta \Rightarrow\sf \llbracket   A \rrbracket$ for sequents corresponding to $\llbracket \sf judgments \rrbracket$ of the calculus modulo $\Delta$ permutations (with $\Delta$ (unlabeled) multiset of ${\sf \llbracket Prop_0 \rrbracket}$). The multiset/ modulo permutation approach is instructed by standard structural properties. 
		All properties are stated formally and proved in the Appendix. 
		
		The  $\llbracket\Gamma\rrbracket \Rightarrow \llbracket A \rrbracket$ relation  is defined directly from  $\vdash$:
		\begin{mdframed}[nobreak=true,frametitle={\footnotesize Sequent Calculus ($\llbracket {\sf Prop_0} \rrbracket$)}]
			$$\begin{array}{l r}
			{\llbracket\Gamma\rrbracket} \Rightarrow \llbracket A\rrbracket:= & \exists \Gamma^{\prime}\in \pi(\llbracket\Gamma\rrbracket)\ \text{s.t} \   
			\Gamma^{\prime}\vdash \llbracket A \rrbracket \end{array}$$
			where $\pi(\llbracket\Gamma\rrbracket)$ is the collection of  permutations of $\llbracket\Gamma\rrbracket$.
		\end{mdframed} 
		
		
		\begin{mdframed}[nobreak=true,frametitle={\footnotesize Sequent Calculus ({\sf Prop})}]
			\mbox{\small
				\begin{mathpar}
					\inferrule*[right=$Id$] { }{\Gamma, A  \Rightarrow A }
					
					\inferrule*[right=$\supset_L$] {{\Gamma, A\supset B, B \Rightarrow  C }\\ {\Gamma, A\supset B \Rightarrow A}} {\Gamma, A\supset B \Rightarrow  C}
					\and
					\inferrule*[right=$\supset_R$] {\Gamma, A \Rightarrow  B} {\Gamma \Rightarrow A\supset B}
					\and
					\inferrule*[right=$\bot_L$] { } {\Gamma, \bot \Rightarrow A}
					\and
					\inferrule*[right=$\Box_{LR}$] {{\Box\Gamma,\Gamma\Rightarrow A}\\{\llbracket\Gamma\rrbracket\Rightarrow \llbracket A \rrbracket }}{\Box\Gamma\Rightarrow \Box A}
					%\inferrule*[right=$\supset$E] {{\Turn {\Gamma} { A\supset  B}}\\{\Turn {\Gamma} { A}}} {\Turn {\Gamma} {   B}}
					%\and
					%\inferrule*[right=$\bot$E] {{\Turn {\Gamma} {\bot}}}{\Turn {\Gamma} {   A}}
				\end{mathpar}}
				%Where the  rule $\Box_{LR}$corresponds to $\Box_{IE}$ and relates the two kinds of sequents 
			\end{mdframed}
			
			Standardly, we  extend the system with the ${\sf Cut}$ rule and we  obtain the extended system $\Gamma \Rightarrow^{+} A:= \Gamma\Rightarrow A + {\sf Cut}$. 
			We show Completeness of $\Rightarrow^{+}$ with respect to Natural Deduction and  Admissibility of Cut that leads to the consistency result
			%\begin{theorem}[Admissibility of Cut]
			%$\Gamma \Rightarrow^{+}A$ implies $\Gamma\Rightarrow A$
			%\end{theorem}
			%Which together with the following theorems:
			%\begin{theorem}[Completeness of $\Rightarrow^{+}$ with respect to Natural Deduction]
			%If  $\Gamma\vdash A$ then $\Gamma \Rightarrow^{+}A$
			%\end{theorem}
			%Which together with the proposition: 
			%\begin{proposition}
			%$\centernot{\Rightarrow^{+}}\bot$
			%\end{proposition}
			\begin{theorem}[Consistency of {\sf Jcalc$^{-}$}]{theorem}{firstcon}
				$\centernot{\vdash}\bot$
			\end{theorem}
			
			
			
			%\begin{mathpar}
			%\inferrule*[right=K]  {  A\supset  B\supset A \in [Prop]_{i>0}}    {{\Delta} \vdash  {{\sf K}[ A, B]:  A\supset B\supset A}}
			%\and
			%\inferrule*[right=S]  { A\supset B\supset C \in [Prop]_{i>0}}    {{\Delta} \vdash  {{\sf S}[ A, B, C]: ( A\supset B\supset C)\supset( A\supset B)\supset( A\supset C)}}
			
			%\and
			%\inferrule*[right= App] {{\Turn {\Gamma, x: A} {M: B}}\\{\Turn {\Gamma} {\sf wf_{i>0}} }} {\Turn {\Gamma} {\lambda  x: A . \   M :  A\supset  B}}
			%\and
			%\inferrule*[right=]{{\Turn {\vec{v}:G} {t: A}}\\{\Turn {\vec{v'}:[G]} {t':[ A]} }} {\Turn {\vec{x}:\Box G}  let\  { \xshlongvec[1] {link(v,v')=x}\  {\sf in} \  link(t,t')}:\Box  A}
			%\end{mathpar}
			
			%Finally, we have the rule that relates judgments of truth to judgments of validity explained in the beginning of this section.
			\section{Order theoretic semantics}
			This chapter started by introducing 
			mappings between deductive systems and 
			motivating the reading of necessity as ``double-proof under a map''.
			As a result, it is unsuprising the the calculus is amenable to order theoretic sematics.
			We present them in this section.

			In order to progress we first define the notion of a \vocab{semi-Heyting  Algebra(semi-HA)}. 
			To define \vocab{semi-HA} we need the notion of a \emph{(meet) semi-lattice}.
			  
			
			\begin{mdframed}
			\textbf{Definition:}
			A \textit{(meet) semi-lattice} is a non-empty \emph{partial order} (i.e. reflexive, antisymmetric and transitive) 
			with finite meets.
			\end{mdframed}
			In addition, we define \emph{meet semi-lattice} as follows: 
			\begin{mdframed}
			\textbf{Definition:}
			A \textit{bounded (meet) semi-lattice} $(L,\le)$ is a (meet) semi-lattice that additionally has 
			a greatest element $1$, which satisfies
			
			$x \le 1$ for every $x$ in $L$
			\end{mdframed}
			Finally, we can define \emph{semi-HA}:
			
			\begin{mdframed}
			\textbf{Definition:}
			A \textit{semi-HA} is a bounded (meet) semi-lattice $(L,\le, 1)$ 
			s.t. for every $a,b\in L$ there exists an \textit{exponential} 
			(we name it $a\supset b$) 
			with the properties: 
			\begin{enumerate}
			\item $a\wedge a\supset b\le b $
			\item $x$ is the greatest such element
			\end{enumerate}
			\end{mdframed}
			\subsubsection{Axiomatization of semi-HAs}
			We can axiomatize the meet (i.e. greatest lower bound)($\wedge$) of $\phi,\psi$ for any  lower bound $\chi$.
			\begin{mdframed}
			\begin{mathpar}
			  \infer{\phi \conj \psi \leq \phi}{
				}
			  \and
			  \infer{\phi \conj \psi \leq \psi}{
				} 
			\end{mathpar}
			\begin{equation*}
			  \infer{\chi \leq \phi \conj \psi}{
				\chi \leq \phi & \chi \leq \psi} 
			\end{equation*}
			\end{mdframed}
			
			We can axiomatize the existence of a greatest element as follows:
			\begin{mdframed}
			\begin{equation*}
			  \infer{\chi \leq 1}{
				} 
			\end{equation*}
			which says that $1$ is the greatest element.
			\end{mdframed}
			Finally, to axiomatize \emph{semi-HAs} we require the existence of exponentials for every $\phi$, $\psi$ as follows:
			
			\begin{mdframed}
			\begin{mathpar}
			  \infer{\phi \wedge  (\phi\supset \psi)\leq\psi}{
				} 
				\and
				\infer{\chi\leq\phi\supset\psi}{\phi\wedge\chi\leq\psi}
			\end{mathpar}
			\end{mdframed}
			
			In addition, given two \emph{semi-HAs}, we are interested in order preserving 
			functions (functors) $F$ that also preserve products and exponentials: 
			\begin{mdframed}
				\textbf{Definition}
			A function $F$ between two (semi)-HAs ($HA_1$, $HA_2$) is order preserving
			and commutes with products and exponentials \emph{iff}
				\begin{enumerate}
				\item $\phi\le_{1}\psi\Rightarrow F\phi \le_{2} F\psi$
				\item{$F(\phi \wedge_{1}\psi) = F(\phi)\wedge_{2}(F(\psi))$} 
				\item{$F(\phi\supset_{1}\psi}) = F(\phi) \supset_{2} F(\psi)$}
				\end{enumerate}
			\end{mdframed}
			
			For the order theoretic models of  Jcalc $^{-}$ the following structures (triplets) 
			are of interest. We define a $J$-triplet as follows:
			\begin{mdframed}
				\textbf{Definition}
			A \emph{$J$-triplet} is 
				
			\begin{enumerate}
			\item A semi-Hayting algebra $HA$
			\item A partial order $J$
			\item An order preserving function $F$ from $HA$ to $J$ s.t.
			\begin{enumerate}
				\item The image $F(HA)$ forms a semi-Heyting Algebra
				\item $F$ preserves products and exponentials
			\end{enumerate}
			\end{enumerate}
		\end{mdframed}

			Given a $J$-triplet we define the induced pair algebra
			$\langle HA, F(HA)\rangle$
			and name it $\Box^{F}HA$ as follows:
			\begin{mdframed}
				\textbf{Definition} Given a $J$-triplet we define the induced 
				$\Box^{F}HA$ as follows:
				
			\begin{enumerate}
				\item Elements are pairs $\langle A, FA\rangle$ (name them $\Box^{F}A$) where $A\in HA$ and $FA$ its image
				\item For every two elements  $\Box^{F}A$, $\Box^{F}B$:
				$$\Box^{F}A \le_{\Box^{F}}B \Box^{F}B \text{iff} A\le{HA}B \text{and} FA\le_J FB $$^{\footnote
				Actually in a $J$-triplet case (
					actuallly in a $ A\le{HA}B$ implies $FA\le_J FB$ but the definition of pair algebras
					of proof systems can be generalized to weaker scenarios as we will sketch in the next section
				)}
			\end{enumerate}
		\end{mdframed}
		\begin{mdframed}
			\begin{theorem}\label{thm:PairAlgebra}
			$\Gamma\vdash_{IPL} \phi \true$ iff for any \vocab{Heyting Algebra} $H$ we have $\Gamma^+\leq\phi^{*}$ where $*$ is  defined as the lifting of any map of $\prop$s to elements of $H$ and $(+)$ is defined inductively on the length of $\Gamma$ as follows
			\begin{alignat*}{2}
			  nil^+  &&\quad = & \quad\top\\
			  (\Gamma,\phi)^+&&\quad = &\quad
			  \Gamma^+\wedge\phi* \
			\end{alignat*}
			\end{theorem}
			\end{mdframed}



\chapter{The computational side of Jcalc $^{-}$}
			In this section we add proof terms to represent natural deduction constructions. The  meaning of these terms emerges naturally from Gentzen's principles that give reduction (computational $\beta$-rules) and expansion (i.e. extensionality $\eta$-rules) equalities for the each construct. We focus on the new constructs of the calculus that emerge from the judgmental interpretation of the $\Box$ connective as explained in 
			section \ref{lsjcalc}.
			
			There will be no computational (reduction) rules on  provability terms. 
			This conforms with our reading of these terms  as \textit{references} to proof constructs of an \textit{abstracted} theory $J$ that can be realized 
			differently for a concrete $J$.  
			%This is, we posit, the logical basis for \textit{dynamic linking} under separate compilation. The linker from a language $I$ to any host language $J$ (that satisfies certain specifications) creates residuals dynamically but is not concerned about the actual execution of such residuals. Execution of such residuals happens at a next phase when the references are dereferenced to code of the host and is not observed by the calculus.\footnote{ Such approach, also conforms with justification logic principles. Justifications are static, purely syntactical constructs that are not further analyzed.} 
			\subsection{Proof term assignment}
			\label{basicpras}
			The following rules and their correspondence with natural deduction  constructs (\ref{jots}) should be obvious to the reader familiar with the simply typed  $\lambda$-calculus and basic justification logic.
			We do not repeat here the corresponding $\beta, \eta$ equality rules since they are standard.
			\begin{mdframed}[nobreak=true,frametitle={\footnotesize Judgments on Truth  $\Gamma, A,B \in {\sf Prop_0}$  and $M := x_i\  |\  <> \ |\ \lambda x:A.\ M\  |\  (M M) $}]
				\label{jot}
				\mbox{\small
					\begin{mathpar}
						\inferrule*[right=$\Gamma_0$-Refl] {x: A \in \Gamma}{\Turn {\Gamma} { x:A}}
						\and
						\inferrule*[right=$\top_0$I] { }{\Turn {\Gamma} { <>:\top}}
						\and
						\inferrule*[right=$\supset_0$I] {{\Turn {\Gamma, x: A} { M:B}}} {\Turn {\Gamma} { \lambda x:A.\  M:  A\supset  B}}
						\and
						\inferrule*[right=$\supset_0$E] {{\Turn {\Gamma} { M:A\supset  B}}\\{\Turn {\Gamma} {M^{\prime}: A}}} {\Turn {\Gamma} { (MM^{'}):  B}}
						\and
						\inferrule*[right=]{}{+\  \beta\eta \text{\ equalities for \ } \top,\supset}
						%\and
						%\inferrule*[right=$\bot$E] {{\Turn {\Gamma} {\bot}}}{\Turn {\Gamma} {   A}}
					\end{mathpar}}
				\end{mdframed}
				
				For  judgments of ${\sf \llbracket Prop_0\rrbracket}$, we assume a countable set of constant names and demand that every combinatorial
				axiom of intuitionistic logic has  a witness under the interpretation 
				$\llbracket\bullet\rrbracket$. This is what justification logicians call ``axiomatically appropriate constant specification''.
				As usual we demand reflection of contexts in $J$
				and preservation of modus ponens -- closedness under some notion of application (which we denote as $*$).
				
				%\begin{mathpar}
				%\inferrule*[right=$CS$] {{\Turn {\Delta} {\sf \llbracket wf \rrbracket}}\\ {C_i:\llbracket  A\rrbracket \in CS}} {\Delta\vdash_{CS}C_i:\llbracket  A\rrbracket}
				%\end{mathpar}
				
				\begin{mdframed}[nobreak=true,frametitle={\footnotesize Judgments on Validity  $\Delta\in {\sf \llbracket Prop_0\rrbracket}$  and ${\sf J} :=  s_i\ |\ C_i\  | \ {\sf J}*{\sf J} $}]
					\label{justf}
					\mbox{\small
						\begin{mathpar}
							\inferrule*[right=$\Delta$-Refl] {  {s:\llbracket  A\rrbracket \in \Delta}}{\Turn {\Delta} {s:\llbracket  A\rrbracket}}
							\and
							\inferrule*[right=  Ax$_1$]{ }   {\Delta\vdash C_{\top}: \llbracket  \top\rrbracket}
							\and
							\inferrule*[right=  Ax$_2$]{{  A, B \in {\sf Prop_0} }}   {\Delta\vdash C_{K^{A,B}}: \llbracket   A \supset (B \supset   A)\rrbracket }
							\and
							\inferrule*[right=  Ax$_3$]{ {  A,B, C \in {\sf Prop_0}}}{\Delta\vdash C_{S^{A,B,C}}:\llbracket   A\supset (B \supset C) \supset ((  A\supset B) \supset (  A \supset C))\rrbracket}
							\and
							\inferrule*[right=App] {{\Turn {\Delta} { {\sf J}: \llbracket  A \supset  B \rrbracket}}\\ {\Turn{\Delta} {{\sf J'}:\llbracket  A \rrbracket}}}{\Turn {\Delta} { {\sf J*J^\prime}:\llbracket  B\rrbracket}}
							
							
						\end{mathpar}
					}
				\end{mdframed}
				If  $J$ is a proof calculus and $\llbracket\bullet \rrbracket_J$ is  an interpretation such that the specifications above  
				are realized, then $J$ can witness intuitionistic provability. This can be shown by the proof relevant version of the lifting  lemma
				that states:
				\begin{lemma}[$\llbracket\bullet\rrbracket$Lifting Lemma]
					\label{bracklift}
					Given  $\Gamma,  A \in {\sf Prop_0}$ s.t. and a term $M$ s.t. $\Gamma\vdash  M: A $ then there exists ${\sf J}$ s.t  $\llbracket \Gamma\rrbracket \vdash {\sf J}:\llbracket   A \rrbracket$. 
				\end{lemma}
				
				
				
				
				\subsubsection{Proof term assignment and Gentzen Equalities for $\Box$ Judgments}
				Before we proceed, we will give a small primer of \textit{let}-bindings as used in modern programming languages to provide for some intuition on how such terms work. 
				Let us assume a rudimentary programming language that supports some basic types, say integers (${\sf int}$), as well as pairs of such types. Moreover, let us define a datatype 
				$\sf{Point}$ as a pair of ${\sf int}$ i.e. as $\sf {(int,int)}$ 
				In  a language with \textit{let}-bindings one can define a simple function that takes a ${\sf Point}$ and ``shifts'' it by adding $1$ to each of its $x$ and $y$ coordinates as follows:
				\begin{lstlisting}
				def shift (p:Point) = 
				let  (x,y) be p
				in
				(x+1,y+1)
				\end{lstlisting}
				If we call this function on the point ${\texttt{(2,3)}}$, then the computation ${\texttt{let (x,y) be (2,3) in (x+1,y+1)}}$ is invoked. This expression reduces following the \textit{let} reduction rule
				(i.e. pattern matching and substitution) to $\texttt{(2+1,3+1)}$; and as a result we obtain the value $\texttt{(3,4)}$.  As we will see, {\textit{let}} bindings -- with appropriate typing restrictions for our system -- 
				are used in the assignment of proof terms for the $\Box_{IE}$ rule. Moreover, the reduction principle for such terms ($\beta$-rule) -- obtained following Gentzen's equalities for the $\Box$ connective --  
				is exactly the one that we just informally described. 
				
				We can now move forward with the  proof term assignment for the $\Box_{IE}$ rule.  We show first the sub-cases for $\Gamma'$ empty (pure $\Box_I$)  and $\Gamma'$ singleton and explain the computational significance 
				utilizing Gentzen's principles appropriated for the $\Box$ connective. We are  directly translating proof tree equalities from \ref{gprinc} to proof term equalities. 
				We generalize for arbitrary $\Gamma'$ in the following subsection. We have, respectively, the following instances:
				
				\begin{mathpar}
					\inferrule*[]{ {\Gamma\in{\sf Prop_1}}\\{\Turn {\circ} { M:B}}\\{\Turn {\dagger} {{\sf J}:\llbracket  B\rrbracket} }} {\Turn {\Gamma} {  M\& {\sf J}:\Box  B}}
					\and
					\inferrule*[]{{ \Turn {\Gamma}{N:\Box  A}}\\{\Turn {x:A} { M:B}}\\{\Turn {s:\llbracket A \rrbracket} {{\sf J}:\llbracket  B\rrbracket} }} {\Turn {\Gamma} {{\sf let} \ (x\& s \ \ {\sf be\ } N) \ {\sf in}\  (M\& {\sf J}):\Box  B}}
				\end{mathpar}
				
				\subsubsection{Gentzen's Equalities for  ($\Box$ terms)}
				Gentzen's reduction and expansion principles give computational meaning (dynamics) and an extensionality principle for linking terms. We omit naming the empty contexts for economy.
				
				\mbox{\small
					\begin{mathpar}
						\inferrule*[Right=$I_{\Box B} E_{\Box A}^{x,s}$]{
							{\inferrule*[Left=$\Box_I$]{{\Gamma\in {\sf Prop_1}}\\ \vdash M:A\\ \vdash {\sf j}:\llbracket A \rrbracket}{\Gamma\vdash M \& {\sf j}:\Box A}}\\{x:A\vdash M':B }\\ {s:\llbracket A\rrbracket \vdash {\sf j'}:\llbracket B \rrbracket}}
						{\Gamma\vdash {\sf let} \ \ (x\& s)\ \ {\sf be}\ \   (M\&  {\sf J}) \ \ {\sf in}\ \  { (M^\prime \& {\sf J^\prime})}:\Box B }
					\end{mathpar}
				}$$\Longrightarrow_{R}$$
				\mbox{\small
					\begin{mathpar}
						\inferrule*[Right=$I_{\Box B}$]{\Gamma \in {\sf Prop_1}\\ \vdash M'[M/x]:B \\ \vdash {\sf J^{\prime}}[{\sf J}/s]:\llbracket B \rrbracket}{\Gamma\vdash { M^{\prime}[M/x]\& {\sf J^{\prime}}[{\sf j}/s]}:\Box B} 
					\end{mathpar}
				}
				Where the expressions $M^\prime[M/x]$ and ${\sf J^\prime[J/s]}$ denote capture avoiding substitution, reflecting proof compositionality of the two calculi.
				
				Following the expansion principle we obtain:
				{\small
					$$\begin{array}{c}
					\Gamma\vdash M:\Box   A
					\end{array} \ \Longrightarrow_{E}$$}
				\mbox{\small
					\begin{mathpar}
						\inferrule*[right=$I_{\Box A}E^{{x},{s}}_{\Box A}$]{{ \Turn {\Gamma}{M:\Box  A}}\\{\Turn {x:A} { x:A}}\\{\Turn {s:\llbracket A \rrbracket} {s:\llbracket  A\rrbracket} }} {\Turn {\Gamma} {{\sf let} \ (x\& s \ {\sf be \ } M) \ {\sf in}\  (x\& s):\Box  A}}
					\end{mathpar}}
					
					That gives an $\eta$-equality as follows:
					{\small
						$$M:\Box A =_{\eta}\ \ {\sf let} \ \ (x \& s\  {\sf be} \ M)\ \ {\sf in}\ \  { (x \& s)}:\Box A$$
					}
					The $\eta$ equality demands that every $M:\Box A$ should be reducible to a form $M'\& {\sf J^{\prime}}$.  
					\subsubsection{Proof term assignment for the $\Box$ rule (Generically)}
					After understanding the computational meaning of let expressions in the $\Box_{IE}$ rule 
					we can now give  proof term assignment  for the rule in the general case(i.e. for $\Gamma'$ of arbitrary length). 
					We define a helper syntactic construct --${\sf let}^{*}\ldots {\ \sf in\ }$-- as syntactic sugar for iterative  let bindings based on the structure  of contexts.
					The ${\sf let}^{*}$ macro takes four arguments: a context $\Gamma\in {\sf Prop_0}$, a  context $\Delta\in{\sf \llbracket Prop_1\rrbracket}$,  
					a possibly empty ($[\ ]$) list of terms  $Ns:=N_1,\ldots,  N_i$ - all three of the same length - and a term $M$. It is defined as follows for the empty and non-empty cases:
					%\footnote{Let us stretch here that when we speak about the structure of $\Gamma$ we imply (given the construction of $\Gamma\vdash{\sf wf}$) a treatment of contexts as lists where $\bullet$ stands for the empty list, singleton $x:A$ for $x:A + \bullet$ and contexts written in the form  $\Gamma,x:A$ for $x:A + \Gamma$ where $\Gamma$ is a list. I.e. contexts are lists but written down inversely with the head in the rightmost position. We treat the list of terms $Ns$ similarly for uniformity}  
					
					{\small
						$$\begin{array}{ll}
						\nonumber {\sf let}^{*}\ (\circ;\ \dagger;\  [\ ]) {\ \sf in\ }  M:= M \  &\\
						\nonumber {\sf let}^{*}\ (x_1:A_1,\ldots, x_i:A_i\ ;\  s_1: \phi_1, \ldots, s_i:\phi_i;\  N_1,\ldots,  N_i) {\ \sf in\ } M:= \  & \\
						{\sf let} \ \{(x_1 \& s_1)\  {\sf be}\  N_1,\ldots,  (x_i \& s_i)\  {\sf be}\  N_i\}\ {\sf in}\  M &\\
						\end{array}$$}
					Using this syntactic definition the rule $\Box_{IE}$ rule  can be written compactly:
					
					\begin{mdframed}[nobreak=true,frametitle= \footnotesize{$\Box_{IE}\ \text{ With}\ \Gamma\in {\sf  Prop_1}\text{,}\  \Gamma^{\prime}\in {\sf Prop_0}\text{,{\ \sf length}}(\Gamma)=i\text{,\ }Ns:=N_1 ...\  N_i\text{,}\ 1\le k\le i$} ]% $\Gamma^\prime\in {\sf Prop_0}$ ]
						\mbox{\small
							\begin{mathpar}
								\inferrule*[right=$I_{\Box B}E^{\vec{x},\vec{s}}_{\Box A_1\ldots \Box A_i}$] %{}
								{{\forall A_k \in \Gamma^\prime . \   \Gamma \vdash N_k:\Box A_k}\\
									{\Turn {\Gamma^\prime} {M:B}}\\{\Turn {\llbracket\Gamma^\prime\rrbracket} {{\sf J}:\llbracket B \rrbracket} }} 
								{\Turn {\Gamma} {{\sf let^{*}}\ (\Gamma^\prime, \llbracket\Gamma^\prime\rrbracket, Ns) \ {\sf in}  \ ( M\& {\sf J}):\Box B}}
							\end{mathpar}
						}
					\end{mdframed}
					It is obvious that all previously mentioned cases are captured with this formulation. The rule of $\beta$-equality can be given  for multi-let bindings directly from Gentzen's reduction principle (\ref{gprinc}) generalized for 
					the multiple intro case shown in the appendix (\ref{redmult}). 
					{\small
						$$\begin{array}{ll}
						{\sf let} \{(x_1 \& s_1) {\ \sf be\ } (M_1\& {\sf J_1}),\ldots,  (x_i \& s_i) {\ {\sf be}\ } (M_i\& {\sf J_i})\}\ {\sf in}\  (M \&  {\sf J})& =_{\beta} \\
						{  M[M_1/x_1, \ldots,  M_i/x_i] \& {\sf J}[{\sf J_1}/s_1,\ldots, {\sf J_i}/s_i]}
						\end{array}$$}
					\subsection{Strong Normalization and small-step semantics}
					In the appendix (\ref{norm}) we provide a proof of normalization for  natural deduction (via cut elimination). 
					This is ``essentially" a strong normalization result for the proof term system also. In general we have shown the congruence obtained from $=_{\beta\eta}$ rules gives a  consistent equational system.
					Nevertheless, we leave this for an extended version of this paper. Instead, we sketch briefly a weaker result: normalization under a  deterministic,``call-by-value" reduction strategy for $\beta$-rules.
					This   gives
					an idea of how the system computes  and we can use it in the applications in the next section. As usual we characterize a subset of the closed terms as values and we provide rules for the reduction of the non-value closed terms.
					Note that for the constants of validity and their applicative closure we do not observe reduction properties but treat them as values -- again conforming with the idea of $J$ (and its reduction principles) being ``black boxed''.
					\begin{mdframed}[nobreak=true,frametitle={\footnotesize Small step, call-by-value reduction $\rightarrow$}]
						\begin{mathpar}
							\inferrule*[] { }{\lambda x. M {\ \sf  value}}
							\and
							\inferrule*[] { } {C_i {\ \sf \ value}}
							\and
							\inferrule*[] {{\sf J_1} {\ \sf value} \\ {{\sf J_2} {\ \sf value}} }  {{\sf J_1*J_2} {\ \sf value}}
							\and
							\inferrule*[] {M\  {\sf value} \\ {{\sf J}\  {\sf value}} }  {M \&{\sf J} {\sf \ value}}
							\and
							\inferrule*[] {M \rightarrow M^\prime}  {M \&{\sf J}\rightarrow M^\prime \& {\sf J}}
							\and
							\inferrule*[] {{N_1 \ {\sf value}\  \ldots\text{ \ }  N_{k-1} \ {\sf value}}\\{N_k\rightarrow N_k^{\prime}}}  {{\sf let} \{(x_1 \& s_1) {\ \sf {be}\ } N_1,\ldots,
								\  (x_{k} \& s_{k}) {\ \sf {be } \ } N_k{\text{,}} \ldots\}    {\ \sf in}\  M \rightarrow \\
								{\sf let} \{(x_1 \& s_1) {\ \sf {be}\ } N_1,\ldots,
								\  (x_{k} \& s_{k}) {\ \sf {be } \ } N_k^{\prime}{\text{,}} \ldots\}    {\ \sf in}\  M }
							\and
							\inferrule*[] {M_1 \& {\sf J_1 \ value\ } \ldots\text{\ }  M_i \& {\sf J_i \ value}}  {{\sf let} \{(x_1 \& s_1) {\ \sf be\ } (M_1\& {\sf J_1}),\ldots,  (x_i \& s_i) {\ {\sf be}\ }
								(M_i\& {\sf J_i})\}\ {\sf in}\  (M \&  {\sf J})\rightarrow \\
								{  M[M_1/x_1, \ldots,  M_i/x_i] \& {\sf J}[{\sf J_1}/s_1,\ldots, {\sf J_i}/s_i]}}
							\and
							\inferrule*[] {M \rightarrow M^{\prime}}  {(MN) \rightarrow (M^\prime N)}
							\and
							\inferrule*[] {N \rightarrow N^{\prime} }  {((\lambda x. M)N) \rightarrow ((\lambda x. M)N^\prime) }
							\and
							\inferrule*[] {N {\ \sf value}}  {((\lambda x. M)N) \rightarrow [N/x]M }
							%\and
							%\inferrule*[right=$\bot$E] {{\Turn {\Gamma} {\bot}}}{\Turn {\Gamma} {   A}}
						\end{mathpar}
					\end{mdframed}
					Using the reducibility candidates proof method \cite{citeulike:993095}) we show:
					\begin{theorem}[Termination Under Small Step Reduction]
						With $\rightarrow^{*}$ being the reflexive transitive closure of $\rightarrow$: for every closed term $M$ and $A \in {\sf Prop}$ if $\vdash M:A$ then there 
						exists $N \ {\sf value}$ s.t. $\vdash N:A$ and  $M\rightarrow^{*} N$.
					\end{theorem}
					
					\section{A programming language view: Dynamic Linking and separate compilation}
					\label{dlinker}
					Our type system can be related to programming language design when considering \textit{Foreign Function Interfaces}. This is a typical scenario in which a language $I$ interfaces another language $J$ which is  essentially ``black boxed''.
					For example, {\sf OCaml} code  might call {\sf C} code to perform certain computations. 
					In such cases $I$ is a client and $J$ is a host that provides implementations for an interface utilized by the client.
					Through software development,  often the implementations of such an interface might change (i.e. a new version of the host language, or more dramatically, a complete switch of host language). 
					We want a language design that satisfies
					two  interconnected 
					properties. First, \textit{separate compilation} i.e. when implementations change we do not have to recompile client code and, yet, 
					secondly, \textit{dynamic linking} we want the client code to be linked dynamically to its new 
					``meaning''.
					
					We will assume that both languages are  functional and based on the lambda calculus. I.e. our interpretation function should have the property $\llbracket A\supset B\rrbracket_J$=
					$\llbracket A\rrbracket_J \llbracket\supset\rrbracket_J \llbracket B\rrbracket_J$ where  $\llbracket\supset\rrbracket_J$ is the implication type constructor in $J$.
					The specifics of the host  $J$ and the concrete implementations are unknown to $I$ but during the linker construction we assume that both languages share  some  basic types
					for otherwise  typed ``communication'' of the two languages would be impossible. 
					Simplifying,  we consider that the only  shared type is  (${\sf int}$), i.e. the linker construction assumes  
					$\bar{n}:\llbracket \sf int \rrbracket$ for every integer $n:{\sf int}$. 
					Let us now assume source code in $I$ that
					is  interfacing   a simple data structure, say an  integer stack,  with the following signature ${\sf \Sigma}$:
					\begin{lstlisting} 
					using type intstack
					empty: intstack, push: int -> intstack -> intstack,
					pop: intstack -> int
					\end{lstlisting}
					
					
					And let us consider a simple program in $I$ that is using the signature say, 
					$${\texttt{pop(push (1+1) empty):int}}$$
					This program involves two kinds of computations: a redex $(1+1)$ that can be reduced using the internal semantics of the language $1+1\rightsquigarrow_{I} 2$ and  
					the signature calls ${{\texttt{pop (push 2 empty)}}}$ 
					that are to be performed
					externally  in whichever host language implements them. 
					We treat  dynamic linkers as ``term re-writers'' that map  a computation to its meaning(s) based on different implementations.
					In the following we consider ${\sf \Sigma}$ to be the signature of the interface. Here are the steps towards the linker construction.
					
					\begin{enumerate}
						\item Reduce the source code based on the operational semantics of $I$ until it doesn't have a redex:
						\small{${\sf \Sigma}; \bullet\mathtt{\vdash_{} pop (push \ (1+1) \ Empty)\rightsquigarrow pop (push \ 2 \ Empty) :int}$}
						\item Contextualize the use of the signature at the final term in step $1$:{\small
							\begin{flalign*}
							& \mathtt{{\sf \Sigma}; x_1:intstack,  x_2:int\rightarrow intstack\rightarrow intstack, x_3:intstack\rightarrow int \vdash x_3 (x_2 \ 2\  x_1):int} &
							\end{flalign*}}
						\item Rewrite the previous judgment assuming (abstract) implementations for the corresponding missing elements
						using the ``known'' specification for the shared elements.
						{\small
							\begin{flalign*}
							& \mathtt{ s_1:\llbracket instack \rrbracket,  s_2:\llbracket int \rightarrow intstack\rightarrow intstack\rrbracket, 
								s_3:\llbracket intstack\rightarrow int \rrbracket\vdash s_3*(s_2* \bar{2}*s_1):\llbracket int\rrbracket}&
							\end{flalign*}}
						\item Combine the two previous judgments using the $\Box_{IE}$ rule.
						{\small
							\begin{flalign*}
							& {\sf \Sigma};\mathtt{x_1^{\prime}:\Box intstack ,x_2^{\prime}:\Box(int\rightarrow intstack\rightarrow intstack), x_3^{\prime}: \Box  (intstack\rightarrow int)\vdash} & \\
							& \mathtt{ let\{ x_1\& s_1 {\ \sf be \ } x_1^{\prime},\ x_2\& s_2 {\ \sf be \ } x_2^{\prime}, \  x_3\& s_3 {\ \sf be \ } x_3^{\prime} \}\  in}\   \mathtt{(x_3 (x_2\ 2 \ x_1)\ \& \ s_3*(s_2* \bar{2}*s_1)):\Box int} &
							\end{flalign*}}
						\item Using $\lambda$-abstraction three times we obtain the dynamic linker:
						{\small
							\begin{flalign*}
							& {\sf \Sigma};\circ\vdash \\
							& \mathtt{linker} = \mathtt{\lambda x_1^{\prime}.\  \lambda x_2^{\prime}}. \lambda x_3^{\prime}. \\
							& \mathtt{let \{ x_1\& s_1 {\ \sf be \ }x_1^{\prime},\ x_2\& s_2 {\ \sf be \ }  x_2^{\prime},\ x_3\& s_3 {\ {\sf be} \ } x_3^{\prime}\}\  in} \mathtt{(x_3(x_2\ 2 \ x_1)\ \&\  s_3*(s_2*\bar{2}*s_1))} &\\
							&\mathtt{:\Box(instack)\rightarrow \Box(int\rightarrow intstack\rightarrow intstack) \rightarrow \Box (intstack\rightarrow int) \rightarrow \Box int}
							\end{flalign*}}
					\end{enumerate}
					Let us see how it can be used in the presence of different implementations:
					\begin{enumerate}
						\item Suppose the developer  responsible for the implementation of the interface is providing an  array based implementation for the stack  in  some language $J$ 
						i.e. we get references to typechecked code fragments of $J$ as follows{\footnote{We have changed the return type of $\mathtt{pop}$ to avoid products. This is just for economy and products can easily be handled.}}:
						{\small
							\begin{flalign*}
							& \mathtt{create():intarray},\  \mathtt{add\_array:int_J \rightarrow_J intarray \rightarrow_J intarray } & \\
							& \mathtt{pop\_array:intarray \rightarrow_J int } &
							\end{flalign*}}
						\item  A unification algorithm check is performed to verify the conformance of the implementations to the signature taking into account 
						fixed type sharing equalities ($\llbracket {\sf int} \rrbracket = {\sf int_J}$). In our case it produces: $$\llbracket\rightarrow\rrbracket = \mathtt{\rightarrow_J}, \llbracket {\sf intstack} \rrbracket= {\sf intarray}$$
						\item
						We thus obtain typechecked links using the $\Box_I$ rule. For example:
						
						\mbox{\small
							\begin{mathpar}
								\inferrule*[]{{\Turn {{\sf \Sigma};\circ} { \mathtt{push: int \rightarrow intstack \rightarrow intstack }}}\\{\Turn {\bullet} {\mathtt{add\_array:\llbracket int \rightarrow intstack \rightarrow intstack\rrbracket  }} }} {\Turn {{\sf \Sigma};\circ} {  \mathtt{ push\  \& {\ \sf add\_array}:\Box (int \rightarrow intstack \rightarrow intstack)}}}
							\end{mathpar}
						}
						And analogously:
						
						\mbox{\small
							\begin{mathpar}
								\inferrule*[]{} {{\sf \Sigma}; \circ\vdash {  \mathtt{ pop\  \& \ {\sf pop\_array}:\Box (intstack\rightarrow int)}}}
								\and
								\inferrule*[]{} {{\sf \Sigma}; \circ\vdash {  \mathtt{ empty\  \& {\sf create()}:\Box intstack}}}	
							\end{mathpar}
						}
						\item Finally we can compute the next step in the computation for the expression  applying the linker to the obtained pairings:{\small\begin{flalign*}
							&{\sf \Sigma}; \bullet\mathtt{\vdash_{}(linker\   (empty\  \& \ create()) \ (push\  \& \ {\sf add\_array}) \ (pop\  \&\  pop\_array)):\Box int}& 
							\end{flalign*}} which reduces to:{\small\begin{flalign*}{\sf \Sigma}; \bullet\vdash&\mathtt{ let\{ (x_1\& s_1) {\ \sf be \ }(empty\  \& {\sf create()}),\ (x_2\& s_2) {\ \sf be\ } (push\  \& \ add\_array),\ \ (x_3\& s_3) {\ \sf be\ } (pop\  \& \ pop\_array)\}}&\\   
							&\mathtt{in \ (x_3(x_2\ 2 \ x_1) \ \&\  s_3*(s_2*\bar{2}*s_1)):\Box int}&
							\end{flalign*}}
						The last expression reduces to ($\beta$-reduction for {\sf let}):{\small\begin{flalign*}
							&{\sf \Sigma}; \bullet\vdash\mathtt{\ pop(push\ 2 \ empty)\ \&\  pop\_array*(add\_array*\bar{2}*empty):\Box int}
							\end{flalign*}}
						giving exactly the next step of the computation for the source expression.
						The good news is that the linker computes correctly the next step given any conforming set of implementations. 
						It is easy to see that given a {\sf list} implementation the very same process would produce a different computation step:{\small\begin{flalign*}
							&{\sf \Sigma}; \bullet\vdash\mathtt{\ pop(push\ 2 \ empty)\ \&\  pop\_list*(Cons*\bar{2}*[]):\Box int}
							\end{flalign*}}
					\end{enumerate}
					We conclude with some remarks that:
					\begin{itemize}
						\item The construction gives a mechanism of abstractions that works not only over different implementations in the
						same language but even for implementations in different (applicative) languages.
						\item We assumed in the example that the  two languages are based on the lambda calculus and implement a curried, higher-order function space. 
						It is easy to see that such host satisfies the requirements for the $\llbracket\bullet\rrbracket$ 
						(with $C_S, C_K$ being the $S, K$ combinators in $\lambda$ form  and $*$ translating to $\lambda$ application).
						\item
						Often, the host language of a foreign call is  not  a language that satisfies  such specifications. This situation occurs  when we have bindings from a functional language to a lower level language \footnote{In this setting the type signature of {\sf push} would be: $\mathtt{\sf int \times intstack\rightarrow instack}$}. 
						Such cases  can be captured by adding conjunction (and pairs), tuning the  specifications of $J$  accordingly and loosening the assumption that $\llbracket \bullet \rrbracket$
						is total on types.
						\item Introduction of  modal types is clearly relative to the $\llbracket\bullet \rrbracket$ function on types. 
						It would be interesting to consider examples where   $\llbracket\bullet \rrbracket$ is realized by non-trivial mappings such as $\llbracket A\supset B \rrbracket= !A \multimap B$
						from the embedding on intuitionistic logic to intuitionistic linear logic \cite{girard1987linear}.
						That will  showcase an example of   modality that works when lifting to a completely different logic or, correspondingly, to an essentially
						different computational model.
						\item Finally, it should be clear from the operational semantics and this example that we did not demand any equalities (or, reduction rules)  
						for the proofs in $J$, but mere existence of specific terms. This is in accordance to justification logic.  Analogously, we did not observe computation 
						in the host language but only the construction of the linkers as program transformers. We were careful, to say that our calculus corresponds to the dynamic 
						linking part of 
						separate compilation. This, of course, does not tell the whole story of program execution in such cases. Foreign function calls, return the control to the 
						client after the result gets calculated 
						in the external language. For example, the execution of the  program ${\texttt{pop (push 2 empty) + 2}}$ should ``escape'' the client 
						to compute the stack calls and then return
						for the last addition. Our modality is concerned  only with passing the control from the client to the host dynamically and, as such, is a $K$ 
						(non-factive) modality. Capturing the continuation of the computation and the return of the control back 
						to the source would  require a factive modality and a notion of ``reverse'' of the mapping $\llbracket\bullet\rrbracket$. 
						We would like to explore  such an extension in  future work.
						%Here the computation  mixes calling the  external implementation  $(\mathtt{st== empty})$ that -- in a full stack implementation -- would  provide for an equality check, computes its truth value externally and returns the control back to the client language  following the semantics of the $\mathtt{if}$ statement. The full logic for this can  be captured with a stronger modality (i.e. ``factive") that is work in progress. Nevertheless, our work is orthogonal and would correspond to dynamic linking aspect for such a system.
					\end{itemize}
					
					
					
					%\textcolor{green}{\texttt{push  empty\&}}}} $
					
					\section{Related and Future Work}
					\label{relat}
					Directly related work with our calculus, in the same fashion that justification logic and LP \cite{Artemov2001} are related to modal logic, is \cite{Bellin2001}.  The work in \cite{Bellin2001} provides a calculus for explicit assignments (substitutions) which is actually a sub-case of 
					{\sf Jcalc$^{-}$} with $\llbracket\bullet \rrbracket$ identity. This  sub-case  captures dynamic linking where the host language is the very same one; such need appears in languages with module mechanisms (i.e. implementation hiding and separate compilation within the very same language). In general, the judgmental approach to modality is heavily influenced by \cite{citeulike:5447115}. In a sense, our treatment of validity-as-explicit-provability also generalizes the approach there without having to commit to a ``factive" modality. Finally,  
					important results on programming paradigms related to justification logic have been obtained in \cite{ArtBon07LFCS,BONELLI2012935, bavera2010justification}. 
					Immediate future developments would be to interpret modal formulas of higher degree under the same principles. 
					This corresponds to dynamic linking in two or more steps (i.e., when the host becomes itself a client of another interface that is implemented dynamically in a third level and, so on). 
					Some preliminary results towards this 
					direction have been developed in \cite{DBLP:journals/entcs/PouliasisP14}. 
					%\nocite{Pfenning2009a, Pfenning2009b}
					

\begin{comment}					
\appendix
					
\label{appen}
\chapter{Appendix}
\subsection{Theorems}
					
					\begin{theorem}[Deduction Theorem for Validity Judgments]
						\label{deduct}
						Given any  $\Gamma,A,B \in {\sf Prop_0}$ then $\Gamma,x:A\vdash B \Longrightarrow \llbracket\Gamma\rrbracket\vdash\llbracket   A\supset B\rrbracket$. 
					\end{theorem}
					\begin{proof}
						The proof proceeds by induction on the derivations $\Gamma,A,B \in {\sf Prop_0}$. Note that the axiomatization of ${\llbracket\sf Prop_0\rrbracket}$ derives the 
						sequents:$\Delta\vdash\llbracket A \supset A\rrbracket$
						for any $\Delta\in {\sf \llbracket Prop_0 \rrbracket}$ (as in combinatory logic the $I$ combinator is derived from $SK$). This handles the reflection case. The rest of the cases are treated exactly as in the proof 
						of completeness of combinatorial axiomatization with respect to the natural deduction in intuitionistic logic. 
						Note that this theorem cannot be proven without the logical specification {\sf $Ax_1$, $Ax_2$}. I.e. it is exactly the requirements of the logical specification that ensure that all  interpretations  
						should be complete with respect to intuitionistic implication.
					\end{proof}
					\begin{lemma}[$\llbracket\bullet\rrbracket$Lifting Lemma]
						\label{bracklift}
						Given  $\Gamma,  A \in {\sf Prop_0}$ then $\Gamma\vdash   A \Longrightarrow \llbracket \Gamma\rrbracket \vdash \llbracket   A \rrbracket$. 
					\end{lemma}
					
					\begin{proof}
						The proof goes by induction on the derivations trivially for all the cases($\supset_{E_0}$ is treated using the ${\sf App}$ rule that internalizes Modus ponens). For the $\supset_{I_0}$ the previous theorem has to be used.
					\end{proof}

					\begin{proof}
						Assuming a derivation $\mathcal {D}$ ::$\Gamma\vdash   A$ from \ref{bracklift} there exists corresponding validity derivation $\mathcal{E}::\llbracket\Gamma\rrbracket\vdash\llbracket   A \rrbracket$. Using the two as premises in the $\Box_{IE}$ with $\Gamma := \Box \Gamma$ we obtain $\Box\Gamma\vdash\Box   A$.
					\end{proof}
					From the previous we get:

					%Let us show an inverse principle to the $\Box$ Lifting Lemma. We define for $A$ in {\sf Prop}:
					%\begin{flalign*}
					%\nonumber \downharpoonright P_i\ & =  P_i \\ \downharpoonright (A_1\supset A_2)&  =  \downharpoonright A_1 \supset \downharpoonright A_2 \\
					%\downharpoonright \Box A & = \downharpoonright A
					%\end{flalign*}
					%And the lifting of the $\downharpoonright$ over $\Gamma\in {\sf Prop}$. We get:
					\begin{theorem}[Collapse $\Box$ Lemma] If $\Box\Gamma\vdash \Box A$ for $\Gamma,A \in {\sf Prop_0}$ then $ \Gamma\vdash  A$.
					\end{theorem}
					\begin{theorem}[Weakening]
						For the N.D. system of {\sf Jcalc$^{-}$}, with $\Gamma, \Gamma^{\prime}\in {\sf Prop_0}$.
						\begin{enumerate}
							\item If  $\Gamma\vdash A$ then $\Gamma,\Gamma^{\prime}\vdash   A$.
							\item If  $\Box\Gamma\vdash \Box   A$ then $\Box\Gamma,\Box\Gamma^{\prime} \vdash \Box A$.
						\end{enumerate}
					\end{theorem} 
					\begin{proof}
						By induction on derivations for the first item. For the second item, given $\Box\Gamma\vdash \Box   A$ by the collapse lemma we get   $\Gamma\vdash   A$ which by the previous item
						gives $\Gamma,\Gamma^{\prime}\vdash   A$.  Using the lifting lemma we get $\llbracket \Gamma,\Gamma^{\prime}\rrbracket \vdash \llbracket  A \rrbracket$.
						Using the last two items we and the $\Box$ rule gives the result.
					\end{proof}
					
					\begin{theorem}[Contraction]
						\begin{enumerate}
							For the N.D. system of Jcalc, with $\Gamma,x:A,B\in {\sf Prop_0}$ 
							\item If  $\Gamma,x:A,x':A,\Gamma^{\prime}\vdash  B$ then $\Gamma,x:A,\Gamma^{\prime}\vdash   B$.
							\item If $\Box\Gamma,x:\Box A,x':\Box A,\Box\Gamma^{\prime}\vdash \Box B$ then $\Box\Gamma,x:\Box A,\Box \Gamma^{\prime}\vdash  \Box B$
						\end{enumerate}  
					\end{theorem}
					\begin{proof}
						Similarly with previous theorem.
					\end{proof}
					\begin{theorem}[Permutation]
						For the N.D. system of Jcalc, with $\Gamma\in {\sf Prop_0}$ and $\pi \Gamma$ the collection of permutations of $\Gamma$.
						\begin{enumerate}
							\item If  $\Gamma\vdash   A$ and $\Gamma^{\prime}\in \pi{\Gamma}$ then $\Gamma'\vdash   A$.
							\item If  $\Box\Gamma\vdash \Box   A$ then  $ \pi\Box\Gamma\vdash \Box   A$.
						\end{enumerate}
					\end{theorem}
					\begin{proof}
						As in the previous item.
					\end{proof}
					\begin{theorem}[Substitution Principle]
						The following hold for both kinds of judgments:
						\begin{enumerate}
							\item If  $\Gamma,x:A\vdash M: B$ and $\Gamma\vdash N: A$ then $\Gamma\vdash M[N/x]: B$ 
							\item If  $\llbracket\Gamma\rrbracket,s:\llbracket A \rrbracket \vdash {\sf J}: \llbracket B\rrbracket$ and 
							$\llbracket\Gamma\rrbracket\vdash {\sf J^{'}}: \llbracket B\rrbracket$ then  $\llbracket\Gamma\rrbracket\vdash  {\sf J[J^{'}/s]}\llbracket B\rrbracket$
						\end{enumerate}
					\end{theorem}
					All previous  theorems can actually be stated for proof terms too. We should discuss the following:
					\begin{theorem}[Deduction Theorem / Emulation of $\lambda$ abstraction]
						\label{deductterms}
						If $\Gamma, A\in {\sf Prop_0}$ and $\Gamma,x:A\vdash M:B$ then there exists ${\sf J}$ s.t.    $\llbracket\Gamma\rrbracket \vdash {\sf J}:\llbracket   A\supset B\rrbracket$.
					\end{theorem}
					\begin{lemma}[$\llbracket\bullet\rrbracket$Lifting Lemma for terms]
						\label{highorder}
						If $\Gamma, A \in {\sf Prop_0}$ and $\Gamma\vdash M: A$ then there exists ${\sf J}$ s.t. $\llbracket \Gamma\rrbracket \vdash {\sf J}:\llbracket   A \rrbracket$. 
					\end{lemma}
					In both theorems the existence of this ${\sf J,J^\prime}$ is algorithmic following the induction proof. 
					\subsection{Linking on the function space}
					The above mentioned algorithms permit  for translating $\lambda$ abstractions to polynomials of $S,K$ combinators which is a standard result in the literature. We do not give the details here but the translation is  syntax driven as it can be seen by the inductive nature of the proofs.
					
					Henceforth, we can generalize the construction in \ref{dlinker} so that it permits for dynamic linking of functions of the client 
					(with missing implementations) such as  $\mathtt{\lambda n:int. push\  n\  empty}$ dynamically given that the host actually implements 
					a higher-order function space (that is it implements the combinators $S,K$ in, say, own lambda calculus $\lambda^{J}$).
					Given implementations of $\mathtt{push\_impl}$, $\mathtt{empty\_impl}$ the linker produces an application expression 
					consisting of $\mathtt{push\_impl}$, $\mathtt{empty\_impl}$, $S$ and $K$.  
					The execution of the target expression will happen in the host after dereferencing  ${\sf push\_impl, empty\_impl}$ (dynamic part) 
					and the combinators $S,K$ (constant part) as, say, lambdas (e.g. $K=\lambda^{J} x.\lambda^{J} y. x$).
					
					\subsection{Gentzen's reduction Principle for $\Box$(General)}
					\label{redmult}
					\mbox{\footnotesize
						\begin{mathpar}
							\inferrule*[right=$  I_{\Box B} E_{\Box   A}^{x,s}$]{
								\inferrule*{}
								{\inferrule*[]{}{ \inferrule*[]	{\inferrule*[]{}{\PrTri{$D_1$} \\ \PrTri{$E_1$}}\\\\
											\inferrule*[]{}{ \ A_1\\ \ \qquad\ \ \  \llbracket   A_1\rrbracket}}{\Box   A_1 }}}\ldots 
								{\inferrule*[]{}{ \inferrule*[]	{\inferrule*[]{}{\PrTri{$D_i$} \\ \PrTri{$E_1$}}\\\\
											\inferrule*[]{}{\    A_i\\ \ \qquad \llbracket   A_i\rrbracket}}{\Box   A_i }}}
								\quad
								\inferrule*{}
								{\inferrule*[vdots=1.0em, right=$\vec{x}$]{ }{ A_1\dots A_i}\\\\
									\inferrule*[]{}{B}} 	   \qquad 
								\inferrule*{}
								{\inferrule*[vdots=1.0em, right=$\vec{s}$]{ }{\llbracket   A_1 \ldots A_i \rrbracket}\\\\
									\inferrule*[]{}{\llbracket B\rrbracket}} 	
							}
							{\Box B} 
						\end{mathpar}
					}
					$$\Longrightarrow_{R}$$
					\mbox{\footnotesize
						\begin{mathpar}
							\inferrule*[right=$I_{\Box B}$]{
								\inferrule*{}
								{\inferrule*[vdots=1.0em]{}{\PrTri{$D_1$}\ \PrTri{$D_i$} \\\\ A_1\ldots\ldots  \ \  A_i}\\\\
									\inferrule*[]{}{B}}
								\qquad 
								\inferrule*{}
								{\inferrule*[vdots=1.0em]
									{}{\PrTri{$E_1$} \PrTri{$E_i$}\\\\\llbracket   A_1\ldots\ldots A_i\rrbracket}\\\\
									\inferrule*[]{}{\llbracket B \rrbracket}} 
							}{\Box B}
							
						\end{mathpar}
					}
					\subsection{Notes on the cut elimination proof and normalization of natural deduction}
					\label{norm}
					Standardly, we add the bottom type and elimination rule in the natural deduction and show that in Jcalc + $\bot$: $\centernot\vdash\bot$. The addition goes as follows:
					
					\begin{mathpar}
						\inferrule*[right= Bot] { } {\bot \in {\sf Prop_0}}	
						\and
						\inferrule*[right= $E_\bot$] {{\Gamma\vdash\bot }\\ A\in {\sf Prop}} {\Gamma \vdash A}
					\end{mathpar}
					Our proof strategy follows directly \cite{pfenning2004automated}. We construct an intercalation calculus \cite{sieg1998normal} corresponding to the ${\sf Prop}$ fragment  with the following two judgments:
					\begin{itemize}
						\item[] $A\Uparrow$ for ``Proposition $A$ has normal deduction".
						\item[] $A^\downarrow$ for ``Proposition $A$ is extracted from hypothesis".
					\end{itemize}
					This calculus is, essentially, restricting the natural deduction to canonical derivations. The $\llbracket {\sf judgments} \rrbracket$ are not annotated and are directly ported from the natural deduction since we observe consistency in ${\sf Prop}$. 
					The construction is identical to \cite{pfenning2004automated} (Chapter 3) for the ${\sf Hypotheses},{\sf Coercion},\supset, \bot$ cases, we add the $\Box$ case.
					\begin{mathpar}
						\inferrule*[right=$\Gamma$-hyp]  {x: A\downarrow \in \Gamma^\downarrow}{ \Gamma^\downarrow\vdash^{-} A\downarrow}
						\and
						\inferrule*[right=$\downarrow\Uparrow$] {\Gamma^\downarrow\vdash^{-} A\downarrow}{\Gamma^\downarrow\vdash^{-} A \Uparrow}
						\and
						\inferrule*[right=$\supset$I$^{x}$] {\Gamma^\downarrow, x: A\downarrow\vdash^{-}  B\Uparrow} {\Gamma^\downarrow \vdash^{-}  A\supset  B\Uparrow}
						\inferrule*[right=$\supset$E] {{\Gamma^\downarrow\vdash^{-} A\supset  B \downarrow}\\{\Gamma^\downarrow\vdash^{-}  A\Uparrow}} {\Gamma^\downarrow\vdash^{-}   B\downarrow}
						%\and
						%\inferrule*[right=$\bot$E] {{\Turn {\Gamma} {\bot}}}{\Turn {\Gamma} {   A}}
						\and
						\inferrule*[right= $E_\bot$] {{\Gamma^{\downarrow}\vdash^{-}\bot\downarrow }\\ A\in {\sf Prop}} {\Gamma^{\downarrow}\vdash^{-} A\Uparrow}
						\and
						\inferrule*[right=$\Box_{IE}$ ] {{\Gamma^\downarrow\vdash\Box \Gamma^{\prime}\downarrow}\\{\Gamma'^\downarrow\vdash A\Uparrow }\\ {\llbracket \Gamma^{\prime} \rrbracket\vdash \llbracket A \rrbracket}}{ {\Gamma^\downarrow\vdash \Box A\Uparrow }}
					\end{mathpar}
					Where $\Gamma^\downarrow\vdash\Box \Gamma^{\prime}$ abbreviates $\forall A_i\in \Gamma'. \ \Gamma^{\downarrow}\vdash\Box A_i\downarrow$.
					We prove simultaneously by induction:
					\begin{theorem}[Soundness of Normal Deductions]
						The following hold:
						\begin{enumerate}
							\item If $\Gamma^\downarrow\vdash^{-} A\Uparrow$ then $\Gamma\vdash A$, and
							\item If $\Gamma^\downarrow\vdash^{-} A\downarrow $ then $\Gamma\vdash A$.
						\end{enumerate}
					\end{theorem}
					\begin{proof}
						Simultaneously by induction on derivations.
					\end{proof}
					It is easy to see that this restricted proof system $\centernot\vdash^{-} \bot\Uparrow$. It is hard to show its completeness to the non-restricted natural deduction ($\vdash + \bot_E$ of Jcalc) directly. For that reason we add a rule to make it complete ($\vdash^{+}$) preserving soundness and get a system of Annotated Deductions. We show the correspondence of the restricted system ($\vdash^{-}$) to a cut-free sequent calculus (${\sf JSeq}$), the correspondence of the extended system ($\vdash^{+}$) to ${\sf Jseq + Cut}$ and show cut elimination.\footnote{ In reality, the sequent calculus formulation is built exactly upon intuitions on the intercalation calculus. We refer the reader to the references.}
					
					To obtain completeness we add the rule:
					\begin{mathpar}
						\inferrule*[right=$\Uparrow\downarrow$] {\Gamma^\downarrow\vdash A\Uparrow} {\Gamma^\downarrow\vdash A\downarrow }
					\end{mathpar}
					We define $\vdash^{+} :=\   \ \ \vdash^{-} {\sf with} {\ \sf \Uparrow\downarrow}{\sf Rule}$.
					We show:
					\begin{theorem}[Soundness of Annotated Deductions]
						The following hold:
						\begin{enumerate}
							\item If $\Gamma^\downarrow\vdash^{+} A\Uparrow$ then $\Gamma\vdash A$, and
							\item If $\Gamma^\downarrow\vdash^{+} A\downarrow $ then $\Gamma\vdash A$.
						\end{enumerate}
					\end{theorem}
					\begin{proof}
						As previous item.
					\end{proof}
					
					\begin{theorem}[Completeness of Annotated Deductions]
						\label{compannot}
						The following hold:
						\begin{enumerate}
							\item If $\Gamma\vdash A$ then $\Gamma\downarrow\vdash^{+} A\Uparrow$, and
							\item If $\Gamma\vdash A$ then $\Gamma\downarrow\vdash^{+} A\downarrow$.
						\end{enumerate}
					\end{theorem}
					\begin{proof}
						By induction over the structure of the $\Gamma\vdash A$ derivation.
					\end{proof}
					
					Next we move with devising a sequent calculus formulation corresponding to normal proofs $\Gamma^{\downarrow}\vdash^{-}A\Uparrow$. The calculus that is given in the main body of this theorem. We repeat it here for completeness.
					\begin{mdframed}[nobreak=true,frametitle={\footnotesize Sequent Calculus ($\llbracket {\sf Prop_0} \rrbracket$)}]
						$$\begin{array}{l r}
						\Delta \Rightarrow \llbracket A\rrbracket:= & \exists \Delta'\in \pi(\Delta)\ \text{s.t} \   
						\Delta'\vdash \llbracket A \rrbracket \end{array}$$
						where $\pi(\Delta)$ is the collection of wellformed  $\llbracket {\sf Prop_0} \rrbracket$ contexts $\Delta'\vdash \llbracket {\sf wf}\rrbracket$  with some permutation of the multiset $\Delta$ as co--domain.
					\end{mdframed} 
					
					
					\begin{mdframed}[nobreak=true,frametitle={\footnotesize Sequent Calculus ({\sf Prop})}]
						\mbox{\small
							\begin{mathpar}
								\inferrule*[right=$Id$] { }{\Gamma, A  \Rightarrow A }
								
								\inferrule*[right=$\supset_L$] {{\Gamma, A\supset B, B \Rightarrow  C }\\ {\Gamma, A\supset B \Rightarrow A}} {\Gamma, A\supset B \Rightarrow  C}
								\and
								\inferrule*[right=$\supset_R$] {\Gamma, A \Rightarrow  B} {\Gamma \Rightarrow A\supset B}
								\and
								\inferrule*[right=$\bot_L$] { } {\Gamma, \bot \Rightarrow A}
								\and
								\inferrule*[right=$\Box_{LR}$] {{\Box\Gamma,\Gamma\Rightarrow A}\\{\llbracket\Gamma\rrbracket\Rightarrow \llbracket A \rrbracket }}{\Box\Gamma\Rightarrow \Box A}
								%\inferrule*[right=$\supset$E] {{\Turn {\Gamma} { A\supset  B}}\\{\Turn {\Gamma} { A}}} {\Turn {\Gamma} {   B}}
								%\and
								%\inferrule*[right=$\bot$E] {{\Turn {\Gamma} {\bot}}}{\Turn {\Gamma} {   A}}
							\end{mathpar}}
							%Where the  rule $\Box_{LR}$corresponds to $\Box_{IE}$ and relates the two kinds of sequents 
						\end{mdframed}
						We want to show correspondence of the sequent calculus  w.r.t normal proofs ($\vdash^{-}$).  Two lemmas are required to show soundness. 
						\begin{lemma}[Substitution principle for extractions]
							The following hold:
							\begin{enumerate}
								\item If $\Gamma_1^\downarrow, x:A^\downarrow,\Gamma_2^\downarrow\vdash^{-} B\Uparrow$ and\\$\Gamma_1^\downarrow\vdash^{-} A\Uparrow$ then  $\Gamma_1^\downarrow,\Gamma_2^\downarrow\vdash^{-} B\Uparrow$
								\item  If $\Gamma_1^\downarrow, x:A^\downarrow,\Gamma_2^\downarrow\vdash^{-} B\downarrow$ and $\Gamma_1^\downarrow\vdash^{-} A\downarrow$ then $\Gamma_1^\downarrow,\Gamma_2^\downarrow\vdash^{-} B\Uparrow$    
							\end{enumerate}
						\end{lemma}
						\begin{proof}
							Simultaneously by induction on the derivations $A\downarrow$ and $A\Uparrow$.
						\end{proof}
						And making use of the previous we can show, with ($\downharpoonright A$ defined previously):
						\begin{lemma}[Collapse principle for normal deductions]
							The following hold:
							\begin{enumerate}
								\item If $\Gamma^\downarrow,\vdash^{-}  A\Uparrow$ then $\downharpoonright\Gamma^\downarrow\vdash^{-} \downharpoonright A\Uparrow$   and,
								\item If $\Gamma^\downarrow\vdash^{-} A\downarrow$ then  $\downharpoonright\Gamma^\downarrow\vdash^{-} \downharpoonright A\downarrow$   
							\end{enumerate}
						\end{lemma}
						Using the previous lemmas and by induction we can show :
						\begin{theorem}[Soundness of the Sequent Calculus] 
							\label{soundnseq}
							If   $\Gamma\Rightarrow B$ then $\Gamma^\downarrow\vdash^{-} B\Uparrow$.
							
							
						\end{theorem}
						\begin{theorem}[Soundness of the Sequent Calculus with Cut] 
							
							If   $\Gamma\Rightarrow^{+} B$ then $\Gamma^\downarrow\vdash^{+} B\Uparrow$.
						\end{theorem}
						
						Next we define the $\Gamma\Rightarrow^{+} A$ as $\Gamma\Rightarrow A$ plus the rule:
						\begin{mathpar}
							\inferrule*[right=Cut]{{\Gamma\Rightarrow^{+} A}\\{\Gamma,A\Rightarrow^{+}B}}{\Gamma\Rightarrow^{+}B}
						\end{mathpar}
						\begin{proof}
							As before. The cut rule case is handled by the $\Uparrow\downarrow$ and substitution for extractions principle showcasing that the correspondence of the cut rule to the coercion from normal to extraction derivations.
						\end{proof}
						Standard structural properties (\textit{Weakening, Contraction}) to show completeness. We do not show these here but they hold.
						\begin{theorem}[Completeness of the Sequent Calculus] 
							\label{compseqcalc}
							The following hold:
							\begin{enumerate}
								\item If   $\Gamma^\downarrow\vdash^{-} B\Uparrow$ then $\Gamma\Rightarrow B$ and,
								\item 	If $\Gamma^\downarrow \vdash^{-} A\downarrow$ and $\Gamma,A\Rightarrow B$ then $\Gamma\Rightarrow B$   
							\end{enumerate}
							\begin{proof}
								Simultaneously by induction on the given derivations making use of the structural properties.
							\end{proof}
							Similarly we show for the extended systems.
							\begin{theorem}[Completeness of the Sequent Calculus with Cut] The following hold:
								\label{compseqcut}
								\begin{enumerate}
									\item If   $\Gamma^\downarrow\vdash^{+} B\Uparrow$ then  $\Gamma\Rightarrow^{+} B$ and,
									\item 	If $\Gamma^\downarrow \vdash^{+} A\downarrow$ and $\Gamma,A\Rightarrow^{+} B$ then $\Gamma\Rightarrow^{+} B$.   
								\end{enumerate}
							\end{theorem}
							\begin{proof}
								As before. The extra case is handled by the Cut rule.
							\end{proof}
						\end{theorem}
						After establishing the correspondence of $\vdash^{-}$ with $\Rightarrow$ and of $\vdash^{+}$ with $\Rightarrow^{+}$ we move on with:
						\begin{theorem}[Admissibility of Cut]
							If $\Gamma\Rightarrow A$ and $\Gamma,A\Rightarrow B$ then $\Gamma\Rightarrow B$.
						\end{theorem}
						The proof is by triple induction on the structure of the formula, and the given derivations and we leave it for a technical report. This gives easily:
						\begin{theorem}[Cut Elimination]
							If $\Gamma\Rightarrow^{+}A$ then $\Gamma\Rightarrow A$.
							
						\end{theorem}
						Which in turn gives us:
						\begin{theorem}[Normalization for Natural Deduction]
							\label{normalization}
							If $\Gamma\vdash A$ then $\Gamma^{\downarrow}\vdash^{-} A\Uparrow$
						\end{theorem}
						\begin{proof}
							From assumption $\Gamma \vdash A$ which by \ref{compannot} gives $\Gamma\vdash^{+} A\Uparrow$. By \ref{compseqcut} and Cut  Elimination we obtain $\Gamma\Rightarrow A$ which by  \ref{soundnseq} completes the proof.
						\end{proof}
						As a result we obtain:

						\begin{proof}
							By contradiction, assume $\vdash\bot$ then $\Rightarrow \bot$ which is not possible.
						\end{proof}
\end{comment}
\chapter{Order theoretic semantics}
\label{jcalcsem}
This chapter started by introducing 
mappings between deductive systems and 
motivating the reading of necessity as ``double-proof under a map''.
As a result, it is unsurprising that the calculus is amenable to order theoretic semantics.
We present them in this chapter.

\section{semi-Heyting algebras}
In order to progress we first define the notion of a 
\vocab{semi-Heyting  Algebra (semi-HA)}. 
To define semi-HA we need the notion of a \vocab{(meet) semi-lattice}.
  

\begin{mdframed}
\textbf{Definition:}
A \textit{(meet) semi-lattice} is a non-empty \emph{partial order} (i.e. reflexive, antisymmetric and transitive) 
with finite meets.
\end{mdframed}
In addition, we define \emph{meet semi-lattice} as follows: 
\begin{mdframed}
\textbf{Definition:}
A \textit{bounded (meet) semi-lattice} $(L,\le)$ is a (meet) 
semi-lattice that additionally has 
a \emph{greatest element} (we name it $1$), which satisfies

$x \le 1$ for every $x$ in $L$
\end{mdframed}
Finally, we can define \emph{semi-HA}:

\begin{mdframed}
\textbf{Definition:}
A \textit{semi-HA} is a bounded (meet) semi-lattice $(L,\le, 1)$ 
s.t. for every $a,b\in L$ there exists an \textit{exponential} 
(we name it $a\rightarrow b$) 
with the properties: 
\begin{enumerate}
\item $a\rightarrow b\times a\le b $
\item $a\rightarrow b$ is the greatest such element
\end{enumerate}
\end{mdframed}

\section{$Jcalc$-triplets}
Given two \emph{semi-HAs}, we are 
interested in order preserving 
functions (functors) $F$ that also preserve products and exponentials: 
\begin{mdframed}
    \textbf{Definition}
A function $F$ between two (semi)-HAs ($HA_1$, $HA_2$) is order preserving
and commutes with top, products and exponentials \emph{iff} for every 
$\phi,\psi \in HA_1$
    \begin{enumerate}
    \item $\phi\le_{HA_1}\psi\Rightarrow F\phi\le_{HA_2}F\psi$
    \item $F\top_{HA_1} = \top_{HA_2}$ 
    \item{$F(\phi \times\psi) = F(\phi)\times(F(\psi)$} 
    \item $F(\phi\rightarrow \psi) = F(\psi)\rightarrow F(\phi)$
    \end{enumerate}
\end{mdframed}

For the order theoretic models of  Jcalc $^{-}$ the following structures (triplets) 
are of interest. We define a $Jcalc$-triplet as follows:
\begin{mdframed}
    \textbf{Definition}
A \emph{$Jcalc$-triplet} is 
    
\begin{enumerate}
\item A semi-Heyting algebra $HA$
\item A partial order $J$
\item An order preserving function $F$ from $HA$ to $J$ s.t.
\begin{enumerate}
    \item The image $F(HA)$ forms a semi-Heyting Algebra
    \item $F$ preserves top, products and exponentials
\end{enumerate}
\end{enumerate}
\end{mdframed}
We are going to utilize the following definition: 
\begin{mdframed}
    \textbf{Definition}
    Given two partial orders $(K,\le_{K})$, $(L,\le_{L})$ and a function ($F: K\rightarrow L$) 
    we can define the algebra of $F$-points  $(F:K \rightarrow L,\le_{F:K\rightarrow L})$
    where:
    \begin{enumerate}
        \item Elements of $F:K\rightarrow L$ are  pairs of the form $\langle k,Fk \rangle$
        \item $\langle k_1,Fk_1 \rangle \le_F \langle k_2, Fk_2\rangle$ \textit{iff}  $k_1\le_{K}k_2$ and $Fk_1\le_{L}Fk_2$ 
    \end{enumerate}
\end{mdframed}

\begin{theorem}
    For any triplet $(K, L, F)$ of $HA$s with an order 
    preserving function $F:K\rightarrow L$
    the algebra of $F$-points is a partial order.
\end{theorem}
\begin{proof}
    It is trivial to show that the algebra of $F$-points ``inherits'' reflexivity, 
    transitivity and antisymmetry from the underlying algebras.
\end{proof}

Given a $Jcalc$-triplet there is an induced $F$-point algebra:
\begin{mdframed}
    \textbf{Definition}
    Given a $Jcalc$-triplet we define the algebra  $\Box^{F}HA$ as the induced $F$-point algebra.
\end{mdframed}
By the definitions, the $\Box^{F}HA$ point algebra has the following properties:
\begin{enumerate}
    \item Elements are pairs $\langle A, FA\rangle$ (name them $\Box^{F}A$) where $A\in HA$ and $FA$ its image
    \item For every two elements $\Box^{F}A$, $\Box^{F}B$:

    $\Box^{F}A \le\Box^{F}B \text{ \textit{iff} } A\le_{HA}B 
    \text{ \textit{and} } FA\le_J FB $ 
    
    \item It is a Heyting algebra with:
    \begin{itemize}
        \item $\Box^F\top := \langle \top_{HA}, F\top_{HA}=\top_J \rangle$ 
        \item Elements of the form $\Box^F (A \times B)$ 
        forming products (we name them $\Box^F A\times \Box^F B$)
        \item Elements of the form $\Box^F ( A\rightarrow B)$ 
        forming exponentials (name them $\Box^F{A}\rightarrow \Box^F B$)
    \end{itemize}
\end{enumerate}
The last property is not obvious so we will sketch the proof. We will be
omitting indexes in the $\le$ relations since they can be trivially inferred:
\begin{theorem}[$\Box^{F}HA$ is Heyting]
\end{theorem}
\begin{proof}
    $\Box^F \top$ is a top element since for any $A\in HA$, 
    $A\le \top$ and thusly $FA\le F\top=\top_J$ and thus by definition
    $\Box^F A\le\Box^F\top $ for any $\Box^F A$.

    For any two elements $\Box^F A$, $\Box^F B$, 
    the element $\Box^F (A \times B)$ forms their product since,
    $A \times B\le A$ in $HA$ and $F(A\times B)=FA\times FB \le FA$
    in $J$, and thusly, 
    $\Box^F (A \times B)\le \Box^FA$ (in $\Box^F HA$).
    Analogously, $\Box^F (A \times B)\le \Box^FA$.

    In addition, $\Box^F (A \times B)$ is the product we need to show 
    that is the greatest element with the previous property.
    I.e. for any $\Box^F C$ s.t. $\Box^F C\le \Box^F A $ and 
    $\Box^F C\le \Box^F B $ we get $\Box^F C\le\Box^F(A\times B)$. 
    By the definition for any such  $\Box^F C$ we have $C\le A\times B$
    and $FC\le F(A\times B)$ which imply that $\Box^F C\le\Box^F(A\times B)$

    To show that  $\Box^F(A\rightarrow B)$ is the  
    exponential of $\Box^FA$, $\Box^FB$, 
    we have to show, first that 
    $\Box^F ( A\rightarrow B)\times\Box^F A\le \Box^F B$.
    By the $\Box^F HA$ product definition 
    $\Box^F ( A\rightarrow B)\times\Box^F A :=\Box^F( (A\rightarrow B)\times A)$. 
    Also
    by the underlying exponentials we have $ (A\rightarrow B)\times A\le B$ 
    and  $F((A\rightarrow B)\times A)= (FA\rightarrow FB)\times FA \le FB$
    which by definition of $\Box^F HA$ gives   
    $\Box^F((A\rightarrow B)\times A)\le \Box^F B$ 
    and hence, by definition, 
    $\Box^F ( A\rightarrow B)\times \Box^F A\le \Box^F B$.

    In addition we have to show that 
    $\Box^F ( A\rightarrow B)$ is the 
    greatest element with the previous property.
    Consider any other $\Box^F C$ s.t. $\Box^F C\times\Box^F A\le \Box^F B$, 
    by definitions of $\Box^F$ and its products then, $C\times A\le B$
     and $FC\times FA\le FB$.
      By the definitions of the underlying exponentials 
     we get $C\le A\rightarrow B$ and 
     $FC \le FA\rightarrow FB =F(A\rightarrow B)$. 
     And again by definition of $\Box^F HA$, $\Box^F C\le \Box^F(A\rightarrow B)$.  
\end{proof}

\section{ $Jcalc$-algebras: Soundness and completeness}
Given a $Jcalc$-triplet we can define a $Jcalc$-algebra:
\begin{mdframed}
    \textbf{Definition}
    Given a $J$-triplet we define the corresponding $Jcalc$
     algebra as the union of the underlying relations
    of   $HA$, $F(HA)$, $\Box^{F}HA$
\end{mdframed}
    \begin{theorem}\label{thm:cmpjtriplet}
        \textbf{Soundness and completeness}\\
    $\Gamma\vdash_{Jcal}\phi$ iff for any \vocab{Jcalc Algebra}  $JC$ ($HA,F,J$)
    and any $*$ map that extends  a map of atomic $\prop$s ($p_i$) to elements of $HA$ 
    with properties shown below and $(+)$ 
    is defined inductively on the length of $\Gamma$ as shown below
    then $\Gamma^+\leq\phi^{*}$.
    \begin{alignat*}{2}
        (\top)* &&\quad= & \quad\top\\
        (A\wedge B \in Prop_0)*  &&\quad = & \quad  A*\times_{HA}B*\\
        (A\supset B \in Prop_0)*  &&\quad = & \quad A*\rightarrow_{HA} B*\\
        (\llbracket A\rrbracket)* && \quad = & \quad F(A*)\\					
        (\Box A)* &&\quad = & \quad\Box^F A* \\
        (\Box A\supset \Box B)*  &&\quad = & \quad\Box^F A* \rightarrow{\Box^F B*}\\
        (\Box A\wedge\Box B)*  &&\quad = & \quad\Box^F A*\times{\Box^F B*}\\
    \end{alignat*}
    
    \begin{alignat*}{2}
      \circ^+  &&\quad = & \quad\top\\
      \dagger^+ &&\quad = & \quad\llbracket\top\rrbracket\\
      (\Box\circ)^+ &&\quad = & \quad\Box^F \top\\
      (\Gamma,\phi\in \sf {Prop_0})^+&&\quad = &\quad
      \Gamma^+\times_{HA}\phi* \\
      (\llbracket \Gamma\rrbracket,\llbracket\phi\rrbracket \in {\sf \llbracket Prop_0\rrbracket})^+&&\quad = &\quad
      \Gamma^+\times_{J}F\phi*\\
      (\Box\Gamma, \Box\phi \in {\sf Prop_1})^+&&\quad = &\quad
      \Gamma^+\times_{\Box^{F}HA}\Box^F (\phi)*\\
    \end{alignat*}
    \end{theorem}
\begin{proof}
    To prove soundness we go by induction on the derivations. 
    For the ${\sf Prop_0}$ fragment
    the proof is well-known from intuitionistic logic semantics 
    ($\Gamma\in{\sf Prop_0}\vdash \phi\in {\sf Prop_0}
    \Rightarrow \Gamma^+\le_{HA}\phi *$)
    . 
    For the ${\sf \llbracket Prop_0\rrbracket}$ part of the calculus again by induction.
    Reflection, of contexts is trivial. For the axiomatic cases, it is a well known result
    that in any Heyting algebra (and thus in $F(HA)$ of any Jcalc algebra) 
    elements of the shape of the axiomatic combinators are equivalent 
    (equiprovable) to $\top$.
    For example in any Heyting algebra  we have $\top\le A\rightarrow(B\rightarrow A)$ 
    (using the definition of exponentials twice from the fact $\top\times A\times B\le A$), 
    the modus ponens case is handled by induction and the properties of $F$ 
    (preserving exponentials). Hence,
    $(\llbracket\Gamma\rrbracket)^{+}+\leq(\llbracket\phi\rrbracket)^{*}$ for any deduction in $\llbracket{\sf Prop_0} \rrbracket$.


    The interesting part of the proof is the $\Box$ rule which we present again here for readability:
    \begin{mdframed}[nobreak=true, frametitle={\footnotesize Judgments on Necessity with 
        $\Gamma\in {\sf Prop_1} \text{,{\ \sf length}}(\Gamma)=i\text{,\ }
        \ 1\le k\le i  \text{\ and, }\Gamma^{\prime},A, A_k,  B\in {\sf Prop_0}$ }]
    \mbox{\footnotesize
        \begin{mathpar}
            
            \inferrule*[right=$I_{\Box B}E^{\vec{x},\vec{s}}_{\Box A_1\ldots \Box A_i}$]{{(\forall  A_i \in \Gamma'. \ \Turn {\Gamma}{\Box  A_i})}\\{\Turn {\Gamma'} { B}}\\{\Turn {\llbracket \Gamma' \rrbracket} {\llbracket  B\rrbracket} }} {\Turn {\Gamma}\Box  B}

        \end{mathpar}}
    \end{mdframed}
    
    By the induction hypothesis we have $(\Gamma^{'})^{+}\le B^{*}$
    and $(\llbracket\Gamma^{'}\rrbracket)^{+}\le FB^{*}$ or equivalently by the properties of $F$
    $F((\Gamma^{'})^{+})\le F(B^{*})$ which gives $\Box^F{(\Gamma^{'})^{+}}\le \Box^F B$
    Additionally from induction hypothesis, for every $A_i$ in $\Gamma^{+}\Box^F A_i$
    and by the product definition $\Gamma^{+}\le(\Gamma^{'})^{+}$ and thus
    $\Gamma^{+}\le\Box^F B*$.

    For the inverse we create a Lindenbaum construction. We sketch the construction:
    \begin{itemize}
        \item Create a preorder \textit{pre-}$HA$ with underlying set (isomorphic to) ${\sf Prop_0}$
        \item Define $\phi\le\psi$ {\textit{iff}} $\phi\vdash\psi$
        \item Define the equivalence relation $\phi\equiv\psi$ {\textit{iff}} 
        $\phi\le\psi$ and $\psi\le\phi$
        \item Define the quotient \textit{pre-}$HA/_{\equiv}$
        \item Show that it is a Heyting Algebra with products the elements of
        of shape $\phi\wedge\psi$, top $\top$ and exponentials $\phi\supset\psi$
        \item Repeat the construction for the syntactic elements of 
        $\llbracket Prop_0 \rrbracket$, with 
        $\llbracket\phi\rrbracket\le\llbracket\psi\rrbracket$ 
        \textit{iff} $\llbracket\phi\rrbracket\vdash\llbracket\psi\rrbracket$
        show that it is a Heyting algebra $J$.
        \item Repeat the construction for the syntactic elements of
         ${\sf Prop_1}$ and $\Box \phi\le\Box\psi$ \textit{iff} $\Box\phi\vdash\Box\psi$
        \item Show that the union of the three relations above forms a 
        Jcalc-algebra with $F:= A\mapsto \llbracket A\rrbracket$. I.e. show that:
        \begin{itemize}
            \item $A\vdash B\Rightarrow \llbracket A\rrbracket\vdash \llbracket B\rrbracket$
            (Holds by the lifting lemma)
            \item $\llbracket A\wedge B\rrbracket $ is product (trivial) and 
            $\llbracket A\supset B \rrbracket$ (trivial given the deduction theorem which we have shown)
            in $J$
            \item $\Box A\vdash\Box B$ \textit{iff} $A\vdash B$ and $\llbracket A\rrbracket \vdash \llbracket B\rrbracket$
            Easy by induction on the derivations and usage of the lifting lemma.
        \end{itemize}
    \end{itemize}
    Now assume that $\Gamma^{+}\le\phi*$ for any Jcalc algebra and 
    mapping $*$; consider  $*$ to extend  the identity mapping  
    into the (free) J-calc algebra defined above. It is trivial to see that in JCalc
    $\Gamma\vdash\phi$.
\end{proof}


\chapter{The computational side of Jcalc }
\label{jcalccom}
In this section we add proof terms to represent natural deduction constructions. 
The  meaning of these terms emerges naturally from Gentzen's principles that give reduction 
(computational $\beta$-rules) and expansion (i.e. extensionality $\eta$-rules) equalities for 
 each construct. We focus on the new constructs of the calculus that emerge from the judgmental interpretation of the $\Box$ connective as explained in 
section \ref{lsjcalc}. In addition we focus on the \textit{implicational} part. The proof term assignment
for $\wedge$ rules is standard and can be added.

There will be no computational (reduction) rules on  provability terms. 
This conforms with our reading of these terms  as \textit{references} to proof constructs of an \textit{abstracted} theory $J$ that can be realized 
differently for a concrete $J$.  
%This is, we posit, the logical basis for \textit{dynamic linking} under separate compilation. The linker from a language $I$ to any host language $J$ (that satisfies certain specifications) creates residuals dynamically but is not concerned about the actual execution of such residuals. Execution of such residuals happens at a next phase when the references are dereferenced to code of the host and is not observed by the calculus.\footnote{ Such approach, also conforms with justification logic principles. Justifications are static, purely syntactical constructs that are not further analyzed.} 
\section{Proof term assignment}
\label{basicpras}
The following rules and their correspondence with natural deduction  constructs (\ref{jots}) should be obvious to the reader familiar with the simply typed  $\lambda$-calculus and basic justification logic.
We do not repeat here the corresponding $\beta, \eta$ equality rules since they are standard.
\begin{mdframed}[nobreak=true,frametitle={\footnotesize Judgments on Truth  $\Gamma, A,B \in {\sf Prop_0}$  and $M := x_i\  |\  <> \ |\ \lambda x:A.\ M\  |\  (M M) $}]
    \label{jot}
    \mbox{\small
        \begin{mathpar}
            \inferrule*[right=$\Gamma_0$-Refl] {x: A \in \Gamma}{\Turn {\Gamma} { x:A}}
            \and
            \inferrule*[right=$\top_0$I] { }{\Turn {\Gamma} { <>:\top}}
            \and
            \inferrule*[right=$\supset_0$I] {{\Turn {\Gamma, x: A} { M:B}}} {\Turn {\Gamma} { \lambda x:A.\  M:  A\supset  B}}
            \and
            \inferrule*[right=$\supset_0$E] {{\Turn {\Gamma} { M:A\supset  B}}\\{\Turn {\Gamma} {M^{\prime}: A}}} {\Turn {\Gamma} { (MM^{'}):  B}}
            \and
            \inferrule*[right=]{}{+\  \beta\eta \text{\ equalities for \ } \top,\supset}
            %\and
            %\inferrule*[right=$\bot$E] {{\Turn {\Gamma} {\bot}}}{\Turn {\Gamma} {   A}}
        \end{mathpar}}
    \end{mdframed}
    
    For  judgments of ${\sf \llbracket Prop_0\rrbracket}$, we assume a countable set of constant names and demand that every combinatorial
    axiom of intuitionistic logic has  a witness under the interpretation 
    $\llbracket\bullet\rrbracket$. This is what justification logicians call ``axiomatically appropriate constant specification''.
    As usual we demand reflection of contexts in $J$
    and preservation of modus ponens -- closedness under some notion of application (which we denote as $*$).
    
    %\begin{mathpar}
    %\inferrule*[right=$CS$] {{\Turn {\Delta} {\sf \llbracket wf \rrbracket}}\\ {C_i:\llbracket  A\rrbracket \in CS}} {\Delta\vdash_{CS}C_i:\llbracket  A\rrbracket}
    %\end{mathpar}
    
    \begin{mdframed}[nobreak=true,frametitle={\footnotesize Judgments on Validity  $\Delta\in {\sf \llbracket Prop_0\rrbracket}$  and ${\sf J} :=  s_i\ |\ C_i\  | \ {\sf J}*{\sf J} $}]
        \label{justf}
        \mbox{\small
            \begin{mathpar}
                \inferrule*[right=$\Delta$-Refl] {  {s:\llbracket  A\rrbracket \in \Delta}}{\Turn {\Delta} {s:\llbracket  A\rrbracket}}
                \and
                \inferrule*[right=  Ax$_1$]{ }   {\Delta\vdash C_{\top}: \llbracket  \top\rrbracket}
                \and
                \inferrule*[right=  Ax$_2$]{{  A, B \in {\sf Prop_0} }}   {\Delta\vdash C_{K^{A,B}}: \llbracket   A \supset (B \supset   A)\rrbracket }
                \and
                \inferrule*[right=  Ax$_3$]{ {  A,B, C \in {\sf Prop_0}}}{\Delta\vdash C_{S^{A,B,C}}:\llbracket   A\supset (B \supset C) \supset ((  A\supset B) \supset (  A \supset C))\rrbracket}
                \and
                \inferrule*[right=App] {{\Turn {\Delta} { {\sf J}: \llbracket  A \supset  B \rrbracket}}\\ {\Turn{\Delta} {{\sf J'}:\llbracket  A \rrbracket}}}{\Turn {\Delta} { {\sf J*J^\prime}:\llbracket  B\rrbracket}}
                
                
            \end{mathpar}
        }
    \end{mdframed}
    If  $J$ is a proof calculus and $\llbracket\bullet \rrbracket_J$ is  an interpretation such that the specifications above  
    are realized, then $J$ can witness intuitionistic provability. This can be shown by the proof relevant version of the lifting  lemma
    that states:
    \begin{lemma}[$\llbracket\bullet\rrbracket$Lifting Lemma]
        \label{bracklift}
        Given  $\Gamma,  A \in {\sf Prop_0}$ s.t. and a term $M$ s.t. $\Gamma\vdash  M: A $ then there exists ${\sf J}$ s.t  $\llbracket \Gamma\rrbracket \vdash {\sf J}:\llbracket   A \rrbracket$. 
    \end{lemma}
    
    
    
    
    \subsection{Proof term assignment and Gentzen Equalities for $\Box$ Judgments}
    Before we proceed, we will give a small primer of \textit{let}-bindings as used in modern programming languages to provide for some intuition on how such terms work. 
    Let us assume a rudimentary programming language that supports some basic types, say integers (${\sf int}$), as well as pairs of such types. Moreover, let us define a datatype 
    $\sf{Point}$ as a pair of ${\sf int}$ i.e. as $\sf {(int,int)}$ 
    In  a language with \textit{let}-bindings one can define a simple function that takes a ${\sf Point}$ and ``shifts'' it by adding $1$ to each of its $x$ and $y$ coordinates as follows:
    \begin{lstlisting}
    def shift (p:Point) = 
    let  (x,y) be p
    in
    (x+1,y+1)
    \end{lstlisting}
    If we call this function on the point ${\texttt{(2,3)}}$, then the computation ${\texttt{let (x,y) be (2,3) in (x+1,y+1)}}$ is invoked. This expression reduces following the \textit{let} reduction rule
    (i.e. pattern matching and substitution) to $\texttt{(2+1,3+1)}$; and as a result we obtain the value $\texttt{(3,4)}$.  As we will see, {\textit{let}} bindings -- with appropriate typing restrictions for our system -- 
    are used in the assignment of proof terms for the $\Box_{IE}$ rule. Moreover, the reduction principle for such terms ($\beta$-rule) -- obtained following Gentzen's equalities for the $\Box$ connective --  
    is exactly the one that we just informally described. 
    
    We can now move forward with the  proof term assignment 
    for the $\Box_{IE}$ rule.  We show first the sub-cases for 
    $\Gamma'$ empty (pure $\Box_I$)  and $\Gamma'$ 
    singleton and explain the computational significance 
    utilizing Gentzen's principles appropriated for the $\Box$ 
    connective. We are  directly translating proof tree equalities 
    from \ref{gprinc} to proof term equalities. 
    We generalize for arbitrary $\Gamma'$ in the following subsection. 
    We have, respectively, the following instances:
    
    \begin{mathpar}
        \inferrule*[]{ {\Gamma\in{\sf Prop_1}}\\{\Turn {\circ} { M:B}}\\{\Turn {\dagger} {{\sf J}:\llbracket  B\rrbracket} }} {\Turn {\Gamma} {  M\& {\sf J}:\Box  B}}
        \and
        \inferrule*[]{{ \Turn {\Gamma}{N:\Box  A}}\\{\Turn {x:A} { M:B}}\\{\Turn {s:\llbracket A \rrbracket} {{\sf J}:\llbracket  B\rrbracket} }} {\Turn {\Gamma} {{\sf let} \ (x\& s \ \ {\sf be\ } N) \ {\sf in}\  (M\& {\sf J}):\Box  B}}
    \end{mathpar}
    
    \subsection{Gentzen's Equalities for  ($\Box$ terms)}
    Gentzen's reduction and expansion principles give computational meaning (dynamics) and an extensionality principle for linking terms. We omit naming the empty contexts for economy.
    
    \mbox{\small
        \begin{mathpar}
            \inferrule*[Right=$I_{\Box B} E_{\Box A}^{x,s}$]{
                {\inferrule*[Left=$\Box_I$]{{\Gamma\in {\sf Prop_1}}\\ \vdash M:A\\ \vdash {\sf j}:\llbracket A \rrbracket}{\Gamma\vdash M \& {\sf j}:\Box A}}\\{x:A\vdash M':B }\\ {s:\llbracket A\rrbracket \vdash {\sf j'}:\llbracket B \rrbracket}}
            {\Gamma\vdash {\sf let} \ \ (x\& s)\ \ {\sf be}\ \   (M\&  {\sf J}) \ \ {\sf in}\ \  { (M^\prime \& {\sf J^\prime})}:\Box B }
        \end{mathpar}
    }$$\Longrightarrow_{R}$$
    \mbox{\small
        \begin{mathpar}
            \inferrule*[Right=$I_{\Box B}$]{\Gamma \in {\sf Prop_1}\\ \vdash M'[M/x]:B \\ \vdash {\sf J^{\prime}}[{\sf J}/s]:\llbracket B \rrbracket}{\Gamma\vdash { M^{\prime}[M/x]\& {\sf J^{\prime}}[{\sf j}/s]}:\Box B} 
        \end{mathpar}
    }
    Where the expressions $M^\prime[M/x]$ and ${\sf J^\prime[J/s]}$ denote capture avoiding substitution, reflecting proof compositionality of the two calculi.
    
    Following the expansion principle we obtain:
    {\small
        $$\begin{array}{c}
        \Gamma\vdash M:\Box   A
        \end{array} \ \Longrightarrow_{E}$$}
    \mbox{\small
        \begin{mathpar}
            \inferrule*[right=$I_{\Box A}E^{{x},{s}}_{\Box A}$]{{ \Turn {\Gamma}{M:\Box  A}}\\{\Turn {x:A} { x:A}}\\{\Turn {s:\llbracket A \rrbracket} {s:\llbracket  A\rrbracket} }} {\Turn {\Gamma} {{\sf let} \ (x\& s \ {\sf be \ } M) \ {\sf in}\  (x\& s):\Box  A}}
        \end{mathpar}}
        
        That gives an $\eta$-equality as follows:
        {\small
            $$M:\Box A =_{\eta}\ \ {\sf let} \ \ (x \& s\  {\sf be} \ M)\ \ {\sf in}\ \  { (x \& s)}:\Box A$$
        }
        The $\eta$ equality demands that every $M:\Box A$ should be reducible to a form $M'\& {\sf J^{\prime}}$.  
        \subsection{Proof term assignment for the $\Box$ rule (Generically)}
        After understanding the computational meaning of let expressions in the $\Box_{IE}$ rule 
        we can now give  proof term assignment  for the rule in the general case(i.e. for $\Gamma'$ of arbitrary length). 
        We define a helper syntactic construct --${\sf let}^{*}\ldots {\ \sf in\ }$-- as syntactic sugar for iterative  let bindings based on the structure  of contexts.
        The ${\sf let}^{*}$ macro takes four arguments: a context $\Gamma\in {\sf Prop_0}$, a  context $\Delta\in{\sf \llbracket Prop_1\rrbracket}$,  
        a possibly empty ($[\ ]$) list of terms  $Ns:=N_1,\ldots,  N_i$ - all three of the same length - and a term $M$. It is defined as follows for the empty and non-empty cases:
        %\footnote{Let us stretch here that when we speak about the structure of $\Gamma$ we imply (given the construction of $\Gamma\vdash{\sf wf}$) a treatment of contexts as lists where $\bullet$ stands for the empty list, singleton $x:A$ for $x:A + \bullet$ and contexts written in the form  $\Gamma,x:A$ for $x:A + \Gamma$ where $\Gamma$ is a list. I.e. contexts are lists but written down inversely with the head in the rightmost position. We treat the list of terms $Ns$ similarly for uniformity}  
        
        {\small
            $$\begin{array}{ll}
            \nonumber {\sf let}^{*}\ (\circ;\ \dagger;\  [\ ]) {\ \sf in\ }  M:= M \  &\\
            \nonumber {\sf let}^{*}\ (x_1:A_1,\ldots, x_i:A_i\ ;\  s_1: \phi_1, \ldots, s_i:\phi_i;\  N_1,\ldots,  N_i) {\ \sf in\ } M:= \  & \\
            {\sf let} \ \{(x_1 \& s_1)\  {\sf be}\  N_1,\ldots,  (x_i \& s_i)\  {\sf be}\  N_i\}\ {\sf in}\  M &\\
            \end{array}$$}
        Using this syntactic definition the rule $\Box_{IE}$ rule  can be written compactly:
        
        \begin{mdframed}[nobreak=true,frametitle= \footnotesize{$\Box_{IE}\ \text{ With}\ \Gamma\in {\sf  Prop_1}\text{,}\  \Gamma^{\prime}\in {\sf Prop_0}\text{,{\ \sf length}}(\Gamma)=i\text{,\ }Ns:=N_1 ...\  N_i\text{,}\ 1\le k\le i$} ]% $\Gamma^\prime\in {\sf Prop_0}$ ]
            \mbox{\small
                \begin{mathpar}
                    \inferrule*[right=$I_{\Box B}E^{\vec{x},\vec{s}}_{\Box A_1\ldots \Box A_i}$] %{}
                    {{\forall A_k \in \Gamma^\prime . \   \Gamma \vdash N_k:\Box A_k}\\
                        {\Turn {\Gamma^\prime} {M:B}}\\{\Turn {\llbracket\Gamma^\prime\rrbracket} {{\sf J}:\llbracket B \rrbracket} }} 
                    {\Turn {\Gamma} {{\sf let^{*}}\ (\Gamma^\prime, \llbracket\Gamma^\prime\rrbracket, Ns) \ {\sf in}  \ ( M\& {\sf J}):\Box B}}
                \end{mathpar}
            }
        \end{mdframed}
        It is obvious that all previously mentioned cases are captured with this formulation. The rule of $\beta$-equality can be given  for multi-let bindings directly from Gentzen's reduction principle (\ref{gprinc}) generalized for 
        the multiple intro case shown in the appendix (\ref{redmult}). 
        {\small
            $$\begin{array}{ll}
            {\sf let} \{(x_1 \& s_1) {\ \sf be\ } (M_1\& {\sf J_1}),\ldots,  (x_i \& s_i) {\ {\sf be}\ } (M_i\& {\sf J_i})\}\ {\sf in}\  (M \&  {\sf J})& =_{\beta} \\
            {  M[M_1/x_1, \ldots,  M_i/x_i] \& {\sf J}[{\sf J_1}/s_1,\ldots, {\sf J_i}/s_i]}
            \end{array}$$}
        \section{Strong Normalization and small-step semantics}
        In the appendix (\ref{norm}) we provide a proof of normalization for  natural deduction (via cut elimination). 
        This is ``essentially" a strong normalization result for the proof term system also. A  weaker result is 
        normalization under a  deterministic,``call-by-value" reduction strategy for $\beta$-rules.
        This   gives
        an idea of how the system computes  and we can use it in the applications in the next section. 
        As usual we characterize a subset of the closed terms as values and we provide rules for the reduction of the non-value closed terms.
        Note that for the constants of validity and their applicative closure we do not observe reduction properties but treat them as values -- again conforming with the idea of $J$ (and its reduction principles) being ``black boxed''.
        \begin{mdframed}[nobreak=true,frametitle={\footnotesize Small step, call-by-value reduction $\rightarrow$}]
            \begin{mathpar}
                \inferrule*[] { }{\lambda x. M {\ \sf  value}}
                \and
                \inferrule*[] { } {C_i {\ \sf \ value}}
                \and
                \inferrule*[] {{\sf J_1} {\ \sf value} \\ {{\sf J_2} {\ \sf value}} }  {{\sf J_1*J_2} {\ \sf value}}
                \and
                \inferrule*[] {M\  {\sf value} \\ {{\sf J}\  {\sf value}} }  {M \&{\sf J} {\sf \ value}}
                \and
                \inferrule*[] {M \rightarrow M^\prime}  {M \&{\sf J}\rightarrow M^\prime \& {\sf J}}
                \and
                \inferrule*[] {{N_1 \ {\sf value}\  \ldots\text{ \ }  N_{k-1} \ {\sf value}}\\{N_k\rightarrow N_k^{\prime}}}  {{\sf let} \{(x_1 \& s_1) {\ \sf {be}\ } N_1,\ldots,
                    \  (x_{k} \& s_{k}) {\ \sf {be } \ } N_k{\text{,}} \ldots\}    {\ \sf in}\  M \rightarrow \\
                    {\sf let} \{(x_1 \& s_1) {\ \sf {be}\ } N_1,\ldots,
                    \  (x_{k} \& s_{k}) {\ \sf {be } \ } N_k^{\prime}{\text{,}} \ldots\}    {\ \sf in}\  M }
                \and
                \inferrule*[] {M_1 \& {\sf J_1 \ value\ } \ldots\text{\ }  M_i \& {\sf J_i \ value}}  {{\sf let} \{(x_1 \& s_1) {\ \sf be\ } (M_1\& {\sf J_1}),\ldots,  (x_i \& s_i) {\ {\sf be}\ }
                    (M_i\& {\sf J_i})\}\ {\sf in}\  (M \&  {\sf J})\rightarrow \\
                    {  M[M_1/x_1, \ldots,  M_i/x_i] \& {\sf J}[{\sf J_1}/s_1,\ldots, {\sf J_i}/s_i]}}
                \and
                \inferrule*[] {M \rightarrow M^{\prime}}  {(MN) \rightarrow (M^\prime N)}
                \and
                \inferrule*[] {N \rightarrow N^{\prime} }  {((\lambda x. M)N) \rightarrow ((\lambda x. M)N^\prime) }
                \and
                \inferrule*[] {N {\ \sf value}}  {((\lambda x. M)N) \rightarrow [N/x]M }
                %\and
                %\inferrule*[right=$\bot$E] {{\Turn {\Gamma} {\bot}}}{\Turn {\Gamma} {   A}}
            \end{mathpar}
        \end{mdframed}
        Using the reducibility candidates proof method \cite{citeulike:993095}) we show:
        \begin{theorem}[Termination Under Small Step Reduction]
            With $\rightarrow^{*}$ being the reflexive transitive closure of $\rightarrow$: for every closed term $M$ and $A \in {\sf Prop}$ if $\vdash M:A$ then there 
            exists $N \ {\sf value}$ s.t. $\vdash N:A$ and  $M\rightarrow^{*} N$.
        \end{theorem}
        
        \section{A programming language view: Dynamic Linking and separate compilation}
        \label{dlinker}
        Our type system can be related to programming language design when considering \textit{Foreign Function Interfaces}. This is a typical scenario in which a language $I$ interfaces another language $J$ which is  essentially ``black boxed''.
        For example, {\sf OCaml} code  might call {\sf C} code to perform certain computations. 
        In such cases $I$ is a client and $J$ is a host that provides implementations for an interface utilized by the client.
        Through software development,  often the implementations of such an interface might change (i.e. a new version of the host language, or more dramatically, a complete switch of host language). 
        We want a language design that satisfies
        two  interconnected 
        properties. First, \textit{separate compilation} i.e. when implementations change we do not have to recompile client code and, yet, 
        secondly, \textit{dynamic linking} we want the client code to be linked dynamically to its new 
        ``meaning''.
        
        We will assume that both languages are  functional and based on the lambda calculus. I.e. our interpretation function should have the property $\llbracket A\supset B\rrbracket_J$=
        $\llbracket A\rrbracket_J \llbracket\supset\rrbracket_J \llbracket B\rrbracket_J$ where  $\llbracket\supset\rrbracket_J$ is the implication type constructor in $J$.
        The specifics of the host  $J$ and the concrete implementations are unknown to $I$ but during the linker construction we assume that both languages share  some  basic types
        for otherwise  typed ``communication'' of the two languages would be impossible. 
        Simplifying,  we consider that the only  shared type is  (${\sf int}$), i.e. the linker construction assumes  
        $\bar{n}:\llbracket \sf int \rrbracket$ for every integer $n:{\sf int}$. 
        Let us now assume source code in $I$ that
        is  interfacing   a simple data structure, say an  integer stack,  with the following signature ${\sf \Sigma}$:
        \begin{lstlisting} 
        using type intstack
        empty: intstack, push: int -> intstack -> intstack,
        pop: intstack -> int
        \end{lstlisting}
        
        
        And let us consider a simple program in $I$ that is using the signature say, 
        $${\texttt{pop(push (1+1) empty):int}}$$
        This program involves two kinds of computations: a redex $(1+1)$ that can be reduced using the internal semantics of the language $1+1\rightsquigarrow_{I} 2$ and  
        the signature calls ${{\texttt{pop (push 2 empty)}}}$ 
        that are to be performed
        externally  in whichever host language implements them. 
        We treat  dynamic linkers as ``term re-writers'' that map  a computation to its meaning(s) based on different implementations.
        In the following we consider ${\sf \Sigma}$ to be the signature of the interface. Here are the steps towards the linker construction.
        
        \begin{enumerate}
            \item Reduce the source code based on the operational semantics of $I$ until it doesn't have a redex:
            \small{${\sf \Sigma}; \bullet\mathtt{\vdash_{} pop (push \ (1+1) \ Empty)\rightsquigarrow pop (push \ 2 \ Empty) :int}$}
            \item Contextualize the use of the signature at the final term in step $1$:{\small
                \begin{flalign*}
                & \mathtt{{\sf \Sigma}; x_1:intstack,  x_2:int\rightarrow intstack\rightarrow intstack, x_3:intstack\rightarrow int \vdash x_3 (x_2 \ 2\  x_1):int} &
                \end{flalign*}}
            \item Rewrite the previous judgment assuming (abstract) implementations for the corresponding missing elements
            using the ``known'' specification for the shared elements.
            {\small
                \begin{flalign*}
                & \mathtt{ s_1:\llbracket instack \rrbracket,  s_2:\llbracket int \rightarrow intstack\rightarrow intstack\rrbracket, 
                    s_3:\llbracket intstack\rightarrow int \rrbracket\vdash s_3*(s_2* \bar{2}*s_1):\llbracket int\rrbracket}&
                \end{flalign*}}
            \item Combine the two previous judgments using the $\Box_{IE}$ rule.
            {\small
                \begin{flalign*}
                & {\sf \Sigma};\mathtt{x_1^{\prime}:\Box intstack ,x_2^{\prime}:\Box(int\rightarrow intstack\rightarrow intstack), x_3^{\prime}: \Box  (intstack\rightarrow int)\vdash} & \\
                & \mathtt{ let\{ x_1\& s_1 {\ \sf be \ } x_1^{\prime},\ x_2\& s_2 {\ \sf be \ } x_2^{\prime}, \  x_3\& s_3 {\ \sf be \ } x_3^{\prime} \}\  in}\   \mathtt{(x_3 (x_2\ 2 \ x_1)\ \& \ s_3*(s_2* \bar{2}*s_1)):\Box int} &
                \end{flalign*}}
            \item Using $\lambda$-abstraction three times we obtain the dynamic linker:
            {\small
                \begin{flalign*}
                & {\sf \Sigma};\circ\vdash \\
                & \mathtt{linker} = \mathtt{\lambda x_1^{\prime}.\  \lambda x_2^{\prime}}. \lambda x_3^{\prime}. \\
                & \mathtt{let \{ x_1\& s_1 {\ \sf be \ }x_1^{\prime},\ x_2\& s_2 {\ \sf be \ }  x_2^{\prime},\ x_3\& s_3 {\ {\sf be} \ } x_3^{\prime}\}\  in} \mathtt{(x_3(x_2\ 2 \ x_1)\ \&\  s_3*(s_2*\bar{2}*s_1))} &\\
                &\mathtt{:\Box(instack)\rightarrow \Box(int\rightarrow intstack\rightarrow intstack) \rightarrow \Box (intstack\rightarrow int) \rightarrow \Box int}
                \end{flalign*}}
        \end{enumerate}
        Let us see how it can be used in the presence of different implementations:
        \begin{enumerate}
            \item Suppose the developer  responsible for the implementation of the interface is providing an  array based implementation for the stack  in  some language $J$ 
            i.e. we get references to typechecked code fragments of $J$ as follows{\footnote{We have changed the return type of $\mathtt{pop}$ to avoid products. This is just for economy and products can easily be handled.}}:
            {\small
                \begin{flalign*}
                & \mathtt{create():intarray},\  \mathtt{add\_array:int_J \rightarrow_J intarray \rightarrow_J intarray } & \\
                & \mathtt{pop\_array:intarray \rightarrow_J int } &
                \end{flalign*}}
            \item  A unification algorithm check is performed to verify the conformance of the implementations to the signature taking into account 
            fixed type sharing equalities ($\llbracket {\sf int} \rrbracket = {\sf int_J}$). In our case it produces: $$\llbracket\rightarrow\rrbracket = \mathtt{\rightarrow_J}, \llbracket {\sf intstack} \rrbracket= {\sf intarray}$$
            \item
            We thus obtain typechecked links using the $\Box_I$ rule. For example:
            
            \mbox{\small
                \begin{mathpar}
                    \inferrule*[]{{\Turn {{\sf \Sigma};\circ} { \mathtt{push: int \rightarrow intstack \rightarrow intstack }}}\\{\Turn {\bullet} {\mathtt{add\_array:\llbracket int \rightarrow intstack \rightarrow intstack\rrbracket  }} }} {\Turn {{\sf \Sigma};\circ} {  \mathtt{ push\  \& {\ \sf add\_array}:\Box (int \rightarrow intstack \rightarrow intstack)}}}
                \end{mathpar}
            }
            And analogously:
            
            \mbox{\small
                \begin{mathpar}
                    \inferrule*[]{} {{\sf \Sigma}; \circ\vdash {  \mathtt{ pop\  \& \ {\sf pop\_array}:\Box (intstack\rightarrow int)}}}
                    \and
                    \inferrule*[]{} {{\sf \Sigma}; \circ\vdash {  \mathtt{ empty\  \& {\sf create()}:\Box intstack}}}	
                \end{mathpar}
            }
            \item Finally we can compute the next step in the computation for the expression  applying the linker to the obtained pairings:{\small\begin{flalign*}
                &{\sf \Sigma}; \bullet\mathtt{\vdash_{}(linker\   (empty\  \& \ create()) \ (push\  \& \ {\sf add\_array}) \ (pop\  \&\  pop\_array)):\Box int}& 
                \end{flalign*}} which reduces to:{\small\begin{flalign*}{\sf \Sigma}; \bullet\vdash&\mathtt{ let\{ (x_1\& s_1) {\ \sf be \ }(empty\  \& {\sf create()}),\ (x_2\& s_2) {\ \sf be\ } (push\  \& \ add\_array),\ \ (x_3\& s_3) {\ \sf be\ } (pop\  \& \ pop\_array)\}}&\\   
                &\mathtt{in \ (x_3(x_2\ 2 \ x_1) \ \&\  s_3*(s_2*\bar{2}*s_1)):\Box int}&
                \end{flalign*}}
            The last expression reduces to ($\beta$-reduction for {\sf let}):{\small\begin{flalign*}
                &{\sf \Sigma}; \bullet\vdash\mathtt{\ pop(push\ 2 \ empty)\ \&\  pop\_array*(add\_array*\bar{2}*empty):\Box int}
                \end{flalign*}}
            giving exactly the next step of the computation for the source expression.
            The good news is that the linker computes correctly the next step given any conforming set of implementations. 
            It is easy to see that given a {\sf list} implementation the very same process would produce a different computation step:{\small\begin{flalign*}
                &{\sf \Sigma}; \bullet\vdash\mathtt{\ pop(push\ 2 \ empty)\ \&\  pop\_list*(Cons*\bar{2}*[]):\Box int}
                \end{flalign*}}
        \end{enumerate}
        We conclude with some remarks that:
        \begin{itemize}
            \item The construction gives a mechanism of abstractions that works not only over different implementations in the
            same language but even for implementations in different (applicative) languages.
            \item We assumed in the example that the  two languages are based on the lambda calculus and implement a curried, higher-order function space. 
            It is easy to see that such host satisfies the requirements for the $\llbracket\bullet\rrbracket$ 
            (with $C_S, C_K$ being the $S, K$ combinators in $\lambda$ form  and $*$ translating to $\lambda$ application).
            \item
            Often, the host language of a foreign call is  not  a language that satisfies  such specifications. This situation occurs  when we have bindings from a functional language to a lower level language \footnote{In this setting the type signature of {\sf push} would be: $\mathtt{\sf int \times intstack\rightarrow instack}$}. 
            Such cases  can be captured by adding conjunction (and pairs), tuning the  specifications of $J$  accordingly and loosening the assumption that $\llbracket \bullet \rrbracket$
            is total on types.
            \item Introduction of  modal types is clearly relative to the $\llbracket\bullet \rrbracket$ function on types. 
            It would be interesting to consider examples where   $\llbracket\bullet \rrbracket$ is realized by non-trivial mappings such as $\llbracket A\supset B \rrbracket= !A \multimap B$
            from the embedding on intuitionistic logic to intuitionistic linear logic \cite{girard1987linear}.
            That will  showcase an example of   modality that works when lifting to a completely different logic or, correspondingly, to an essentially
            different computational model.
            \item Finally, it should be clear from the operational semantics and this example that we did not demand any equalities (or, reduction rules)  
            for the proofs in $J$, but mere existence of specific terms. This is in accordance to justification logic.  Analogously, we did not observe computation 
            in the host language but only the construction of the linkers as program transformers. We were careful, to say that our calculus corresponds to the dynamic 
            linking part of 
            separate compilation. This, of course, does not tell the whole story of program execution in such cases. Foreign function calls, return the control to the 
            client after the result gets calculated 
            in the external language. For example, the execution of the  program ${\texttt{pop (push 2 empty) + 2}}$ should ``escape'' the client 
            to compute the stack calls and then return
            for the last addition. Our modality is concerned  only with passing the control from the client to the host dynamically and, as such, is a $K$ 
            (non-factive) modality. Capturing the continuation of the computation and the return of the control back 
            to the source would  require a factive modality and a notion of ``reverse'' of the mapping $\llbracket\bullet\rrbracket$. 
            We would like to explore  such an extension in  future work.
            %Here the computation  mixes calling the  external implementation  $(\mathtt{st== empty})$ that -- in a full stack implementation -- would  provide for an equality check, computes its truth value externally and returns the control back to the client language  following the semantics of the $\mathtt{if}$ statement. The full logic for this can  be captured with a stronger modality (i.e. ``factive") that is work in progress. Nevertheless, our work is orthogonal and would correspond to dynamic linking aspect for such a system.
        \end{itemize}
        
        
        
        %\textcolor{green}{\texttt{push  empty\&}}}} $
        
        \section{Related and Future Work}
        \label{relat}
        Directly related work with our calculus, in the same fashion that justification logic and LP \cite{Artemov2001} are related to modal logic, is \cite{Bellin2001}.  The work in \cite{Bellin2001} provides a calculus for explicit assignments (substitutions) which is actually a sub-case of 
        {\sf Jcalc} with $\llbracket\bullet \rrbracket$ identity. This  sub-case  captures dynamic linking where the host language is the very same one; such need appears in languages with module mechanisms (i.e. implementation hiding and separate compilation within the very same language). In general, the judgmental approach to modality is heavily influenced by \cite{citeulike:5447115}. In a sense, our treatment of validity-as-explicit-provability also generalizes the approach there without having to commit to a ``factive" modality. Finally,  
        important results on programming paradigms related to justification logic have been obtained in \cite{ArtBon07LFCS,BONELLI2012935, bavera2010justification}. 
        Immediate future developments would be to interpret modal formulas of higher degree under the same principles. 
        This corresponds to dynamic linking in two or more steps (i.e., when the host becomes itself a client of another interface that is implemented dynamically in a third level and, so on). 
        Some preliminary results towards this 
        direction have been developed in \cite{DBLP:journals/entcs/PouliasisP14} and we sketch them in the next section. 
        %\nocite{Pfenning2009a, Pfenning2009b}
        

\begin{comment}					
\appendix
        
\label{appen}
\chapter{Appendix}
\subsection{Theorems}
        
        \begin{theorem}[Deduction Theorem for Validity Judgments]
            \label{deduct}
            Given any  $\Gamma,A,B \in {\sf Prop_0}$ then $\Gamma,x:A\vdash B \Longrightarrow \llbracket\Gamma\rrbracket\vdash\llbracket   A\supset B\rrbracket$. 
        \end{theorem}
        \begin{proof}
            The proof proceeds by induction on the derivations $\Gamma,A,B \in {\sf Prop_0}$. Note that the axiomatization of ${\llbracket\sf Prop_0\rrbracket}$ derives the 
            sequents:$\Delta\vdash\llbracket A \supset A\rrbracket$
            for any $\Delta\in {\sf \llbracket Prop_0 \rrbracket}$ (as in combinatory logic the $I$ combinator is derived from $SK$). This handles the reflection case. The rest of the cases are treated exactly as in the proof 
            of completeness of combinatorial axiomatization with respect to the natural deduction in intuitionistic logic. 
            Note that this theorem cannot be proven without the logical specification {\sf $Ax_1$, $Ax_2$}. I.e. it is exactly the requirements of the logical specification that ensure that all  interpretations  
            should be complete with respect to intuitionistic implication.
        \end{proof}
        \begin{lemma}[$\llbracket\bullet\rrbracket$Lifting Lemma]
            \label{bracklift}
            Given  $\Gamma,  A \in {\sf Prop_0}$ then $\Gamma\vdash   A \Longrightarrow \llbracket \Gamma\rrbracket \vdash \llbracket   A \rrbracket$. 
        \end{lemma}
        
        \begin{proof}
            The proof goes by induction on the derivations trivially for all the cases($\supset_{E_0}$ is treated using the ${\sf App}$ rule that internalizes Modus ponens). For the $\supset_{I_0}$ the previous theorem has to be used.
        \end{proof}

        \begin{proof}
            Assuming a derivation $\mathcal {D}$ ::$\Gamma\vdash   A$ from \ref{bracklift} there exists corresponding validity derivation $\mathcal{E}::\llbracket\Gamma\rrbracket\vdash\llbracket   A \rrbracket$. Using the two as premises in the $\Box_{IE}$ with $\Gamma := \Box \Gamma$ we obtain $\Box\Gamma\vdash\Box   A$.
        \end{proof}
        From the previous we get:

        %Let us show an inverse principle to the $\Box$ Lifting Lemma. We define for $A$ in {\sf Prop}:
        %\begin{flalign*}
        %\nonumber \downharpoonright P_i\ & =  P_i \\ \downharpoonright (A_1\supset A_2)&  =  \downharpoonright A_1 \supset \downharpoonright A_2 \\
        %\downharpoonright \Box A & = \downharpoonright A
        %\end{flalign*}
        %And the lifting of the $\downharpoonright$ over $\Gamma\in {\sf Prop}$. We get:
        \begin{theorem}[Collapse $\Box$ Lemma] If $\Box\Gamma\vdash \Box A$ for $\Gamma,A \in {\sf Prop_0}$ then $ \Gamma\vdash  A$.
        \end{theorem}
        \begin{theorem}[Weakening]
            For the N.D. system of {\sf Jcalc}, with $\Gamma, \Gamma^{\prime}\in {\sf Prop_0}$.
            \begin{enumerate}
                \item If  $\Gamma\vdash A$ then $\Gamma,\Gamma^{\prime}\vdash   A$.
                \item If  $\Box\Gamma\vdash \Box   A$ then $\Box\Gamma,\Box\Gamma^{\prime} \vdash \Box A$.
            \end{enumerate}
        \end{theorem} 
        \begin{proof}
            By induction on derivations for the first item. For the second item, given $\Box\Gamma\vdash \Box   A$ by the collapse lemma we get   $\Gamma\vdash   A$ which by the previous item
            gives $\Gamma,\Gamma^{\prime}\vdash   A$.  Using the lifting lemma we get $\llbracket \Gamma,\Gamma^{\prime}\rrbracket \vdash \llbracket  A \rrbracket$.
            Using the last two items we and the $\Box$ rule gives the result.
        \end{proof}
        
        \begin{theorem}[Contraction]
            \begin{enumerate}
                For the N.D. system of Jcalc, with $\Gamma,x:A,B\in {\sf Prop_0}$ 
                \item If  $\Gamma,x:A,x':A,\Gamma^{\prime}\vdash  B$ then $\Gamma,x:A,\Gamma^{\prime}\vdash   B$.
                \item If $\Box\Gamma,x:\Box A,x':\Box A,\Box\Gamma^{\prime}\vdash \Box B$ then $\Box\Gamma,x:\Box A,\Box \Gamma^{\prime}\vdash  \Box B$
            \end{enumerate}  
        \end{theorem}
        \begin{proof}
            Similarly with previous theorem.
        \end{proof}
        \begin{theorem}[Permutation]
            For the N.D. system of Jcalc, with $\Gamma\in {\sf Prop_0}$ and $\pi \Gamma$ the collection of permutations of $\Gamma$.
            \begin{enumerate}
                \item If  $\Gamma\vdash   A$ and $\Gamma^{\prime}\in \pi{\Gamma}$ then $\Gamma'\vdash   A$.
                \item If  $\Box\Gamma\vdash \Box   A$ then  $ \pi\Box\Gamma\vdash \Box   A$.
            \end{enumerate}
        \end{theorem}
        \begin{proof}
            As in the previous item.
        \end{proof}
        \begin{theorem}[Substitution Principle]
            The following hold for both kinds of judgments:
            \begin{enumerate}
                \item If  $\Gamma,x:A\vdash M: B$ and $\Gamma\vdash N: A$ then $\Gamma\vdash M[N/x]: B$ 
                \item If  $\llbracket\Gamma\rrbracket,s:\llbracket A \rrbracket \vdash {\sf J}: \llbracket B\rrbracket$ and 
                $\llbracket\Gamma\rrbracket\vdash {\sf J^{'}}: \llbracket B\rrbracket$ then  $\llbracket\Gamma\rrbracket\vdash  {\sf J[J^{'}/s]}\llbracket B\rrbracket$
            \end{enumerate}
        \end{theorem}
        All previous  theorems can actually be stated for proof terms too. We should discuss the following:
        \begin{theorem}[Deduction Theorem / Emulation of $\lambda$ abstraction]
            \label{deductterms}
            If $\Gamma, A\in {\sf Prop_0}$ and $\Gamma,x:A\vdash M:B$ then there exists ${\sf J}$ s.t.    $\llbracket\Gamma\rrbracket \vdash {\sf J}:\llbracket   A\supset B\rrbracket$.
        \end{theorem}
        \begin{lemma}[$\llbracket\bullet\rrbracket$Lifting Lemma for terms]
            \label{highorder}
            If $\Gamma, A \in {\sf Prop_0}$ and $\Gamma\vdash M: A$ then there exists ${\sf J}$ s.t. $\llbracket \Gamma\rrbracket \vdash {\sf J}:\llbracket   A \rrbracket$. 
        \end{lemma}
        In both theorems the existence of this ${\sf J,J^\prime}$ is algorithmic following the induction proof. 
        \subsection{Linking on the function space}
        The above mentioned algorithms permit  for translating $\lambda$ abstractions to polynomials of $S,K$ combinators which is a standard result in the literature. We do not give the details here but the translation is  syntax driven as it can be seen by the inductive nature of the proofs.
        
        Henceforth, we can generalize the construction in \ref{dlinker} so that it permits for dynamic linking of functions of the client 
        (with missing implementations) such as  $\mathtt{\lambda n:int. push\  n\  empty}$ dynamically given that the host actually implements 
        a higher-order function space (that is it implements the combinators $S,K$ in, say, own lambda calculus $\lambda^{J}$).
        Given implementations of $\mathtt{push\_impl}$, $\mathtt{empty\_impl}$ the linker produces an application expression 
        consisting of $\mathtt{push\_impl}$, $\mathtt{empty\_impl}$, $S$ and $K$.  
        The execution of the target expression will happen in the host after dereferencing  ${\sf push\_impl, empty\_impl}$ (dynamic part) 
        and the combinators $S,K$ (constant part) as, say, lambdas (e.g. $K=\lambda^{J} x.\lambda^{J} y. x$).
        
        \subsection{Gentzen's reduction Principle for $\Box$(General)}
        \label{redmult}
        \mbox{\footnotesize
            \begin{mathpar}
                \inferrule*[right=$  I_{\Box B} E_{\Box   A}^{x,s}$]{
                    \inferrule*{}
                    {\inferrule*[]{}{ \inferrule*[]	{\inferrule*[]{}{\PrTri{$D_1$} \\ \PrTri{$E_1$}}\\\\
                                \inferrule*[]{}{ \ A_1\\ \ \qquad\ \ \  \llbracket   A_1\rrbracket}}{\Box   A_1 }}}\ldots 
                    {\inferrule*[]{}{ \inferrule*[]	{\inferrule*[]{}{\PrTri{$D_i$} \\ \PrTri{$E_1$}}\\\\
                                \inferrule*[]{}{\    A_i\\ \ \qquad \llbracket   A_i\rrbracket}}{\Box   A_i }}}
                    \quad
                    \inferrule*{}
                    {\inferrule*[vdots=1.0em, right=$\vec{x}$]{ }{ A_1\dots A_i}\\\\
                        \inferrule*[]{}{B}} 	   \qquad 
                    \inferrule*{}
                    {\inferrule*[vdots=1.0em, right=$\vec{s}$]{ }{\llbracket   A_1 \ldots A_i \rrbracket}\\\\
                        \inferrule*[]{}{\llbracket B\rrbracket}} 	
                }
                {\Box B} 
            \end{mathpar}
        }
        $$\Longrightarrow_{R}$$
        \mbox{\footnotesize
            \begin{mathpar}
                \inferrule*[right=$I_{\Box B}$]{
                    \inferrule*{}
                    {\inferrule*[vdots=1.0em]{}{\PrTri{$D_1$}\ \PrTri{$D_i$} \\\\ A_1\ldots\ldots  \ \  A_i}\\\\
                        \inferrule*[]{}{B}}
                    \qquad 
                    \inferrule*{}
                    {\inferrule*[vdots=1.0em]
                        {}{\PrTri{$E_1$} \PrTri{$E_i$}\\\\\llbracket   A_1\ldots\ldots A_i\rrbracket}\\\\
                        \inferrule*[]{}{\llbracket B \rrbracket}} 
                }{\Box B}
                
            \end{mathpar}
        }
        \subsection{Notes on the cut elimination proof and normalization of natural deduction}
        \label{norm}
        Standardly, we add the bottom type and elimination rule in the natural deduction and show that in Jcalc + $\bot$: $\centernot\vdash\bot$. The addition goes as follows:
        
        \begin{mathpar}
            \inferrule*[right= Bot] { } {\bot \in {\sf Prop_0}}	
            \and
            \inferrule*[right= $E_\bot$] {{\Gamma\vdash\bot }\\ A\in {\sf Prop}} {\Gamma \vdash A}
        \end{mathpar}
        Our proof strategy follows directly \cite{pfenning2004automated}. We construct an intercalation calculus \cite{sieg1998normal} corresponding to the ${\sf Prop}$ fragment  with the following two judgments:
        \begin{itemize}
            \item[] $A\Uparrow$ for ``Proposition $A$ has normal deduction".
            \item[] $A^\downarrow$ for ``Proposition $A$ is extracted from hypothesis".
        \end{itemize}
        This calculus is, essentially, restricting the natural deduction to canonical derivations. The $\llbracket {\sf judgments} \rrbracket$ are not annotated and are directly ported from the natural deduction since we observe consistency in ${\sf Prop}$. 
        The construction is identical to \cite{pfenning2004automated} (Chapter 3) for the ${\sf Hypotheses},{\sf Coercion},\supset, \bot$ cases, we add the $\Box$ case.
        \begin{mathpar}
            \inferrule*[right=$\Gamma$-hyp]  {x: A\downarrow \in \Gamma^\downarrow}{ \Gamma^\downarrow\vdash^{-} A\downarrow}
            \and
            \inferrule*[right=$\downarrow\Uparrow$] {\Gamma^\downarrow\vdash^{-} A\downarrow}{\Gamma^\downarrow\vdash^{-} A \Uparrow}
            \and
            \inferrule*[right=$\supset$I$^{x}$] {\Gamma^\downarrow, x: A\downarrow\vdash^{-}  B\Uparrow} {\Gamma^\downarrow \vdash^{-}  A\supset  B\Uparrow}
            \inferrule*[right=$\supset$E] {{\Gamma^\downarrow\vdash^{-} A\supset  B \downarrow}\\{\Gamma^\downarrow\vdash^{-}  A\Uparrow}} {\Gamma^\downarrow\vdash^{-}   B\downarrow}
            %\and
            %\inferrule*[right=$\bot$E] {{\Turn {\Gamma} {\bot}}}{\Turn {\Gamma} {   A}}
            \and
            \inferrule*[right= $E_\bot$] {{\Gamma^{\downarrow}\vdash^{-}\bot\downarrow }\\ A\in {\sf Prop}} {\Gamma^{\downarrow}\vdash^{-} A\Uparrow}
            \and
            \inferrule*[right=$\Box_{IE}$ ] {{\Gamma^\downarrow\vdash\Box \Gamma^{\prime}\downarrow}\\{\Gamma'^\downarrow\vdash A\Uparrow }\\ {\llbracket \Gamma^{\prime} \rrbracket\vdash \llbracket A \rrbracket}}{ {\Gamma^\downarrow\vdash \Box A\Uparrow }}
        \end{mathpar}
        Where $\Gamma^\downarrow\vdash\Box \Gamma^{\prime}$ abbreviates $\forall A_i\in \Gamma'. \ \Gamma^{\downarrow}\vdash\Box A_i\downarrow$.
        We prove simultaneously by induction:
        \begin{theorem}[Soundness of Normal Deductions]
            The following hold:
            \begin{enumerate}
                \item If $\Gamma^\downarrow\vdash^{-} A\Uparrow$ then $\Gamma\vdash A$, and
                \item If $\Gamma^\downarrow\vdash^{-} A\downarrow $ then $\Gamma\vdash A$.
            \end{enumerate}
        \end{theorem}
        \begin{proof}
            Simultaneously by induction on derivations.
        \end{proof}
        It is easy to see that this restricted proof system $\centernot\vdash^{-} \bot\Uparrow$. It is hard to show its completeness to the non-restricted natural deduction ($\vdash + \bot_E$ of Jcalc) directly. For that reason we add a rule to make it complete ($\vdash^{+}$) preserving soundness and get a system of Annotated Deductions. We show the correspondence of the restricted system ($\vdash^{-}$) to a cut-free sequent calculus (${\sf JSeq}$), the correspondence of the extended system ($\vdash^{+}$) to ${\sf Jseq + Cut}$ and show cut elimination.\footnote{ In reality, the sequent calculus formulation is built exactly upon intuitions on the intercalation calculus. We refer the reader to the references.}
        
        To obtain completeness we add the rule:
        \begin{mathpar}
            \inferrule*[right=$\Uparrow\downarrow$] {\Gamma^\downarrow\vdash A\Uparrow} {\Gamma^\downarrow\vdash A\downarrow }
        \end{mathpar}
        We define $\vdash^{+} :=\   \ \ \vdash^{-} {\sf with} {\ \sf \Uparrow\downarrow}{\sf Rule}$.
        We show:
        \begin{theorem}[Soundness of Annotated Deductions]
            The following hold:
            \begin{enumerate}
                \item If $\Gamma^\downarrow\vdash^{+} A\Uparrow$ then $\Gamma\vdash A$, and
                \item If $\Gamma^\downarrow\vdash^{+} A\downarrow $ then $\Gamma\vdash A$.
            \end{enumerate}
        \end{theorem}
        \begin{proof}
            As previous item.
        \end{proof}
        
        \begin{theorem}[Completeness of Annotated Deductions]
            \label{compannot}
            The following hold:
            \begin{enumerate}
                \item If $\Gamma\vdash A$ then $\Gamma\downarrow\vdash^{+} A\Uparrow$, and
                \item If $\Gamma\vdash A$ then $\Gamma\downarrow\vdash^{+} A\downarrow$.
            \end{enumerate}
        \end{theorem}
        \begin{proof}
            By induction over the structure of the $\Gamma\vdash A$ derivation.
        \end{proof}
        
        Next we move with devising a sequent calculus formulation corresponding to normal proofs $\Gamma^{\downarrow}\vdash^{-}A\Uparrow$. The calculus that is given in the main body of this theorem. We repeat it here for completeness.
        \begin{mdframed}[nobreak=true,frametitle={\footnotesize Sequent Calculus ($\llbracket {\sf Prop_0} \rrbracket$)}]
            $$\begin{array}{l r}
            \Delta \Rightarrow \llbracket A\rrbracket:= & \exists \Delta'\in \pi(\Delta)\ \text{s.t} \   
            \Delta'\vdash \llbracket A \rrbracket \end{array}$$
            where $\pi(\Delta)$ is the collection of wellformed  $\llbracket {\sf Prop_0} \rrbracket$ contexts $\Delta'\vdash \llbracket {\sf wf}\rrbracket$  with some permutation of the multiset $\Delta$ as co--domain.
        \end{mdframed} 
        
        
        \begin{mdframed}[nobreak=true,frametitle={\footnotesize Sequent Calculus ({\sf Prop})}]
            \mbox{\small
                \begin{mathpar}
                    \inferrule*[right=$Id$] { }{\Gamma, A  \Rightarrow A }
                    
                    \inferrule*[right=$\supset_L$] {{\Gamma, A\supset B, B \Rightarrow  C }\\ {\Gamma, A\supset B \Rightarrow A}} {\Gamma, A\supset B \Rightarrow  C}
                    \and
                    \inferrule*[right=$\supset_R$] {\Gamma, A \Rightarrow  B} {\Gamma \Rightarrow A\supset B}
                    \and
                    \inferrule*[right=$\bot_L$] { } {\Gamma, \bot \Rightarrow A}
                    \and
                    \inferrule*[right=$\Box_{LR}$] {{\Box\Gamma,\Gamma\Rightarrow A}\\{\llbracket\Gamma\rrbracket\Rightarrow \llbracket A \rrbracket }}{\Box\Gamma\Rightarrow \Box A}
                    %\inferrule*[right=$\supset$E] {{\Turn {\Gamma} { A\supset  B}}\\{\Turn {\Gamma} { A}}} {\Turn {\Gamma} {   B}}
                    %\and
                    %\inferrule*[right=$\bot$E] {{\Turn {\Gamma} {\bot}}}{\Turn {\Gamma} {   A}}
                \end{mathpar}}
                %Where the  rule $\Box_{LR}$corresponds to $\Box_{IE}$ and relates the two kinds of sequents 
            \end{mdframed}
            We want to show correspondence of the sequent calculus  w.r.t normal proofs ($\vdash^{-}$).  Two lemmas are required to show soundness. 
            \begin{lemma}[Substitution principle for extractions]
                The following hold:
                \begin{enumerate}
                    \item If $\Gamma_1^\downarrow, x:A^\downarrow,\Gamma_2^\downarrow\vdash^{-} B\Uparrow$ and\\$\Gamma_1^\downarrow\vdash^{-} A\Uparrow$ then  $\Gamma_1^\downarrow,\Gamma_2^\downarrow\vdash^{-} B\Uparrow$
                    \item  If $\Gamma_1^\downarrow, x:A^\downarrow,\Gamma_2^\downarrow\vdash^{-} B\downarrow$ and $\Gamma_1^\downarrow\vdash^{-} A\downarrow$ then $\Gamma_1^\downarrow,\Gamma_2^\downarrow\vdash^{-} B\Uparrow$    
                \end{enumerate}
            \end{lemma}
            \begin{proof}
                Simultaneously by induction on the derivations $A\downarrow$ and $A\Uparrow$.
            \end{proof}
            And making use of the previous we can show, with ($\downharpoonright A$ defined previously):
            \begin{lemma}[Collapse principle for normal deductions]
                The following hold:
                \begin{enumerate}
                    \item If $\Gamma^\downarrow,\vdash^{-}  A\Uparrow$ then $\downharpoonright\Gamma^\downarrow\vdash^{-} \downharpoonright A\Uparrow$   and,
                    \item If $\Gamma^\downarrow\vdash^{-} A\downarrow$ then  $\downharpoonright\Gamma^\downarrow\vdash^{-} \downharpoonright A\downarrow$   
                \end{enumerate}
            \end{lemma}
            Using the previous lemmas and by induction we can show :
            \begin{theorem}[Soundness of the Sequent Calculus] 
                \label{soundnseq}
                If   $\Gamma\Rightarrow B$ then $\Gamma^\downarrow\vdash^{-} B\Uparrow$.
                
                
            \end{theorem}
            \begin{theorem}[Soundness of the Sequent Calculus with Cut] 
                
                If   $\Gamma\Rightarrow^{+} B$ then $\Gamma^\downarrow\vdash^{+} B\Uparrow$.
            \end{theorem}
            
            Next we define the $\Gamma\Rightarrow^{+} A$ as $\Gamma\Rightarrow A$ plus the rule:
            \begin{mathpar}
                \inferrule*[right=Cut]{{\Gamma\Rightarrow^{+} A}\\{\Gamma,A\Rightarrow^{+}B}}{\Gamma\Rightarrow^{+}B}
            \end{mathpar}
            \begin{proof}
                As before. The cut rule case is handled by the $\Uparrow\downarrow$ and substitution for extractions principle showcasing that the correspondence of the cut rule to the coercion from normal to extraction derivations.
            \end{proof}
            Standard structural properties (\textit{Weakening, Contraction}) to show completeness. We do not show these here but they hold.
            \begin{theorem}[Completeness of the Sequent Calculus] 
                \label{compseqcalc}
                The following hold:
                \begin{enumerate}
                    \item If   $\Gamma^\downarrow\vdash^{-} B\Uparrow$ then $\Gamma\Rightarrow B$ and,
                    \item 	If $\Gamma^\downarrow \vdash^{-} A\downarrow$ and $\Gamma,A\Rightarrow B$ then $\Gamma\Rightarrow B$   
                \end{enumerate}
                \begin{proof}
                    Simultaneously by induction on the given derivations making use of the structural properties.
                \end{proof}
                Similarly we show for the extended systems.
                \begin{theorem}[Completeness of the Sequent Calculus with Cut] The following hold:
                    \label{compseqcut}
                    \begin{enumerate}
                        \item If   $\Gamma^\downarrow\vdash^{+} B\Uparrow$ then  $\Gamma\Rightarrow^{+} B$ and,
                        \item 	If $\Gamma^\downarrow \vdash^{+} A\downarrow$ and $\Gamma,A\Rightarrow^{+} B$ then $\Gamma\Rightarrow^{+} B$.   
                    \end{enumerate}
                \end{theorem}
                \begin{proof}
                    As before. The extra case is handled by the Cut rule.
                \end{proof}
            \end{theorem}
            After establishing the correspondence of $\vdash^{-}$ with $\Rightarrow$ and of $\vdash^{+}$ with $\Rightarrow^{+}$ we move on with:
            \begin{theorem}[Admissibility of Cut]
                If $\Gamma\Rightarrow A$ and $\Gamma,A\Rightarrow B$ then $\Gamma\Rightarrow B$.
            \end{theorem}
            The proof is by triple induction on the structure of the formula, and the given derivations and we leave it for a technical report. This gives easily:
            \begin{theorem}[Cut Elimination]
                If $\Gamma\Rightarrow^{+}A$ then $\Gamma\Rightarrow A$.
                
            \end{theorem}
            Which in turn gives us:
            \begin{theorem}[Normalization for Natural Deduction]
                \label{normalization}
                If $\Gamma\vdash A$ then $\Gamma^{\downarrow}\vdash^{-} A\Uparrow$
            \end{theorem}
            \begin{proof}
                From assumption $\Gamma \vdash A$ which by \ref{compannot} gives $\Gamma\vdash^{+} A\Uparrow$. By \ref{compseqcut} and Cut  Elimination we obtain $\Gamma\Rightarrow A$ which by  \ref{soundnseq} completes the proof.
            \end{proof}
            As a result we obtain:

            \begin{proof}
                By contradiction, assume $\vdash\bot$ then $\Rightarrow \bot$ which is not possible.
            \end{proof}
\end{comment}
\chapter{Notes on extending the calculus}
\label{ext}
In this chapter we will make an informal case about the scalability
of the presented system. We will sketch how the calculus can
quite easily be extended in different ways 
and make a case that such extensions are of interest from 
the trinitarian (logic/ type theory/ category theory)
point of view.

\section{Extending on higher order modal types}
We saw in Chapter $6$ how the calculus corresponds to Jcalc algebras
which are essentially pairs of Heyting algebras under an order preserving function.
The points of such functions correspond to $\Box$ed types.

\begin{tikzpicture}[description/.style={fill=white,inner sep=2pt}]
    \matrix (m) [matrix of math nodes, row sep=3em,
    column sep=2.5em, text height=1.5ex, text depth=0.25ex]
    { A & & FA  \\
    B & & FB \\ };
    %\draw[double,double distance=5pt] (m-1-1) – (m-1-3);
    \path[-,font=\scriptsize]
    (m-1-1) edge[double,thick,double distance=5pt] node[auto] {$ \Box A $} (m-1-3)
    (m-2-1) edge[double,thick,double distance=5pt] node[auto] {$ \Box B $} (m-2-3)
    (m-1-1) edge node[auto] {$ \vdash $} (m-2-1)
    (m-1-3) edge node[auto] {$ \vdash $} (m-2-3);
    \end{tikzpicture}

This structure is easily
extensible to account for $\Box$ed types of higher degree. Instead of a pair of 
Heyting algebras we could have a stack of Heyting algebras 
related with order preserving
functions as shown in the schema.

\begin{tikzpicture}[description/.style={fill=white,inner sep=2pt}]
    \matrix (m) [matrix of math nodes, row sep=3em,
    column sep=2.5em, text height=1.5ex, text depth=0.25ex]
    { A & & F_0A & & F_1F_0A \\
      B & & F_0B & & F_1F_0B\\ };
    %\draw[double,double distance=5pt] (m-1-1) – (m-1-3);
    \path[-,font=\scriptsize]
    (m-1-1) edge[double,thick,double distance=5pt] node[auto] {$ \Box A $} (m-1-3)
    (m-2-1) edge[double,thick,double distance=5pt] node[auto] {$ \Box B $} (m-2-3)
    (m-1-3) edge[double,thick,double distance=5pt] node[auto] {$ \Box FA $} (m-1-5)
    (m-2-3) edge[double,thick,double distance=5pt] node[auto] {$ \Box FB $} (m-2-5)
    (m-1-1) edge[bend left = 30] node[auto] {$ \Box\Box A $} (m-1-5)
    (m-2-1) edge[bend right = 30] node[auto] {$ \Box\Box B $} (m-2-5)
    (m-1-1) edge node[auto] {$ \vdash $} (m-2-1)
    (m-1-3) edge node[auto] {$ \vdash $} (m-2-3)
    (m-1-5) edge node[auto] {$ \vdash $} (m-2-5);
    \end{tikzpicture}


In a nutshell, instead 
of one function symbol $\llbracket \rrbracket$ 
the system can 
be axiomatized to reason about chains of 
composable (provability) preserving functions. 
The modifications required are minor to obtain such a system.
Instead of a function symbol $\llbracket \rrbracket$ we have $F_0, F_1\ldots F_j$ and we 
define for any formula $\Box A\in {\sf Prop_i}$, 
$F_i\Box A:= \Box F_i A$ (
    and similarly, lifting over the connectives: 
$F_i(\Box A\supset \Box B):= \Box F_i A\supset\Box F_i B$)
the rule can then be written:

\begin{mdframed}[nobreak=true, frametitle={\footnotesize 
    Judgments on Necessity with $\Gamma\in {\sf Prop_i} \text{,{\ \sf length}}(\Gamma)=i\text{,\ }
    \ 1\le k\le j  \text{\ and, }\Gamma^{\prime},A, A_k,  B\in {\sf Prop_{i-1}}$ }]
\mbox{\footnotesize
    \begin{mathpar}
        \inferrule*[right=$I_{\Box B}E^{\vec{x},\vec{s}}_{\Box A_1\ldots \Box A_i}$]
        {{(\forall  A_i \in \Gamma'. \ \Turn {\Gamma}{\Box  A_i})}\\{\Turn {\Gamma'} { B}}\\{\Turn {F_i\Gamma'} {F_i  B} }} {\Turn {\Gamma}\Box  B}
        
    \end{mathpar}}
\end{mdframed}




\section{From order theory to category theory}

There is a classic passage from orders to categories, 
which corresponds to the passage of provability to 
proof relevance. In addition, order preserving functions become 
functors in the categorical scenario. But functors behave functionally
on terms (i.e. preserve proof equalities, or essentially, normalization principles of the cut elimination process).
To account, hence, for 
a categorical semantics of the system one has to account for equality
in the higher level of the system (i.e. on justifications).

This idea is actually not foreign in the litarature that explores the relation
between lambda calculus and (typed) combinatory logic and, in addition,
it is tempting to introduce equality between justifications so that one
could more accurately describe computational phenomena arising when 
a language interacts with another language (or, its own modules).

Generalizing, from the order theoretic semantics, we would expect a system
in which $\llbracket\rrbracket$ would correspond to functors 
(preserving the connectives and hence, $\beta\eta$ equalities). 
We are expecting the extension of Jcalc with rules for $\beta\eta$ equalities 
on the level of justifications to fit exactly the bill. The required 
equalities that correspond to $\beta$ reduction are standard in presentations of 
the $SK$ calculus. 
\begin{mdframed}
    \begin{mathpar}
        \inferrule*[right=$K_\beta$]{}{\Delta\vdash K*J_1*J_2 = J_1}
        \and
        \inferrule*[right=$S_\beta$]{} {\Delta\vdash S*J_1*J_2*J_3 = J_1*J_3(J_2*J_3)}
    \end{mathpar}
    \end{mdframed}
Rules corresponding to $\eta$ equality are less obvious 
but can be found in the excellent \cite{}. Similarly, equalities regarding (\wedge)
combinators can be introduced. 
We also need an axiom  that expressed 
that $\llbracket\rrbracket$ (or, $F_0$ in an extended system) is functorial (functional)
on terms
\begin{mdframed}
    \begin{mathpar}
        \inferrule*[right=$K_\beta$]{\Gamma\vdash M =_{\beta\eta} N :A}{\Delta\vdash \llbracket M\rrbracket=\rrbracket N\rrbracket :
        \llbracket A \rrbracket}    
    \end{mathpar}
\end{mdframed}
We expect that such a system is a full axiomatization  of functors
that preserve products and exponentials (of cartesian closed categories) 
Which are common in the litarature in proofs of the (syntactical) 
embedding of the lambda calculus into combinatory logic.



\section {Factivity and adjunctions}
Having an understanding of the system in order theoretic/ categorical terms 
helps thinking about extensions. In any category theoretic
textbook, the next ``tighter" relation between categories is that of an adjunction.
Interestingly, the notion of adjunction, is also central in the relation
between classical and intuitionistic proofs (the two parts of the adjunction
are inclusion, and double negation translation).  The notion of adjunction, plays
an important role in functional programming theory as the backbone 
of monadic computation. 

We would expect that the view of necessity as 
relating two proof systems could be extended to cover the notion of an adjunction.
In order theoretic terms an adjunction between two preorders $\mathcal{C},\mathcal{D}$
is a pair of orderd preserving functions $L:\mathcal{D}\rightarrow C$, $R:\mathcal{C}\rightarrow{D}$
such that there is an isomorphism:
$\forall d\in  D, c\in C Ld\ le c \longleftrightarrow d\le R c$. A logically interesting example of 
an adjunction is that between intuitionistic proofs and classical proofs where the $L$ adjunct is inclusion
and the right adjunct is double negation translation and we have:
\begin{mathpar}
    \inferrule*[right=$\downarrow\uparrow$]{I(\Gamma)\vdash\phi}{\Gamma\vdash\neg\neg \phi} 
\end{mathpar}
To axiomatize such notions in $Jcalc$ one should add another function symbol to 
correspond to $R$ and add the rule:  
\begin{mathpar}
    \inferrule*[right=$R$]{\llbracket\Gamma\rrbracket\vdash_J j:\phi}
    {\Gamma\vdash return{j}:R\phi} 
\end{mathpar}

This is enough to obtain a generalized notion of Factivity that is:
\begin{mathpar}
    \inferrule*[right=$R$]{\llbracket\Gamma\rrbracket\vdash M:\Box\phi}
    {\Gamma\vdash let \_&s=M in return (s):R\llbracket\phi\rrbracket} 
\end{mathpar}
Such an extension gives the standard factivity rule if $R\llbracket\rrbracket=id$.
Such extensions can capture phenomena in which a language is giving control to another language
to perform a computation externally but then it retains to control to calculate the continuation
of the client program. Refine example above.






\begin{appendices}
\chapter{Theorems}

\begin{theorem}[Deduction Theorem for Validity Judgments]
	\label{deduct}
With $\Delta\vdash \llbracket\sf wf\rrbracket$, if $\Delta, s:\llbracket   A\rrbracket \vdash\llbracket B \rrbracket $ then $\Delta \vdash\llbracket   A \supset B \rrbracket $. 
\end{theorem}
\begin{proof}
The proof is essentially the deduction theorem for a Hilbert style formulation of the corresponding fragment of  propositional logic and we do not show it here for economy. Note that this theorem cannot be proven without the logical specification {\sf $Ax_1$, $Ax_2$}. I.e. it is exactly the requirements of the logical specification that ensure that all  interpretations  should adequately embed intuitionistic implication.
\end{proof}
\begin{lemma}[$\llbracket\cdot\rrbracket$Lifting Lemma]
\label{bracklift}
Given any wellformed context of assumptions $\Gamma \vdash {\sf wf}$ and $\Gamma,  A \in {\sf Prop_0}$ then $\Gamma\vdash   A \Longrightarrow \llbracket \Gamma\rrbracket \vdash \llbracket   A \rrbracket$. 
\end{lemma}

\begin{proof}
The proof goes by induction on the derivations trivially for all the cases($\supset_E$ is treated using the ${\sf App}$ rule that internalizes Modus ponens). For the $\supset I$ the previous theorem has to be used.
\end{proof}
\begin{lemma}[$\Box$Lifting Lemma]
\label{boxlift}
For $\Gamma,   A \in {\sf Prop_0}$, then $\Gamma\vdash   A$ implies $\Box\Gamma \vdash \Box   A$.
\end{lemma}
\begin{proof}
 Assuming a derivation $\mathcal {D}$ ::$\Gamma\vdash   A$, 
 from the  previous item, there exists corresponding validity derivation $\mathcal{E}::\llbracket\Gamma\rrbracket\vdash\llbracket   A \rrbracket$. Using the two as premises in the $\Box_{IE}$ with $\Gamma := \Box \Gamma$ we obtain $\Box\Gamma\vdash\Box   A$.
\end{proof}
Let us show an inverse principle to the $\Box$ Lifting Lemma. 
We define for $A$ in {\sf Prop$_1$}:
\begin{flalign*}
\nonumber  \downharpoonright (A_1\supset A_2)&  =  \downharpoonright A_1 \supset \downharpoonright A_2 \\
\downharpoonright \Box A & = \downharpoonright A
\end{flalign*}
And the lifting of the $\downharpoonright$ over $\Gamma\in {\sf Prop}$. We get:
\begin{theorem}[Collapse $\Box$ Lemma] If $\Gamma\vdash A$ for $\Gamma,A \in {\sf Prop_1}$ then $\downharpoonright \Gamma\vdash \downharpoonright$ A.
\end{theorem}
\begin{theorem}[Weakening]
For the N.D. system of Jcalc, with $\Gamma,\Gamma^{\prime}\vdash \sf wf$ and $\Delta,\Delta^{'}\vdash\llbracket {\sf wf}\rrbracket $.
\begin{enumerate}
\item If  $\Gamma\vdash   A$ then $\Gamma,\Gamma'\vdash   A$.
\item If  $\Delta\vdash \llbracket   A\rrbracket$ then $\Delta,\Delta^{'} \vdash \llbracket   A\rrbracket$.
\end{enumerate}

\end{theorem} 
\begin{proof}
By induction on derivations.
\end{proof}
\begin{theorem}[Contraction]
For the N.D. system of Jcalc, with $\Gamma,x:A,x':A,\Gamma'\vdash \sf wf$ and $\Delta,s:\llbracket A \rrbracket, s':\llbracket A\rrbracket,\Delta^{'}\vdash\llbracket {\sf wf}\rrbracket $.
\begin{enumerate}
\item If  $\Gamma,x:A,x':A,\Gamma'\vdash  B$ then $\Gamma,x:A,\Gamma'\vdash   B$.
\item If  $\Delta,s:\llbracket A \rrbracket, s':\llbracket A\rrbracket,\Delta^{'}\vdash\llbracket B \rrbracket $ then $\Delta,s:\llbracket A \rrbracket, \Delta^{'} \vdash \llbracket   B\rrbracket$.
\end{enumerate}  
\end{theorem}
\begin{proof}
By induction on derivations.
\end{proof}
\begin{theorem}[Permutation]
For the N.D. system of Jcalc, with $\Gamma\vdash \sf wf$ and $\Delta\vdash\llbracket {\sf wf}\rrbracket $ and $\pi \Gamma$ and $\pi \Delta$ the collection of well-formed contexts of assumptions with the same co-domain of $\Gamma$, $\Delta$ we get
\begin{enumerate}
\item If  $\Gamma\vdash   A$ and $\Gamma'\in \pi{\Gamma}$ then $\Gamma'\vdash   A$.
\item If  $\Delta\vdash \llbracket   A\rrbracket$ and $\Delta^{'}\in \pi \Delta$ then  $ \Delta'\vdash \llbracket   A\rrbracket$.
\end{enumerate}
\end{theorem}
\begin{proof}
By induction on derivations.
\end{proof}
\begin{theorem}[Substitution Principle]
The following hold for both kinds of judgments:
\begin{enumerate}
\item If  $\Gamma,x:A\vdash B$ and $\Gamma\vdash  A$ then $\Gamma\vdash B$ 
\item If  $\Delta,s:\llbracket A \rrbracket \vdash \llbracket B\rrbracket$ and$\Delta\vdash  \llbracket A\rrbracket$ then  $\Delta\vdash\llbracket B\rrbracket$ 
\end{enumerate}
\end{theorem}
All previous  theorems can be stated for proof terms too. Specifically:
\begin{theorem}[Deduction Theorem / Emulation of $\lambda$ abstraction]
	\label{deductterms}
	With $\Delta\vdash \llbracket\sf wf\rrbracket$, if $\Delta, s:\llbracket   A\rrbracket \vdash {\sf j}:\llbracket B \rrbracket $ then there exists {\sf j'} s.t. $\Delta \vdash {\sf j'}:\llbracket   A \supset B \rrbracket $. 
\end{theorem}
\begin{lemma}[$\llbracket\cdot\rrbracket$Lifting Lemma for terms]
\label{highorder}
	If $\Gamma A \in {\sf Prop_0}$ and $\Gamma\vdash M: A$ then there exists ${\sf j}$ s.t. $\llbracket \Gamma\rrbracket \vdash {\sf j}:\llbracket   A \rrbracket$. 
\end{lemma}
In both theorems the existence of this ${\sf j,j'}$ is algorithmic following the induction
principle of the proof. 


\chapter{Notes on the cut elimination proof and normalization of natural deduction}
\label{norm}
Standardly, we add the bottom type and elimination rule in the natural deduction and show that in Jcalc + $\bot$: $\centernot\vdash\bot$. The addition goes as follows:

\begin{mathpar}
\inferrule*[right= Bot] { } {\bot \in {\sf Prop_0}}	
\and
\inferrule*[right= $E_\bot$] {{\Gamma\vdash\bot }\\ A\in {\sf Prop}} {\Gamma \vdash A}
\end{mathpar}
Our proof strategy follows directly \cite{pfenning2004automated}. 
We construct an intercalation calculus \cite{Sieg1998} corresponding to the ${\sf Prop}$ fragment  with the following two judgments:
\begin{itemize}
	\item[] $A\Uparrow$ for ``Proposition $A$ has normal deduction".
	\item[] $A^\downarrow$ for ``Proposition $A$ is extracted from hypothesis".
\end{itemize}
This calculus is, essentially, restricting the natural deduction to canonical derivations. The $\llbracket {\sf judgments} \rrbracket$ are not annotated and are directly ported from the natural deduction since we observe consistency in ${\sf Prop}$. 
The construction is identical to \cite{pfenning2004automated} (Chapter 3) for the ${\sf Hypotheses},{\sf Coercion},\supset, \bot$ cases, we add the $\Box$ case.
\begin{mathpar}
	\inferrule*[right=$\Gamma$-hyp]  {x: A\downarrow \in \Gamma^\downarrow}{ \Gamma^\downarrow\vdash^{-} A\downarrow}
	\and
	\inferrule*[right=$\downarrow\Uparrow$] {\Gamma^\downarrow\vdash^{-} A\downarrow}{\Gamma^\downarrow\vdash^{-} A \Uparrow}
	\and
	\inferrule*[right=$\supset$I$^{x}$] {\Gamma^\downarrow, x: A\downarrow\vdash^{-}  B\Uparrow} {\Gamma^\downarrow \vdash^{-}  A\supset  B\Uparrow}
	\inferrule*[right=$\supset$E] {{\Gamma^\downarrow\vdash^{-} A\supset  B \downarrow}\\{\Gamma^\downarrow\vdash^{-}  A\Uparrow}} {\Gamma^\downarrow\vdash^{-}   B\downarrow}
	%\and
	%\inferrule*[right=$\bot$E] {{\Turn {\Gamma} {\bot}}}{\Turn {\Gamma} {   A}}
\and
\inferrule*[right= $E_\bot$] {{\Gamma^{\downarrow}\vdash^{-}\bot\downarrow }\\ A\in {\sf Prop}} {\Gamma^{\downarrow}\vdash^{-} A\Uparrow}
\and
 	\inferrule*[right=$\Box_{IE}$ ] {{\Gamma\downarrow\vdash A\Uparrow }\\ {\llbracket \Gamma \rrbracket\vdash \llbracket A \rrbracket}}{ {\Box\Gamma\downarrow\vdash \Box A\Uparrow }}
 \end{mathpar}
We prove simultaneously by induction:
\begin{theorem}[Soundness of Normal Deductions]
The following hold:
\begin{enumerate}
\item If $\Gamma^\downarrow\vdash^{-} A\Uparrow$ then $\Gamma\vdash A$, and
\item If $\Gamma^\downarrow\vdash^{-} A\downarrow $ then $\Gamma\vdash A$.
\end{enumerate}
\end{theorem}
\begin{proof}
	Simultaneously by induction on derivations.
\end{proof}
It is easy to see that this restricted proof system $\centernot\vdash^{-} \bot\Uparrow$. It is hard to show its completeness to the non-restricted natural deduction ($\vdash + \bot_E$ of Jcalc) directly. For that reason we add a rule to make it complete ($\vdash^{+}$) preserving soundness and get a system of Annotated Deductions. We show the correspondence of the restricted system ($\vdash^{-}$) to a cut-free sequent calculus (${\sf JSeq}$), the correspondence of the extended system ($\vdash^{+}$) to ${\sf Jseq + Cut}$ and show cut elimination.\footnote{ In reality, the sequent calculus formulation is built exactly upon intuitions on the intercalation calculus. We refer the reader to the references.}

To obtain completeness we add the rule:
 \begin{mathpar}
 	\inferrule*[right=$\Uparrow\downarrow$] {\Gamma^\downarrow\vdash A\Uparrow} {\Gamma^\downarrow\vdash A\downarrow }
 \end{mathpar}
We define $\vdash^{+} :=\   \ \ \vdash^{-} {\sf with} {\ \sf \Uparrow\downarrow}{\sf Rule}$.
We show:
\begin{theorem}[Soundness of Annotated Deductions]
	The following hold:
	\begin{enumerate}
		\item If $\Gamma^\downarrow\vdash^{+} A\Uparrow$ then $\Gamma\vdash A$, and
		\item If $\Gamma^\downarrow\vdash^{+} A\downarrow $ then $\Gamma\vdash A$.
	\end{enumerate}
\end{theorem}
\begin{proof}
	As previous item.
\end{proof}

\begin{theorem}[Completeness of Annotated Deductions]
	\label{compannot}
	The following hold:
	\begin{enumerate}
		\item If $\Gamma\vdash A$ then, $\Gamma\downarrow\vdash^{+} A\Uparrow$, and
		\item If $\Gamma\vdash A$, then $\Gamma\downarrow\vdash^{+} A\downarrow$.
	\end{enumerate}
\end{theorem}
\begin{proof}
	By induction over the structure of the $\Gamma\vdash A$ derivation.
\end{proof}

Next we move with devising a sequent calculus formulation corresponding to normal proofs $\Gamma^{\downarrow}\vdash^{-}A\Uparrow$. The calculus that is given in the main body of this theorem. We repeat it here for completeness.
\begin{mdframed}[nobreak=true,frametitle={\footnotesize Sequent Calculus ($\llbracket {\sf Prop_0} \rrbracket$)}]
	$$\begin{array}{l r}
	\Delta \Rightarrow \llbracket A\rrbracket:= & \exists \Delta'\in \pi(\Delta)\ \text{s.t} \   
	\Delta'\vdash \llbracket A \rrbracket \end{array}$$
	where $\pi(\Delta)$ is the collection of wellformed  $\llbracket {\sf Prop_0} \rrbracket$ contexts $\Delta'\vdash \llbracket {\sf wf}\rrbracket$  with some permutation of the multiset $\Delta$ as co--domain.
\end{mdframed} 


\begin{mdframed}[nobreak=true,frametitle={\footnotesize Sequent Calculus ({\sf Prop})}]
	\mbox{\small
		\begin{mathpar}
			\inferrule*[right=$Id$] { }{\Gamma, A  \Rightarrow A }
			
			\inferrule*[right=$\supset_L$] {{\Gamma, A\supset B, B \Rightarrow  C }\\ {\Gamma, A\supset B \Rightarrow A}} {\Gamma, A\supset B \Rightarrow  C}
			\and
			\inferrule*[right=$\supset_R$] {\Gamma, A \Rightarrow  B} {\Gamma \Rightarrow A\supset B}
			\and
			\inferrule*[right=$\bot_L$] { } {\Gamma, \bot \Rightarrow A}
			\and
			\inferrule*[right=$\Box_{LR}$] {{\Gamma\Rightarrow A}\\{\llbracket\Gamma\rrbracket\Rightarrow \llbracket A \rrbracket }}{\Box\Gamma\Rightarrow \Box A}
			%\inferrule*[right=$\supset$E] {{\Turn {\Gamma} { A\supset  B}}\\{\Turn {\Gamma} { A}}} {\Turn {\Gamma} {   B}}
			%\and
			%\inferrule*[right=$\bot$E] {{\Turn {\Gamma} {\bot}}}{\Turn {\Gamma} {   A}}
		\end{mathpar}}
		%Where the  rule $\Box_{LR}$corresponds to $\Box_{IE}$ and relates the two kinds of sequents 
	\end{mdframed}
We want to show correspondence of the sequent calculus  w.r.t normal proofs ($\vdash^{-}$).  Two lemmas are required to show soundness. 
\begin{lemma}[Substitution principle for extractions]
The following hold:
\begin{enumerate}
\item If $\Gamma_1^\downarrow, x:A^\downarrow,\Gamma_2^\downarrow\vdash^{-} B\Uparrow$ and\\$\Gamma_1^\downarrow\vdash^{-} A\Uparrow$ then  $\Gamma_1^\downarrow,\Gamma_2^\downarrow\vdash^{-} B\Uparrow$
\item  If $\Gamma_1^\downarrow, x:A^\downarrow,\Gamma_2^\downarrow\vdash^{-} B\downarrow$ and $\Gamma_1^\downarrow\vdash^{-} A\downarrow$ then $\Gamma_1^\downarrow,\Gamma_2^\downarrow\vdash^{-} B\Uparrow$    
\end{enumerate}
\end{lemma}
\begin{proof}
Simultaneously by induction on the derivations $A\downarrow$ and $A\Uparrow$.
\end{proof}
And making use of the previous we can show, with ($\downharpoonright A$ defined previously):
\begin{lemma}[Collapse principle for normal deductions]
The following hold:
\begin{enumerate}
		\item If $\Gamma^\downarrow,\vdash^{-}  A\Uparrow$ then $\downharpoonright\Gamma^\downarrow\vdash^{-} \downharpoonright A\Uparrow$   and,
		\item If $\Gamma^\downarrow\vdash^{-} A\downarrow$ then  $\downharpoonright\Gamma^\downarrow\vdash^{-} \downharpoonright A\downarrow$   
	\end{enumerate}
\end{lemma}
Using the previous lemmas and by induction we can show :
\begin{theorem}[Soundness of the Sequent Calculus] 
	\label{soundnseq}
If   $\Gamma\Rightarrow B$ then $\Gamma^\downarrow\vdash^{-} B\Uparrow$.

	
\end{theorem}
\begin{theorem}[Soundness of the Sequent Calculus with Cut] 
	
	If   $\Gamma\Rightarrow^{+} B$ then $\Gamma^\downarrow\vdash^{+} B\Uparrow$.
\end{theorem}

Next we define the $\Gamma\Rightarrow^{+} A$ as $\Gamma\Rightarrow A$ plus the rule:
\begin{mathpar}
\inferrule*[right=Cut]{{\Gamma\Rightarrow^{+} A}\\{\Gamma,A\Rightarrow^{+}B}}{\Gamma\Rightarrow^{+}B}
\end{mathpar}
\begin{proof}
As before. The cut rule case is handled by the $\Uparrow\downarrow$ and substitution for extractions principle showcasing  the correspondence of the cut rule to the coercion from normal to extraction derivations.
\end{proof}
Standard structural properties (\textit{Weakening, Contraction}) to show completeness. We do not show these here but they hold.
\begin{theorem}[Completeness of the Sequent Calculus] 
	\label{compseqcalc}
	The following hold:
	\begin{enumerate}
		\item If   $\Gamma^\downarrow\vdash^{-} B\Uparrow$ then $\Gamma\Rightarrow B$ and,
		\item 	If $\Gamma^\downarrow \vdash^{-} A\downarrow$ and $\Gamma,A\Rightarrow B$ then $\Gamma\Rightarrow B$   
	\end{enumerate}
\begin{proof}
	Simultaneously by induction on the given derivations making use of the structural properties.
\end{proof}
Similarly we show for the extended systems.
\begin{theorem}[Completeness of the Sequent Calculus with Cut] The following hold:
	\label{compseqcut}
	\begin{enumerate}
		\item If   $\Gamma^\downarrow\vdash^{+} B\Uparrow$ then  $\Gamma\Rightarrow^{+} B$ and,
		\item 	If $\Gamma^\downarrow \vdash^{+} A\downarrow$ and $\Gamma,A\Rightarrow^{+} B$ then $\Gamma\Rightarrow^{+} B$.   
	\end{enumerate}
\end{theorem}
\begin{proof}
	As before. The extra case is handled by the Cut rule.
\end{proof}
\end{theorem}
After establishing the correspondence of $\vdash^{-}$ with $\Rightarrow$ and of $\vdash^{+}$ with $\Rightarrow^{+}$ we move on with:
\begin{theorem}[Admissibility of Cut]
	If $\Gamma\Rightarrow A$ and $\Gamma,A\Rightarrow B$ then $\Gamma\Rightarrow B$.
\end{theorem}
The proof is by double induction on the structure of the formula, its (sub-)derivations. This gives easily:
\begin{theorem}[Cut Elimination]
	If $\Gamma\Rightarrow^{+}A$ then $\Gamma\Rightarrow A$.
	
\end{theorem}
Which in turn gives us:
\begin{theorem}[Normalization for Natural Deduction]
	\label{normalization}
	If $\Gamma\vdash A$ then $\Gamma^{\downarrow}\vdash^{-} A\Uparrow$
\end{theorem}
\begin{proof}
	From assumption $\Gamma \vdash A$ which by completeness of annotated deductions gives $\Gamma\vdash^{+} A\Uparrow$. 
	Then by completeness of sequent calculus  and Cut  Elimination we obtain $\Gamma\Rightarrow A$ 
	which by  soundness of sequent calculus completes the proof.
\end{proof}
As a result we obtain:
\begin{proof}
	By contradiction, assume $\vdash\bot$ then $\Rightarrow \bot$ which is not possible.
\end{proof}
\chapter{Makam Implementation}
A \textit{Github} repo is preserved for $Jcalc$ which includes an implementation in the
metaprogramming framework {\sf Makam} ~\cite{stampoulis}. 
The implementation is currently developed by Antonis Stampoulis and
the author. A type checker for $Jcalc$ terms has been implemented and the call-by-value evaluation
strategy is under current development. 

The interested reader should install {\sf Makam}
(in a unix environment) and then she can run the \url{run.sh} 
file in the repository. 
A successful run will verify a number of Jcalc theorems encoded in the file \url{just.md} 
and will  additionally output an html file 
(already included in the repo as \url{just.html}).
The obtained file is a notebook of (pretty-printed) {\sf Latex} and {\sf Makam} code 
that can be opened in any modern browser.
It showcases how the implementation faithfully follows the rules 
of the formal system.  
\end{appendices}




\nocite{Pfenning2009a, Pfenning2009b}


\bibliographystyle{plain}
\bibliography{secondexam}

\end{document}
