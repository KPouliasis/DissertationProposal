\chapter{Curry -- Howard view of justification logic}
\label{proposal}
In this chapter we suggest reading a constructive necessity  of a formula ($\Box A$) as  internalizing a notion of 
constructive truth of $A$ 
(a proof within a deductive system $I$) and validity of $A$
(a proof under an interpretation  $\llbracket A \rrbracket_J$ within some system $J$).  
An example of such a relation is provided by the simply typed lambda calculus
(as $I$) and its implementation in $SK$ combinators (as $J$). 
We utilize justification logic to axiomatize the notion of 
validity-under-interpretation and, hence, treat  a  ``semantical'' notion in a purely proof-theoretic manner. 
We present the system  in 
Gentzen-style  natural deduction formulation  and provide reduction and expansion rules for the $\Box$ connective. 
In the subsequent chapter, we add proof-terms and proof-term equalities
to obtain a corresponding calculus ({\sf{Jcalc$^{-}$}}) that can be viewed as an extension of the Curry--Howard isomorphism with justifications.
We provide standard metatheoretic results  and suggest a 
programming language  interpretation in  languages with foreign function interfaces (\textit{FFI}s).

\section{Introduction: Necessity and Constructive Semantics}
In his seminal ``Explicit Provability and Constructive Semantics'' \cite{Artemov2001} 
Artemov developed a constructive, proof-theoretic semantics for 
\acs{BHK} proofs ~\cite{Troelstra1988} 
in what turned out to be the first development of a family of logics that we now call justification logic.
The general idea, upon which we build our calculus, is that semantics of a deductive system $I$ can be 
viewed in a solely proof-theoretic manner 
as mappings of proof constructs of $I$ into another proof system $J$ (which we call justifications).
As an example one could think  $I$  being  Heyting arithmetic and $J$ some  ``stronger'' system 
(e.g. a classical axiomatization of Peano arithmetic, a classical or intuitionistic set theory etc). 
 In Artemov's work $I$ is assumed to be
based on intuitionistic logic and $J$  on classical logic. 
We, initially,  mute such assumptions to focus exclusively on the mechanics of necessity in this framework.
We recover them later and study  their relation  to  the Rule of Necessitation for our system.
What's more,  such a semantic relation can be treated logically giving  rise to a modality of explicit necessity. 
Different sorts of necessity
($K$, $D$, $S4$, $S5$) have been offered  an explicit counterpart under the umbrella of justification logic. 
Some of them have been studied within a
Curry--Howard setting ~\cite{ArtBon07LFCS}. Our paper
focuses on  $K$ modality and  should be viewed as the  counterpart of ~\cite{Bellin2001} with justifications as we explain in \ref{relat}.
\subsection{Deductive Systems, Validity and Necessity}
Following a framework championed by Lambek \cite{Lambek1968,Lambek1969}, let us  assume two deductive systems $I$ 
(with propositional universe $U_I$, 
a possibly non-empty signature of axioms $\Sigma_I$ and an entailment relation $\Sigma_I;\Gamma\vdash_{I}A$) and $J$ 
(resp. with  $U_J$, $\Sigma_J$ and $\Sigma_J;\Delta\vdash_J \phi$). We will be using Latin letters for the formulae of $I$ and Greek letters for the formulae of $J$.
We will be omitting the $\Sigma$ signatures when they are not relevant.

For the  entailment relations of the two systems we require the following elementary principles\footnote{We are not excluding other connectives but by imposing
such minimal requirements we show that ``necessity'' ($\Box$) connective can be treated generically and orthogonally of the presence of other connectives}:
\begin{enumerate}
	\item \textit{Reflexivity}. In both relations $\Gamma$ and $\Delta$ are multisets of formulas (contexts) that enjoy reflexivity:
	$$A \in \Gamma \Longrightarrow \Gamma\vdash_{I}A$$ $$\phi \in \Delta \Longrightarrow \Delta\vdash_{J}\phi$$
	\item \textit{Compositionality}.  Both relations are closed under deduction composition:  
	$$\Gamma\vdash_I A \text{\ and\ } \Gamma^{'},A\vdash_{I} B \Longrightarrow \Gamma,\Gamma^{'}\vdash_I B $$  
	$$\Delta\vdash_J\phi \text{\ and\ } \Delta^{'},\phi\vdash_{J} \psi \Longrightarrow \Delta,\Delta^{'}\vdash_J \psi$$ 
	\item \textit{Top}. Both systems have a distinguished top formula $\top$ for which under any $\Gamma$, $\Delta$: $$\Gamma\vdash_{I}\top_I \text{\ and \ }
	\Delta \vdash_J\top_J$$
\end{enumerate}

Now we can define: 
\begin{definition}Given a deductive system $I$, an   \textit{interpretation for $I$}, noted by $\llbracket\bullet \rrbracket_J$, is a pair
	$(J,\llbracket\bullet\rrbracket)$ of a deductive  system $J$ together
	with a (functional) mapping $\llbracket \bullet \rrbracket: U_I\rightarrow U_J$ on propositions of $I$ into propositions of $J$ extended to multisets of 
	formulae of $U_I$ with the following properties:
	\begin{enumerate}
		\item \textit{Top preservation}. $\llbracket\top_I \rrbracket = \top_J$
		% \item\textit{structural interpretation of connectives}. The interpretation of compound formulas built upon primitive connectives of $I$ is structural.
		% E.g. Given $Con$ is a binary primitive connective of $I$ then in prefix notation we want:  
		% $\llbracket  Con(A, B) \rrbracket_J = \llbracket\ Con\rrbracket (\llbracket A \rrbracket , \llbracket \ B \rrbracket$ where $\llbracket Con\rrbracket$ is a connective in $J$.
		\item \textit{Structural interpretation of contexts}. For  $\Gamma$ contexts of the form $A_1,\ldots, A_n$:
		$$\llbracket\Gamma \rrbracket=\llbracket A_1  \rrbracket,\ldots,  \llbracket A_n\rrbracket$$ (trivially empty contexts map to empty contexts. 
		As in \cite{Lambek1968} they can be treated as 
		the $\top$ element).
	\end{enumerate}
\end{definition}
\begin{definition}Given a deductive system $I$  and an  interpretation $\llbracket\bullet\rrbracket_J$ for $I$ we define
	a \textit{corresponding validation of a deduction $\Sigma_I;\Gamma\vdash_I A$}  
	as a deduction $\Sigma_J;\Delta\vdash_{J} \phi$ in $J$ such that $\llbracket A \rrbracket=\phi$ and $\Delta=\llbracket \Gamma \rrbracket $ . We will be writing
	$ \llbracket \Sigma_I;\Gamma\vdash_I A\rrbracket_J$ to denote such a validation.
\end{definition}
\begin{definition}
	Given a deductive system $I$, we say that an interpretation  $\llbracket\bullet \rrbracket_J$  is \textit{logically complete} when  for
	all purely logical deductions $\mathcal{D}$ (i.e. deductions that make no use of $\Sigma_I$) in $I$ 
	there exists a corresponding (purely logical) validation $\llbracket\mathcal{D}\rrbracket$ in $J$.
	i.e. $$\forall \mathcal{D}. \ \mathcal{D}:\Gamma\vdash_I A \Longrightarrow \exists \llbracket\mathcal{D}\rrbracket: \llbracket \Gamma\vdash A\rrbracket_J$$
\end{definition}
\footnote{Note, that we require existence but not uniqueness. Nevertheless, if we treat deductive systems  in a proof irrelevant manner as preorders 
the above definition gives uniqueness vacuously. In a more refined approach where $I$ and $J$ are viewed as  categories of proofs  the above ``logical completeness''  
translates to the requirement that if the set of (purely logical) arrows $Hom_I(\Gamma,A)$ is non empty  then $Hom_J(\llbracket\Gamma\rrbracket,\llbracket A\rrbracket_J)$ 
cannot be empty (i.e. that $\llbracket\bullet \rrbracket_J$ can be extended to a functor). We leave a complete categorical semantics of our logic 
for future work but we expect   a generalization of the 
endofunctorial interpretations of $K$ modality appearing in \cite{Bellin2001,kavvos2016system}.} 


Examples of triplets ($I$, $J$, $\llbracket\bullet \rrbracket_J$) of logical systems that fall under the definition above are: any intuitionistic system mapped 
to a classical one under the embedding $\llbracket A \supset B\rrbracket= \tilde{\neg} A \tilde{\vee} B$ where  $\tilde{\neg}$
and $\tilde{\vee}$ are classical connectives, the opposite direction under double negation translation, 
an  intuitionistic system  mapped to another intuitionistic system (i.e. a mapping of atomic formulas of $I$ to atomic formulas of $J$ extended naturally to the intuitionistic connectives 
or, simply, the identity mapping) etc.  A vacuous validation  (when $\llbracket\bullet  \rrbracket_J$ maps everything to $\top$) gives another example. 

The main thesis that is exposed in this chapter is that this notion of ``double proof'' (reasoning about proofs that exists in two related systems) 
provides for an understanding of necessity in proof theoretic terms. In addition, we argue, 
that this is the driver of (at least) the simplest
form of necessity ($K$) that appears in justification logic (\textit{necessity as internalization}).
We will focus on the case where $I$ (the propositional part of our logic) is based on the implicative fragment of intuitionistic logic and show how justification logic
provides for an axiomatization of such logically complete interpretations $\llbracket\bullet\rrbracket_J$ of  implicative intuitionistic logic.
In what follows we provide  a natural deduction for
an intuitionistic system $I$ (truth), an axiomatization/specification of   $\llbracket \bullet \rrbracket_J$ (treated abstractly as a function symbol on types) and a treatment 
of basic necessity that relates the two deductions by internalizing  a  notion of ``double truth'' (proof in $I$  and  existence of corresponding validation in  $J$).
\section{Judgments of Jcalc$^{-}$}
\label{lsjcalc}

We aim for a reading of necessity that internalizes a notion of ``double proof''  in two deductive systems.
Motivated by the discussion and definitions in the previous section we will treat the notion of interpretation abstractly -- as a function symbol on types -- 
and axiomatize in accordance. Intuitively we want:
$$\Box  A\  {\sf true} :=  A\ {\sf true}\   \& \ A\ {\sf valid} = A{\sf\  true\  in\  I}\  \& \  \llbracket  A \rrbracket \ {\sf true\  in\  J}$$
We will be dropping indexes $I$, $J$ since they can be inferred by the different kinds of assumption  contexts. In addition, we  omit signatures 
$\Sigma$ since they do not offer anything from a logical perspective.

Logical entailment for the proposed  $\Box$ connective can be  summarized  easily given our previous discussion.
Given a deduction $\mathcal{D}:A \vdash B$ and  the existence of validation $\llbracket\mathcal{D}\rrbracket:\llbracket A \rrbracket\vdash \llbracket B\rrbracket$ then 
given $\Box A$ (i.e.  a proof of a  $\vdash A$ and a validation $\vdash \llbracket A \rrbracket$)
we obtain a double proof of $B$ (and hence, $\Box B$) by \textit{compositionality} of the underlying systems.  
Using standard, proof tree notation with labeled assumptions we formulate our rule of the connective in natural deduction:

\mbox{\small
	\begin{mathpar}
		\inferrule* [right=$I_{\Box B} E^{x,s}_{\Box A}$]{{\infer*{\Box  A}{}}
			\\{\inferrule*{}
				{\inferrule*[vdots=1.5em, right=$x$]{ }{ A}\\\\
					\inferrule*[]{}{ B}}}\\
			{\inferrule*{}
				{\inferrule*[vdots=1.5em, right=${s}$]{ }{\llbracket   A\rrbracket}\\\\
					\inferrule*[]{}{\llbracket  	B\rrbracket}}}
		}{\Box  B} 
		
	\end{mathpar}
}
We can, easily, generalize to $\Box$ed contexts (of the form $\Box A_1, \ldots,  \Box A_i$) of arbitrary length:
\\
\mbox{\small
	\begin{mathpar}
		\inferrule*[right=$I_{\Box B}E^{\vec{x},\vec{s}}_{\Box A_1\ldots \Box A_i} $] {{\infer*{\Box  A_1}{}}\\{\ldots}\\
			{\infer*{\Box  A_i}{}}\\	{\inferrule*{}
				{\inferrule*[vdots=1.5em, right=$\vec{x}$]{ }{\Gamma': A_1,\ldots  A_i}\\\\
					\inferrule*[]{}{ B}}}\\
			{\inferrule*{}
				{\inferrule*[vdots=1.5em, right=$\vec{s}$]{ }{\llbracket \Gamma'\rrbracket:\llbracket  A_1\rrbracket,\ldots \llbracket  A_i\rrbracket}\\\\
					\inferrule*[]{}{\llbracket  B\rrbracket}}}
		}{\Box  B}
		
	\end{mathpar}
}
\\
We read as ``Introducing $\Box B$ after eliminating $\Box A_1 \ldots \Box A_i $ crossing out (vectors of) labels $\vec{x}, \vec{s}$ ". 
Interestingly, the same rule eliminates boxes  and introduces new ones. 
This  is not surprising for $K$ modality (it is a left-right rule as we will see (\ref{seqcalc}).
See also discussion in \cite{Bellin2001,Bierman2000}). We will be referring to this rule  as ``$\Box$ Intro--After--Elim'' or, simply $\Box_{IE}$, from now on.
%Moreover, it has a very clear computational reading in our system which we will be explaining. 
%In a nutshell, theorems of $K$ correspond to programs that consume (eliminate) congruences on subterms to construct construct congruences on larger terms. In that sense, the elim-intro mix is quite natural.

Note that we define the $\Box$ connective negatively, yet (pure) introduction rules for the $\Box$ 
connective are  derivable. 
Such are instances of the previous Intro--After--Elim rule when 
$\Gamma'$ is empty which conforms exactly with the idea  of necessity internalizing double theoremhood.

\mbox{
	\begin{mathpar}	
		\inferrule*[right=$I_{\Box  B}$]{{\vdash B }\\{\vdash \llbracket B \rrbracket}}{\Box B}
	\end{mathpar}}
	
	In the next section, we provide the whole calculus in natural deduction format. As expected
	we will extend the implicational fragment of intuitionistic logic with 
	\begin{itemize}
		\item Judgments about validity (justification logic).
		\item Judgments that relate truth and validity (modal judgments).
	\end{itemize}
	%The former judgments could be viewed as a "Hilbert-style" encoding of the underlying intuitionistic one. Justification logic uses a combinatory calculus (extended with classical combinators) to represent logical principles (constants) in this second level. We follow the same route, obtaining a Curry Howard Correspondance for Justification logic under this new ``necessity as double proof" doctrine. 
	\subsection{Natural Deduction for Jcalc$^{-}$}
	The treatment of necessity in the previous section is completely orthogonal to the underlying systems \
	(it just assumes the basic requirements stated for the behavior $\llbracket\dot\rrbracket$). 
	In this section we will provide a full calculus and in congruence with justification logic we will
	assume that the underlying system ($I$) is a fragment of intuitionistic logic (the `negative' to be precise).
	The host theory $J$ can still remain unspecified, but the choice of $I$ informs for some specifications (to preserve
	completeness of logical deductions).   

	Following type theory conventions,  we first provide rules underlying type construction, then  rules for  well-formedness of (labeled) assumption contexts and rules  introducing and eliminating connectives. 
	The rules below should be obvious except for small caveat.
	 On the one hand, the type universe of $U_I$ and the proof trees of $I$ are 
	inductively defined as usual; on the other 
	hand, the host theory $J$ (its corresponding universe, connectives and  proof trees) is  ``black boxed''. What we actually axiomatize are the properties that all
	(logic preserving) interpretations of $I$ should conform to, independently of the specifics of the host theory. 
	Validity judgments should thus be read  as specifications
	of provability (existence of proofs) of any candidate $J$. 
	
	When we write $\llbracket \Gamma\rrbracket\vdash\llbracket\phi\rrbracket$ it reads as there exists derivation
	$\mathcal{D}$ : $\Delta\rrbracket\vdash_{J}\psi\rrbracket$ s.t. $\Delta=\llbracket\Gamma\rrbracket$ and $\psi = \llbracket \phi\rrbracket$ )
	
	We use ${\sf Prop_0}$  to denote  the type universe of $I$ and $\llbracket \sf Prop_0\rrbracket $ to denote its image under an interpretation, ${\sf Prop_1}$ denotes   modal (``boxed'') types
	and ${\sf Prop}$  the union of ${\sf Prop_0, Prop_1}$. We write $P_k$ with $k$ ranging in some subset of natural numbers to denote atomic propositions in $I$. 
	
	
	
	\begin{mdframed}[nobreak=true,frametitle={\footnotesize Judgments on Type Universe(s)}]
		\mbox{\small
			\begin{mathpar}
				\inferrule*[right= Atom] { } {P_k \in {\sf Prop_0}}
				\and
				\inferrule*[right=Top] { } {\top \in {\sf Prop_0}}
				\and
				\inferrule*[right=Conj] {{ A \in {\sf Prop_i }}\\ { B \in {\sf Prop_j}}} {  A \wedge B \in {\sf Prop_{max(i,j)}} } 
				%\inferrule*[right=Bot] { } {\bot \in {\sf Prop_0}} 
				%\and
				\and
				\inferrule*[right=Box] { A \in{\sf Prop_{0} }} {\Box  A\in{\sf Prop_{1}} }
				%\and
				%\inferrule*[right= Arr] {{ A \in {\sf Prop_{i} }}\\ { B \in {\sf Prop_{j}}}} { A\supset  B\in {\sf Prop_{max(i,j)}}}
				\and
				\inferrule*[right= Arr] {{ A \in {\sf Prop_i }}\\ { B \in {\sf Prop_j}}} { A\supset  B\in {\sf Prop_{max(i,j)}}}
				\and
				\inferrule*[right=Brc] { A \in {\sf Prop_0 }} {\llbracket  A\rrbracket \in {\sf \llbracket Prop_{0}\rrbracket}}
				%\and
				%\inferrule*[right=Brc $\supset$ Eq] {\llbracket A\supset\psi\rrbracket \in {\sf \llbracket Prop_{0}\rrbracket }} {\llbracket A\supset \psi\rrbracket=\llbracket A\rrbracket\supset \llbracket\psi\rrbracket :{\llbracket\sf Prop_{0}\rrbracket} }
			\end{mathpar}
		}
	\end{mdframed}
	
	%From now on we will be omitting the type construction derivations and write, informally, $ A\in U$( or, for multisets, $\Gamma\in U$) denoting the unique (\ref{bftu}) such derivation(s). 
	
	
	
	
	For labeled contexts of assumptions we require standard wellformedness conditions (i.e. uniqueness of labels).
	We use letters $x_i$, or simply $x$, for labels of  contexts with assumptions in $\sf Prop_0$, $x_i'$ or simply $x'$ for contexts with assumptions in   $\sf Prop_1$ and $s_i$, or simply $s$,  
	for $\sf \llbracket Prop_0 \rrbracket$ contexts. 
	We use $\circ$ for the empty context of ${\sf Prop_0}$ and ${\sf Prop_1}$ and  $\dagger$ for the empty context of ${ \llbracket {\sf Prop_0}\rrbracket}$.
	%We ``overload" the use of $\in$ symbol and write $x\not\in\Gamma$ to denote that ``the label $x$ is not present in the domain of $\Gamma$".
	We abuse notation and write $x:A\in\Gamma$ (or, similarly, $s:\llbracket A\rrbracket\in\Delta$) to denote that the label $x$ is assigned type $A$ in $\Gamma$; or $\Gamma\in {\sf Prop_0}$ 
	(resp. $\Gamma \in {\sf Prop_1}$, $\Delta\in {\sf \llbracket Prop_0\rrbracket}$)  
	to denote that  $\Gamma$ is a wellformed context  with  co--domain of elements in ${\sf Prop_0}$ (resp. in ${\sf Prop_1}$, $\llbracket {\sf Prop_0}\rrbracket$).
	For $\Gamma \in {\sf Prop_0}$ we define  $\llbracket \Gamma\rrbracket$ as  the lifting of the context $\Gamma$ through the $\llbracket \bullet \rrbracket$  symbol 
	(with appropriate renaming of variables -- e.g. $x_i\rightsquigarrow s_i$). For the vacuous case  when $\Gamma$  is  empty 
	we require $\llbracket\circ \rrbracket = \dagger$ to be well formed.
	
	% \begin{mdframed}[nobreak=true,frametitle={\footnotesize Judgments on Context Wellformedness}]
	% \mbox{\footnotesize
	% \begin{mathpar}
	% \inferrule*[right=\small{Nil}] { }{\Turn {\circ} {\sf wf}}
	% \and
	% \inferrule*[right=$\Gamma$-Ext] { {\Gamma\in {\sf Prop_0}}\\ { A \in {\sf Prop_0}} \\{x\not\in \Gamma}}{{\Gamma, x: A}\in {\sf Prop_0}}
	% \and
	% %\and
	% %\inferrule*[right=\small{Nil}] { } {\Turn {\circ} {\sf\llbracket wf\rrbracket}}
	% \inferrule*[right=$\llbracket$\small{ Nil} $\rrbracket$]{ }{\Turn {\dagger= \llbracket \circ \rrbracket} {\sf\llbracket wf\rrbracket}}
	% \and
	% \inferrule*[right=$\llbracket\Gamma\rrbracket$] { {  \Gamma  \in {\sf  Prop_0}}} {\llbracket \Gamma \rrbracket \in { \llbracket \sf Prop_0 \rrbracket}}
	% \end{mathpar}}
	%\end{mdframed}
	%Certain basic facts about well formed contexts are shown in the appendix \ref{bfcw}. 
	In the following entry we define proof trees (in turnstile representation) of the intuitionistic source theory $I$.  For all following rules we assume $\Gamma, A,B \in {\sf Prop_0}$:
	\begin{mdframed}[nobreak=true,frametitle={\footnotesize Judgments on Truth  $\Gamma, A,B \in {\sf Prop_0}$ }]
		\label{jots}
		\mbox{\small
			\begin{mathpar}
				\inferrule*[right=$\Gamma_0$-Refl] {x: A \in \Gamma}{\Turn {\Gamma} { A}}
				\and
				\inferrule*[right=$\top_0$I] { }{\Turn {\Gamma} { \top}}
				\and
				\inferrule*[right=$\supset_0$I] {{\Turn {\Gamma, x: A} { B}}} {\Turn {\Gamma} {   A\supset  B}}
				\and
				\inferrule*[right=$\supset_0$E] {{\Turn {\Gamma} { A\supset  B}}\\{\Turn {\Gamma} { A}}} {\Turn {\Gamma} {   B}}
				
				%\inferrule*[right=$\bot$E] {{\Turn {\Gamma} {\bot}}}{\Turn {\Gamma} {   A}}
			\end{mathpar}}
		\end{mdframed}
		
		For the calculus of interpretation (validity) we demand context reflexivity, compositionality and logical completeness with respect to  intuitionistic implication.
		Logical completeness is specified axiomatically, since the host theory is ``black boxed''. 
		Following justification logic, we use an axiomatic characterization of combinatory logic (for $\supset$) together 
		with the requirement that the interpretation preserves modus ponens:
		\begin{mdframed}[nobreak=true,frametitle={\footnotesize Judgments on Validity with {$\Delta\in \llbracket {\sf Prop_0} \rrbracket$}}]
			\label{jov}
			\mbox{\footnotesize
				\begin{mathpar}
					\inferrule*[right=$\Delta$-Refl] {s:\llbracket  A\rrbracket \in \Delta}{\Turn {\Delta} {\llbracket  A\rrbracket}}
					\and
					\inferrule*[right=Ax$_1$] { } {\Turn {\Delta} {\llbracket  \top\rrbracket}}
					\and
					\inferrule*[right=  Ax$_2$]{  A, B \in {\sf Prop_0} }   {\Delta\vdash \llbracket   A \supset (B \supset   A)\rrbracket }
					\and
					\inferrule*[right=  Ax$_3$]{ {  A,B, C \in {\sf Prop_0}}}{\Delta\vdash\llbracket   A\supset (B \supset C) \supset ((  A\supset B) \supset (  A \supset C))\rrbracket}
					\and
					\inferrule*[right=MP] {{\Turn {\Delta} { \llbracket  A \supset  B \rrbracket}}\\ {\Turn{\Delta} {\llbracket  A \rrbracket}}}{\Turn {\Delta} {\llbracket  B\rrbracket}}
				\end{mathpar}}
			\end{mdframed}
			
			Finally, we have judgments in the $\Box$ed universe (${\sf Prop_1}$). These are context reflection, the $\Box$ Intro-After-Elim rule, and the rules for intuitionistic implication between $\Box$ed types
			\footnote{The implication and elimination rules in ${\sf Prop_1}$ actually coincide with the ones
				in ${\sf Prop_0}$ since we are focusing on the case where $I$ is intuitionistic. This need not necessarily be the case as we have explained. Intuitionistic  implication among $\Box$ types should be read as
				``double proof of $A$ implies double
				proof of $B$'' and would still be defined even if we did not observe any kind of  implication in $I$. Similarly, one could provide intuitionistic conjunction or disjunction between $\Box$ types independently of 
				$I$ and, vice versa, one could add connectives in $I$ that are not observed between $\Box$ed types.}. 
			\begin{mdframed}[nobreak=true, frametitle={\footnotesize Judgments on Necessity with $\Gamma\in {\sf Prop_1} \text{,{\ \sf length}}(\Gamma)=i\text{,\ }
					\ 1\le k\le i  \text{\ and, }\Gamma^{\prime},A, A_k,  B\in {\sf Prop_0}$ }]
				\mbox{\footnotesize
					\begin{mathpar}
						\inferrule*[right=$\Gamma_1$-Refl] {x^{\prime}: \Box A \in \Gamma}{\Turn {\Gamma} {\Box A}}
						\and
						\inferrule*[right=$I_{\Box B}E^{\vec{x},\vec{s}}_{\Box A_1\ldots \Box A_i}$]{{(\forall  A_i \in \Gamma'. \ \Turn {\Gamma}{\Box  A_i})}\\{\Turn {\Gamma'} { B}}\\{\Turn {\llbracket \Gamma' \rrbracket} {\llbracket  B\rrbracket} }} {\Turn {\Gamma}\Box  B}
						\and
						\inferrule*[right=$\supset_1$I] {{\Turn {\Gamma, x^{\prime}: \Box A} { \Box B}}} {\Turn {\Gamma} {   \Box A\supset  \Box B}}
						\and
						\inferrule*[right=$\supset_1$E] {{\Turn {\Gamma} { \Box A\supset  \Box B}}\\{\Turn {\Gamma} { 
									\Box A}}} {\Turn {\Gamma} {  \Box B}}
					\end{mathpar}}
				\end{mdframed}
				
				%Observe, that as a consequence of our restrictions in the universe of propositions, 
				%all instances of the previous rule have $\Gamma'\in {\sf Prop_0}$. 
				%This is not a premise of the rule since it is can be proven, easily, meta--theoretically.
				\subsubsection{(Pure) $\Box I$ as derivable rule}
				We stress here that $\Box$ can be introduced positively with the previous rule with $\Gamma^{'}=\circ$. The first premise reduces to a simple requirement that $\Gamma\in{\sf Prop_1}$.
				
				\mbox{\small
					\begin{mathpar}
						\inferrule*[right=$I_{\Box A}$]{{\Turn {\circ} { A}}\\{\Turn {\dagger} {\llbracket  A\rrbracket} }} {\Turn {\Gamma}\Box  A}
					\end{mathpar}}
					\subsubsection{A simple derivation}
					\label{smpdv}
					We show here that the $K$ axiom of modal logic is a theorem (omitting some obvious steps). In the following {\small $$\Gamma := x_1^{\prime}:\Box (A\supset B),x_2^{\prime}:\Box A, \ \Gamma^{\prime}=x_1:A\supset B, x_2:A,
						\ \llbracket \Gamma^\prime\rrbracket= s_1:\llbracket A \supset B\rrbracket, s_2:\llbracket A\rrbracket$$}
					\mbox{\small
						\begin{mathpar}
							\inferrule*[right=$\supset_1$I]{\inferrule*[right=$\supset_1$I]{
									\inferrule*[right= $I_{\Box A}E^{x_1,x_2,s_1,s_2}_{\Box A\supset B, \Box A}$]{{\inferrule*[]{}{\Gamma \vdash \Box ( A \supset B)}}\\
										{\inferrule*[]{}{ \Gamma\vdash \Box  A }}\\{\inferrule*[]{}{\Gamma^\prime\vdash B}}\\{\inferrule*[]{}{\llbracket\Gamma^\prime\rrbracket\vdash \llbracket B\rrbracket}}} {\Box ( A\supset B), \Box  A \vdash \Box B }} {\Box ( A\supset B) \vdash \Box  A \supset \Box B}}{\circ\vdash \Box ( A\supset B)\supset \Box  A \supset \Box B}
						\end{mathpar}
					}
\subsection{Logical Completeness, Admissibility of Necessitation and Completeness with respect to Hilbert Axiomatization}
\label{completness}
Here we give a Hilbert  axiomatization  of the $\supset$ fragment of intuitionistic $K$ logic in order to compare it with our system. Here  $\vdash^{\mathcal{H}}$ captures the textbook (metatheoretic)
notion of ``deduction from assumptions'' in a Hilbert style axiomatization. We assume the restriction of the system to formulas up to modal degree $1$.
\begin{mdframed}[nobreak=true,frametitle={\footnotesize Hilbert Style Formulation}]
	\mbox{\footnotesize
		\begin{mathpar}
			\inferrule*[left=ax1.]{}{ A\supset (B \supset  A)}
			\and
			\inferrule*[left=ax2.]{}{ (A\supset (B \supset C) )\supset (( A\supset B) \supset ( A \supset C))}
			\and
			\inferrule*[left=K.]{}{\Box ( A\supset B)\supset\Box  A \supset \Box B}
			\and
			\inferrule*[left=MP]{ { A \supset B}\\ { A}}{B}
			\and
			\inferrule*[left=Nec]{\vdash^{\mathcal{H}}  A}{\Box  A}
		\end{mathpar}}
	\end{mdframed}
	
	It is easy to verify that axioms $1$, $2$ are derived theorems of {\sf Jcalc$^{-}$} in ${\sf Prop_0}$. The rule Modus Ponens is also admissible trivially, whereas axiom $K$  was  
	shown to be a theorem in the previous section (\ref{smpdv}). The rule of Necessitation is not obviously admissible though. In our reading of necessity the
	admissibility of this rule is directly related to the requirement of ``logical completeness of the interpretation'' i.e. preservation of logical theoremhood.  
	In general, adding more connectives in $I$ would require additional specifications for the host theory to obtain necessitation.
	
	The steps of the proof are given in the Appendix, but this is essentially  the ``lifting lemma'' in justification logic \cite{Artemov2001}. 
	The proof fully depends on
	the provability requirements imposed in the $\llbracket{\sf Prop_0}\rrbracket$ fragment.
	\begin{theorem}[$\Box$Lifting Lemma]
		In {\sf Jcalc$^{-}$}, for every  $\Gamma,   A \in {\sf Prop_0}$ if  $\Gamma\vdash A$ then  $\llbracket \Gamma\rrbracket\vdash\llbracket A\rrbracket $ and, hence, $\Box\Gamma \vdash \Box   A$.
	\end{theorem}
	We get admissibility of necessitation as a lemma for $\Gamma$ empty:
    \begin{theorem}[Admissibility of Necessitation]
		
     For $  A \in {\sf Prop_0}$, if $\circ\vdash   A$ then $\circ\vdash \Box A$. 
	\end{theorem}
	As a result:
	\begin{theorem}[Completeness]
		{\sf Jcalc$^{-}$} is complete with respect to the Hilbert style formulation of degree-$1$ intuitionistic $K$ modal logic. 
	\end{theorem}
	
	\subsection{Harmony: Local Soundness and Local Completeness}
	\label{gprinc}
	Before we move on to show (Global) Soundness we provide evidence for the so called ``local soundness" and ``local completeness'' of the $\Box$ connective
	following Gentzen's dictum. The local soundness and completeness for the $\supset$ connective 
	is given elsewhere (e.g. \cite{prawitz10natural}) and in Gentzen's original \cite{gentzen1935untersuchungen}. Gentzen's program can be described with the following two slogans:\begin{itemize} \item[a.] Elim is left-inverse to Intro \item[b.] Intro is right-inverse to Elim\end{itemize}   
	Applied to the $\Box$ connective, the first principle says that introducing a $\Box   A$ (resp. many $\Box A_1, \ldots, \Box A_i$) only to eliminate it (resp. them) directly is redundant. 
	In other words, the elimination rule cannot give you more data than what were inserted in the introduction rule(s)  (``elimination rules are not \textit{too} strong").
	We show first the ``Elim-After-Singleton-Intro" sub-case.
	
	\mbox{\footnotesize
		\begin{mathpar}
			\\
			\inferrule*[right=\large{$\quad\Longrightarrow_{R}\quad$}]{
				\inferrule*{}
				{\inferrule*[]{}{ \inferrule*[]	{\inferrule*[]{}{\PrTri{D} \\ \PrTri{E}}\\\\
							\inferrule*[]{}{\    A\\ \ \qquad \llbracket   A\rrbracket}}{\Box   A }}} \quad
				\inferrule*{}
				{\inferrule*[vdots=1.0em, right=$x$]{ }{  A}\\\\
					\inferrule*[]{}{B}} 	   \qquad 
				\inferrule*{}
				{\inferrule*[vdots=1.0em, right=$s$]{ }{\llbracket   A \rrbracket}\\\\
					\inferrule*[]{}{\llbracket B\rrbracket}} 	
			}
			{\Box B}  \inferrule*[]{
				\inferrule*{}
				{\inferrule*[vdots=1.0em]{}{\PrTri{D} \\\\  A}\\\\
					\inferrule*[]{}{B}}
				\qquad 
				\inferrule*{}
				{\inferrule*[vdots=1.0em]
					{}{\PrTri{E}\\\\\llbracket   A \rrbracket}\\\\
					\inferrule*[]{}{\llbracket B \rrbracket}} 
			}{\Box B}
			
		\end{mathpar}
	}
	The exact same principle applies in the ``Elim-after-Intro''  of   multiple $\Box$s:
	
		\mbox{\footnotesize
			\begin{mathpar}
				\inferrule*[right=$  I_{\Box B} E_{\Box   A}^{x,s}$]{
					\inferrule*{}
					{\inferrule*[]{}{ \inferrule*[]	{\inferrule*[]{}{\PrTri{$D_1$} \\ \PrTri{$E_1$}}\\\\
								\inferrule*[]{}{ \ A_1\\ \ \qquad\ \ \  \llbracket   A_1\rrbracket}}{\Box   A_1 }}}\ldots 
					{\inferrule*[]{}{ \inferrule*[]	{\inferrule*[]{}{\PrTri{$D_i$} \\ \PrTri{$E_1$}}\\\\
								\inferrule*[]{}{\    A_i\\ \ \qquad \llbracket   A_i\rrbracket}}{\Box   A_i }}}
					\quad
					\inferrule*{}
					{\inferrule*[vdots=1.0em, right=$\vec{x}$]{ }{ A_1\dots A_i}\\\\
						\inferrule*[]{}{B}} 	   \qquad 
					\inferrule*{}
					{\inferrule*[vdots=1.0em, right=$\vec{s}$]{ }{\llbracket   A_1 \ldots A_i \rrbracket}\\\\
						\inferrule*[]{}{\llbracket B\rrbracket}} 	
				}
				{\Box B} 
			\end{mathpar}
		}
		$$\Longrightarrow_{R}$$
		\mbox{\footnotesize
			\begin{mathpar}
				\inferrule*[right=$I_{\Box B}$]{
					\inferrule*{}
					{\inferrule*[vdots=1.0em]{}{\PrTri{$D_1$}\ \PrTri{$D_i$} \\\\ A_1\ldots\ldots  \ \  A_i}\\\\
						\inferrule*[]{}{B}}
					\qquad 
					\inferrule*{}
					{\inferrule*[vdots=1.0em]
						{}{\PrTri{$E_1$} \PrTri{$E_i$}\\\\\llbracket   A_1\ldots\ldots A_i\rrbracket}\\\\
						\inferrule*[]{}{\llbracket B \rrbracket}} 
				}{\Box B}
				
			\end{mathpar}
		}
	These equalities are of importance since they dictate (together with the corresponding principles for the $\supset$, $\wedge$ connectives) the proof dynamics of the calculus. 
	The proof term assignment and the corresponding computational ($\beta$-)rules  are directly instructed by these reduction principles. We see that eliminating (using) an introduced $\Box$ 
	corresponds to double substitution in the corresponding judgments. 
	
	Dually, the second principle says eliminating a $\Box A$ , should give enough information to directly reintroduce it (``elimination rules are not \textit{too weak}"). This is an expansion principle.
	
	\mbox{\small
		\begin{mathpar}
			\inferrule*[]{}{ \inferrule*[lab=$\ \mathcal{D}$]{}{\Box   A }}\quad  \Longrightarrow_{E}\quad
			\inferrule*[Right=$I_{\Box A}E^{x,s}_{\Box  A}$]{
				\inferrule*{}
				{\inferrule*[]{}{  }\\\\ \quad\inferrule*[]{}{ \inferrule*[lab=$\ \mathcal{D}$]{}{\Box   A }}\\
				} \qquad
				\inferrule*{}
				{\inferrule*[right=$x$]{ }{  A}}\\
				\inferrule*{}
				{\inferrule*[right=$s$]{ }{\llbracket A\rrbracket}
				} 	
			}
			{\Box   A} 
		\end{mathpar}}
		\subsection{(Global) Soundness}
		\label{seqcalc}
		Soundness is shown by proof theoretic techniques. Standardly, we add the bottom type ($\bot$) to {\sf Jcalc$^{-}$} together with its elimination rule and show  that the system is consistent ($\not\vdash \bot$) by   devising a sequent calculus  and showing admissibility of cut. We 
		only present the calculus here and collect the theorems towards consistency  in the Appendix. 
		
		In the following we use $\Gamma\Rightarrow   A$ (where $\Gamma,  A \in {\sf Prop_0}\cup{\sf Prop_1})$  to denote 
		sequents modulo $\Gamma$ permutations where $\Gamma$ is a multiset of ${\sf Prop}$ (no labels)  and $\Delta \Rightarrow\sf \llbracket   A \rrbracket$ for sequents corresponding to $\llbracket \sf judgments \rrbracket$ of the calculus modulo $\Delta$ permutations (with $\Delta$ (unlabeled) multiset of ${\sf \llbracket Prop_0 \rrbracket}$). The multiset/ modulo permutation approach is instructed by standard structural properties. 
		All properties are stated formally and proved in the Appendix. 
		
		The  $\llbracket\Gamma\rrbracket \Rightarrow \llbracket A \rrbracket$ relation  is defined directly from  $\vdash$:
		\begin{mdframed}[nobreak=true,frametitle={\footnotesize Sequent Calculus ($\llbracket {\sf Prop_0} \rrbracket$)}]
			$$\begin{array}{l r}
			{\llbracket\Gamma\rrbracket} \Rightarrow \llbracket A\rrbracket:= & \exists \Gamma^{\prime}\in \pi(\llbracket\Gamma\rrbracket)\ \text{s.t} \   
			\Gamma^{\prime}\vdash \llbracket A \rrbracket \end{array}$$
			where $\pi(\llbracket\Gamma\rrbracket)$ is the collection of  permutations of $\llbracket\Gamma\rrbracket$.
		\end{mdframed} 
		
		
		\begin{mdframed}[nobreak=true,frametitle={\footnotesize Sequent Calculus ({\sf Prop})}]
			\mbox{\small
				\begin{mathpar}
					\inferrule*[right=$Id$] { }{\Gamma, A  \Rightarrow A }
					
					\inferrule*[right=$\supset_L$] {{\Gamma, A\supset B, B \Rightarrow  C }\\ {\Gamma, A\supset B \Rightarrow A}} {\Gamma, A\supset B \Rightarrow  C}
					\and
					\inferrule*[right=$\supset_R$] {\Gamma, A \Rightarrow  B} {\Gamma \Rightarrow A\supset B}
					\and
					\inferrule*[right=$\bot_L$] { } {\Gamma, \bot \Rightarrow A}
					\and
					\inferrule*[right=$\Box_{LR}$] {{\Box\Gamma,\Gamma\Rightarrow A}\\{\llbracket\Gamma\rrbracket\Rightarrow \llbracket A \rrbracket }}{\Box\Gamma\Rightarrow \Box A}
					%\inferrule*[right=$\supset$E] {{\Turn {\Gamma} { A\supset  B}}\\{\Turn {\Gamma} { A}}} {\Turn {\Gamma} {   B}}
					%\and
					%\inferrule*[right=$\bot$E] {{\Turn {\Gamma} {\bot}}}{\Turn {\Gamma} {   A}}
				\end{mathpar}}
				%Where the  rule $\Box_{LR}$corresponds to $\Box_{IE}$ and relates the two kinds of sequents 
			\end{mdframed}
			
			Standardly, we  extend the system with the ${\sf Cut}$ rule and we  obtain the extended system $\Gamma \Rightarrow^{+} A:= \Gamma\Rightarrow A + {\sf Cut}$. 
			We show Completeness of $\Rightarrow^{+}$ with respect to Natural Deduction and  Admissibility of Cut that leads to the consistency result
			%\begin{theorem}[Admissibility of Cut]
			%$\Gamma \Rightarrow^{+}A$ implies $\Gamma\Rightarrow A$
			%\end{theorem}
			%Which together with the following theorems:
			%\begin{theorem}[Completeness of $\Rightarrow^{+}$ with respect to Natural Deduction]
			%If  $\Gamma\vdash A$ then $\Gamma \Rightarrow^{+}A$
			%\end{theorem}
			%Which together with the proposition: 
			%\begin{proposition}
			%$\centernot{\Rightarrow^{+}}\bot$
			%\end{proposition}
			\begin{theorem}[Consistency of {\sf Jcalc$^{-}$}]
				$\centernot{\vdash}\bot$
			\end{theorem}
			
			
			
			%\begin{mathpar}
			%\inferrule*[right=K]  {  A\supset  B\supset A \in [Prop]_{i>0}}    {{\Delta} \vdash  {{\sf K}[ A, B]:  A\supset B\supset A}}
			%\and
			%\inferrule*[right=S]  { A\supset B\supset C \in [Prop]_{i>0}}    {{\Delta} \vdash  {{\sf S}[ A, B, C]: ( A\supset B\supset C)\supset( A\supset B)\supset( A\supset C)}}
			
			%\and
			%\inferrule*[right= App] {{\Turn {\Gamma, x: A} {M: B}}\\{\Turn {\Gamma} {\sf wf_{i>0}} }} {\Turn {\Gamma} {\lambda  x: A . \   M :  A\supset  B}}
			%\and
			%\inferrule*[right=]{{\Turn {\vec{v}:G} {t: A}}\\{\Turn {\vec{v'}:[G]} {t':[ A]} }} {\Turn {\vec{x}:\Box G}  let\  { \xshlongvec[1] {link(v,v')=x}\  {\sf in} \  link(t,t')}:\Box  A}
			%\end{mathpar}
			
			%Finally, we have the rule that relates judgments of truth to judgments of validity explained in the beginning of this section.
			\section{Order theoretic semantics}
			This chapter started by introducing 
			mappings between deductive systems and 
			motivating the reading of necessity as ``double-proof under a map''.
			As a result, it is unsuprising the the calculus is amenable to order theoretic sematics.
			We present them in this section.

			In order to progress we first define the notion of a \vocab{semi-Heyting  Algebra(semi-HA)}. 
			To define \vocab{semi-HA} we need the notion of a \emph{(meet) semi-lattice}.
			  
			
			\begin{mdframed}
			\textbf{Definition:}
			A \textit{(meet) semi-lattice} is a non-empty \emph{partial order} (i.e. reflexive, antisymmetric and transitive) 
			with finite meets.
			\end{mdframed}
			In addition, we define \emph{meet semi-lattice} as follows: 
			\begin{mdframed}
			\textbf{Definition:}
			A \textit{bounded (meet) semi-lattice} $(L,\le)$ is a (meet) semi-lattice that additionally has 
			a greatest element $1$, which satisfies
			
			$x \le 1$ for every $x$ in $L$
			\end{mdframed}
			Finally, we can define \emph{semi-HA}:
			
			\begin{mdframed}
			\textbf{Definition:}
			A \textit{semi-HA} is a bounded (meet) semi-lattice $(L,\le, 1)$ 
			s.t. for every $a,b\in L$ there exists an \textit{exponential} 
			(we name it $b^{a}$) 
			with the properties: 
			\begin{enumerate}
			\item $b^{a}\times a\le b $
			\item $b^{a}$ is the greatest such element
			\end{enumerate}
			\end{mdframed}
			\subsubsection{Axiomatization of semi-HAs}
			We can axiomatize the meet (i.e. greatest lower bound)($\times$) of $\phi,\psi$ for any  lower bound $\chi$.
			\begin{mdframed}
			\begin{mathpar}
			  \infer{\phi \conj \psi \leq \phi}{
				}
			  \and
			  \infer{\phi \conj \psi \leq \psi}{
				} 
			\end{mathpar}
			\begin{equation*}
			  \infer{\chi \leq \phi \conj \psi}{
				\chi \leq \phi & \chi \leq \psi} 
			\end{equation*}
			\end{mdframed}
			
			We can axiomatize the existence of a greatest element as follows:
			\begin{mdframed}
			\begin{equation*}
			  \infer{\chi \leq 1}{
				} 
			\end{equation*}
			which says that $1$ is the greatest element.
			\end{mdframed}
			Finally, to axiomatize \emph{semi-HAs} we require the existence of exponentials for every $\phi$, $\psi$ as follows:
			
			\begin{mdframed}
			\begin{mathpar}
			  \infer{\phi \times  (\psi^{\phi})\leq\psi}{
				} 
				\and
				\infer{\chi\leq\psi^{\phi}}{\phi\times\chi\leq\psi}
			\end{mathpar}
			\end{mdframed}
			
			In addition, given two \emph{semi-HAs}, we are interested in order preserving 
			functions (functors) $F$ that also preserve products and exponentials: 
			\begin{mdframed}
				\begin{enumerate}
				\item{$F(\phi \times\psi) = F(\phi)\times(F(\psi))$} 
				\item{$F(\psi^{\phi}}) = F(\psi)^{F(\phi)}$}
				\end{enumerate}
			\end{mdframed}
			
			For the order theoretic models of  Jcalc $^{-}$ the following structures (triplets) 
			are of interest. We define a $Jcalc$-triplet as follows:
			\begin{mdframed}
				\textbf{Definition}
				
			\begin{enumerate}
			\item A semi-Hayting algebra $HA$
			\item A partial order $J$
			\item A function order preserving function $F$ from $HA$ to $J$ s.t.
			\begin{enumerate}
				\item The image $F(HA)$ forms a semi-Heyting Algebra
				\item $F$ preserves products and exponentials
			\end{enumerate}
			\end{enumerate}
			\end{mdframed}
			We are going to utilize the following definition: 
			\begin{mdframed}
				\textbf{Definition}
				Given two partial orders $(K,\le_{K})$, $(L,\le_{L})$ and a function ($F: K\rightarrow L$) 
				we can define the algebra of $F$-points  $(F:K \rightarrow L,\le_{F:K\rightarrow L})$
				where:
				\begin{enumerate}
					\item Elements of $F:K\rightarrow L$ are  pairs of the form $\langle k,Fk \rangle$
					\item $\langle k_1,Fk_1 \rangle \le \langle k_2, Fk_2\rangle$ \textit{iff}  $k_1\le_{K}k_2$ and $Fk_1\le_{L}Fk_2$ 
				\end{enumerate}
			\end{mdframed}

			\begin{theorem}
				For any triplet $(K,L,F)$ the algebra of $F$-points is a partial order.
			\end{theorem}

			Given a $Jcalc$-triplet there is an induced $F$-point algebra:
			\begin{mdframed}
				\textbf{Definition}
				Given a $J$-triplet we define the algebra  $\Box^{F}HA$ as the induced $F$-point algebra.
			\end{mdframed}
			By the definitions, the $\Box^{F}HA$ point algebra has the following properties:
			\begin{enumerate}
				\item Elements are pairs $\langle A, FA\rangle$ (name them $\Box^{F}A$) where $A\in HA$ and $FA$ its image
				\item For every two elements $\Box^{F}A$, $\Box^{F}B$:

				$\Box^{F}A \le_\Box^{F}B \Box^{F}B \text{iff} A\le{HA}B \text{and} FA\le_J FB $ 
				
				\item It is a Heyting algebra with:
				\begin{itemize}
					\item Elements of the form $\Box^F (A \times B)$ forming products 
					(name them $\Box^F A\times \Box^F B$)
					\item Elements of the form $\Box^F ( B^{A})$ 
					forming exponentials (name them $\Box^FB^{\Box^F{A}}$)
				\end{itemize}
			\end{enumerate}
			The last property is not obvious so we will sketch the proof:
			\begin{theorem}[$\Box^{F}HA$ is Heyting]
			\end{theorem}
			\begin{proof}
				For any two elements $\Box^F A$, $\Box^F B$, the element $\Box^F (A \times B)$ forms their product since,
				$A \times B\le A$ in $HA$ and $F(A\times B)=FA\times FB \le FA$ and thusly, 
				$\Box^F (A \times B)\le \Box^FA$.
				Analogously, $\Box^F (A \times B)\le Box^FA$. 
				In addition we need to show that, for any $\Box^F C$ s.t. $\Box^F C\le \Box^F A $ and 
				$\Box^F C\le \Box^F B $ we get $\Box^F C\le\Box^F(A\times B)$. 
				By the definition for any such  $\Box^F C$ we have $C\le A\times B$
				and $FC\le F(A\times B)$ which imply that $\Box^F C\le\Box^F(A\times B)$

				To show that  $\Box^F ( B^{A})$ is the  exponential of $\Box^FA$, $\Box^FB$, 
				we have to show, first that $\Box^F ( B^{A})\times\Box^F A\le \Box^F B$.
				By definition $\Box^F ( B^{A})\times\Box^F A :=\Box^F( B^{A}\times A)$. Also
				for the underlying exponentials we have $ B^{A}\times A\le B$ and 
				$F(B^{A}\times A)= FB^{FA}\times FA \le FB$ which by definition of $\Box^F HA$
				gives   $\Box^F(B^{A}\times A)\le \Box^F(B)$ and hence, by definition, $\Box^F ( B^{A})\times \Box^F A\le \Box^F B$.

				In addition we have to show that $\Box^F ( B^{A})$ is the greatest element with the previous property.
				Consider any other $\Box^F C$ s.t. $\Box^F C\times\Box^F A\le \Box^F B$, by definitions, then, $C\times A\le B$
				 and $FC\times FA\le FB$. By the definitions of the underlying exponentials we get $C\le B^A$ and 
				 $FC \le FA^FB =F(B^A)$. Thusly, $\Box^F C\le \Box^F(B^A)$  
			\end{proof}


			Given a $Jcalc$-triplet we can define a $Jcalc$-algebra:
			\begin{mdframed}
				\textbf{Definition}
				Given a $J$-triplet we define the corresponding $Jcalc$ algebra as the union of the underlying relations
				of   $HA$, $F(HA)$, $\Box^{F}HA$
			\end{mdframed}
			
				\begin{theorem}\label{thm:cmpha}[Soundness and completness]
				$\Gamma\vdash_{Jcal}\phi \true$ iff for any \vocab{Jcalc Algebra} $JC$ ($HA,F,J$)we have 
				$\Gamma^+\leq\phi^{*}$ where $*$ is  defined as 
				the extension of any map of atomic $\prop$s to elements of $HA$ and $(+)$ 
				is defined inductively on the length of $\Gamma$ as follows
				\begin{alignat*}{2}
					(\top)* &&\quad= \top_J\\
					(A\wedge B \in Prop_0)*  &&\quad = & A*\times_{HA}B*\\
					(A\supset B \in Prop_0)*  &&\quad = & B*^{A*}\\
					(\llbracket A\rrbracket)* && \quad = & FA*\\
					(\Box A)* &&\quad = & \Box^F A* \\
					(\Box A\supset \Box B)*  &&\quad = & \Box^F B*^{\Box^F A}\\
					(\Box A\wedge\Box B)*  &&\quad = & \Box^F A*\times{\Box^F B}\\
				\end{alignat*}
				
				\begin{alignat*}{2}
				  \circ^+  &&\quad = & \quad\top\\
				  \dagger^+ &&\quad = & \quad\llbracket\top\rrbracket\\
				  (\Gamma,\phi\in \sf {Prop_0})^+&&\quad = &\quad
				  \Gamma^+\times_{HA}\phi* \\
				  (\llbracket \Gamma\rrbracket,\llbracket\phi\rrbracket \in {\sf \llbracket Prop_0\rrbracket})^+&&\quad = &\quad
				  \Gamma^+\times_{J}F\phi*\\
				  (\Box\Gamma, \Box\phi \in {\sf Prop_1})^+&&\quad = &\quad
				  \Gamma^+\times_{\Box^{F}HA}\Box^F (\phi)*\\
				\end{alignat*}
				\end{theorem}
			\begin{proof}
				To prove soundness we got by induction on the derivations. For the ${\sf Prop_0}$
				the proof is well-known. For the ${\sf Prop_1}$ part of the calculus, 
			\end{proof}
			To prove soundness and compl
			\begin{theorem}[Soundness and Completeness]
				Define a syntatic
			\end{theorem}
		\begin{theorem}
		\end{theorem}
