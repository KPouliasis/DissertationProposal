\chapter{Justification Logic}\label{jllogic}
In this chapter I will give an overview of \ac {JL}. 
I will emphasize LP, 
the very first logic of justification, and its deep relation with 
\ac{IPL}. My scaffolding will be based upon \cite{Art01BSL},~\cite{Art95TR}  
that reflect this relation. Beforehand, I will allow for a more general discussion on 
\ac{JL} following \cite{sep-logic-justification} and other 
relevant papers.

%It is well known that the provability predicate can be axiomatized using a modality \cite{citeulike:214701}, \cite{ArtBek05HPL}. The Logic of Proofs {\sf LP} \cite{Art94APAL} goes further and provides explicit proof terms (\textit{proof polynomials}) to inhabit judgments on validity. By translating reasoning in Intuitionistic Propositional Calculus ({\sf IPC}) to classical proofs, {\sf LP} obtains classical semantics for {\sf IPC} through a modality (inducing a {\sf BHK} semantics).

\section{A bird's eye view}
According to \cite{sep-logic-justification}\say {Justification logics are epistemic logics which allow knowledge and belief modalities to be ``unfolded'' into justification terms.}
 More specifically, in \ac{JL} the modality in question is witnessed by a reason and propositions of the kind $\Box\phi$ become $t:\phi$ that reads ``$\phi$ is justified by reason t". Witnesses in \ac{JL} have structure and operations. Different choices of operators result in logics that explicate different modalities 
 ({\sf {$K$, $T$, $S4$, $S5$}}). 
 In general, there is an infinite family of justification logics.
 For our purposes, and in addition to type theoretic approaches to logic, \ac{JL} reveals a computational content for \emph{validity} in classical terms. As we will see following~\cite{artemov97un}, \ac{JL} and especially its {\sf $S4$} counterpart \ac{LP}, can provide a unified classical \emph{semantics} for type theoretic 
 formulations of intuitionistic logic. In addition, following \cite{Artemov2007a, DBLP:journals/entcs/PouliasisP14}, JL mechanics can be viewed type
 theoretically to provide for modal typed systems that enrich computational
 type theories with \say{semantical} notions such as explicit reflection and modular
 binding.
 

\section{Minimal Justification Logic $J_0$}~\label{min:jo}
To permit for an account of reasons, the logic is enriched with an extra sort ($j$) for justifications. The sort of propositions is then enriched with propositions of the kind $j:\phi$ with $\phi$ being a proposition. Here is the abstract syntax:

\begin{mdframed}
\begin{align*}
j := &s_i|\ C_i| j_1*j_2| j_2 + j_2\\
 \phi:=& P_i|\ \bot|\ \phi_1\wedge\phi_2|\ \phi_1\vee\phi_2| \ \phi_2\supset\phi_2|\ \neg\phi|\ j:\phi   
\end{align*}
\end{mdframed}
Constants $C_i$ are symbols that can be assigned to logic axioms that are assumed to be necessary. Weaker justification logics exist without any assignment of constants (empty \emph{constant specifications}) or with partial constant specifications. Nevertheless, in order for the  \emph{rule of necessitation} to be admissible each axiom instance of the underlying propositional logic has to be assigned a constant. We will be coming back to this topic in later sections. Symbols $s_i$ stand for variables.

A Hilbert--style axiomatization of $J_0$ is given below. Its components are Hilbert's axioms for propositional logic together with two basic rules for justification: \emph{applicativity} and \emph{concatenation}. Concatenation internalizes weakening of proofs.
\begin{mdframed}
\sf{Propositional Axioms}
\begin{align*}
&\sf{P1}.  \vdash \phi\supset(\psi\supset\phi)\\
& \sf{P2}. \vdash (\phi\supset(\psi\supset\chi))\supset((\phi\supset\psi)\supset(\phi\supset\chi))\\
& \sf{P3}. \vdash \phi\supset\psi\supset\phi\wedge\psi\\
&\sf{P4}. \vdash \phi\supset\psi\supset\psi\wedge\phi\\
&\sf{P5}.  \vdash \phi\supset\phi\vee\psi\\
&\sf{P6}. \vdash \psi\supset\phi\vee\psi\\
&\sf{P7}. \vdash (\phi\supset\psi)\supset(\neg\psi\supset\neg\phi)\\
\end{align*}
\end{mdframed}


\begin{mdframed}
\sf{Justification Axioms}
\begin{align*}
& \sf{Times}. \vdash j:(\phi\supset\psi)\supset(j':\phi\supset j*j':\psi)\\
& \sf{PlusL}. \vdash j:\phi\supset(j+j':\phi)\\
& \sf{PlusR}. \vdash j:\phi\supset(j'+j:\phi)\\
\end{align*}
\end{mdframed}
The rule of the system is \emph{Modus Ponens}. 
\begin{mdframed}
\sf{Modus Ponens}
\begin{mathpar}
\inferrule*[right=\sf{MP}]{{\phi\supset\psi}\\{\phi}}{\psi}
\end{mathpar}
\end{mdframed}
For the rule of necessitation to be admissible, we need necessitation of axioms to be admissible. 
For that reason a constant specification is required. 
We focus here on axiomatically appropriate constant specification $\sf{CS}$ because of its relation to combinatorial calculi. 
An axiomatization of axiomatically appropriate $\sf{CS}$ given below. 
Elements of $\sf{CS}$ are pairs $(C,\phi)$ of polymorphic 
(i.e. \textit{parametrized} over propositions) constants and propositions. The $!$ operator relates to
a concept of internalization of justified statements, i.e. witnessing the existence of a justified statement with
a (higher order) justification. We demand that all justified axiomatic schemes can be internalized.
\begin{mdframed} 
\sf{Axiomatic CS}
\begin{mathpar}
\inferrule*[right=$\sf{C_1}$]  { }   {\vdash({{\sf C_1}[\phi,\psi],\  \phi\rightarrow(\psi\rightarrow\phi))\in \sf{CS}}}
\and
\inferrule*[right=$\sf{C_2}$]  { }   {\vdash({{\sf C_2}[\phi,\psi,\chi],\ (\phi\supset(\psi\supset\chi))\supset((\phi\supset\psi)\supset(\phi\supset\chi))  ) )\in \sf{CS}}}
\and
\inferrule*[right=$\sf{C_3}$]  { }   {\vdash(  {{\sf C_3}[\phi,\psi],\  \phi\supset\psi\supset\phi\wedge\psi  )\in \sf{CS}}}
\and
\inferrule*[right=$\sf{C_4}$]  { }   {\vdash(  {{\sf C_4}[\phi,\psi],\  \phi\supset\psi\supset\psi\wedge\phi  )\in \sf{CS}}}
\and
\inferrule*[right=$\sf{C_5}$]  { }   {\vdash(  {{\sf C_5}[\phi,\psi],\  \phi\supset\phi\vee\psi  )\in \sf{CS}}}
\and
\inferrule*[right=$\sf{C_6}$]  { }   {\vdash(  {{\sf C_6}[\phi,\psi],\  \psi\supset\phi\vee\psi  )\in \sf{CS}}}
\and
\inferrule*[right=$\sf{C_7}$]  { }   {\vdash(  {{\sf C_7}[\phi,\psi],\  (\phi\supset\psi)\supset(\neg\psi\supset\neg\phi)\in \sf{CS}}}
\and
\inferrule*[right=$\sf{C_8}$] { } {\vdash(  {{\sf C_8}[\phi,\psi,j,j'],\  j:(\phi\supset\psi) \supset (j':\phi \supset j*j':\psi  ))\in \sf{CS}}}
\and
\inferrule*[right=$C!$]
{\vdash ( {\sf C},\phi)\in \sf{CS}} {\vdash(\sf{C!} ,\ \sf{C}:\phi )\in \sf{CS}}
\end{mathpar}
\end{mdframed}

Finally we require reflection on $\sf{CS}$: 
\begin{mdframed}
\sf{Specification Reflection}
\begin{mathpar}
\inferrule*[right=CSR]
{\vdash ( {\sf C},\phi)\in \sf{CS}} {\vdash\sf{C}:\phi}
\end{mathpar}

\end{mdframed}

The system can be given a Natural Deduction formulation \textit{\`a la} \ac{IPL} since the following theorem holds:
\begin{mdframed}
\textbf{Deduction Theorem}
For any set of propositional assumptions $\Gamma$, \\ $\Gamma,\phi\vdash\psi$ implies $\Gamma\vdash\phi\supset\psi$ 
\end{mdframed}
\section{Epistemic motivation} 
 \ac{JL} as an epistemic logic departs from previous traditions of logic of knowledge based on  universality judgments. From \cite{sep-logic-justification}
\begin{quotation}
The modal approach to the logic of knowledge is, in a sense, built around the universal quantifier: X is known in a situation if X is true in all situations indistinguishable from that one. Justifications, on the other hand, bring an existential quantifier into the picture: X is known in a situation if there exists a justification for X in that situation
\end{quotation}

This fresh approach to the epistemic tradition has been utilized to solve many problems in formal epistemology (see \cite{Artemov2014-ARTLOA}). We sketch 
here the solution to the famous \textit{Red barn problem} 
that, also, provides a pedagogical example 
on how deduction in the system works.

The red barn problem can be stated as follows:
\begin{quote}
Suppose I am driving through a neighborhood in which, unbeknownst to me, papier-mâché barns are scattered, and I see that the object in front of me is a barn. Because I have barn-before-me percepts, I believe that the object in front of me is a barn. Our intuitions suggest that I fail to know barn. But now suppose that the neighborhood has no fake red barns, and I also notice that the object in front of me is red, so I know a red barn is there. This juxtaposition, being a red barn, which I know, entails there being a barn, which I do not, “is an embarrassment”
\end{quote}

The red barn example can be represented in a system of modal logic where $\Box \phi$ represents knowledge of $\phi$ that, in contrast to the the justified approach, is forgetful with respect to reasons. The formalization and the accompanying problem go as follows:

\begin{enumerate}
    \item $\Box B$, ‘I believe that the object in front of me is a red barn’.
    \item  $\Box(B \wedge R),$ ‘I believe that the object in front of me is a red barn’. 

At the metalevel, 2 is actually knowledge, whereas by the problem description, 1 is not knowledge.

   \item $\Box(B\wedge R\supset B)$, a knowledge assertion of a logical axiom.
	\end{enumerate}
\begin{quote}	
Within this formalization, it appears that epistemic closure in its modal form (2) is violated:line 2, $\Box(B \wedge R )$, and line 3, $(B \wedge R \supset B)$ are cases of knowledge whereas $\Box B$ (line 1) is not knowledge. The modal language here does not seem to help resolving this issue.
\end{quote}
Of course, one can resolve this by introducing a second modality(e.g. for \say{I believe that}). But then similar problems can occur (e.g. by adding a third modality read as `it should be'). Indexing of modalities with reasons solves this problem in its generality: by permitting the applicative closure only on reasons of the same sort one can overcome this defect.
\begin{enumerate}
   \item $u:B$, ‘$u$ is a reason to believe that the object in front of me is a barn’;
   \item $v:(B \wedge R)$, ‘$v$ is a reason to believe that the object in front of me is a red barn’;
    \item $a:(B \wedge R \supset B)$, because of logical awareness.


 On the metalevel, the problem description states that 2 and 3 are cases of knowledge, and not merely belief, whereas 1 is belief which is not knowledge. Here is how the formal reasoning goes:

    \item $a:(B \wedge R \supset B)\supset(v:(B \wedge R) \supset  a*v:B)$, by {\sf Times} 
    \item $v:(B \wedge R) \supset a*v:B$, from 3 and 4, by propositional logic;
    $a*v:B$, from 2 and 5, by propositional logic.

\end{enumerate}
\section{Proof theoretic view}
	In  Chapter ~\ref{intui} we gave an analytic account of the \ac{BHK} principles of  constructive proofs. In the paper ``Eine Interpretation des
	intuitionistischen Aussagenkalküls ", G\"oedel gave a classical provability interpretation of \ac{BHK} using the modal system ${\sf S4}$.
	
	The standard axiomatization of ${\sf S4}$ is given below: 
	\begin{mdframed}
	The system $\sf{S4}$ 
	\begin{align*}
	& \sf{P1-P7}&\\
	 \sf{K}.& \vdash \Box(\phi\supset\psi)\supset(\Box\phi\supset\Box\psi)\\
	 \sf{T}.& \vdash \Box\phi\supset\phi\\
	\sf{4}. &\vdash \Box\phi\supset\Box\Box\phi\\
	\end{align*}


	\sf{Modus Ponens}
	\begin{mathpar}
	\inferrule*[right=\sf{MP}]{{\phi\supset\psi}\\{\phi}}{\psi}
	\end{mathpar}
	\end{mdframed}
	
	G\"oedel's result can be summarized in the following theorem:

	\begin{mdframed}
		\textbf{G\"odel-Tarski Translation of Intuitionistic Logic}
	$$\Gamma\vdash_{\sf IPL}\phi \rightharpoondown \Gamma\vdash_{\sf S4}\operatorname{tr}(\phi)$$
		where $\operatorname{tr}(\phi)$ is obtained by $\phi$ by $\Box$-ing its subformulas. 
	\end{mdframed}

After this result the state of the project of a classical interpretation of \ac{BHK} semantics was as follows:
     IPC $\hookrightarrow$ S4 $\hookrightarrow$ ? $\hookrightarrow$ CLASSICAL PROOFS. Filling the missing part was 
     the motivation behind \ac{LP}, the first Justification Logic.

	
\section{The Logic of Proofs}
An axiomatization of \ac{LP} with axiomatically appropriate constant specification as defined in \ref{min:jo} can be given
as follows:
	\begin{mdframed}
	The system $\sf{LP}$ 
	\begin{align*}
	& \sf{P1-P7}\\
	 \sf{Times}.& \vdash j:(\phi\supset\psi)\supset(j':\phi\supset j*j':\psi)&\\
	  \sf{PlusL}.& \vdash j:\phi\supset(j+j':\phi)&\\
	  \sf{PlusR}.& \vdash j:\phi\supset(j'+j:\phi)&\\
	 \sf{T}.& \vdash j:\phi\supset\phi\\
	\sf{4}. &\vdash j:\phi\supset (j!:j:\phi)
		\end{align*}
	
   \end{mdframed}
\section{Metatheoretic Results}
The \emph{Deduction Theorem} holds for \ac{LP}
\begin{mdframed}
\textbf{Deduction Theorem}
Any deduction of the kind $\Gamma,\phi\vdash\psi$ implies $\Gamma\vdash \phi\supset\psi$.
\end{mdframed}
Also, the lifting property can be obtained:
\begin{mdframed}
\textbf{Lifting Lemma}

Any deduction of the kind $\vec{j}:\Gamma,\Delta\vdash\phi$ implies $\vec{j}:\Gamma,\vec{s}:\Delta\vdash j'(\vec{j},\vec{s}):\phi$ where $\vec{j}$ is a vector metavariables to be substituted for arbitrary polynomials and $\vec{s}$ is a vector of (object) variables. 

\end{mdframed}
In addition, \ac{LP} is the forgetful projection of {\sf $S4$}. 
More specifically, consider a formula of \ac{LP} $\phi$ and the transformation $F_\Box(\phi)$ that replaces all subformulae of $\phi$ of the kind $j:\phi'$ with $\Box\phi'$. The following theorem holds:
\begin{mdframed}
\textbf{Forgetful Projection Property}

$\Gamma\vdash_{\sf LP}\phi$ implies $\Gamma\vdash_{\sf{ S4}}F_{\Box}(\phi)$ 

\end{mdframed}
  The inverse also holds as the realization theorem says. Before introducing the realization procedure we give a motivating example.
  
  \begin{mdframed}
  \textbf{Example}: Realization of $\vdash_{\sf S4}\Box\phi\vee\Box\psi\supset\Box(\phi\vee\psi)$
  \begin{enumerate}
     \item $\phi\supset\phi\vee\psi$, $\psi\supset\phi\vee\psi$ Prop. Axioms;
     \item $C:(\phi\supset \phi\vee \psi)$, $C':(\psi\supset \phi\vee \psi)$ From {\sf CS} rules.
      \item $s:\phi\supset C*s:\phi\vee\psi$, From 1,2 and{\sf Times} and {\sf MP}
      \item $t:\psi\supset C'*t:\phi\vee\psi$, Similarly 
      \item $C*s:\phi\vee\psi\supset(C*s+C'*t):\phi\vee\psi$ and $C'*t:\phi\vee\psi\supset(C*s+C'*t):\phi\vee\psi$, From {\sf Rplus, Lplus} 
      \item $s:\phi\supset(C*s+C'*t):\phi\vee\psi$, 
      From 3,5 by Propositional Logic.
      \item $t:\psi\supset(C*s+C'*t):\phi\vee\psi$,
      From 4,5 by Propositional Logic.
      \item  $s:\phi\vee t:\psi\supset(C*s+C'*t):\phi\vee\psi$,
      From 6,7 and Propositional Logic.
      
  \end{enumerate}
  
  \end{mdframed}

\subsection{Realization}							
The realization theorem gives an algorithmic process for transforming cut-free deductions in{\sf S4} to {\sf LP}.
By an {\sf LP}-realization of a modal formula $\phi$ we mean an assignment of proof polynomials to
all occurrences of the modality in$ \phi$. Let $\phi^{r}$  be the image of $\phi$ under a realization $r$. 

The polarity of $\Box$s in a formula is relevant in realizations.  
We define positive
and negative occurrences of modality in a formula and a sequent.
\begin{mdframed}

\textbf{$\Box$ Polarities}
\begin{enumerate}
\item The indicated occurrence of $\Box$ in $\Box\phi$ is of positive polarity; 
\item any occurrence of $\Box$ in the subformula $\phi$ of $\psi\supset\phi$,$\psi\wedge\phi$, $\phi\wedge\psi$, $\psi\vee\phi$, $\phi\vee\psi$, $\Box\phi$,$\Gamma\Rightarrow\Delta,\phi$ -- we will be defining $\Rightarrow$ momentarily -- has the same polarity as the same occurrence of $\Box$ in $\phi$.
\item any occurrence of $\Box$ in the subformula $\phi$ of $\neg\phi$,  $\phi\supset\psi$, $\Gamma,\phi\Rightarrow\Delta$, has polarity opposite to the polarity of the very same  occurrence of $\Box$ in $\phi$.

\end{enumerate}

\end{mdframed}


Next we give a a cut-free sequent formulation of ${\sf S4}$ (reference) with sequents $\Gamma\vdash\Delta$, where $\Gamma$ and $\Delta$ are finite multisets of modal formulas. The left hand multisets are to be read conjunctively and the right hand ones disjunctively. The rules are the rules given below together with the typical structural ones.
\begin{mdframed}

\begin{mathpar}

\inferrule*[right=Refl]  { }   {\Gamma,\phi\vdash\phi,\Delta}
\and
\inferrule*[right=$\neg$L]  {\Gamma\vdash\phi,\Delta}   {\Gamma,\neg\phi\vdash\Delta}
\and
\inferrule*[right=$\neg$R]  {\phi,\Gamma\vdash\Delta}   {\Gamma\vdash\neg\phi,\Delta}
\and
\inferrule*[right=$\wedge$L]  {\Gamma,\phi,\psi\vdash\Delta}    {\Gamma,\phi\wedge\psi\vdash\Delta}
\and
\inferrule*[right=$\wedge$L]  {{\Gamma\vdash\phi,\Delta}\\   {\Gamma\vdash\psi,\Delta}}{\Gamma\vdash\phi\wedge\psi,\Delta}
\and
\inferrule*[right=$\vee$L]  {{\Gamma,\phi\vdash\Delta}\\   {\Gamma,\psi\vdash\Delta}}{\Gamma,\phi\vee\psi\vdash\Delta}
\and
\inferrule*[right=$\vee$R]  {\Gamma\vdash\phi,\psi,\Delta}   {\Gamma\vdash\phi\vee\psi, \Delta}
\and
\inferrule*[right=$\supset$L]  {{\Gamma\vdash\phi,\Delta}\\   {\Gamma, \psi\vdash\Delta}}{\Gamma,\phi\supset\psi\vdash\Delta}
\and
\inferrule*[right=$\supset$R]  {{\Gamma,\phi\vdash\psi,\Delta}}{\Gamma\vdash\phi\supset\psi, \Delta}
\end{mathpar}
\begin{mathpar}
\inferrule*[right=$\Box$L]  {\phi,\Gamma\vdash\Delta} {\Box\phi,\Gamma\vdash\Delta}
\and
\inferrule*[right=$\Box$R]  {\Box\Gamma\vdash\phi}  {\Box\Gamma\vdash\Box\phi}
\end{mathpar}

\end{mdframed}

Relevant in the realization proof is the sequent formulation of {\sf LP}, the system {\sf LPG} which enjoys the cut-elimination property resulting in the system ${\sf LPG^{-}}$. The rules relevant to justifications are given below.  

\begin{mdframed}

\begin{mathpar}

\inferrule*[right=$:$L]  {\Gamma,\phi\vdash\phi,\Delta}   {\Gamma,t:\phi\vdash\phi,\Delta}
\and
\inferrule*[right=$!$R]  {\Gamma\vdash t:\phi,\Delta}   {\Gamma\vdash!t:t:\phi,\Delta}
\and
\inferrule*[right=$+$L]  {\Gamma\vdash t:\phi,\Delta}   {\Gamma\vdash(t+s):\phi,\Delta}
\and
\inferrule*[right=$+$R]  {\Gamma\vdash t:\phi,\Delta}    {\Gamma\vdash (s+t):\phi,\Delta}
\and
\inferrule*[right=$*$R]  {{\Gamma\vdash s:\phi\supset\psi,\Delta}\\   {\Gamma\vdash t:\phi,\Delta}}{\Gamma\vdash s*t:\psi,\Delta}
\and
\inferrule*[right=$c$R]  {\Gamma\vdash\phi,\Delta}   {\Gamma\vdash c:\phi,\Delta}

\end{mathpar}

\end{mdframed}
Utilizing the previous systems the realization theorem shows:
\begin{mdframed}
\textbf{Realization Theorem}
If $\Gamma\vdash_{\sf S4} \phi$ then there is a \emph{normal} realization s.t.  $\Gamma\vdash_{\sf LP} \phi^{r}$. By normal we mean a realization for which all occurrences of $\Box$ are realized by proof variables and the corresponding constant specification is injective.
   
\end{mdframed}
\subsection{Kripke -  Fitting Semantics}
In this section I will be discussing Kripke -- Fitting Semantics\cite{fitting2005logic} for Justification Logic $\sf{J_0} + CS$ very briefly.


  A  possible world justification logic model for the system ${\sf J_0 + CS}$  is a structure $M=\langle G, R, E, V\rangle$. $\langle G,R\rangle$ is a standard $K$ frame, where $G$ is a set of possible worlds and $R$ is a binary relation on it. $V$ is a mapping from propositional variables to subsets of $G$, specifying atomic truth at possible worlds.
$E$ is an evidence function that maps pairs of justification terms and formulas to sets of worlds.  

Given such a model, we define the $\models$ relation as follows:
\begin{mdframed}
$\forall \Gamma\in G$
\begin{itemize}
    

    \item[] $M, \Gamma \models P$ iff $\Gamma \in V(P)$ for $P$ a propositional letter
    \item  It is not the case that $M, \Gamma \models \bot$
     \item   $M, \Gamma \models \phi \supset \psi$ iff it is not the case that $M, \Gamma \models \phi$ or$ M, \Gamma\models Y$
     \item  $M,\Gamma \models (j:\phi)$ if and only if $\Gamma \in E(j,\phi)$ and, $\forall \Delta\in G$ with $\Gamma R \Delta$, we have that $M,\Delta\models \phi$.
     
\end{itemize}
\end{mdframed}


 The following conditions on evidence functions are assumed:
\begin{itemize}
\item[] $E(j,\phi\supset\psi)\cap E(j',\phi) \subseteq E(j*j',\psi)$
\item[]  $E(j,\phi) \cup E(j',\phi)\subseteq E(j + j',\phi) $
\end{itemize}
   


Finally, the Constant Specification CS should be taken into account. Recall that constants are intended to represent reasons for basic assumptions that are accepted outright. A model $M = \langle G,R,E,V\rangle$ meets Constant Specification CS provided: if $(C,\phi) \in CS$ then $E(c,\phi) = G$.

Typical, soundness and completeness results can be shown for such models. They can also be extended for all other justification logics.