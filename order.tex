\chapter{Order theoretic semantics}
\label{jcalcsem}
This chapter started by introducing 
mappings between deductive systems and 
motivating the reading of necessity as ``double-proof under a map''.
As a result, it is unsurprising that the calculus is amenable to order theoretic semantics.
We present them in this chapter.

\section{semi-Heyting algebras}
In order to progress we first define the notion of a 
\vocab{semi-Heyting  Algebra (semi-HA)}. 
To define semi-HA we need the notion of a \vocab{(meet) semi-lattice}.
  

\begin{mdframed}
\textbf{Definition:}
A \textit{(meet) semi-lattice} is a non-empty \emph{partial order} (i.e. reflexive, antisymmetric and transitive) 
with finite meets.
\end{mdframed}
In addition, we define \emph{meet semi-lattice} as follows: 
\begin{mdframed}
\textbf{Definition:}
A \textit{bounded (meet) semi-lattice} $(L,\le)$ is a (meet) 
semi-lattice that additionally has 
a \emph{greatest element} (we name it $1$), which satisfies

$x \le 1$ for every $x$ in $L$
\end{mdframed}
Finally, we can define \emph{semi-HA}:

\begin{mdframed}
\textbf{Definition:}
A \textit{semi-HA} is a bounded (meet) semi-lattice $(L,\le, 1)$ 
s.t. for every $a,b\in L$ there exists an \textit{exponential} 
(we name it $a\rightarrow b$) 
with the properties: 
\begin{enumerate}
\item $a\rightarrow b\times a\le b $
\item $a\rightarrow b$ is the greatest such element
\end{enumerate}
\end{mdframed}

\section{$Jcalc$-triplets}
Given two \emph{semi-HAs}, we are 
interested in order preserving 
functions (functors) $F$ that also preserve products and exponentials: 
\begin{mdframed}
    \textbf{Definition}
A function $F$ between two (semi)-HAs ($HA_1$, $HA_2$) is order preserving
and commutes with top, products and exponentials \emph{iff} for every 
$\phi,\psi \in HA_1$
    \begin{enumerate}
    \item $\phi\le_{HA_1}\psi\Rightarrow F\phi\le_{HA_2}F\psi$
    \item $F\top_{HA_1} = \top_{HA_2}$ 
    \item{$F(\phi \times\psi) = F(\phi)\times(F(\psi)$} 
    \item $F(\phi\rightarrow \psi) = F(\psi)\rightarrow F(\phi)$
    \end{enumerate}
\end{mdframed}

For the order theoretic models of  Jcalc $^{-}$ the following structures (triplets) 
are of interest. We define a $Jcalc$-triplet as follows:
\begin{mdframed}
    \textbf{Definition}
A \emph{$Jcalc$-triplet} is 
    
\begin{enumerate}
\item A semi-Heyting algebra $HA$
\item A partial order $J$
\item An order preserving function $F$ from $HA$ to $J$ s.t.
\begin{enumerate}
    \item The image $F(HA)$ forms a semi-Heyting Algebra
    \item $F$ preserves top, products and exponentials
\end{enumerate}
\end{enumerate}
\end{mdframed}
We are going to utilize the following definition: 
\begin{mdframed}
    \textbf{Definition}
    Given two partial orders $(K,\le_{K})$, $(L,\le_{L})$ and a function ($F: K\rightarrow L$) 
    we can define the algebra of $F$-points  $(F:K \rightarrow L,\le_{F:K\rightarrow L})$
    where:
    \begin{enumerate}
        \item Elements of $F:K\rightarrow L$ are  pairs of the form $\langle k,Fk \rangle$
        \item $\langle k_1,Fk_1 \rangle \le_F \langle k_2, Fk_2\rangle$ \textit{iff}  $k_1\le_{K}k_2$ and $Fk_1\le_{L}Fk_2$ 
    \end{enumerate}
\end{mdframed}

\begin{theorem}
    For any triplet $(K, L, F)$ of $HA$s with an order 
    preserving function $F:K\rightarrow L$
    the algebra of $F$-points is a partial order.
\end{theorem}
\begin{proof}
    It is trivial to show that the algebra of $F$-points ``inherits'' reflexivity, 
    transitivity and antisymmetry from the underlying algebras.
\end{proof}

Given a $Jcalc$-triplet there is an induced $F$-point algebra:
\begin{mdframed}
    \textbf{Definition}
    Given a $Jcalc$-triplet we define the algebra  $\Box^{F}HA$ as the induced $F$-point algebra.
\end{mdframed}
By the definitions, the $\Box^{F}HA$ point algebra has the following properties:
\begin{enumerate}
    \item Elements are pairs $\langle A, FA\rangle$ (name them $\Box^{F}A$) where $A\in HA$ and $FA$ its image
    \item For every two elements $\Box^{F}A$, $\Box^{F}B$:

    $\Box^{F}A \le\Box^{F}B \text{ \textit{iff} } A\le_{HA}B 
    \text{ \textit{and} } FA\le_J FB $ 
    
    \item It is a Heyting algebra with:
    \begin{itemize}
        \item $\Box^F\top := \langle \top_{HA}, F\top_{HA}=\top_J \rangle$ 
        \item Elements of the form $\Box^F (A \times B)$ 
        forming products (we name them $\Box^F A\times \Box^F B$)
        \item Elements of the form $\Box^F ( A\rightarrow B)$ 
        forming exponentials (name them $\Box^F{A}\rightarrow \Box^F B$)
    \end{itemize}
\end{enumerate}
The last property is not obvious so we will sketch the proof. We will be
omitting indexes in the $\le$ relations since they can be trivially inferred:
\begin{theorem}[$\Box^{F}HA$ is Heyting]
\end{theorem}
\begin{proof}
    $\Box^F \top$ is a top element since for any $A\in HA$, 
    $A\le \top$ and thusly $FA\le F\top=\top_J$ and thus by definition
    $\Box^F A\le\Box^F\top $ for any $\Box^F A$.

    For any two elements $\Box^F A$, $\Box^F B$, 
    the element $\Box^F (A \times B)$ forms their product since,
    $A \times B\le A$ in $HA$ and $F(A\times B)=FA\times FB \le FA$
    in $J$, and thusly, 
    $\Box^F (A \times B)\le \Box^FA$ (in $\Box^F HA$).
    Analogously, $\Box^F (A \times B)\le \Box^FA$.

    In addition, $\Box^F (A \times B)$ is the product we need to show 
    that is the greatest element with the previous property.
    I.e. for any $\Box^F C$ s.t. $\Box^F C\le \Box^F A $ and 
    $\Box^F C\le \Box^F B $ we get $\Box^F C\le\Box^F(A\times B)$. 
    By the definition for any such  $\Box^F C$ we have $C\le A\times B$
    and $FC\le F(A\times B)$ which imply that $\Box^F C\le\Box^F(A\times B)$

    To show that  $\Box^F(A\rightarrow B)$ is the  
    exponential of $\Box^FA$, $\Box^FB$, 
    we have to show, first that 
    $\Box^F ( A\rightarrow B)\times\Box^F A\le \Box^F B$.
    By the $\Box^F HA$ product definition 
    $\Box^F ( A\rightarrow B)\times\Box^F A :=\Box^F( (A\rightarrow B)\times A)$. 
    Also
    by the underlying exponentials we have $ (A\rightarrow B)\times A\le B$ 
    and  $F((A\rightarrow B)\times A)= (FA\rightarrow FB)\times FA \le FB$
    which by definition of $\Box^F HA$ gives   
    $\Box^F((A\rightarrow B)\times A)\le \Box^F B$ 
    and hence, by definition, 
    $\Box^F ( A\rightarrow B)\times \Box^F A\le \Box^F B$.

    In addition we have to show that 
    $\Box^F ( A\rightarrow B)$ is the 
    greatest element with the previous property.
    Consider any other $\Box^F C$ s.t. $\Box^F C\times\Box^F A\le \Box^F B$, 
    by definitions of $\Box^F$ and its products then, $C\times A\le B$
     and $FC\times FA\le FB$.
      By the definitions of the underlying exponentials 
     we get $C\le A\rightarrow B$ and 
     $FC \le FA\rightarrow FB =F(A\rightarrow B)$. 
     And again by definition of $\Box^F HA$, $\Box^F C\le \Box^F(A\rightarrow B)$.  
\end{proof}

\section{ $Jcalc$-algebras: Soundness and completeness}
Given a $Jcalc$-triplet we can define a $Jcalc$-algebra:
\begin{mdframed}
    \textbf{Definition}
    Given a $J$-triplet we define the corresponding $Jcalc$
     algebra as the union of the underlying relations
    of   $HA$, $F(HA)$, $\Box^{F}HA$
\end{mdframed}
    \begin{theorem}\label{thm:cmpjtriplet}
        \textbf{Soundness and completeness}\\
    $\Gamma\vdash_{Jcal}\phi$ iff for any \vocab{Jcalc Algebra}  $JC$ ($HA,F,J$)
    and any $*$ map that extends  a map of atomic $\prop$s ($p_i$) to elements of $HA$ 
    with properties shown below and $(+)$ 
    is defined inductively on the length of $\Gamma$ as shown below
    then $\Gamma^+\leq\phi^{*}$.
    \begin{alignat*}{2}
        (\top)* &&\quad= & \quad\top\\
        (A\wedge B \in Prop_0)*  &&\quad = & \quad  A*\times_{HA}B*\\
        (A\supset B \in Prop_0)*  &&\quad = & \quad A*\rightarrow_{HA} B*\\
        (\llbracket A\rrbracket)* && \quad = & \quad F(A*)\\					
        (\Box A)* &&\quad = & \quad\Box^F A* \\
        (\Box A\supset \Box B)*  &&\quad = & \quad\Box^F A* \rightarrow{\Box^F B*}\\
        (\Box A\wedge\Box B)*  &&\quad = & \quad\Box^F A*\times{\Box^F B*}\\
    \end{alignat*}
    
    \begin{alignat*}{2}
      \circ^+  &&\quad = & \quad\top\\
      \dagger^+ &&\quad = & \quad\llbracket\top\rrbracket\\
      (\Box\circ)^+ &&\quad = & \quad\Box^F \top\\
      (\Gamma,\phi\in \sf {Prop_0})^+&&\quad = &\quad
      \Gamma^+\times_{HA}\phi* \\
      (\llbracket \Gamma\rrbracket,\llbracket\phi\rrbracket \in {\sf \llbracket Prop_0\rrbracket})^+&&\quad = &\quad
      \Gamma^+\times_{J}F\phi*\\
      (\Box\Gamma, \Box\phi \in {\sf Prop_1})^+&&\quad = &\quad
      \Gamma^+\times_{\Box^{F}HA}\Box^F (\phi)*\\
    \end{alignat*}
    \end{theorem}
\begin{proof}
    To prove soundness we go by induction on the derivations. 
    For the ${\sf Prop_0}$ fragment
    the proof is well-known from intuitionistic logic semantics 
    ($\Gamma\in{\sf Prop_0}\vdash \phi\in {\sf Prop_0}
    \Rightarrow \Gamma^+\le_{HA}\phi *$)
    . 
    For the ${\sf \llbracket Prop_0\rrbracket}$ part of the calculus again by induction.
    Reflection, of contexts is trivial. For the axiomatic cases, it is a well known result
    that in any Heyting algebra (and thus in $F(HA)$ of any Jcalc algebra) 
    elements of the shape of the axiomatic combinators are equivalent 
    (equiprovable) to $\top$.
    For example in any Heyting algebra  we have $\top\le A\rightarrow(B\rightarrow A)$ 
    (using the definition of exponentials twice from the fact $\top\times A\times B\le A$), 
    the modus ponens case is handled by induction and the properties of $F$ 
    (preserving exponentials). Hence,
    $(\llbracket\Gamma\rrbracket)^{+}+\leq(\llbracket\phi\rrbracket)^{*}$ for any deduction in $\llbracket{\sf Prop_0} \rrbracket$.


    The interesting part of the proof is the $\Box$ rule which we present again here for readability:
    \begin{mdframed}[nobreak=true, frametitle={\footnotesize Judgments on Necessity with 
        $\Gamma\in {\sf Prop_1} \text{,{\ \sf length}}(\Gamma)=i\text{,\ }
        \ 1\le k\le i  \text{\ and, }\Gamma^{\prime},A, A_k,  B\in {\sf Prop_0}$ }]
    \mbox{\footnotesize
        \begin{mathpar}
            
            \inferrule*[right=$I_{\Box B}E^{\vec{x},\vec{s}}_{\Box A_1\ldots \Box A_i}$]{{(\forall  A_i \in \Gamma'. \ \Turn {\Gamma}{\Box  A_i})}\\{\Turn {\Gamma'} { B}}\\{\Turn {\llbracket \Gamma' \rrbracket} {\llbracket  B\rrbracket} }} {\Turn {\Gamma}\Box  B}

        \end{mathpar}}
    \end{mdframed}
    
    By the induction hypothesis we have $(\Gamma^{'})^{+}\le B^{*}$
    and $(\llbracket\Gamma^{'}\rrbracket)^{+}\le FB^{*}$ or equivalently by the properties of $F$
    $F((\Gamma^{'})^{+})\le F(B^{*})$ which gives $\Box^F{(\Gamma^{'})^{+}}\le \Box^F B$
    Additionally from induction hypothesis, for every $A_i$ in $\Gamma^{+}\Box^F A_i$
    and by the product definition $\Gamma^{+}\le(\Gamma^{'})^{+}$ and thus
    $\Gamma^{+}\le\Box^F B*$.

    For the inverse we create a Lindenbaum construction. We sketch the construction:
    \begin{itemize}
        \item Create a preorder \textit{pre-}$HA$ with underlying set (isomorphic to) ${\sf Prop_0}$
        \item Define $\phi\le\psi$ {\textit{iff}} $\phi\vdash\psi$
        \item Define the equivalence relation $\phi\equiv\psi$ {\textit{iff}} 
        $\phi\le\psi$ and $\psi\le\phi$
        \item Define the quotient \textit{pre-}$HA/_{\equiv}$
        \item Show that it is a Heyting Algebra with products the elements of
        of shape $\phi\wedge\psi$, top $\top$ and exponentials $\phi\supset\psi$
        \item Repeat the construction for the syntactic elements of 
        $\llbracket Prop_0 \rrbracket$, with 
        $\llbracket\phi\rrbracket\le\llbracket\psi\rrbracket$ 
        \textit{iff} $\llbracket\phi\rrbracket\vdash\llbracket\psi\rrbracket$
        show that it is a Heyting algebra $J$.
        \item Repeat the construction for the syntactic elements of
         ${\sf Prop_1}$ and $\Box \phi\le\Box\psi$ \textit{iff} $\Box\phi\vdash\Box\psi$
        \item Show that the union of the three relations above forms a 
        Jcalc-algebra with $F:= A\mapsto \llbracket A\rrbracket$. I.e. show that:
        \begin{itemize}
            \item $A\vdash B\Rightarrow \llbracket A\rrbracket\vdash \llbracket B\rrbracket$
            (Holds by the lifting lemma)
            \item $\llbracket A\wedge B\rrbracket $ is product (trivial) and 
            $\llbracket A\supset B \rrbracket$ (trivial given the deduction theorem which we have shown)
            in $J$
            \item $\Box A\vdash\Box B$ \textit{iff} $A\vdash B$ and $\llbracket A\rrbracket \vdash \llbracket B\rrbracket$
            Easy by induction on the derivations and usage of the lifting lemma.
        \end{itemize}
    \end{itemize}
    Now assume that $\Gamma^{+}\le\phi*$ for any Jcalc algebra and 
    mapping $*$; consider  $*$ to extend  the identity mapping  
    into the (free) J-calc algebra defined above. It is trivial to see that in JCalc
    $\Gamma\vdash\phi$.
\end{proof}

