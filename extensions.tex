\chapter{Notes on extending the calculus}

In this chapter we will make an informal case about the scalability
of the presented system. We will sketch how the calculus can
quite easily be extended in different ways 
and make a case that such extensions are of interest from 
the trinitarian (logic/ type theory/ category theory)
point of view.

\section{Extending on higher order modal types}
We saw in Chapter $6$ how the calculus corresponds to Jcalc algebras
which are essentially pairs of Heyting algebras under an order preserving function.
The points of such functions correspond to $\Box$ed types.

\begin{tikzpicture}[description/.style={fill=white,inner sep=2pt}]
    \matrix (m) [matrix of math nodes, row sep=3em,
    column sep=2.5em, text height=1.5ex, text depth=0.25ex]
    { A & & FA  \\
    B & & FB \\ };
    %\draw[double,double distance=5pt] (m-1-1) – (m-1-3);
    \path[-,font=\scriptsize]
    (m-1-1) edge[double,thick,double distance=5pt] node[auto] {$ \Box A $} (m-1-3)
    (m-2-1) edge[double,thick,double distance=5pt] node[auto] {$ \Box B $} (m-2-3)
    (m-1-1) edge node[auto] {$ \vdash $} (m-2-1)
    (m-1-3) edge node[auto] {$ \vdash $} (m-2-3);
    \end{tikzpicture}

This structure is easily
extensible to account for $\Box$ed types of higher degree. Instead of a pair of 
Heyting algebras we could have a stack of Heyting algebras 
related with order preserving
functions as shown in the schema.

\begin{tikzpicture}[description/.style={fill=white,inner sep=2pt}]
    \matrix (m) [matrix of math nodes, row sep=3em,
    column sep=2.5em, text height=1.5ex, text depth=0.25ex]
    { A & & F_0A & & F_1F_0A \\
      B & & F_0B & & F_1F_0B\\ };
    %\draw[double,double distance=5pt] (m-1-1) – (m-1-3);
    \path[-,font=\scriptsize]
    (m-1-1) edge[double,thick,double distance=5pt] node[auto] {$ \Box A $} (m-1-3)
    (m-2-1) edge[double,thick,double distance=5pt] node[auto] {$ \Box B $} (m-2-3)
    (m-1-3) edge[double,thick,double distance=5pt] node[auto] {$ \Box FA $} (m-1-5)
    (m-2-3) edge[double,thick,double distance=5pt] node[auto] {$ \Box FB $} (m-2-5)
    (m-1-1) edge[bend left = 30] node[auto] {$ \Box\Box A $} (m-1-5)
    (m-2-1) edge[bend right = 30] node[auto] {$ \Box\Box B $} (m-2-5)
    (m-1-1) edge node[auto] {$ \vdash $} (m-2-1)
    (m-1-3) edge node[auto] {$ \vdash $} (m-2-3)
    (m-1-5) edge node[auto] {$ \vdash $} (m-2-5);
    \end{tikzpicture}


In a nutshell, instead 
of one function symbol $\llbracket \rrbracket$ 
the system can 
be axiomatized to reason about chains of 
composable (provability) preserving functions. 
The modifications required are minor to obtain such a system.
Instead of a function symbol $\llbracket \rrbracket$ we have $F_0, F_1\ldots F_j$ and we 
define for any formula $\Box A\in {\sf Prop_i}$, 
$F_i\Box A:= \Box F_i A$ (
    and similarly, lifting over the connectives: 
$F_i(\Box A\supset \Box B):= \Box F_i A\supset\Box F_i B$)
the rule can then be written:

\begin{mdframed}[nobreak=true, frametitle={\footnotesize 
    Judgments on Necessity with $\Gamma\in {\sf Prop_i} \text{,{\ \sf length}}(\Gamma)=i\text{,\ }
    \ 1\le k\le j  \text{\ and, }\Gamma^{\prime},A, A_k,  B\in {\sf Prop_{i-1}}$ }]
\mbox{\footnotesize
    \begin{mathpar}
        \inferrule*[right=$I_{\Box B}E^{\vec{x},\vec{s}}_{\Box A_1\ldots \Box A_i}$]
        {{(\forall  A_i \in \Gamma'. \ \Turn {\Gamma}{\Box  A_i})}\\{\Turn {\Gamma'} { B}}\\{\Turn {F_i\Gamma'} {F_i  B} }} {\Turn {\Gamma}\Box  B}
        
    \end{mathpar}}
\end{mdframed}




\section{From order theory to category theory}

There is a classic passage from orders to categories, 
which corresponds to the passage of provability to 
proof relevance. In addition, order preserving functions become 
functors in the categorical scenario. But functors behave functionally
on terms (i.e. preserve proof equalities, or essentially, normalization principles of the cut elimination process).
To account, hence, for 
a categorical semantics of the system one has to account for equality
in the higher level of the system (i.e. on justifications).

This idea is actually not foreign in the litarature that explores the relation
between lambda calculus and (typed) combinatory logic and, in addition,
it is tempting to introduce equality between justifications so that one
could more accurately describe computational phenomena arising when 
a language interacts with another language (or, its own modules).

Generalizing, from the order theoretic semantics, we would expect a system
in which $\llbracket\rrbracket$ would correspond to functors 
(preserving the connectives and hence, $\beta\eta$ equalities). 
We are expecting the extension of Jcalc with rules for $\beta\eta$ equalities 
on the level of justifications to fit exactly the bill. The required 
equalities that correspond to $\beta$ reduction are standard in presentations of 
the $SK$ calculus. 
\begin{mdframed}
    \begin{mathpar}
        \inferrule*[right=$K_\beta$]{}{\Delta\vdash K*J_1*J_2 = J_1}
        \and
        \inferrule*[right=$S_\beta$]{} {\Delta\vdash S*J_1*J_2*J_3 = J_1*J_3(J_2*J_3)}
    \end{mathpar}
    \end{mdframed}
Rules corresponding to $\eta$ equality are less obvious 
but can be found in the excellent \cite{}. Similarly, equalities regarding (\wedge)
combinators can be introduced. 
We also need an axiom  that expressed 
that $\llbracket\rrbracket$ (or, $F_0$ in an extended system) is functorial (functional)
on terms
\begin{mdframed}
    \begin{mathpar}
        \inferrule*[right=$K_\beta$]{\Gamma\vdash M =_{\beta\eta} N :A}{\Delta\vdash \llbracket M\rrbracket=\rrbracket N\rrbracket :
        \llbracket A \rrbracket}    
    \end{mathpar}
\end{mdframed}
We expect that such a system is a full axiomatization  of functors
that preserve products and exponentials (of cartesian closed categories) 
Which are common in the litarature in proofs of the (syntactical) 
embedding of the lambda calculus into combinatory logic.



\section {Factivity and adjunctions}
Having an understanding of the system in order theoretic/ categorical terms 
helps thinking about extensions. In any category theoretic
textbook, the next ``tighter" relation between categories is that of an adjunction.
Interestingly, the notion of adjunction, is also central in the relation
between classical and intuitionistic proofs (the two parts of the adjunction
are inclusion, and double negation translation).  The notion of adjunction, plays
an important role in functional programming theory as the backbone 
of monadic computation. 

We would expect that the view of necessity as 
relating two proof systems could be extended to cover the notion of an adjunction.
In order theoretic terms an adjunction between two preorders $\mathcal{C},\mathcal{D}$
is a pair of orderd preserving functions $L:\mathcal{D}\rightarrow C$, $R:\mathcal{C}\rightarrow{D}$
such that there is an isomorphism:
$\forall d\in  D, c\in C Ld\ le c \longleftrightarrow d\le R c$. A logically interesting example of 
an adjunction is that between intuitionistic proofs and classical proofs where the $L$ adjunct is inclusion
and the right adjunct is double negation translation and we have:
\begin{mathpar}
    \inferrule*[right=$\downarrow\uparrow$]{I(\Gamma)\vdash\phi}{\Gamma\vdash\neg\neg \phi} 
\end{mathpar}
To axiomatize such notions in $Jcalc$ one should add another function symbol to 
correspond to $R$ and add the rule:  
\begin{mathpar}
    \inferrule*[right=$R$]{\llbracket\Gamma\rrbracket\vdash_J j:\phi}
    {\Gamma\vdash return{j}:R\phi} 
\end{mathpar}

This is enough to obtain a generalized notion of Factivity that is:
\begin{mathpar}
    \inferrule*[right=$R$]{\llbracket\Gamma\rrbracket\vdash M:\Box\phi}
    {\Gamma\vdash let \_&s=M in return (s):R\llbracket\phi\rrbracket} 
\end{mathpar}
Such an extension gives the standard factivity rule if $R\llbracket\rrbracket=id$.
Such extensions can capture phenomena in which a language is giving control to another language
to perform a computation externally but then it retains to control to calculate the continuation
of the client program. Refine example above.





