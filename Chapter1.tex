\chapter{Intuitionistic Logic}\label{intui}
\section{Intuitionism}\label{sec:intrui}
In this and the subsequent chapter, I will be presenting foundational work in the intersection of \emph{Intuitionistic Logic} and \emph{Type Theory}. 
The presentation is scaffolding following Robert Harper's lecture videos in \emph{Homotopy Type Theory}~\cite{HarperHOTT} and the accompanying notes by students of the class~\cite{HOTTNotes1}. I will often deviate to standard textbooks in the field ~\cite{Barendregt1984-BARTLC},~\cite{citeulike:993095},~\cite{Pierce:2002:TPL:509043} to present further important results. 
\subsection{A bird's eye view}  
In a nutshell, \emph{Intuitionistic mathematics}  is a program in foundations 
of mathematics  that extends \emph{Brouwer's program}~\cite{brouwer1975collected}.
Brouwer, in an almost Kantian fashion, viewed mathematical reasoning as a human faculty 
and mathematics as a language of the ``creative subject"
aiming to communicate mathematical concepts. 
The concept of \emph{algorithm} as a step--by--step constructive process is brought in the 
foreground in Brouwer's program. 
As a result, intuitionistic theories are amenable to 
computational interpretations. We are  using the terms intutionistic 
and \emph{constructive} interchangeably.  

For the purposes of this paper, the main diverging point of
 Brouwer's program, 
later explicated by Heyting~\cite{heyting1966intuitionism} 
and Kolmogorov~\cite{kolmogorov1925principe}~\cite{artemov2004kolmogorov}, lies in the treatment of proofs. In contrast to classical approaches to foundations 
that treat proof objects as external to theories, 
the constructive approach treats proofs 
as the fundamental forms of construction and hence, 
as first class citizens. 
As a result, the constructive view of logic draws heavily 
from proof theory
and Gentzen's developments~\cite{gentzen1970collected}. 
For the reader interested also in the philosophical implications  
of constructive foundations and \emph{antirealism}, 
Dummet's treatment is a classic in the field~\cite{dummett2000elements}.    
 


It has to be emphasized that proofs in the intuitionistic approach 
 are treated as stand--alone and are not bound to formal systems 
(i.e the notion of proof \textit{precedes} that of a 
formal system). 
It is necessary, hence, to draw a distinction between
 the notion of 
 \emph{proof as construction} and the 
  notion of \emph{proof in a formal system} 
 ~\cite{Harper2013,Harper2012}.

A \emph{formal proof}
is a proof given in a fixed formal system and it is constructed
by the application of the rules in that system, recursively. 
Formal proofs can, thus, be viewed as strings or g\"{o}delizations of textual derivation in some fixed system. 

As Harper puts it \say{Although every formal proof (in a specific system)
is also a proof (assuming soundness of the system) the converse is not true}.  
This conforms with G\"{o}del's Incompleteness Theorem, which precisely states that there
exist true propositions (with a proof in \emph{some} formal system), but for which 
there cannot be given a formal proof within the formal system in question. 
This \emph{openness} of the nature of proofs is necessary for a foundational 
treatment of proofs that respects  G\"{o}delian phenomena.

Following the same line of thought, and adopting the doctrine of \emph{proof relevance} 
for obtaining true judgments, leads to another main difference of the constructive 
approach and the classical one i.e the (default) absence of the 
\emph{law of excluded middle}. 


\section{IPL}
\ac{IPL} can be viewed as ``the logic of \emph{proof relevance}" conforming with the intuitionistic view described in \ref{sec:intrui}. To judge a fact as \emph{true} one may provide a \emph{proof}
appropriate of the fact. \emph{Proofs} can be synthesized to obtain 
proofs for more complex facts (\emph{introduction rules}) and consumed
 to provide proofs relevant for other  facts (\emph{elimination rules}). The importance of the interplay between introduction and elimination rules was developed by Gentzen. 
 A discussion on the meaning of the logical connectives that is prevalent in \emph{MLTT} can be found in \cite{martin1996meanings} 
Following the presentation style  by   Martin-L\"{o}f we split the notions of \emph{judgment} and \emph{proposition}. We have two main kinds of judgments:
\begin{itemize}
\item  \emph{Judgments} that are logical arguments about the truth(or, equivalently, proof) of a \emph{proposition}. They might, optionally, involve assumptions on the truth (or, equivalently, proof) of other propositions. We might call these \emph{logical judgments}. 

\item Judgments on \emph{propositionality} or typeability. \emph{Propositions} are the \emph{subjects}  of \emph{logical judgments}. If something is judged to be a proposition then it belongs to the universe of discourse and can be mentioned in \emph{logical judgments}. 
\end{itemize} 
In addition, since a \emph{logical judgment} might involve a set $\Gamma$ of assumptions (or a \emph{context}), it is convenient to add a third kind of judgment of the form $\Gamma \context$ 
Thus, in \ac{IPL}, we get the judgments $\phi\ \in \prop$, $\phi \true$ and $\Gamma \context$:
\begin{alignat*}{2}
  \phi \in \prop &&\quad& \text{$
  \phi$ is a (well-formed) proposition} \\
  \phi\  \true &&& \text{\begin{tabular}[t]{@{}l@{}}
                Proposition $\phi$ is true \\
                i.e., has a proof.
              \end{tabular}}\\
  \Gamma \context &&\quad& \text{$\Gamma$ is a (well-formed) context of assumptions} \\
\end{alignat*}

The natural deduction system of \ac{IPL} is given below:


\begin{mdframed}
\textbf{Prop Formation}
\begin{mathpar}
\inferrule*[right=Atom] { } {P_i \in {\sf Prop}}
\and
\inferrule*[right=Top] { } {\top \in {\sf Prop}}
\and
\inferrule*[right=Bottom] { } {\bot \in {\sf Prop}}
\and
\inferrule*[right=Arr] {{\phi_1 \in {\sf Prop }}\\ {\phi_2 \in {\sf Prop}}} {\phi_1\supset\phi_2\in {\sf Prop}}
\and
\inferrule*[right=Conj] {{\phi_1 \in {\sf Prop }}\\ {\phi_2 \in {\sf Prop}}} {\phi_1\wedge\phi_2\in {\sf Prop}}
\and
\inferrule*[right=Disj] {{\phi_1 \in {\sf Prop }}\\ {\phi_2 \in {\sf Prop}}} {\phi_1\vee\phi_2\in {\sf Prop}}

\end{mathpar}
\end{mdframed}

\begin{mdframed}
\textbf{Context Formation}
\begin{mathpar}
\inferrule*[right=Nil] { } {{\sf \circ}\  \context}
\and
\inferrule*[right=$\Gamma$-Ext] {{\Gamma\ } {\sf \context}  \\ {\phi \in {\sf Prop}}} {{\Gamma, \phi \true} \ \context}
\end{mathpar}
\end{mdframed}

\begin{mdframed}
\textbf{Context Reflection}
\begin{mathpar}
\inferrule*[right=$\Gamma$-Refl] { {\Gamma}\  {\sf \context}\\ {\phi \true \in \Gamma}}{\Turnsi {\Gamma} {\phi \true}}
\end{mathpar}
\end{mdframed}

\begin{mdframed}
\textbf{Top Introduction -- Bottom Elimination}
\begin{mathpar}
\inferrule*[right=$\top$I] { } {\Turnsi {\Gamma} { \top \true}}
\and
\inferrule*[right=$\bot$E] {\Turnsi {\Gamma} {\bot \true} } {\Turnsi {\Gamma} {  \phi \true}}
\end{mathpar}
\end{mdframed}

\begin{mdframed}
\textbf{Implication Introduction and Elimination}
\begin{mathpar}
\inferrule*[right=$\supset$I] {\Turnsi {\Gamma, \phi_1 \true} {\phi_2 \true}} {\Turnsi {\Gamma} { \phi_1\supset \phi_2 \true}}
\and
\inferrule*[right=$\supset$E] {\Turnsi {\Gamma} {\phi_1\supset\phi_2 \true}\\{\Turnsi {\Gamma} {\phi_1 \true}}} {\Turnsi {\Gamma} {  \phi_2 \true}}
\end{mathpar}
\end{mdframed}
\begin{mdframed}
\textbf{Conjunction Introduction and Elimination}
\begin{mathpar}
\inferrule*[right=$\wedge$I] {\Turnsi {\Gamma} {\phi_1\true}\\{\Turnsi {\Gamma} {\phi_2 \true}}} {\Turnsi {\Gamma} {  \phi_1 \wedge\phi_2 \true}}
\end{mathpar}
\begin{mathpar}
\inferrule*[right=$\wedge$El] {\Turnsi {\Gamma} {\phi_1\wedge\phi_2 \true}} {\Turnsi {\Gamma} {  \phi_1\true}}
\and
\inferrule*[right=$\wedge$Er] {\Turnsi {\Gamma} {\phi_1\wedge\phi_2 \true}} {\Turnsi {\Gamma} {  \phi_2\true}}
\end{mathpar}
\end{mdframed}
\begin{mdframed}
\textbf{Disjunction Introduction and Elimination}
\begin{mathpar}
\inferrule*[right=$\vee$Il] {\Turnsi {\Gamma} {\phi_1 \true}} {\Turnsi {\Gamma} {  \phi_1\vee\phi_2\true}}
\and
\inferrule*[right=$\vee$Ir] {\Turnsi {\Gamma} {\phi_2 \true}} {\Turnsi {\Gamma} {  \phi_1\vee\phi_2\true}}
\end{mathpar}


\begin{mathpar}
\inferrule*[right=$\vee$E] 
{ {\Turnsi {\Gamma} {  \phi_1\vee\phi_2\true}}\\
{\Turnsi {\Gamma,\phi_1 \true} {\phi \true}}\\
{\Turnsi {\Gamma,\phi_2 \true} {\phi \true}}
}
 {\Turnsi {\Gamma} {\phi \true}}
\end{mathpar}

\end{mdframed}


\subsection{Basic Properties of Intuitionistic Entailment}
\label{ssec:entail}
			\begin{mdframed}
			\textbf{Reflexivity}
			    
				\begin{mathpar}
			   \inferrule*[] 
			    { }
			    {\Turnsi {\Gamma,\phi\true} {\phi \true}} 
				\end{mathpar}
		  \end{mdframed}

		\begin{mdframed}
		\textbf{Transitivity}
			\begin{mathpar}
					   \inferrule*[] 
					    {\Turnsi {\Gamma} {\psi \true}\\ {\Turnsi {\Gamma,\psi\true}{\phi\true}}}
					    {\Turnsi {\Gamma,\phi\true} {\phi \true}} 
						\end{mathpar}
				\end{mdframed}
   			
   			\begin{mdframed}
   			\textbf{Contraction}
   						\begin{mathpar}
   								   \inferrule*[] 
   								    {\Turnsi {\Gamma,\phi\true,\phi \true
   								   } {\psi \true}} {\Turnsi {\Gamma,\phi \true}{\psi\true}}
   								    
   									\end{mathpar}
   			\end{mdframed}
		\begin{mdframed}{Exchange}
					\begin{mathpar}
							   \inferrule*[] 
							    {\Turnsi {\Gamma
							   } {\phi \true}} {\Turnsi {\operatorname{\pi}(\Gamma)}{\phi\true}}
							    \end{mathpar}
                  Where $\pi(\Gamma)$ is a meta-symbol standing for any permutation of $\Gamma$.
                \end{mdframed}

\section{Order Theoretic Semantics: \vocab{Heyting Algebras}}\label{ha:ax}
\vocab{IPL} viewed order theoretically gives rise to a \vocab{Heyting  Algebra(HA)}. 
To define \vocab{HA} we need the notion of a \emph{lattice}.
 For our purposes we define it as follows\footnote{One can take a lattice being a partial order. The same results hold with slight modifications.}: 
  

\begin{mdframed}
\textbf{Definition:}
A \textit{lattice} is a non-empty \emph{pre--order} with finite meets and joins.
\end{mdframed}
In addition, we define \emph{bounded lattice} as follows: 
\begin{mdframed}
\textbf{Definition:}
A \textit{bounded lattice} $(L,\le)$ is a lattice that additionally has a greatest element 1 and a least element 0, which satisfy

$0\le x \le 1$ for every $x$ in $L$
\end{mdframed}
Finally, we can define \emph{HA}:

\begin{mdframed}
\textbf{Definition:}
A \textit{HA} is a bounded lattice $(L,\le,0,1)$ s.t. for every $a,b\in L$ there exists an $x$ (we name it $a\rightarrow b$) with the properties: 
\begin{enumerate}
\item $a\wedge x\le b $
\item $x$ is the greatest such element
\end{enumerate}
\end{mdframed}
\subsubsection{Axiomatization of HAs}
We can axiomatize the meet (i.e. greatest lower bound)($\wedge$) of $\phi,\psi$ for any  lower bound $\chi$.
\begin{mdframed}
\begin{mathpar}
  \infer{\phi \conj \psi \leq \phi}{
    }
  \and
  \infer{\phi \conj \psi \leq \psi}{
    } 
\end{mathpar}
\begin{equation*}
  \infer{\chi \leq \phi \conj \psi}{
    \chi \leq \phi & \chi \leq \psi} 
\end{equation*}
\end{mdframed}

We can axiomatize the join ($\vee$)(i.e. the least upper bound) of $\phi,\psi$ for any upper bound $\chi$ as follows .
\begin{mdframed}
\begin{mathpar}
  \infer{\phi  \leq \phi\vee \psi}{
    }
  \and
  \infer{\psi \leq \phi \vee \psi}{
    } 
\end{mathpar}
\begin{equation*}
  \infer{\phi \vee \psi \leq \chi}{
    \phi \leq \chi & \psi \leq \chi} 
\end{equation*}
\end{mdframed}
We can axiomatize the existence of a greatest element as follows:
\begin{mdframed}
\begin{equation*}
  \infer{\chi \leq 1}{
    } 
\end{equation*}
which says that $1$ is the greatest element.
\end{mdframed}

We can axiomatize the existence of a least element as follows:
\begin{mdframed}
\begin{equation*}
  \infer{0 \leq \chi}{
    } 
\end{equation*}
which says that $0$ is the least element.
\end{mdframed}
Finally, to axiomatize \emph{HAs} we require the existence of exponentials for every $\phi$, $\psi$ as follows:

\begin{mdframed}
\begin{mathpar}
 

  \infer{\phi \wedge  (\phi\supset \psi)\leq\psi}{
    } 
    \and
    \infer{\chi\leq\phi\supset\psi}{\phi\wedge\chi\leq\psi}
\end{mathpar}
\end{mdframed}

\subsubsection{Soundness and Completeness}

\begin{mdframed}
\begin{theorem}\label{thm:cmpha}
$\Gamma\vdash_{IPL} \phi \true$ iff for any \vocab{Heyting Algebra} $H$ we have $\Gamma^+\leq\phi^{*}$ where $*$ is  defined as the lifting of any map of $\prop$s to elements of $H$ and $(+)$ is defined inductively on the length of $\Gamma$ as follows
\begin{alignat*}{2}
  {\sf \circ}^+  &&\quad = & \quad\top\\
  (\Gamma,\phi)^+&&\quad = &\quad
  \Gamma^+\wedge\phi* \
\end{alignat*}
\end{theorem}
\end{mdframed}
