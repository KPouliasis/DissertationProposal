\chapter{Intuitionistic Logic}\label{intui}
\section{Intuitionism}\label{sec:intrui}
In this Chapter, I will be presenting foundational work in the intersection of \emph{Intuitionistic Logic} and \emph{Type Theory}. The presentation is scaffolding following Prof. Robert Harper's lecture videos in \emph{Homotopy Type Theory}~\cite{HarperHOTT} and the accompanying notes by students of the class~\cite{HOTTNotes1}. I will often diverge to standard textbooks in the field ~\cite{Barendregt1984-BARTLC},~\cite{girard1989proofs},~\cite{Pierce:2002:TPL:509043} to present further important results. 
\subsection{A bird's eye view}  
In a nutshell, \emph{Intuitionistic mathematics}  is a program in foundations of mathematics  that extends \emph{Brouwer's program}~\cite{brouwer1975collected}.
Brouwer, in an almost Kantian fashion, viewed mathematical reasoning as a human faculty and mathematics as a language of the ``creative subject'
aiming to communicate mathematical concepts. The concept of \emph{algorithm} as a step--by--step constructive process is brought in the foreground in Brouwer's program. As a result, intuitionistic theories adhere to computational interpretations. In the following I will be using the terms intutionistic and \emph{constructive} interchangeably.  

 For the purposes of this paper, the main diverging point of Brouwer's program, later explicated by Heyting ~\cite{heyting1966intuitionism} and Kolmogorov~\cite{kolmogorov1925principe}~\cite{artemov2004kolmogorov}, lies in the treatment of proofs. In contrast to classical approaches to foundations that treat proof objects as external to theories, the constructive approach treats proofs as the
 fundamental forms of construction and hence, as first class citizens. As a result, the constructive view of logic draws heavily from proof theory and Gentzen's developments~\cite{gentzen1970collected}. For the reader interested also in the philosophical implications  of constructive foundations and \emph{antirealism}, Dummet's treatment is a classic in the field~\cite{dummett2000elements}.    
 


It has to be emphasized that proofs in the intuitionistic approach  are treated as stand--alone and are not bound to formal systems (i.e the notion of proof \textit{precedes} that of a formal system). It is necessary, hence, to draw a distinction  between the notion of \emph{proof as construction} and the typical notion of \emph{proof in a formal system} ~\cite{Harper2013,Harper2012}.

  A formal proof
is a proof given in a fixed formal system, such as the axiomatic theory of sets, and arises
from the application of the inductively defined rules in that system. Formal proofs can, thus, be viewed as g\"{o}delizations of textual derivation in some fixed system. 

Although every formal proof (in a specific system)
 is also a proof (assuming soundness of the system) the converse is not true.  This conforms with 
 G\"{o}del's Incompleteness Theorem, which precisely states that there
exist true propositions (with a proof in \emph{some} formal system), but for which there cannot be given a formal proof in the formal system that is at stake. This \emph{openness} of the nature of proofs is necessary for a foundational treatment of proofs that respects  G\"{o}delian phenomena. This is often coined as ``Axiomatic Freedom"
of intuitionistic foundations.

Following the same line of thought, and adopting the doctrine of \emph{proof relevance} for obtaining true judgments, leads to another main difference of the constructive approach and the classical one i.e the (default) absence of the \emph{law of excluded middle}. Current developments in constructive foundations like \emph{Homotopy Type Theory} and in general systems that rely on \emph{Martin-L\"{o}f Type Theory}~\cite{martin1984intuitionistic} do not necessarily rule out \emph{LEM} but they might permit its usage locally, if needed, in a proof. 


\section{IPL}
\ac{IPL} can be viewed as ``the logic of \emph{proof relevance}" conforming with the intuitionistic view described in \ref{sec:intrui}. To judge a fact as \emph{true} one may provide a \emph{proof}
appropriate of the fact. \emph{Proofs} can be synthesized to obtain proofs for more complex facts (\emph{introduction rules}) and consumed to provide proofs relevant for other  facts (\emph{elimination rules}). The importance of the interplay between introduction and elimination rules was developed by Gentzen. A discussion on the meaning of the logical connectives that is prevalent in \emph{MLTT} can be found in \cite{martin1996meanings} 
Following the presentation style  by   Martin-L\"{o}f we split the notions of \emph{judgment} and \emph{proposition}. We have two main kinds of judgments:
\begin{itemize}
\item  \emph{Judgments} that are logical arguments about the truth(or, equivalently, proof) of a \emph{proposition}. They might, optionally, involve assumptions on the truth (or, equivalently, proof) of other propositions. We might call these \emph{logical judgments}. 

\item Judgments on \emph{propositionality} or typeability. \emph{Propositions} are the \emph{subjects}  of \emph{logical judgments}. If something is judged to be a proposition then it belongs to the universe of discourse and can be mentioned in \emph{logical judgments}. 
\end{itemize} 
In addition, since a \emph{logical judgment} might involve a set $\Gamma$ of assumptions (or a \emph{context}), it is convenient to add a third kind of judgment of the form $\Gamma \context$ 
Thus, in \ac{IPL}, we get the judgments $\phi\ \in \prop$, $\phi \true$ and $\Gamma \context$:
\begin{alignat*}{2}
  \phi \in \prop &&\quad& \text{$
  \phi$ is a (well-formed) proposition} \\
  \phi\  \true &&& \text{\begin{tabular}[t]{@{}l@{}}
                Proposition $\phi$ is true \\
                i.e., has a proof.
              \end{tabular}}\\
  \Gamma \context &&\quad& \text{$\Gamma$ is a (well-formed) context of assumptions} \\
\end{alignat*}

The natural deduction system of \ac{IPL} is given below:


\begin{mdframed}
\textbf{Prop Formation}
\begin{mathpar}
\inferrule*[right=Atom] { } {P_i \in {\sf Prop}}
\and
\inferrule*[right=Top] { } {\top \in {\sf Prop}}
\and
\inferrule*[right=Bottom] { } {\bot \in {\sf Prop}}
\and
\inferrule*[right=Arr] {{\phi_1 \in {\sf Prop }}\\ {\phi_2 \in {\sf Prop}}} {\phi_1\supset\phi_2\in {\sf Prop}}
\and
\inferrule*[right=Conj] {{\phi_1 \in {\sf Prop }}\\ {\phi_2 \in {\sf Prop}}} {\phi_1\wedge\phi_2\in {\sf Prop}}
\and
\inferrule*[right=Disj] {{\phi_1 \in {\sf Prop }}\\ {\phi_2 \in {\sf Prop}}} {\phi_1\vee\phi_2\in {\sf Prop}}

\end{mathpar}
\end{mdframed}

\begin{mdframed}
\textbf{Context Formation}
\begin{mathpar}
\inferrule*[right=Nil] { } {{\sf nil}\  \context}
\and
\inferrule*[right=$\Gamma$-Add] {{\Gamma\ } {\sf \context}  \\ {\phi \in {\sf Prop}}} {{\Gamma, \phi \true} \ \context}
\end{mathpar}
\end{mdframed}

\begin{mdframed}
\textbf{Context Reflection}
\begin{mathpar}
\inferrule*[right=$\Gamma$-Refl] { {\Gamma}\  {\sf \context}\\ {\phi \true \in \Gamma}}{\Turnsi {\Gamma} {\phi \true}}
\end{mathpar}
\end{mdframed}

\begin{mdframed}
\textbf{Top Introduction -- Bottom Elimination}
\begin{mathpar}
\inferrule*[right=$\top$I] { } {\Turnsi {\Gamma} { \top \true}}
\and
\inferrule*[right=$\bot$E] {\Turnsi {\Gamma} {\bot \true} } {\Turnsi {\Gamma} {  \phi \true}}
\end{mathpar}
\end{mdframed}

\begin{mdframed}
\textbf{Implication Introduction and Elimination}
\begin{mathpar}
\inferrule*[right=$\supset$I] {\Turnsi {\Gamma, \phi_1 \true} {\phi_2 \true}} {\Turnsi {\Gamma} { \phi_1\supset \phi_2 \true}}
\and
\inferrule*[right=$\supset$E] {\Turnsi {\Gamma} {\phi_1\supset\phi_2 \true}\\{\Turnsi {\Gamma} {\phi_1 \true}}} {\Turnsi {\Gamma} {  \phi_2 \true}}
\end{mathpar}
\end{mdframed}
\begin{mdframed}
\textbf{Conjunction Introduction and Elimination}
\begin{mathpar}
\inferrule*[right=$\wedge$I] {\Turnsi {\Gamma} {\phi_1\true}\\{\Turnsi {\Gamma} {\phi_2 \true}}} {\Turnsi {\Gamma} {  \phi_1 \wedge\phi_2 \true}}
\end{mathpar}
\begin{mathpar}
\inferrule*[right=$\wedge$El] {\Turnsi {\Gamma} {\phi_1\wedge\phi_2 \true}} {\Turnsi {\Gamma} {  \phi_1\true}}
\and
\inferrule*[right=$\wedge$Er] {\Turnsi {\Gamma} {\phi_1\wedge\phi_2 \true}} {\Turnsi {\Gamma} {  \phi_2\true}}
\end{mathpar}
\end{mdframed}
\begin{mdframed}
\textbf{Disjunction Introduction and Elimination}
\begin{mathpar}
\inferrule*[right=$\vee$Il] {\Turnsi {\Gamma} {\phi_1 \true}} {\Turnsi {\Gamma} {  \phi_1\vee\phi_2\true}}
\and
\inferrule*[right=$\vee$Ir] {\Turnsi {\Gamma} {\phi_2 \true}} {\Turnsi {\Gamma} {  \phi_1\vee\phi_2\true}}
\end{mathpar}


\begin{mathpar}
\inferrule*[right=$\vee$E] 
{ {\Turnsi {\Gamma} {  \phi_1\vee\phi_2\true}}\\
{\Turnsi {\Gamma,\phi_1 \true} {\phi \true}}\\
{\Turnsi {\Gamma,\phi_2 \true} {\phi \true}}
}
 {\Turnsi {\Gamma} {\phi \true}}
\end{mathpar}

\end{mdframed}


\subsection{Basic Properties of Intuitionistic Entailment}
\label{ssec:entail}
			\begin{mdframed}
			\textbf{Reflexivity}
			    
				\begin{mathpar}
			   \inferrule*[] 
			    { }
			    {\Turnsi {\Gamma,\phi\true} {\phi \true}} 
				\end{mathpar}
		  \end{mdframed}

		\begin{mdframed}
		\textbf{Transitivity}
			\begin{mathpar}
					   \inferrule*[] 
					    {\Turnsi {\Gamma} {\psi \true}\\ {\Turnsi {\Gamma,\psi\true}{\phi\true}}}
					    {\Turnsi {\Gamma,\phi\true} {\phi \true}} 
						\end{mathpar}
				\end{mdframed}
   			
   			\begin{mdframed}
   			\textbf{Contraction}
   						\begin{mathpar}
   								   \inferrule*[] 
   								    {\Turnsi {\Gamma,\phi\true,\phi \true
   								   } {\psi \true}} {\Turnsi {\Gamma,\phi \true}{\psi\true}}
   								    
   									\end{mathpar}
   			\end{mdframed}
		\begin{mdframed}{Exchange}
					\begin{mathpar}
							   \inferrule*[] 
							    {\Turnsi {\Gamma
							   } {\phi \true}} {\Turnsi {\operatorname{\pi}(\Gamma)}{\phi\true}}
							    
								\end{mathpar}
						\end{mdframed}

\section{Order Theoretic Semantics: \vocab{Hayting Algebras}}\label{ha:ax}
\vocab{IPL} viewed order theoretically gives rise to a \vocab{Hayting  Algebra(HA)}. To define \vocab{HA} we need the notion of a \emph{lattice}. For our purposes we define it as follows\footnote{One can take a lattice being a partial order. The same results hold with slight modifications.}: 
  

\begin{mdframed}
\textbf{Definition:}
A \textit{lattice} is a \emph{pre--order} with finite meets and joins.
\end{mdframed}
In addition, we define \emph{bounded lattice} as follows: 
\begin{mdframed}
\textbf{Definition:}
A \textit{bounded lattice} $(L,\le)$ is a lattice that additionally has a greatest element 1 and a least element 0, which satisfy

$0\le x \le 1$ for every $x$ in $L$
\end{mdframed}
Finally, we can define \emph{HA}:

\begin{mdframed}
\textbf{Definition:}
A \textit{HA} is a bounded lattice $(L,\le,0,1)$ s.t. for every $a,b\in L$ there exists an $x$ (we name it $a\rightarrow b$) with the properties: 
\begin{enumerate}
\item $a\wedge x\le b $
\item $x$ is the greatest such element
\end{enumerate}
\end{mdframed}
\subsubsection{Axiomatization of HAs}
We can axiomatize the meet (i.e. greatest lower bound)($\wedge$) of $\phi,\psi$ for any  lower bound $\chi$.
\begin{mdframed}
\begin{mathpar}
  \infer{\phi \conj \psi \leq \phi}{
    }
  \and
  \infer{\phi \conj \psi \leq \psi}{
    } 
\end{mathpar}
\begin{equation*}
  \infer{\chi \leq \phi \conj \psi}{
    \chi \leq \phi & \chi \leq \psi} 
\end{equation*}
\end{mdframed}

We can axiomatize the join ($\vee$)(i.e. the least upper bound) of $\phi,\psi$ for any upper bound $\chi$ as follows .
\begin{mdframed}
\begin{mathpar}
  \infer{\phi  \leq \phi\vee \psi}{
    }
  \and
  \infer{\psi \leq \phi \vee \psi}{
    } 
\end{mathpar}
\begin{equation*}
  \infer{\phi \vee \psi \leq \chi}{
    \phi \leq \chi & \psi \leq \chi} 
\end{equation*}
\end{mdframed}
We can axiomatize the existence of a greatest element as follows:
\begin{mdframed}
\begin{equation*}
  \infer{\chi \leq 1}{
    } 
\end{equation*}
which says that $1$ is the greatest element.
\end{mdframed}

We can axiomatize the existence of a least element as follows:
\begin{mdframed}
\begin{equation*}
  \infer{0 \leq \chi}{
    } 
\end{equation*}
which says that $0$ is the least element.
\end{mdframed}
Finally, to axiomatize \emph{HAs} we require the existence of exponentials for every $\phi$, $\psi$ as follows:

\begin{mdframed}
\begin{mathpar}
 

  \infer{\phi \wedge  (\phi\supset \psi)\leq\psi}{
    } 
    \and
    \infer{\chi\leq\phi\supset\psi}{\phi\wedge\chi\leq\psi}
\end{mathpar}
\end{mdframed}

\subsubsection{Soundness and Completeness}

\begin{mdframed}
\begin{theorem}\label{thm:cmpha}
$\Gamma\vdash_{IPL} \phi \true$ iff for any \vocab{Heyting Algebra} $H$ we have $\Gamma^+\leq\phi^{*}$ where $*$ is  defined as the lifting of any map of $\prop$s to elements of $H$ and $(+)$ is defined inductively on the length of $\Gamma$ as follows
\begin{alignat*}{2}
  nil^+  &&\quad = & \quad\top\\
  (\Gamma,\phi)^+&&\quad = &\quad
  \Gamma^+\wedge\phi* \
\end{alignat*}
\end{theorem}
\end{mdframed}
\chapter{Lambda Calculus With Types}\label{lambda}
\section{From intuitionistic provability to proof trees}
\vocab{IPL} can be viewed as a declarative axiomatization of proof constructs. Take the introduction rule for conjunction as an example: 
\begin{mathpar}
	\inferrule*[right=$\wedge$I] {\Turnsi {\Gamma} {\phi_1\true}\\{\Turnsi {\Gamma} {\phi_2 \true}}} {\Turnsi {\Gamma} {  \phi_1 \wedge\phi_2 \true}}
\end{mathpar}

The rule says, ``given the existence a proof of $\phi$ and a proof of $\psi$ from assumptions $\Gamma$, there exists a proof of $\phi\wedge\psi$ from assumptions $\Gamma$ at hand ".

We used the description ``declarative" because in this format \vocab{IPL} sequents $\Gamma\vdash \true $ do not describe how such existentials are realized. It is in essence a logic of ``proof relevant truth" but it does not involve the proofs themselves as first class objects. 

An alternative presentation is to explicate proof constructs by directly providing a system of ``proof trees". Such, an approach was actually championed in Gentzen's natural deduction systems and is the necessary move to obtain proof calculi. Once we have proof explicit  proof objects (either as trees, or as we will, see as terms) the system is enriched with equality principles involving such objects. Such rules give computational value (``proof dynamics") to the constructs  and are the driver idea in the  ``Curry--Howard Isomorphism" and its extensions.

Here we provide such a  formulation in proof trees of judgments together with the equality rules on trees, essentially following Gentzen. Proof trees of judgments have the following shape:

\begin{mathpar}
\inferrule*[vdots=1.5em]{}{ J_1,\ldots,  J_i}
\end{mathpar}
\begin{mathpar}
\inferrule*[]{}{J}
\end{mathpar}
We focus one judgments $J$ of the form ${A \true}$.
Here are the the rules for constructing proof trees with labeled assumptions\footnote{Essentially the constructs are directed acyclic graphs since assumptions with the same label are ``bind" and substitutable together but we will be cavalier with such a details}.
First, the deductions using reflection on hypothesis are valid:
\begin{mathpar}
\begin{array}[b]{c}{ x_1:A_1 \true,\ldots,  x_i:A_i\true}\\{\vdots}\\{A_{j\in 1\dots i} \true}
\end{array}
\and
\begin{array}[b]{c}{ x_1:A_1 \true, \ldots,   x_i:A_i\true}\\{\vdots}\\{\top\true}
\end{array}
\end{mathpar}

\begin{mathpar}
	\inferrule[]{{\begin{array}[b]{c} {\mathcal{D}} \\ A\true  \end{array}}\\ {\begin{array}[b]{c} {\mathcal{E}} \\ B\true  \end{array}}}{A \wedge B \true}
	
\end{mathpar}
\begin{mathpar}
	\inferrule[]{{\begin{array}[b]{c} {\mathcal{D}} \\ A\wedge B\true  \end{array}} }{A \true}
	\and
	\inferrule[]{{\begin{array}[b]{c} {\mathcal{D}} \\ A\wedge B\true  \end{array}} }{B \true}
\end{mathpar}
\begin{mathpar}
	
\end{mathpar}


\begin{mathpar}
	\inferrule[]{{\begin{array}[b]{c} {\mathcal{D}} \\ A \true \end{array}} }{A\vee B \true}
	\and
	\inferrule[]{{\begin{array}[b]{c} {\mathcal{D}} \\ B\true  \end{array}} }{A\vee B \true}
\end{mathpar}
\begin{mathpar}
	
\end{mathpar}
\begin{mathpar}
	\inferrule[]{{\begin{array}[b]{c}  {\mathcal{D}} \\ A\vee B \true \end{array}} \\ {\begin{array}[b]{c} {A\true}\\ {\mathcal{E}} \\ C \true \end{array}}\\
	{\begin{array}[b]{c} {B\true}\\ {\mathcal{F}} \\ C \true \end{array}		 }}{C\true}
\end{mathpar}

\begin{mathpar}
	\inferrule[]{
	{\begin{array}[b]{c}  {\mathcal{D}} \\ \bot \true \end{array}}}{C \true}
\end{mathpar}
%\begin{mathpar}
%\inferrule*[]{\PrTri{D}} {\PrTri{E} }
		
%\end{mathpar}
%\begin{mathpar}
%{\inferrule*[]{}{ \inferrule*[]	{\inferrule*[]{}{{D} \\ {E}}\\\\
			%\inferrule*[]{}{\    A\\ \ \qquad %\llbracket   A\rrbracket}}{\Box   A }}}
%\end{mathpar}
\subsection{Properties of Intuitionistic Entailment Redux}
Proof trees by their nature satisfy the properties of entailment in \ref{ssec:entail}. We will not bother with reflection and contraction. The first is trivial and the second can be shown by simple induction on the structure of trees with the proof highlighting  that reflection on hypothesis is order-irrelevant. Transitivity is established by \textit{compositionality} of proof trees and reflects the essence of hypothetical reasoning: proof trees of the appropriate proposition can be ``plugged in" for assumptions to create new valid trees.  
\begin{mdframed}
	\begin{theorem}\label{thm:cmpha}
		If {$\begin{array}{c}{ x:A }\\{\mathcal{D}}
			\\{B \true}
			\end{array}$}
		and {$\begin{array}{c}\PrTri{$\mathcal{E}$}\\{A \true}\end{array}$} are valid proof trees their composition denoted as {$\begin{array}{c}\PrTri{$\mathcal{E}$}\\{A \true}\\{\mathcal{D}}\\{B\true}\end{array}$},  defined by substituting all occurrences of $x:A$ for $E$ in $\mathcal{D}$, is a valid proof tree for $B \true$. 
	\end{theorem}
\end{mdframed}
\subsection{Equating Proof Trees}
Having proof objects as first class citizens, permits for developing logics, essentially, as theories of (typed) equality among such objects. This idea was  stemmed from Gentzen's work on natural deduction and cut elimination and it is what gives to proofs  computational content. Here are the proposed  equalities for the proof relevant \vocab{IPL} introduced initially by Gentzen as the driver of the proof  cut elimination. We will be revisiting these very same equalities and reframe them as equalities among proof terms in the next section. Nevertheless, they can be expressed in proof tree form.  We show indicatively the equalities regarding the $\supset$ connective proofs reserving the rest for the more concise notation.

\begin{mathpar}
	\begin{array}{c c c}
	{\inferrule{
	{\inferrule[]{
		{\begin{array}[b]{c} x:A\\ {\mathcal{D}} \\ B \true \end{array}}}{A \supset B \true}}\\
{\begin{array}{c}\mathcal{E}\\{A \true}\end{array}}
}
{ B \true}} & = &
	
			{\begin{array}[b]{c} \mathcal{E} \\ A \true \\ {\mathcal{D}} \\ B \true \end{array}} 


\end{array}
\end{mathpar}

\begin{mathpar}
	\begin{array}{c c c}
		{\inferrule{
				{\inferrule[]{
						{\begin{array}[b]{c}  {\mathcal{D}} \\ A\supset B \true \end{array}}\\{\overline{x:A}}} { B \true}}}{A \supset B}
			}
			& = &
		
		{\begin{array}[b]{c}  {\mathcal{D}} \\ A\supset B \true \end{array}}
		
		
	\end{array}
\end{mathpar}










\section{Linear representation of trees  with proof terms: $\lambda$ calculus}
Proof terms provide an alternative linear representation for proof trees. The simply typed lambda calculus and its equational system can, thus, be viewed as a calculus for proof trees and proof reductions for  intuitionistic logic. What's more, following the doctrine of proof relevance and of characterizing connectives by their proof reductions, i.e. working in the realm of Curry -- Howard Isomorphism, we hit two birds with one stone: we both develop proof relevant logics and we get typed programming languages that reflect their computational content. The ``simplest" language obtained within this program is the simply typed lambda calculus, but we will see that the same doctrine  extends to different logics with different judgmental constructs.
  
\subsubsection{Simply typed lambda calculus}
\begin{mdframed}
\textbf{Type Formation}
\begin{mathpar}
\inferrule*[right=Atom] { } {P_i \in {\sf Type}}
\and
\inferrule*[right=Top] { } {\top \in {\sf Type}}
\and
\inferrule*[right=Bottom] { } {\bot \in {\sf Type}}
\and
\inferrule*[right=Arr] {{\phi_1 \in {\sf Type }}\\ {\phi_2 \in {\sf Type}}} {\phi_1\rightarrow\phi_2\in {\sf Type}}
\and
\inferrule*[right=Prod] {{\phi_1 \in {\sf Type }}\\ {\phi_2 \in {\sf Type}}} {\phi_1\times \phi_2\in {\sf Type}}
\and
\inferrule*[right=Union] {{\phi_1 \in {\sf Type }}\\ {\phi_2 \in {\sf Type}}} {\phi_1 + \phi_2\in {\sf Type}}
\end{mathpar}
\end{mdframed}

\begin{mdframed}
\textbf{Context Formation}
\begin{mathpar}
\inferrule*[right=Nil] { } {{\sf nil}\  \context}
\and
\inferrule*[right=$\Gamma$-Add] {{\Gamma\ } {\sf \context}  \\ {\phi \in {\sf Type}}\\ x\text{ fresh  in }\Gamma} {{\Gamma,\  x:\phi} \ \context}
\end{mathpar}
\end{mdframed}

\begin{mdframed}
\textbf{Context Reflection}
\begin{mathpar}
\inferrule*[right=$\Gamma$-Refl] { {\Gamma}\ {\sf \context}\\ {x:\phi  \in \Gamma}}
{\Turnsi {\Gamma} {x:\phi}}
\end{mathpar}
\end{mdframed}

\begin{mdframed}
\textbf{Top Introduction -- Bottom Elimination}
\begin{mathpar}
\inferrule*[right=$\top$I] { } {\Turnsi {\Gamma} { \langle \rangle:\top }}
\and
\inferrule*[right=$\bot$E] {\Turnsi {\Gamma} {M:\bot } } {\Turnsi {\Gamma} { abort[\phi](M) :\phi}}
\end{mathpar}

\end{mdframed}

\begin{mdframed}
\textbf{Function Construction and Application}
\begin{mathpar}
\inferrule*[right=$\lambda-$Abs] {\Turnsi {\Gamma, x:\phi_1 } {M:\phi_2 }} {\Turnsi {\Gamma} { \lambda x. M:\phi_1\rightarrow \phi_2 }}
\and
\inferrule*[right=App] {\Turnsi {\Gamma} {M:\phi_1\rightarrow\phi_2 }\\{\Turnsi {\Gamma} {M^{'}:\phi_1}}} {\Turnsi {\Gamma} {  (M M^{'}):\phi_2 }}
\end{mathpar}
\end{mdframed}
\begin{mdframed}
\textbf{Tuple Construction and Projections}
\begin{mathpar}
\inferrule*[right=Tup] {\Turnsi {\Gamma} {M:\phi_1}\\{\Turnsi {\Gamma} { M^{'}:\phi_2}}} {\Turnsi {\Gamma} {  
\langle M,M^{'}\rangle:\phi_1\times \phi_2 }}
\end{mathpar}
\begin{mathpar}
\inferrule*[right=LPrj] {\Turnsi {\Gamma} {M:\phi_1\times\phi_2 }} {\Turnsi {\Gamma} {\operatorname{fst}(M):  \phi_1}}
\and
\inferrule*[right=RPrj] {\Turnsi {\Gamma} {M:\phi_1\times\phi_2}} {\Turnsi {\Gamma} {\operatorname{snd}(M):  \phi_2}}
\end{mathpar}
\end{mdframed}
\begin{mdframed}
\textbf{Union Construction and Elimination}
\begin{mathpar}
\inferrule*[right=InjL] {\Turnsi {\Gamma} {M:\phi_1}} {\Turnsi {\Gamma} {  inj_l[\phi_2](M):\phi_1+\phi_2}}
\and
\inferrule*[right=InjR] {\Turnsi {\Gamma} {M:\phi_2 }} {\Turnsi {\Gamma} {inj_r[\phi_1](M) : \phi_1+\phi_2}}
\end{mathpar}


\begin{mathpar}
\inferrule*[right=$\vee$E] 
{ {\Turnsi {\Gamma} {  M:\phi_1 + \phi_2}}\\
{\Turnsi {\Gamma,x:\phi_1 } {N:\phi }}\\
{\Turnsi {\Gamma,y:\phi_2} {O:\phi }}
}
 {\Turnsi {\Gamma} {\text{case }M \text{ of } inj_l(  x)\longmapsto N  |\  inj_r (y)\longmapsto O : \phi}}
\end{mathpar}

\end{mdframed}
\subsection{Definitional Equality: Proof tree equalities as term equalities}

We want to think about when two proofs (resp. proof trees) $M:A$ and $M':A$ are the same. In the following  we elaborate Gentzen's principles introducing an equivalence relation called \emph{definitional equality} that respects these principles, denoted $M\equiv M':A$. We want definitional equality $\equiv$ to be the least congruence closed under the $\beta$ rules. We will define what this means:

A \emph{congruence} is an equivalence relation (i.e. reflexive transitive and antisymmetric) that respects our operators as formulated below. 

\begin{mdframed}
	\begin{mathpar}
		\inferrule*[right=R] 
		{ }
		{\Turnsi {\Gamma} {M\equiv M:\phi }}
		\and
		\inferrule*[right=S] 
		{\Turnsi {\Gamma} {  M\equiv M':\phi}}
		{\Turnsi {\Gamma} {M'\equiv M:\phi }}
		\and
		\inferrule*[right=T] 
		{{\Turnsi {\Gamma} {  M\equiv M':\phi}}\\
			{\Turnsi {\Gamma} {M'\equiv M'':\phi }}} {{\Turnsi {\Gamma} {  M\equiv M'':\phi} }}
	\end{mathpar}
	
\end{mdframed}

For the equivalence relation to respect operators we  means that if $M\equiv M':A$, then that their image under any operator should be equivalent. In other words, we should be able to replace $M$ with $M'$ everywhere.For example:

\begin{mdframed}
	\textbf{Congruence over $fst$}
	\begin{mathpar}
		
		\inferrule*[]
		{\Gamma \entails M\equiv M':A\wedge B}
		{\Gamma\entails \operatorname{fst}(M)\equiv\operatorname{fst}(M'):A}
		
		
	\end{mathpar}
	
\end{mdframed}

\subsubsection{Inversion Principle}\label{ge:in}

Gentzen's Inversion Principle captures the idea that ``elim is post-inverse to intro," which is the informal notion that the elimination rules should cancel the introduction rules.

The $\beta$ rules are as follows:


\begin{mdframed}
	\begin{mathpar}
		
		\inferrule*[right=$\beta\wedge_1$] 
		{{\Turnsi {\Gamma} {  M:\phi_1}}\\
			{\Turnsi {\Gamma} {N :\phi_2 }}  } {{\Turnsi {\Gamma} {\operatorname{fst}(  \langle M,N\rangle)\equiv M:\phi_1} }}
		
		\and
		\inferrule*[right=$\beta\wedge_2$] 
		{{\Turnsi {\Gamma} {  M:\phi_1}}\\
			{\Turnsi {\Gamma} {N :\phi_2 }}} { {\Turnsi {\Gamma} {\operatorname{snd}(  \langle M,N\rangle)\equiv N:\phi_2} }}
		\and
		\inferrule*[right=$\beta\supset$] 
		{{\Gamma,x:A\vdash M:B}\\ {\Gamma\vdash N:A}}
		{\Gamma\vdash(\lambda x.M)(N)\equiv [N/x]M:B}
		\and
		\inferrule*[right=$\beta\vee_1$]
		{ {\Gamma,x
				:\phi_1\vdash N:\psi}\\{\Gamma,y:\phi_2 \vdash O:\psi}\\{\Gamma\vdash P:\phi_1} }
		{\Gamma\vdash  ({\text{case } inj_l(P) \text{ of } inj_l(x) \longmapsto N  |\  inj_r (y)\longmapsto O) \equiv[P/x]N: \psi}}
		\and
		\inferrule*[right=$\beta\vee_2$]
		{ {\Gamma,x
				:\phi_1\vdash N:\psi}\\{\Gamma,y:\phi_2 \vdash O:\psi}\\{\Gamma\vdash Q:\phi_2} }
		{\Gamma\vdash  ({\text{case } inr_r(Q) \text{ of } inj_l(x)  \longmapsto N  |\  inj_r(y) \longmapsto O) \equiv[Q/y]O: \psi}}
	\end{mathpar}
\end{mdframed}

\subsubsection{Unicity Principle}

Gentzen's Unicity Principles on the other hand capture the idea that ``intro is post-inverse to elim". 	There should be only one way -- modulo definitional equivalence -- to prove something. The ``$\beta$" rules give rise to computational interpretation. The ``$\eta$" rules impose properties that the computational model should satisfy (e.g. Church-Rosser property).

The $\eta$ rules are given below:

\begin{mdframed}
	\begin{mathpar}
		\inferrule*[right=$\eta\top$]{\Gamma\vdash M:\top}{\Gamma\vdash M\equiv \langle \ \rangle: \top} 
		\and
		
		\inferrule*[right=$\eta\wedge$]{\Gamma\vdash M\equiv \langle \operatorname{fst}(M),\operatorname{snd}(M)\rangle :A\wedge B }{\Gamma\vdash M:A\wedge B}
		\and
		\inferrule*[right=$\eta\supset$]{\Gamma\vdash M:\phi\supset \psi}{\Gamma\vdash M\equiv \lambda x. M x: \phi\supset\psi} 
		\and
		\inferrule*[right=$\eta\vee$]
		{\Gamma,z
			:\phi_1 + \phi_2\vdash M:\psi\\\Gamma\vdash N:\phi_1 + \phi_2}
		{{
				\begin{tabular}[b]{ rl}
					$\Gamma\vdash M[N/z] \equiv \text{case }  N \text{ of }$  & $| inj_l(x) \mapsto M[inj_l(x)/z]$
					\\
					& $ inj_r (y)\mapsto M[inj_r(y)/z]) : \psi$
					\\
				\end{tabular}
			}
		}
		
	\end{mathpar}
\end{mdframed}

\section{Operational (a.k.a ``term") Semantics}
Given the formulation of the $\lambda$-calculus we can think of \emph{formulae-as-types} and \textit{proof terms-as-programs}. This enriches logic with a computational aspect(\emph{proof dynamics}) that is absent from other formulations. Dynamics stems from Gentzen's insight to give an equational system for proofs equipped with $\beta$$\eta$ rules for proof-tree conversion. These insights, give rise to $\lambda$-calculus dynamics if we devise an execution strategy (an \emph{operational semantics}) for program reduction that respects these rules. 

The first step toward operational semantics is to break the symmetry of the definitional equivalence and construct a one-way reduction relation  on lambda terms. Towards this definition we first define the notion of a  \textit{redex}: 
\begin{mdframed}
\begin{definition}	
\begin{itemize}
\item A $\beta$-redex is every term of the form:  $$(\lambda x:\phi. N) M |  \operatorname{fst}\langle M, N \rangle| \operatorname{snd}\langle M, N \rangle $$
\item  A $\eta$-redex is every term of the form:
$$\lambda x: M x | \langle \operatorname{fst} M\operatorname{snd} M \rangle  $$    
\end{itemize}
\end{definition}	
\end{mdframed}
A \textit{normal form} is a term where no redex occurs. To make the
term surrounding the redex explicit, we can use a \textit{term context}, i.e. a term with a single term hole, such as $\lambda x:[]$, $(e[])$, $[]e$, where  a hole can be substituted for a term to give a larger term. To be strict all single hole terms have the following diagram:
$$H:= [\bullet]|\ (M[\bullet])|\  ([\bullet]M)|\ \lambda x:A. H|\  \langle H, M \rangle |\  \langle  M, H \rangle $$

Now we have enough tools to define the \textit{(one-step) $\beta\eta$-reduction} between two terms can be defined on terms that include redexes as subterms as follows :
\begin{mdframed}
	\textbf{One-step $\mapsto_{\beta\eta}$  reduction}
	\begin{mathpar}
		\inferrule*[right=($\beta$)]{} {(\lambda x:\phi. N) M \mapsto [M/x]N }
		\and
		\inferrule*[right=($\beta$)] {}{\operatorname{fst}\langle M, N \rangle\mapsto M}
		\and
		\inferrule*[right=($\beta$)] {}{\operatorname{snd}\langle M, N \rangle\mapsto N}
		\and
		\inferrule*[right=($\eta$)]{}{\lambda x: M x\mapsto M}
		\and
		\inferrule*[right=($\eta$)]{} { \langle \operatorname{fst} M\operatorname{snd} M\rangle\mapsto M }  
		\and
		\inferrule*[right=(subterm)]{M\mapsto M^{'}}{H[M]\mapsto H[M^{'}]}
	\end{mathpar}
\end{mdframed}

Now we can define the  reflexive, transitive closure of the previous relation as $\mapsto_{\beta\eta}^{*}$ to denote zero or more reduction steps. The following facts -- leading to a computational proof of consistency -- hold:

\begin{mdframed}
	\begin{theorem}	\textbf{Church -- Rosser for $\mapsto_{\beta\eta}$}
		For every term $M$, if $M\mapsto_{\beta\eta} N_1$ and $M \mapsto_{\beta\eta} N_2$ then there exists $N^{'}$ s.t. $N_1 \mapsto N^{'}$ and $N_2\mapsto N^{'}$
		
	\end{theorem}	
	\begin{theorem}	\textbf{Church -- Rosser for $\mapsto_{\beta\eta}^{*}$}
		For every term $M$, if $M\mapsto_{\beta\eta}^{*} N_1$ and $M \mapsto_{\beta\eta}^{*} N_2$ then there exists $N^{'}$ s.t. $N_1 \mapsto_{\beta\eta}^{*} N^{'}$ and $N_2\mapsto_{\beta\eta}^{*} N^{'}$
		
	\end{theorem}	
\end{mdframed}
 The first consistency result for the equational system comes straight from the Church--Rosser properties. Since, it is  easy to show that for any terms $M$, $N$ s.t where $\Gamma\vdash M\equiv N: A$ based on the $\equiv$ axiomatization there exists a finite sequence of terms $N_0,\ldots, N_i$ such that $M\mapsto_{\beta\eta}^{*}N_0\mapsfrom_{\beta\eta}^{*} N_1\mapsto_{\beta\eta}^{*} N_2\mapsfrom_{\beta\eta}^{*}\ldots \mapsfrom_{\beta\eta} N_i$. Now we can obtain:
 \begin{mdframed}
 \begin{theorem}	\textbf{Definitional equality implies common contractum}
 	For any terms, $M$,$N$ if $\Gamma\vdash M\equiv N:A$ then there exists tern $L$ s.t. $M,N\mapsto_{\beta\eta}^{*} L$
 	
 \end{theorem}
\end{mdframed}
 And as a result:
 \begin{mdframed}	
 \begin{theorem}	\textbf{Consistency of definitional equality of terms}
 	The definitional equality $\equiv$ is not trivial i.e. it won't equate any two terms.
 \end{theorem}	
\end{mdframed}
Moving toward consistency of the whole system (i.e. there is not term of type $\bot$), we prove a theorem for the existence of normal forms.
\begin{mdframed}	
\begin{theorem}	\textbf{Weak normalization theorem}
For any term $M$, there exists a finite sequence of terms s.t. $M\mapsto_{\beta\eta}N_0\mapsto_{\beta\eta} N_1\mapsto_{\beta\eta} N_2\mapsto_{\beta\eta}\ldots \mapsto_{\beta\eta} N_i$ where $N_i$ is a $\beta\eta$ normal form.
\end{theorem}	
\end{mdframed}
It is common place in metatheotic proofs for such systems that induction on the structure of the term does not ``go through''. Intuitively, a reduction can be ``enlarging" the term but, yet, it is doing progress based on a different kind of metric. We build this metric based on the following definitions and facts. The idea is that we can choose a reduction strategy such that the number of \textit{redexes of a specific type} (to be defined soon) reduce. Here are the steps towards the proof. We omit redexes related to disjunction but the proof extends to such cases pretty easily.


\begin{mdframed}
	\textbf{Definition}
	The \textit{degree of a type $A$} is defined as follows:
	\begin{itemize}
		\item $\theta(P_i)=1$ if $P_i$ is atomic
		\item $\theta(A\times B)=\theta(A \rightarrow B)= \theta(A)+\theta(B)+1$
	\end{itemize}
\end{mdframed}
\begin{mdframed}
	\textbf{Definition}
	The \textit{degree of a redex} is defined as follows:
	\begin{itemize}
		\item  Given that the type of $\lambda x. M$ is of type $A\rightarrow B$ then $\theta(A\rightarrow B)$ then $d(\lambda x. M)N=\theta(A\rightarrow B)$
		\item Similarly, $d(\pi\operatorname{fst}\langle M, N\rangle)=\theta(A\times B)$
		where $A\times B$ is the type of $\langle M, N\rangle$
		
		\item Similarly for the other kinds of redexes.
	\end{itemize}
\end{mdframed}
\begin{mdframed}
	\textbf{Definition}
	The \textit{degree of a term} $d(t)$ is defined as the supremum of the degrees of its redexes.
\end{mdframed}
Now we can prove the following facts:
\begin{theorem}
	\begin{mdframed}
\begin{enumerate}
	\item The degree of redex $r$ is strictly larger than the degree of its type $A$: 
	$\theta(A)<d(r)$
	\item The degree of a redex ($r$) seen as term ($t$)  can be smaller than its redex degree since it might include other redexes: $d(r)\le d(t)$.
	\item The term resulting from a substitution $M[N/x]$  has degree : $d(M[N/x])\le max(d(M),d(N),\theta(A))$ where $A$ is the declared type of $x$ in the type context.
\end{enumerate}
	\end{mdframed}
\end{theorem}
Which give us the following fact that suggests the induction principle that succeeds toward the proof.
\begin{mdframed}
	\begin{theorem}
If $M\mapsto M^{\prime}$ then $d(M)<d(N)$ and hence, if $M\mapsto^{+} N$ then $d(M)<d(N)$.
\end{theorem}
\end{mdframed}
As a result we get a weak  normalization theorem by induction on pairs $(d(M),k)$ where $k$ is the number of redexes with degree $d(M)$.
\begin{mdframed}
\begin{theorem}
	\textbf{Weak Normalization Theorem}
For every term $\Gamma\vdash M:A$ there exists a normalization strategy such that $M\mapsto_{\beta\eta}^{*}N$ and $N$ is a normal form.
\end{theorem}
\end{mdframed} 

Combining with previous results we get:
\begin{theorem}
\begin{mdframed}
\textbf{Consistency}
There is no (closed) term $M$ for which $\vdash M:\bot$ 
\end{mdframed}
\end{theorem}
 Suppose the opposite and obtain a contradiction using the previous theorem: there is no way to obtain a normal form of a bottom type from the rules.  
 
 A stronger result is the strong normalization theorem that says that \textit{every} strategy in normalizing. This result is important in concurrent implementations of reduction since it implies that the order in which redexes are consumed does not matter during the evaluation of  expression. 
 %The strong normalization, as we will see, is effectively isomorphic (in  the usage of induction principles) with the proof of admissibility of cut for the sequent calculus formulation of the system. Again we sketch the basic steps toward the proof which is one intricate application of the mathematics of induction.
 
 The important idea behind the technique - that generalizes to proof strong normalization for more complex calculi- is the concept of a reducibility predicate. Reducibility predicates give a stronger induction principle that is able to give the desired result as a lemma. Similarly with the weak normalization reducibility predicates are sets of terms classified by their type.
 They are defined by induction on the structure of their type. We show strong normalization only for $\beta$ reduction but and only for the $\rightarrow, \times$ fragment of the calculus but all techniques generalize (see, e.g. PRAWITZ). 
 
\begin{mdframed}
	\textbf{Definition}
	We define the predicate $Red_A$ for a type $A$ as follows and for closed terms $M$ as follows:
	\begin{itemize}
		\item $M\in Red_{P_i}$ iff $M$ is of atomic type $P_i$  and $M$ is normalizable. 
		\item $M \in Red_{A\rightarrow B}$ iff $M$ is of type $A\rightarrow B$ and $\forall N. N \in Red_{A}\Longrightarrow (MN) \in Red_B$
		
		\item $M \in Red_{A\wedge B}$ iff $M$ is of type $A\wedge B$,  $\operatorname{fst}(M)\in Red_A$ and $\operatorname{snd}(M) \in Red_B$
	%	\item $M in Red_{A\vee B}$ iff $\forall N_3, N_4 s.t.  N_1 \in Red_{A} implies N_3[N_1\x]\in Red_C and N_2 \in Red_{B} implies  N_4[N_2\y]\in Red_{C} the term $case M of inl(x) M[x]| inr(y)N[y] \in Red_{C}$		\
		
	\end{itemize}
\end{mdframed}
Reducibility has the important following properties below where  by \textit{neutral} we mean terms of the form $\langle M, N \rangle$, $\lambda x. M$ 
\begin{theorem}
	\begin{enumerate}
		\item $M\in Red_A$ implies that $M$ is normalizable. 
		\item $M \in Red_A$ and $M\mapsto_\beta^{*} N$ then $N\in Red_A$
		\item If $M$ is neutral then,  $\forall N {\text s.t} M\mapsto^{\beta} N. N\in Red_A\Longrightarrow M \in Red_A $ 
		\item  If $M$ is neutral and normal, then $M \in Red_A$
	\end{enumerate}
\end{theorem}
The proof goes  by induction on the structure of type $A$.
Now we are able to prove the following theorems:

\begin{mdframed}
	\begin{theorem}
	\begin{itemize}
		\item If $M,N$ are reducible then so is $\langle M, N \rangle$
		\item For all reducible $M$ of type $A$ if $N[M/x]$ is reducible then so is $\lambda x. N$
	\end{itemize}

	\end{theorem}
	\end{mdframed}

These results suffice to show the following theorem:
Given $x_1:A_1, \ldots, x_i: A_i\vdash M: A$ then and reducible terms $v_1\in Red_{A_1},\ldots v_i\in Red_{A_i}$ then  $M[v_1/x_1\ldots v_i/x_i]\in Red_A$ 
Out of which we get:
\begin{mdframed}
 	\begin{theorem}
 		\textbf{Strong Normalization Theorem}

 			Every term $\vdash M:A$ is strongly normalizing

 		
 	\end{theorem}
 \end{mdframed}
 
\subsection{The essence of Proofs as program }
The proofs of normalization above are essentially of the same "proof strength" (induction principles) as the logical proof of cut elimination. In a nutshell, eliminating cuts is the same as normalizing proof terms (and the corresponding proof trees).

In reality, the slogan of the Curry-Howard isomorphism and, in general, of a type theoretic treatment to logic should be "Normalization as Cut Elimination". This aspect of the isomorphism can be shown explicitly follow Siegl's extraction method, that essentially showcases how a construction of the Cut-free sequent calculus comes naturally from an analysis of normal proofs in the natural deduction. This result has been extremely useful through my research and I am expecting to revisit it in more detail at my thesis work.

 
\subsection{Propositions as Types}



There is a correspondence between propositions and types:
\begin{center}
  \begin{tabular}{@{} cc @{}}

    Propositions & Types \\

    $\top$ & $1$ \\
    $A\wedge B$ & $A\times B$ \\
    $A\supset B$ & function $A\to B$ or $B^A$ \\
    $\bot$ & $0$ \\
    $A\vee B$ & $A+B$\\

  \end{tabular}
\end{center}



\section{Categories for proof relevant \vocab{IPL}}

In a Heyting Algebra, we have a preorder (or, partial order in the ``textbook" definition) $\phi\leq \psi$ when $\phi$ implies $\psi$. \emph{HAs} are insufficient, however, for the treatment of proof objects (there can be at most one instance of $\phi\leq\psi$ for specific $\phi$,$\psi$). We can  keep track of proofs, so if $M$ is a proof from  $\Gamma$ to $\psi$, we want to think of it as a map $M:\Gamma+\to \psi+$. The category theoretic analog of a Heyting 	Algebra is a Bi--Cartesian Closed Category (\emph{BiCCC}). That is a category with all finite products, co--products and exponentials. The axiomatization of a category (in general), finite (and nullary) products and co-products and exponentials is given in this section. 

\subsection{Definitions and Axioms of a Category}
A category has \emph{objects} $\phi,\psi, \ldots$ and \emph{arrows} $f,g,h\ldots$ Each arrow goes from an object to an object. To say that $g$ goes from $\phi$ to $\psi$ we write $g:\phi\rightarrow\psi$, or say that $\phi$ is the domain of $g$, and $\psi$ the \emph{co--domain}. We write $Dom(g)=\phi$ and $Cod(g)=\psi$.
We say that two arrows $f$ and $g$ are \emph{composable} with $Dom(f)=Cod(g)$. If $f$ and $g$ are composable, they have a \emph{composite}, an arrow called $f\circ g$. There is an \emph{identity} for every object $\phi$.
\begin{mdframed}
\begin{mathpar}
\inferrule*[right=$\text{ID}_{ex}$]{ }
{\operatorname{id}:\phi\to \phi}
\and
\inferrule*[right=Comp]{{f:\phi\to\psi } \\{g:\psi\to\chi}}
{g\circ f:\phi\to \chi}
\and
\inferrule*[right=$\text{ID}_l$]{f:\phi\to\psi }
{id_{\psi}\circ f=f:\phi\to \psi}
\and
\inferrule*[right=idr]{f:\phi\to\psi }
{f\circ id_{\phi} =f:\phi\to \psi}
\and
\inferrule*[right=idr]{{f:\phi\to\psi }\\{g:\psi\to\chi}\\{h:\chi\to\omega}}
{h\circ (g\circ f) =(h\circ g)\circ f:\phi\to \omega}
\end{mathpar}
\end{mdframed}



\subsection{Terminal, Co-Terminal objects, Products and Co-Products}
Now we can think about objects in the category that correspond to propositions given in the correspondence.

\paragraph{Terminal Object}
$1$ is the terminal object, also called the final object, which corresponds to $\top$. For any object $\Gamma$ there is a unique map $\Gamma\to 1$. 
\begin{mdframed}
\begin{mathpar}

\inferrule*[right=Existence]
{ } {\langle \ \rangle:\phi\to 1}
\and
\inferrule*[right=Unicity($\eta$)]
{M:\Gamma\to 1}{M=\langle \ \rangle:\Gamma\to 1}
\end{mathpar}
\end{mdframed}



\paragraph{Product} 
For any objects $\phi$ and $\psi$ there is an object $\chi=\phi\times \psi$ equipped with arrows $\operatorname{fst}:\phi\times\psi\to\phi$ and $\operatorname{snd}:\phi\times\psi\to\psi$ that is the \emph{product} of $\phi$ and $\psi$, which corresponds to the join $\phi\wedge \psi$. For any other object $\Gamma$ with arrows $M:\Gamma\to\phi$ and $\Gamma\to\psi$ there  exists \emph{unique} arrow, $\langle M,N \rangle$ s.t. $\operatorname{fst}\circ\langle M, N\rangle=M(\beta\times_1)$ and
$\operatorname{snd}\circ\langle M, N\rangle=N(\beta\times_2)$.
\begin{mdframed}
\begin{mathpar}

\inferrule*[right=Exist$_1$]{{M:\Gamma\to \phi} \\{ N:\Gamma\to \psi}} {\langle M,N\rangle : \Gamma\to \phi\times \psi} 
\and
\inferrule*[right=Exist$_2$($\beta_1$)]{{M:\Gamma\to \phi} \\{ N:\Gamma\to \psi}} {\operatorname{fst}\circ\langle M,N\rangle : \Gamma\to \phi}
\and 
\inferrule*[right=Exist$_3$($\beta_2$)]{{M:\Gamma\to \phi} \\{ N:\Gamma\to \psi}} {\operatorname{snd}\circ\langle M,N\rangle : \Gamma\to \phi} 
\and
\inferrule*[right=Un($\eta$)]{{P:\Gamma\to \phi\times \psi} \\{ \operatorname{fst}\circ P=M:\Gamma\to \phi}\\ {\operatorname{snd}\circ P=N:\Gamma\to \psi}} {P=\langle M,N\rangle : \Gamma\to \phi\times \psi}
\end{mathpar}
Diagrammatically:
\begin{equation*}
  \begin{tikzcd}
    {} & \Gamma \arrow[bend right]{ddl}[swap]{M} \arrow[bend left]{ddr}{N}\dar[dashed]{\langle M,N\rangle} & {} \\
    {} & \phi \times \psi \dlar{\text{fst}}\drar[swap]{\text{snd}} & {} \\
    \phi & {} & \psi
  \end{tikzcd}
\end{equation*}
\end{mdframed}




\paragraph{Exponentials}
Given objects $A$ and $B$, an exponential $B^A$ (which corresponds to $A\supset B$) is an object with the following universal property:
\[
\begin{tikzcd}
C \arrow[dashed]{dd}[swap]{\lambda(h)}&{C\times A}\arrow{ddrr}{h} \arrow[dashed]{dd}[swap]{\lambda(h)\times \operatorname{id}_A} &&\\
&& {}	&{}\\
B^A&{B^A\times A}\arrow{rr}[swap]{\operatorname{ap}} &{}&{B}
\end{tikzcd}
\]
such that the diagram commutes.

This means that there exists a map $\operatorname{ap}:B^A\times A\to B$ (application map) that corresponds to implication elimination.

The universal property is that for all objects $C$ that have a map $h:C\times A\to B$, there exists a unique map $\lambda(h):C\to B^A$ such that
\[
\operatorname{ap}\circ (\lambda(h)\times \operatorname{id}_A )=h:C\times A\to B
\]
This means that the diagram commutes. Another way to express the induced map is $\lambda(h)\times \operatorname{id}_A = \langle \lambda(h)\circ\operatorname{fst},\operatorname{snd}\rangle$.

The map $\lambda(h):C\to B^A$ is unique, meaning that
\begin{mathpar}
\inferrule*{\operatorname{ap}\circ(g\times \operatorname{id_A})=h:C\times A\to B}{g=\lambda(h):C\to B^A}
\end{mathpar}

\paragraph{Co--Products}
For any objects $\phi$ and $\psi$ there is an object $\chi=\phi + \psi$ equipped with arrows $\operatorname{inl}:\phi\to\phi+\psi$ and $\operatorname{inr}:\psi\to\phi+\psi$ that is the \emph{co-product} of $\phi$ and $\psi$, which corresponds to the meet $\phi\wedge \psi$. For any other object $\omega$ with arrows $M:\omega\to\phi\vee\psi$ and $N:\omega\to\phi\vee\psi$ there  exists \emph{unique} arrow, ${ M,N }$ s.t. $\{M,N\}\circ\operatorname{inl}=M$ and
$\{M,N\}\circ\operatorname{inr}=N$.

\begin{mdframed}
\begin{mathpar}

\inferrule*[right=Exist$_1$]{O:\Gamma\to \phi}{ \operatorname{inl}\circ O:\Gamma\to\phi + \psi}  
\and

\inferrule*[right=Exist$_2$]{P:\Gamma\to \psi}{ \operatorname{inr}\circ P:\Gamma\to\phi + \psi}
\and
\inferrule*[right=Exist$_3$($\beta_1$)]{{O:\Gamma\to \phi} \\{M:\phi\to\omega}\\{N:\psi\to\omega}} {\{M,N\}\circ\operatorname{inl}\circ O=M\circ O : \Gamma\to \omega}
\and 
\inferrule*[right=Exist$_3$($\beta_2$)]{{P:\Gamma\to \psi} \\{M:\phi\to\omega}\\{N:\psi\to\omega}} {\{M,N\}\circ\operatorname{inr}\circ P=N\circ P : \Gamma\to \omega}
\and
\inferrule*[right=Exist$_3$($\beta_2$)]{{M:\Gamma\to \phi} \\{ N:\Gamma\to \psi}} {\operatorname{snd}\circ\langle M,N\rangle : \Gamma\to \phi} 
\and
\inferrule*[right=Un($\eta$)]{ { O :\Gamma\to \phi} \\ {P:\Gamma\to \psi}\\{U:\phi + \psi\to\omega}\\{M:\phi\to\omega}\\{N:\psi\to\omega} \\{U\circ \operatorname{inl}\circ O=M} \\
 \\{U\circ \operatorname{inr}\circ N =M}} {U=\{M,N\} }  
\end{mathpar}
Diagrammatically:
\begin{tikzcd}
\phi\drar{\inl}\arrow[bend right]{ddr}{(\beta)}[swap]{M}  & {} & \psi\dlar[swap] {\inr}\arrow[bend left]{ddl}{N}[swap]{(\beta)}	\\
{} & \phi+\psi\dar[bend right]{(\eta)}[swap]{M}\dar[bend left]{\{M,N\}} & {}	\\
{} & C & {}
\end{tikzcd}
\end{mdframed}



