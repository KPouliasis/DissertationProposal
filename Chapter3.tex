\chapter{Modal Type Theory}
In this chapter I will give a short, example-driven introduction to typed modality and its applications. 
The discussion will remain informal when it comes to metatheory and will be focused on usage of modal typed calculi in applications. Apropos, I will discuss Operational Semantic as the essence of the computational approach to logic.  
\section{A bird's eye view}
Significant in the development of modal logic within a type-theoretic framework is the work of Moggi \cite{moggi1991notions}.  In his seminal work he mentioned the need of shifting from simply typed calculi to systems that can capture notions of computation such exceptions, partial functions,  binding constructs etc. Moggi's initiative stems from a categorical point of view. 

In the realm of deductive systems with explicit witnesses, Artemov's work on Operational Modal Logic\cite{artemovy1995operational}, and the ``Gang of four" \cite{benton1992term},\cite{Benton:1998:CTL:969611.96961} revived the interest in constructive modality and its computational view. Of course earlier work in the intersection of modal and constructive logic exists (cf. \cite{prawitz10natural},\cite{fitting1983proof} ) but its relation with programming languages is not yet explicit. It's worth mentioning that the work by DePaiva  initiates from a categorical view too. That is Categorical Semantics for Linear Logic. 

Since it is of great importance in my work, I will be presenting here the \emph{judgmental} approach followed by Pfenning's ``Judgmental Reconstruction of Modal Logic" \cite{pfenning2001judgmental} which is a foundational approach that captures previous work on $\Box$ Calculi for $S4$. Although the system presented is a judgmental reconstruction of the system {\sf $S4$} the approach can be used to host other modalities. 

\section{Judgmental Reconstruction of Modal Logic}
The $\Box$ fragment of Pfenning's Judgmental reconstruction, consists of the judgments of \ac{IPL} as developed in the first Chapter together with judgments of validity. The definition of validity is given in a proof theoretic manner. In a nutshell, judgments of validity internalize judgments of proof without assumptions. ``Evidence for validity of $\phi$, is simply unconditional evidence of truth of $\phi$".   

\begin{mdframed}
\begin{enumerate}
\item If ${\sf{nil}}\vdash \phi \true$ then $\phi\valid$
\item If $\phi\valid$  then $\Gamma\vdash \phi\true$

\end{enumerate}
\end{mdframed}

The logical judgments are now extended in the form:
\begin{mdframed}
$$\phi_1'\valid, \phi_2'\valid,\ldots,\phi_n'\valid;\phi_1\true\phi_2\true,\ldots\phi_m\true\vdash\phi\true $$
\end{mdframed}


In the rules, we restrict ourselves to proving judgments of the form $\phi \true$ (rather
than
$\phi \valid$), which is possible since the latter is directly defined  in terms of the former.
The meaning of hypothetical judgments yields the general substitution principle.

\begin{mdframed}
$\Delta\vdash\phi\valid$ and $\Delta,\phi\valid\vdash J$ then $\Delta\vdash J$ 
\end{mdframed}

This principle can be rewritten utilizing the definition of validity:
\begin{mdframed}
\textbf{Substitution Principle For Validity}
$\Delta;\nil\vdash\phi \true$ and $\Delta,\phi \valid;\Gamma\vdash J$ then $\Delta;\Gamma\vdash J$ 
\end{mdframed}


Additionally, we add the following hypothesis reflection rule for valid contexts:


\begin{mdframed}
\textbf{Validity Context Reflection}

\begin{mathpar}
\inferrule*[right=$\Delta$-Refl] { {\Delta \ {\sf \context}}\\{\Gamma\context} \\{\phi \valid \in \Delta}}{\Turnsi {\Delta;\Gamma} {\phi \true}}
\end{mathpar}
\end{mdframed}

In the typability rules we add the following:
\begin{mdframed}
\begin{mathpar}
\inferrule*[right=$\Box$F]{\phi\prop}{\Box\phi\prop}
\end{mathpar}
\end{mdframed}

The $\Box$ introduction rule just allows the internalization of the validity of
$\phi$ as truth of $\Box \phi$, according to the definition of validity.
\begin{mdframed}
\textbf{Necessity Introduction}
\begin{mathpar}
\inferrule*[right=$\Box$I]{\Delta;\nil\vdash\phi\true}{\Delta;\Gamma\vdash\Box\phi\true}
\end{mathpar}
\end{mdframed}
The elimination rule is harder. A simplified version like the one below is unsound  since the hypotheses in  $\Gamma$  are unjustified.:
\begin{mathpar}
\inferrule*[right=$\Box$E-Unsound]{\Delta;\Gamma\vdash\Box\phi\true}{\Delta;\vdash\Box\phi\true}
\end{mathpar}

Another approach  would be the rule below. Which is locally sound but not complete  Gentzen's inversion principle is not satisfied. After eliminating the $\Box$ we cannot re-introduce it.

\begin{mathpar}
\inferrule*[right=$\Box$E-Incomplete]{\Delta;\vdash\Box\phi\true}{\Delta;\Gamma\vdash\phi\true}
\end{mathpar}


The proposed rule that satisfies local reduction and local expansion is :
\begin{mdframed}
\begin{mathpar}
\inferrule*[right=$\Box$E]{{\Delta;\Gamma\vdash\Box\phi\true}\\{\Delta,\phi\valid;\Gamma\vdash\psi\true}}{\Delta;\Gamma\vdash\psi\true}
\end{mathpar}
\end{mdframed}

The negative fragment of the system is, thus, as follows:

\begin{mdframed}
\textbf{Prop Formation}
\begin{mathpar}
\inferrule*[right=Atom] { } {P_i \in {\sf Prop}}
\and
\inferrule*[right=Top] { } {\top \in {\sf Prop}}
\and
\inferrule*[right=Arr] {{\phi_1 \in {\sf Prop }}\\ {\phi_2 \in {\sf Prop}}} {\phi_1\supset\phi_2\in {\sf Prop}}


\end{mathpar}
\end{mdframed}

\begin{mdframed}
\textbf{Context $\Gamma$ Formation}
\begin{mathpar}
\inferrule*[right=Nil] { } {{\sf nil}\  \context}
\and
\inferrule*[right=$\Gamma$-Add] {{\Gamma\ } {\sf \context}  \\ {\phi \in {\sf Prop}}} {{\Gamma, \phi\true } \ \context}
\end{mathpar}
\end{mdframed}
\begin{mdframed}
\textbf{Context $\Delta$ Formation}
\begin{mathpar}
\inferrule*[right=Nil] { } {{\sf nil}\  \context}
\and
\inferrule*[right=$\Delta$-Add] {{\Delta\ } {\sf \context}  \\ {\phi \in {\sf Prop}}} {{\Delta, \phi\valid } \ \context}
\end{mathpar}
\end{mdframed}
\begin{mdframed}
\textbf{Compound $\Gamma;\Delta$ Context}
\begin{mathpar}
\inferrule*[right=$\Gamma;\Delta$-F] { {\Delta}\ {\sf \context}\\ {\Gamma \context}}
{\Turnsi {\Delta;\Gamma} {\context}}
\end{mathpar}
\end{mdframed}
\begin{mdframed}
\textbf{Context  $\Gamma$ Reflection}
\begin{mathpar}
\inferrule*[right=$\Gamma$-Refl] { {\Delta;\Gamma}\ {\sf \context}\\ {\phi \true \in \Gamma}}
{\Turnsi {\Delta;\Gamma} {\phi \true}}
\end{mathpar}
\end{mdframed}
\begin{mdframed}
\textbf{Context  $\Delta$ Reflection}
\begin{mathpar}
\inferrule*[right=$\Delta$-Refl] { {\Delta;\Gamma}\ {\sf \context}\\ {\phi \valid \in \Delta}}
{\Turnsi {\Delta;\Gamma} {\phi \true}}
\end{mathpar}
\end{mdframed}
\begin{mdframed}

\textbf{Top Introduction}
\begin{mathpar}
\inferrule*[right=$\top$I] { } {\Turnsi {\Delta;\Gamma} { \top \true}}
\end{mathpar}
\end{mdframed}
\begin{mdframed}

\textbf{Implication Introduction and Elimination}
\begin{mathpar}
\inferrule*[right=$\supset$I] {\Turnsi {\Delta;\Gamma, \phi_1 \true} {\phi_2 \true}} {\Turnsi {\Delta;\Gamma} { \phi_1\supset \phi_2 \true}}
\and
\inferrule*[right=$\supset$E] {\Turnsi {\Delta;\Gamma} {\phi_1\supset\phi_2 \true}\\{\Turnsi {\Delta;\Gamma} {\phi_1 \true}}} {\Turnsi {\Delta;\Gamma} {  \phi_2 \true}}
\end{mathpar}
\end{mdframed}
\begin{mdframed}
\textbf{Necessity Introduction and Elimination}
\begin{mathpar}
\inferrule*[right=$\Box$I]{\Delta;\nil\vdash\phi\true}{\Delta;\Gamma\vdash\Box\phi\true}
\and
\inferrule*[right=$\Box$E]{{\Delta;\Gamma\vdash\Box\phi\true}\\{\Delta,\phi\valid;\Gamma\vdash\psi\true}}{\Delta;\Gamma\vdash\psi\true}
\end{mathpar}
\end{mdframed}

\subsection{Properties of Entailment For the Judgmental Reconstruction}
The guiding transitivity principles, weakening, contraction, and exchange can be proved for the system:
\begin{mdframed}
\textbf{Transitivity}
The guiding substitution principle can be expressed as a property of this formal
system and also be proven by induction over the structure of derivations
\begin{itemize}
\item If $\Delta;\Gamma, \phi\true,\Gamma'\vdash\psi\true$ and $\Delta;\Gamma\vdash\phi\true$
then $\Delta;\Gamma,\Gamma'\vdash\psi \true$
\item If $\Delta,\phi \valid,\Delta';\Gamma\vdash\psi\true$ and $\Delta;\nil\vdash\phi \true$
then $\Delta\Delta';\Gamma\vdash\psi \true$
\end{itemize}
\textbf{Weakening, Contraction and Exchange properties can be also shown to hold.}
\end{mdframed}

\subsection{Adding proof terms}
The system can be assigned proof terms as follows:
\begin{mdframed}
\textbf{Context $\Gamma$ Formation}
\begin{mathpar}
\inferrule*[right=Nil] { } {{\sf nil}\  \context}
\and
\inferrule*[right=$\Gamma$-Add] {{\Gamma\ } {\sf \context}  \\ {\phi \in {\sf Prop}}\\x\not\in\Gamma} {{\Gamma, x:\phi} \ \context}
\end{mathpar}
\end{mdframed}
\begin{mdframed}
\textbf{Context $\Delta$ Formation}
\begin{mathpar}
\inferrule*[right=Nil] { } {{\sf nil}\  \context}
\and
\inferrule*[right=$\Delta$-Add] {{\Delta\ } {\sf \context}  \\ {\phi \in {\sf Prop}}\\s\not\in\Delta} {{\Delta, s::\phi} \ \context}
\end{mathpar}
\end{mdframed}
\begin{mdframed}
\textbf{Compound $\Gamma;\Delta$ Context}
\begin{mathpar}
\inferrule*[right=$\Gamma;\Delta$-F] { {\Delta}\ {\sf \context}\\ {\Gamma \context}}
{\Turnsi {\Delta;\Gamma} {\context}}
\end{mathpar}
\end{mdframed}
\begin{mdframed}
\textbf{Context  $\Gamma$ Reflection}
\begin{mathpar}
\inferrule*[right=$\Gamma$-Refl] {  }
{\Turnsi {\Delta;\Gamma,x:\phi,\Gamma'} {x:\phi }}
\end{mathpar}
\end{mdframed}
\begin{mdframed}
\textbf{Context  $\Delta$ Reflection}
\begin{mathpar}
\inferrule*[right=$\Delta$-Refl] {  }
{\Turnsi {\Delta,s::\phi,\Delta';\Gamma} {s:\phi }}
\end{mathpar}
\end{mdframed}
\begin{mdframed}

\textbf{Top Introduction}
\begin{mathpar}
\inferrule*[right=$\top$I] { } {\Turnsi {\Delta;\Gamma} { \langle\rangle:\top }}
\end{mathpar}
\end{mdframed}
\begin{mdframed}

\textbf{Implication Introduction and Elimination}
\begin{mathpar}
\inferrule*[right=$\supset$I] {\Turnsi {\Delta;\Gamma, x:\phi_1 } {M:\phi_2 }} {\Turnsi {\Delta;\Gamma} { \lambda x:\phi_1.M:\phi_1\supset \phi_2}}
\and
\inferrule*[right=$\supset$E] {\Turnsi {\Delta;\Gamma} {M:\phi_1\supset\phi_2 }\\{\Turnsi {\Delta;\Gamma} {N:\phi_1}}} {\Turnsi {\Delta;\Gamma} {(M N) : \phi_2 }}
\end{mathpar}
\end{mdframed}
\begin{mdframed}
\textbf{Necessity Introduction and Elimination}
\begin{mathpar}
\inferrule*[right=$\Box$I]{\Delta;\nil\vdash M:\phi}{\Delta;\Gamma\vdash\operatorname{box}(M) :\Box\phi\true}
\and
\inferrule*[right=$\Box$E]{{\Delta;\Gamma\vdash M:\Box\phi\true}\\{\Delta,s::\phi;\Gamma\vdash N:\psi}}{\Delta;\Gamma\vdash  \operatorname{let}\operatorname{box}(s)=M \operatorname{in} N: \psi}
\end{mathpar}
\end{mdframed}

\section{Computational Interpretation}
One of the possible ways to read $\Box\phi$ is as representing \emph{source} code of type $\phi$. This 
makes the $Box$ calculus given a framework for typing programs with explicit \emph{staged computation}.
Explicit staging exists in many languages. One of its most characteristic implementations is the \emph{quote} constructs in Lisp \cite{bawden1999quasiquotation}. We introduce the concept and the application of the calculus with a motivating example following \cite{PfenningCompMod}.

Consider the exponential function $\operatorname{exp}:nat\rightarrow nat\rightarrow nat$  and the two definitions
\begin{code}
exp(0) = lm x -> 1
exp(s(n)) = lm x -> x* exp n x
\end{code}
\begin{code}
exp'(0) = lm x -> 1
exp(s(n)) = let f = exp'n in lm x -> x* f x 
 \end{code}
The two functions although behaviorally equivalent have a completely different operational behavior.
For the first function applied to a $s(s(0))$ will unfold to 
\begin{code}
lm x-> x* exp(s(0)) x 
\end{code} 
the second though recurs completely on its argument unfolding  to:
\begin{code}
lm x-> x* (lm x-> x *( lm x->1) x) x
\end{code}
which after reduction under $\lambda$ can be reduced to:
\begin{code}
lm x-> x* x*1
\end{code}
We can see that the second version does a lot more computation than
the first. However, if the resulting function is applied many times, to many
different bases, then the second can be more efficient. We want to extend our language with types 
that discriminate between the two cases. 

We will try to explore this and show the how $\Box$ types can be useful to discriminate between the two.
Prior to this we have to speak about about operational semantics.
\subsection{Small Step Semantics}
Small steps semantics, is a transition system that describes how an abstract state machine would execute well typed programs expressed in a $\lambda$ calculus. Small steps semantics gives local reductions and follows a deterministic evaluation principle. Other kind of operational semantics exist (e.g big step or non-deterministic. The Church-Rosser theorem for a calculus can give a proof that all evaluation strategies are equivalent for a calculus). We also need a notion of value. That is a term that accepts no more reductions under our strategy. We work here with call-by-value semantics. That is functions in a $\lambda$ form are not further reduced and when the term is a function call the arguments are reduced to values before application.

 Here is an example of  transition system for small step semantics of the negative fragment of \ac{IPL}:
 \begin{mdframed}
 \begin{mathpar}
  \inferrule*{ } {\langle \rangle \ value }
  \and
    \inferrule*{ } {(\lambda x:\phi. M) \ value} 
    \and
 \inferrule*{ M	\ value} {(\lambda x:\phi. N) M \mapsto [M/x]N }
 
\and
 \inferrule*{{ M	\ value}\\{N \ value}} {\operatorname{fst}\langle M, N \rangle\mapsto M}
 \and
 \inferrule*{{ N	\ value}\\{N \ value}} {\operatorname{snd}\langle M, N \rangle\mapsto M}
\and
\inferrule*{M\mapsto M'} {\langle M, N\rangle\mapsto \langle M',M \rangle}
\and
  \inferrule*{{M \ value}\\{N\mapsto N'}} {\langle M, N\rangle\mapsto \langle M,N' \rangle}
 \and
 \inferrule*{M\mapsto M'} {\operatorname{fst}(M) \mapsto fst( M')}
 \and
 \inferrule*{M\mapsto M'} {\operatorname{snd}(M) \mapsto snd( M')}
 \and
 \inferrule*{M\mapsto M'} { (M N)  \mapsto ( M' N)}
 \and
  \inferrule*{ {M value} \\{N \mapsto N'}} {(M N)  \mapsto ( M N')}
 \end{mathpar}
 \end{mdframed}

Now we have \emph{preservation} and \emph{progress} property. Those are standard properties for any small step semantics transition system. They are formulated
only on closed terms because, unlike the process of proof reduction, we
only evaluate expressions that are closed.

\begin{mdframed}
\textbf{Preservation}
If $\nil\vdash M:\phi$ and $ M\mapsto M'$ then $\nil\vdash M':\phi$ \\
\textbf{Progress}
If $\nil\vdash M:\phi$ then either $\exists M'. M\mapsto M'$ or $M \ value$
\end{mdframed}

Finally we have the  \emph{weak normalization theorem} or \emph{termination theorem}. That is pertinent to the specific choice of semantics. In the simply typed lambda calculus the reduction strategy does not change the normalization property. That is \emph{strong normalization} can be shown. Moreover, from the Church--Rosser theorem and the normalization property one can deduce the existence and unicity of canonical forms.  For other systems this might not be the case.

\begin{mdframed}  
\textbf{Termination}
 If $\nil\vdash M:\phi$ then $\exists V. V \ value\  and \  M\mapsto ^{*} V$ . Where $\mapsto^{*}$ is the reflexive, transitive closure of $\mapsto$ 
 \end{mdframed}
 \subsection{Operational Semantics for Source Expressions}
 We extend the computational interpretation sketched above to encompass the necessity modality
 The interpretation  goes as follows:
 \begin{mdframed}
 \begin{tabular}{l l}
	 $x:\phi$ & $x$ stands for value of type $\phi$ \\
	$s::\phi$ & $s$ stands for a source expression $\phi$ \\
    $[\![ M/s ]\!]N$ & substitute the source expression $M$ for $s$ in $N$  \\
    $\operatorname{box} M$ & quote the closed term $M$ as a source expression \\
    $\operatorname{let}\operatorname{box} (s)=M \operatorname{in} N$ & evaluate $M$ up to the (quoted expression of the)\\ 
    & form $\operatorname{box}(M')$ and then evaluate   $[\![ M/s ]\!]N$\\ 
\end{tabular}
\end{mdframed}

We add the following values, a congruence rule and a reduction rule:
 \begin{mdframed}
 \begin{mathpar}
 \inferrule*{ } { \operatorname{box}{(M)\  value}}
 \and
 \inferrule*{M \mapsto M'}  { \operatorname{let}\operatorname{box}(s)=M \operatorname{in} N \mapsto  \operatorname{let}\operatorname{box}(u)=M' \operatorname{in} N }
 \and
  \inferrule*{  }  { \operatorname{let}\operatorname{box}(s)= \operatorname{box}(M)\operatorname{in} N \mapsto  [\![ M/s ]\!]N}
\end{mathpar}
 \end{mdframed}
 The importance of the typing is explained by Pfenning:
 \begin{quote}
 The crucial restriction of the typing rules ensures that in an expression $\operatorname{box}(M)$, the term
 $M$ does not refer to  any free variables $x$ that stand for values. It can, however, mention variables
 $s$ that stand for source expressions. So when we substitute
 $[\![N/s]\!] \operatorname{box}(M)$ then we are building a larger source expression from two smaller ones,
 $N$ and $M$.  Conversely, when we substitute a value
 $[V/x]\operatorname{box}M=\operatorname{box}(M)$ the source expression is not
 affected.
 \end{quote}
 
 Returning to our example, the system is able to discriminate between the two examples.  The first version of $\operatorname{exp}$ still has type $nat\rightarrow (nat \rightarrow nat)$ whereas the second can be rewritten in the new syntax and has type $nat\rightarrow \Box(nat\rightarrow nat)$.  It is crucial that the first version fails to be written as a source code generator since if we try to re-implement the definition as:
 \begin{code}
 exp(s(n))= box(lm x-> x* exp n x) 
 \end{code}   
 the expression is ill-typed due to the reference to the value variable $n$. 
 The second version can be written in our extension of the language as a code generator and it is well typed:
  \begin{code}
  exp'(0)= box(lm x-> 1): Box(nat -> nat)
  exp(s(n))= let box(f)= exp'(n) in box(lm x-> x* f  x): Box(nat -> nat) 
  \end{code}   
 
 The computational reading sheds new light to modal theorems as programming combinators.
 The canonical inhabitant of Axiom $4$ of modal logic (seen in the Curry--Howard fashion) is the type of the polymorphic metaprogram that quotes a quoted source code expression: 
 \begin{mathpar}
  \inferrule*{D}  { quote= \lambda x:\Box\phi. \operatorname{let}\operatorname{box}(s)= x  \operatorname{in} \operatorname{box}\operatorname{box}(x):\Box\phi\supset\Box\Box\phi }
 \end{mathpar}
 The $K$ axiom corresponds to the combinator that applies applicative source expressions and results to a larger source expression:
  \begin{mathpar}
   \inferrule*{}  {\lambda x:\Box(\phi\supset\psi).\lambda y:\Box\phi. \operatorname{let}\operatorname{box}(s)= x  \operatorname{in}  \operatorname{let}\operatorname{box}(t)=y \operatorname{in}\operatorname{box}(st):\Box(\phi\supset\psi)\supset\Box \phi\supset\Box \psi }
  \end{mathpar}
  Finally the factivity axiom corresponds to unquoting a quoted source code expression:
    \begin{mathpar}
     \inferrule*{D}  { unquote= \lambda x:\Box\phi. \operatorname{let}\operatorname{box}(s)= x  \operatorname{in} s: \Box\phi\supset\phi }
    \end{mathpar}
    
    These operations resemble monadic combinators in languages like Haskell or Scala (cf. \cite{wadler1992comprehending},\cite{Wadler:1992:EFP:143165.143169}). The connection of modality and monadic computation is thoroughly explored in \cite{kobayashi1997monad}. This discussion, pushes towards a judgmental reconstruction of the possibility modality which we will not be discussing here.
 