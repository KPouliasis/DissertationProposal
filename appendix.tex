\section{Theorems}
\begin{proposition}[Basic Facts on Context Wellformedness] 
\label{bfcw}
For every derived  judgment of the scheme $\mathcal{ CX}\vdash\mathcal{ WF}$ with $\mathcal{WF}:= {\sf wf}\  |\  {\sf \llbracket  wf \rrbracket }$ and $\mathcal{ CX}$ a metavariable ranging over lists of labeled formulae.
\begin{enumerate}
\item If $ \mathcal{ WF}= {\sf wf}$ then $\mathcal{CX}=\Gamma$ where, exclusively, either $\Gamma\in {\sf Prop}$ or $\Gamma=\bullet$, and $\Gamma\centernot\vdash \llbracket {\sf wf}\rrbracket$.
\item If $ \mathcal{ WF}= {\sf\llbracket wf\rrbracket}$ then $\mathcal{CX}=\Delta$ where exclusively, either $\Delta\in {\sf \llbracket Prop_0\rrbracket}$ or $\Delta = \circ$,  and $\Delta\centernot\vdash {\sf wf}$.
\item If $\mathcal{CX}= \Gamma, x:A$ then $ \mathcal{ WF}= {\sf wf}$, $A\in {\sf Prop}$ and,  $\Gamma\vdash {\sf wf}$.
\item If $\mathcal{CX}= \Delta, s:A $ then $ \mathcal{ WF}= { \llbracket {\sf wf} \rrbracket}$, $A\in { \llbracket {\sf Prop_0}\rrbracket}$  and $\Delta\vdash{ \llbracket {\sf wf}\rrbracket}$.
\item If $\mathcal{CX}=\llbracket \Gamma \rrbracket$ for some $\Gamma\vdash{\sf wf}$ then  $\mathcal{ WF}={\llbracket {\sf wf}\rrbracket}$ and either, $\Gamma = \bullet$ or $\Gamma \in {\sf Prop_0}$.
%\in Prop$ and   
%$Ctx\vdash WF$ is of the form $Ctx\vdash {\sf wf}$ with then $Ctx=\Gamma$ with $\Gamma \in Prop$ and $\Gamma\centernot $
\end{enumerate}
\end{proposition}
\begin{proof}
\begin{itemize}

\item
Items $1,2$ are trivial by case analysis on the derivations for context wellformedness. For items $3,4$ notice that $x,x_i$ labels can only be part of a ${\sf Prop}$ context and similarly for $s,s_1$ labels.
\item
By contradiction, assume $A\in\Gamma$ s.t. $A\not\in {\sf Prop_0}$ from $1$ it can only be $A\in {\sf Prop}$.
Hence $A\in {\sf Prop_1}$ (using item $1$ in \ref{bftu}) but then $\llbracket A \rrbracket$ is not an element of any universe (using item $8$ in \ref{bftu}) and hence it cannot be in the co--domain of any well formed context.
\end{itemize}
\end{proof}
\begin{lemma}[Guarded $\Box_{IE}$ rule application]
	All instances of the $\Box_{IE}$  rule apply with context $\Gamma^{\prime}\in {\sf Prop_0}$.
\end{lemma}
\begin{proof}
Trivially by the previous item.
\end{proof}
\begin{lemma}[Separation of Judgments]
\label{spjp}
For every judgment of the scheme $Ctx\vdash A$ with a derivation in JCalc one of the two holds exclusively:
\begin{enumerate}
\item (\textbf{${\sf Prop}$ Judgments}) $Ctx =\Gamma$  with $\Gamma\vdash {\sf wf}$ and $\Gamma,  A\in {\sf Prop}$
or, \item(\textbf{${\sf\llbracket  Prop_0\rrbracket}$ Judgments}) $Ctx=\Delta$ with $\Delta\vdash {\sf \llbracket wf\rrbracket}$ and $\Delta,  A \in \llbracket {\sf Prop_0}\rrbracket$
\end{enumerate}
\end{lemma}
\begin{proof}
By induction on the derivations of this scheme. Using some obvious inversion facts such as 
if $\llbracket   A_1\supset   A_2\rrbracket \in \llbracket{\sf Prop_0}\rrbracket$ then $  A\supset   A_2\in {\sf Prop_0}$
and whence, $  A_2 \in {\sf Prop_0}$ and $\llbracket   A_2 \rrbracket \in \llbracket {\sf Prop_0}\rrbracket$ and facts in \ref{bfcw}.
\end{proof}
\begin{theorem}[Deduction Theorem for Validity Judgments]
	\label{deduct}
With $\Delta\vdash \llbracket\sf wf\rrbracket$, if $\Delta, s:\llbracket   A\rrbracket \vdash\llbracket B \rrbracket $ then $\Delta \vdash\llbracket   A \supset B \rrbracket $. 
\end{theorem}
\begin{proof}
The proof is essentially the deduction theorem for a Hilbert style formulation of the corresponding fragment of  propositional logic and we do not show it here for economy. Note that this theorem cannot be proven without the logical specification {\sf $Ax_1$, $Ax_2$}. I.e. it is exactly the requirements of the logical specification that ensure that all  interpretations  should adequately embed intuitionistic implication.
\end{proof}
\begin{lemma}[$\llbracket\cdot\rrbracket$Lifting Lemma]
\label{bracklift}
Given any wellformed context of assumptions $\Gamma \vdash {\sf wf}$ and $\Gamma,  A \in {\sf Prop_0}$ then $\Gamma\vdash   A \Longrightarrow \llbracket \Gamma\rrbracket \vdash \llbracket   A \rrbracket$. 
\end{lemma}

\begin{proof}
The proof goes by induction on the derivations trivially for all the cases($\supset_E$ is treated using the ${\sf App}$ rule that internalizes Modus ponens). For the $\supset I$ the previous theorem has to be used.
\end{proof}
\begin{lemma}[$\Box$Lifting Lemma]
\label{boxlift}
For $\Gamma,   A \in {\sf Prop_0}$, then $\Gamma\vdash   A$ implies $\Box\Gamma \vdash \Box   A$.
\end{lemma}
\begin{proof}
 Assuming a derivation $\mathcal {D}$ ::$\Gamma\vdash   A$ from \ref{bracklift} there exists corresponding validity derivation $\mathcal{E}::\llbracket\Gamma\rrbracket\vdash\llbracket   A \rrbracket$. Using the two as premises in the $\Box_{IE}$ with $\Gamma := \Box \Gamma$ we obtain $\Box\Gamma\vdash\Box   A$.
\end{proof}
From the previous we get:
Let us show an inverse principle to the $\Box$ Lifting Lemma. We define for $A$ in {\sf Prop}:
\begin{flalign*}
\nonumber \downharpoonright P_i\ & =  P_i \\ \downharpoonright (A_1\supset A_2)&  =  \downharpoonright A_1 \supset \downharpoonright A_2 \\
\downharpoonright \Box A & = \downharpoonright A
\end{flalign*}
And the lifting of the $\downharpoonright$ over $\Gamma\in {\sf Prop}$. We get:
\begin{theorem}[Collapse $\Box$ Lemma] If $\Gamma\vdash A$ for $\Gamma,A \in {\sf Prop}$ then $\downharpoonright \Gamma\vdash \downharpoonright$ A.
\end{theorem}
\begin{theorem}[Weakening]
For the N.D. system of JCalc, with $\Gamma,\Gamma^{\prime}\vdash \sf wf$ and $\Delta,\Delta^{'}\vdash\llbracket {\sf wf}\rrbracket $.
\begin{enumerate}
\item If  $\Gamma\vdash   A$ then $\Gamma,\Gamma'\vdash   A$.
\item If  $\Delta\vdash \llbracket   A\rrbracket$ then $\Delta,\Delta^{'} \vdash \llbracket   A\rrbracket$.
\end{enumerate}

\end{theorem} 
\begin{proof}
By induction on derivations.
\end{proof}
\begin{theorem}[Contraction]
For the N.D. system of JCalc, with $\Gamma,x:A,x':A,\Gamma'\vdash \sf wf$ and $\Delta,s:\llbracket A \rrbracket, s':\llbracket A\rrbracket,\Delta^{'}\vdash\llbracket {\sf wf}\rrbracket $.
\begin{enumerate}
\item If  $\Gamma,x:A,x':A,\Gamma'\vdash  B$ then $\Gamma,x:A,\Gamma'\vdash   B$.
\item If  $\Delta,s:\llbracket A \rrbracket, s':\llbracket A\rrbracket,\Delta^{'}\vdash\llbracket B \rrbracket $ then $\Delta,s:\llbracket A \rrbracket, \Delta^{'} \vdash \llbracket   B\rrbracket$.
\end{enumerate}  
\end{theorem}
\begin{proof}
By induction on derivations.
\end{proof}
\begin{theorem}[Permutation]
For the N.D. system of JCalc, with $\Gamma\vdash \sf wf$ and $\Delta\vdash\llbracket {\sf wf}\rrbracket $ and $\pi \Gamma$ and $\pi \Delta$ the collection of well-formed contexts of assumptions with the same co-domain of $\Gamma$, $\Delta$ we get
\begin{enumerate}
\item If  $\Gamma\vdash   A$ and $\Gamma'\in \pi{\Gamma}$ then $\Gamma'\vdash   A$.
\item If  $\Delta\vdash \llbracket   A\rrbracket$ and $\Delta^{'}\in \pi \Delta$ then  $ \Delta'\vdash \llbracket   A\rrbracket$.
\end{enumerate}
\end{theorem}
\begin{proof}
By induction on derivations.
\end{proof}
\begin{theorem}[Substitution Principle]
The following hold for both kinds of judgments:
\begin{enumerate}
\item If  $\Gamma,x:A\vdash B$ and $\Gamma\vdash  A$ then $\Gamma\vdash B$ 
\item If  $\Delta,s:\llbracket A \rrbracket \vdash \llbracket B\rrbracket$ and$\Delta\vdash  \llbracket A\rrbracket$ then  $\Delta\vdash\llbracket B\rrbracket$ 
\end{enumerate}
\end{theorem}
All previous $9$ theorems can be stated for proof terms too. We should discuss the following:
\begin{theorem}[Deduction Theorem / Emulation of $\lambda$ abstraction]
	\label{deductterms}
	With $\Delta\vdash \llbracket\sf wf\rrbracket$, if $\Delta, s:\llbracket   A\rrbracket \vdash {\sf j}:\llbracket B \rrbracket $ then there exists {\sf j'} s.t. $\Delta \vdash {\sf j'}:\llbracket   A \supset B \rrbracket $. 
\end{theorem}
\begin{lemma}[$\llbracket\cdot\rrbracket$Lifting Lemma for terms]
\label{highorder}
	If $\Gamma A \in {\sf Prop_0}$ and $\Gamma\vdash M: A$ then there exists ${\sf j}$ s.t. $\llbracket \Gamma\rrbracket \vdash {\sf j}:\llbracket   A \rrbracket$. 
\end{lemma}
In both theorems the existence of this ${\sf j,j'}$ is algorithmic following the induction proof. 
\section{Linking on the function space}
The above mentioned algorithms permit  for translating $\lambda$ abstractions to polynomials of $S,K$ combinators which is a standard result in the literature. We do not give the details here but the translation is  syntax driven as it can be seen by the inductive nature of the proofs.

Henceforth, we can generalize the construction in \ref{dlinker} so that it permits for dynamic linking of functions of the client (with missing implementations) such as  $\mathtt{\lambda n:int. push\  n\  empty}$ dynamically given that the host actually implements a higher-order function space (that is it implements the combinators $S,K$ in, say, own lambda calculus $\lambda^{J}$). Given implementations of $\mathtt{push\_impl}$, $\mathtt{empty\_impl}$ the linker produces an application expression consisting of $\mathtt{push\_impl}$, $\mathtt{empty\_impl}$, $S$ and $K$.   The execution of the target expression will happen in the host after dereferencing  ${\sf push\_impl, empty\_impl}$ (dynamic part) and the combinators $S,K$ (constant part) as, say, lambdas (e.g. $K=\lambda^{J} x.\lambda^{J} y. x$).

\section{Gentzen's reduction Principle for $\Box$(General)}
\label{redmult}
\mbox{\footnotesize
	\begin{mathpar}
		\inferrule*[right=$  I_{\Box B} E_{\Box   A}^{x,s}$]{
			\inferrule*{}
			{\inferrule*[]{}{ \inferrule*[]	{\inferrule*[]{}{\PrTri{$D_1$} \\ \PrTri{$E_1$}}\\\\
						\inferrule*[]{}{ \ A_1\\ \ \qquad\ \ \  \llbracket   A_1\rrbracket}}{\Box   A_1 }}}\ldots 
			{\inferrule*[]{}{ \inferrule*[]	{\inferrule*[]{}{\PrTri{$D_i$} \\ \PrTri{$E_1$}}\\\\
						\inferrule*[]{}{\    A_i\\ \ \qquad \llbracket   A_i\rrbracket}}{\Box   A_i }}}
			\quad
			\inferrule*{}
			{\inferrule*[vdots=1.0em, right=$\vec{x}$]{ }{ A_1\dots A_i}\\\\
				\inferrule*[]{}{B}} 	   \qquad 
			\inferrule*{}
			{\inferrule*[vdots=1.0em, right=$\vec{s}$]{ }{\llbracket   A_1 \ldots A_i \rrbracket}\\\\
				\inferrule*[]{}{\llbracket B\rrbracket}} 	
		}
		{\Box B} 
	\end{mathpar}
}
$$\Longrightarrow_{R}$$
\mbox{\footnotesize
	\begin{mathpar}
		\inferrule*[right=$I_{\Box B}$]{
			\inferrule*{}
			{\inferrule*[vdots=1.0em]{}{\PrTri{$D_1$}\ \PrTri{$D_i$} \\\\ A_1\ldots\ldots  \ \  A_i}\\\\
				\inferrule*[]{}{B}}
			\qquad 
			\inferrule*{}
			{\inferrule*[vdots=1.0em]
				{}{\PrTri{$E_1$} \PrTri{$E_i$}\\\\\llbracket   A_1\ldots\ldots A_i\rrbracket}\\\\
				\inferrule*[]{}{\llbracket B \rrbracket}} 
		}{\Box B}
		
	\end{mathpar}
}
\section{Notes on the cut elimination proof and normalization of natural deduction}
\label{norm}
Standardly, we add the bottom type and elimination rule in the natural deduction and show that in Jcalc + $\bot$: $\centernot\vdash\bot$. The addition goes as follows:

\begin{mathpar}
\inferrule*[right= Bot] { } {\bot \in {\sf Prop_0}}	
\and
\inferrule*[right= $E_\bot$] {{\Gamma\vdash\bot }\\ A\in {\sf Prop}} {\Gamma \vdash A}
\end{mathpar}
Our proof strategy follows directly \cite{pfenning2004automated}. We construct an intercalation calculus \cite{sieg1998normal} corresponding to the ${\sf Prop}$ fragment  with the following two judgments:
\begin{itemize}
	\item[] $A\Uparrow$ for ``Proposition $A$ has normal deduction".
	\item[] $A^\downarrow$ for ``Proposition $A$ is extracted from hypothesis".
\end{itemize}
This calculus is, essentially, restricting the natural deduction to canonical derivations. The $\llbracket {\sf judgments} \rrbracket$ are not annotated and are directly ported from the natural deduction since we observe consistency in ${\sf Prop}$. 
The construction is identical to \cite{pfenning2004automated} (Chapter 3) for the ${\sf Hypotheses},{\sf Coercion},\supset, \bot$ cases, we add the $\Box$ case.
\begin{mathpar}
	\inferrule*[right=$\Gamma$-hyp]  {x: A\downarrow \in \Gamma^\downarrow}{ \Gamma^\downarrow\vdash^{-} A\downarrow}
	\and
	\inferrule*[right=$\downarrow\Uparrow$] {\Gamma^\downarrow\vdash^{-} A\downarrow}{\Gamma^\downarrow\vdash^{-} A \Uparrow}
	\and
	\inferrule*[right=$\supset$I$^{x}$] {\Gamma^\downarrow, x: A\downarrow\vdash^{-}  B\Uparrow} {\Gamma^\downarrow \vdash^{-}  A\supset  B\Uparrow}
	\inferrule*[right=$\supset$E] {{\Gamma^\downarrow\vdash^{-} A\supset  B \downarrow}\\{\Gamma^\downarrow\vdash^{-}  A\Uparrow}} {\Gamma^\downarrow\vdash^{-}   B\downarrow}
	%\and
	%\inferrule*[right=$\bot$E] {{\Turn {\Gamma} {\bot}}}{\Turn {\Gamma} {   A}}
\and
\inferrule*[right= $E_\bot$] {{\Gamma^{\downarrow}\vdash^{-}\bot\downarrow }\\ A\in {\sf Prop}} {\Gamma^{\downarrow}\vdash^{-} A\Uparrow}
\and
 	\inferrule*[right=$\Box_{IE}$ ] {{\Gamma\downarrow\vdash A\Uparrow }\\ {\llbracket \Gamma \rrbracket\vdash \llbracket A \rrbracket}}{ {\Box\Gamma\downarrow\vdash \Box A\Uparrow }}
 \end{mathpar}
We prove simultaneously by induction:
\begin{theorem}[Soundness of Normal Deductions]
The following hold:
\begin{enumerate}
\item If $\Gamma^\downarrow\vdash^{-} A\Uparrow$ then $\Gamma\vdash A$, and
\item If $\Gamma^\downarrow\vdash^{-} A\downarrow $ then $\Gamma\vdash A$.
\end{enumerate}
\end{theorem}
\begin{proof}
	Simultaneously by induction on derivations.
\end{proof}
It is easy to see that this restricted proof system $\centernot\vdash^{-} \bot\Uparrow$. It is hard to show its completeness to the non-restricted natural deduction ($\vdash + \bot_E$ of Jcalc) directly. For that reason we add a rule to make it complete ($\vdash^{+}$) preserving soundness and get a system of Annotated Deductions. We show the correspondence of the restricted system ($\vdash^{-}$) to a cut-free sequent calculus (${\sf JSeq}$), the correspondence of the extended system ($\vdash^{+}$) to ${\sf Jseq + Cut}$ and show cut elimination.\footnote{ In reality, the sequent calculus formulation is built exactly upon intuitions on the intercalation calculus. We refer the reader to the references.}

To obtain completeness we add the rule:
 \begin{mathpar}
 	\inferrule*[right=$\Uparrow\downarrow$] {\Gamma^\downarrow\vdash A\Uparrow} {\Gamma^\downarrow\vdash A\downarrow }
 \end{mathpar}
We define $\vdash^{+} :=\   \ \ \vdash^{-} {\sf with} {\ \sf \Uparrow\downarrow}{\sf Rule}$.
We show:
\begin{theorem}[Soundness of Annotated Deductions]
	The following hold:
	\begin{enumerate}
		\item If $\Gamma^\downarrow\vdash^{+} A\Uparrow$ then $\Gamma\vdash A$, and
		\item If $\Gamma^\downarrow\vdash^{+} A\downarrow $ then $\Gamma\vdash A$.
	\end{enumerate}
\end{theorem}
\begin{proof}
	As previous item.
\end{proof}

\begin{theorem}[Completeness of Annotated Deductions]
	\label{compannot}
	The following hold:
	\begin{enumerate}
		\item If $\Gamma\vdash A$ then, $\Gamma\downarrow\vdash^{+} A\Uparrow$, and
		\item If $\Gamma\vdash A$, then $\Gamma\downarrow\vdash^{+} A\downarrow$.
	\end{enumerate}
\end{theorem}
\begin{proof}
	By induction over the structure of the $\Gamma\vdash A$ derivation.
\end{proof}

Next we move with devising a sequent calculus formulation corresponding to normal proofs $\Gamma^{\downarrow}\vdash^{-}A\Uparrow$. The calculus that is given in the main body of this theorem. We repeat it here for completeness.
\begin{mdframed}[nobreak=true,frametitle={\footnotesize Sequent Calculus ($\llbracket {\sf Prop_0} \rrbracket$)}]
	$$\begin{array}{l r}
	\Delta \Rightarrow \llbracket A\rrbracket:= & \exists \Delta'\in \pi(\Delta)\ \text{s.t} \   
	\Delta'\vdash \llbracket A \rrbracket \end{array}$$
	where $\pi(\Delta)$ is the collection of wellformed  $\llbracket {\sf Prop_0} \rrbracket$ contexts $\Delta'\vdash \llbracket {\sf wf}\rrbracket$  with some permutation of the multiset $\Delta$ as co--domain.
\end{mdframed} 


\begin{mdframed}[nobreak=true,frametitle={\footnotesize Sequent Calculus ({\sf Prop})}]
	\mbox{\small
		\begin{mathpar}
			\inferrule*[right=$Id$] { }{\Gamma, A  \Rightarrow A }
			
			\inferrule*[right=$\supset_L$] {{\Gamma, A\supset B, B \Rightarrow  C }\\ {\Gamma, A\supset B \Rightarrow A}} {\Gamma, A\supset B \Rightarrow  C}
			\and
			\inferrule*[right=$\supset_R$] {\Gamma, A \Rightarrow  B} {\Gamma \Rightarrow A\supset B}
			\and
			\inferrule*[right=$\bot_L$] { } {\Gamma, \bot \Rightarrow A}
			\and
			\inferrule*[right=$\Box_{LR}$] {{\Box\Gamma',\Gamma'\Rightarrow A}\\{\llbracket\Gamma'\rrbracket\Rightarrow \llbracket A \rrbracket } \\{\Gamma \in{\sf Prop}}}{\Gamma,\Box\Gamma'\Rightarrow \Box A}
			%\inferrule*[right=$\supset$E] {{\Turn {\Gamma} { A\supset  B}}\\{\Turn {\Gamma} { A}}} {\Turn {\Gamma} {   B}}
			%\and
			%\inferrule*[right=$\bot$E] {{\Turn {\Gamma} {\bot}}}{\Turn {\Gamma} {   A}}
		\end{mathpar}}
		%Where the  rule $\Box_{LR}$corresponds to $\Box_{IE}$ and relates the two kinds of sequents 
	\end{mdframed}
We want to show correspondence of the sequent calculus  w.r.t normal proofs ($\vdash^{-}$).  Two lemmas are required to show soundness. 
\begin{lemma}[Substitution principle for extractions]
The following hold:
\begin{enumerate}
\item If $\Gamma_1^\downarrow, x:A^\downarrow,\Gamma_2^\downarrow\vdash^{-} B\Uparrow$ and\\$\Gamma_1^\downarrow\vdash^{-} A\Uparrow$ then  $\Gamma_1^\downarrow,\Gamma_2^\downarrow\vdash^{-} B\Uparrow$
\item  If $\Gamma_1^\downarrow, x:A^\downarrow,\Gamma_2^\downarrow\vdash^{-} B\downarrow$ and $\Gamma_1^\downarrow\vdash^{-} A\downarrow$ then $\Gamma_1^\downarrow,\Gamma_2^\downarrow\vdash^{-} B\Uparrow$    
\end{enumerate}
\end{lemma}
\begin{proof}
Simultaneously by induction on the derivations $A\downarrow$ and $A\Uparrow$.
\end{proof}
And making use of the previous we can show, with ($\downharpoonright A$ defined previously):
\begin{lemma}[Collapse principle for normal deductions]
The following hold:
\begin{enumerate}
		\item If $\Gamma^\downarrow,\vdash^{-}  A\Uparrow$ then $\downharpoonright\Gamma^\downarrow\vdash^{-} \downharpoonright A\Uparrow$   and,
		\item If $\Gamma^\downarrow\vdash^{-} A\downarrow$ then  $\downharpoonright\Gamma^\downarrow\vdash^{-} \downharpoonright A\downarrow$   
	\end{enumerate}
\end{lemma}
Using the previous lemmas and by induction we can show :
\begin{theorem}[Soundness of the Sequent Calculus] 
	\label{soundnseq}
If   $\Gamma\Rightarrow B$ then $\Gamma^\downarrow\vdash^{-} B\Uparrow$.

	
\end{theorem}
\begin{theorem}[Soundness of the Sequent Calculus with Cut] 
	
	If   $\Gamma\Rightarrow^{+} B$ then $\Gamma^\downarrow\vdash^{+} B\Uparrow$.
\end{theorem}

Next we define the $\Gamma\Rightarrow^{+} A$ as $\Gamma\Rightarrow A$ plus the rule:
\begin{mathpar}
\inferrule*[right=Cut]{{\Gamma\Rightarrow^{+} A}\\{\Gamma,A\Rightarrow^{+}B}}{\Gamma\Rightarrow^{+}B}
\end{mathpar}
\begin{proof}
As before. The cut rule case is handled by the $\Uparrow\downarrow$ and substitution for extractions principle showcasing that the correspondence of the cut rule to the coercion from normal to extraction derivations.
\end{proof}
Standard structural properties (\textit{Weakening, Contraction}) to show completeness. We do not show these here but they hold.
\begin{theorem}[Completeness of the Sequent Calculus] 
	\label{compseqcalc}
	The following hold:
	\begin{enumerate}
		\item If   $\Gamma^\downarrow\vdash^{-} B\Uparrow$ then $\Gamma\Rightarrow B$ and,
		\item 	If $\Gamma^\downarrow \vdash^{-} A\downarrow$ and $\Gamma,A\Rightarrow B$ then $\Gamma\Rightarrow B$   
	\end{enumerate}
\begin{proof}
	Simultaneously by induction on the given derivations making use of the structural properties.
\end{proof}
Similarly we show for the extended systems.
\begin{theorem}[Completeness of the Sequent Calculus with Cut] The following hold:
	\label{compseqcut}
	\begin{enumerate}
		\item If   $\Gamma^\downarrow\vdash^{+} B\Uparrow$ then  $\Gamma\Rightarrow^{+} B$ and,
		\item 	If $\Gamma^\downarrow \vdash^{+} A\downarrow$ and $\Gamma,A\Rightarrow^{+} B$ then $\Gamma\Rightarrow^{+} B$.   
	\end{enumerate}
\end{theorem}
\begin{proof}
	As before. The extra case is handled by the Cut rule.
\end{proof}
\end{theorem}
After establishing the correspondence of $\vdash^{-}$ with $\Rightarrow$ and of $\vdash^{+}$ with $\Rightarrow^{+}$ we move on with:
\begin{theorem}[Admissibility of Cut]
	If $\Gamma\Rightarrow A$ and $\Gamma,A\Rightarrow B$ then $\Gamma\Rightarrow B$.
\end{theorem}
The proof is by double induction on the structure of the formula, its (sub-)derivations. This gives easily:
\begin{theorem}[Cut Elimination]
	If $\Gamma\Rightarrow^{+}A$ then $\Gamma\Rightarrow A$.
	
\end{theorem}
Which in turn gives us:
\begin{theorem}[Normalization for Natural Deduction]
	\label{normalization}
	If $\Gamma\vdash A$ then $\Gamma^{\downarrow}\vdash^{-} A\Uparrow$
\end{theorem}
\begin{proof}
	From assumption $\Gamma \vdash A$ which by \ref{compannot} gives $\Gamma\vdash^{+} A\Uparrow$. By \ref{compseqcut} and Cut  Elimination we obtain $\Gamma\Rightarrow A$ which by  \ref{soundnseq} completes the proof.
\end{proof}
As a result we obtain:
\begin{proof}
	By contradiction, assume $\vdash\bot$ then $\Rightarrow \bot$ which is not possible.
\end{proof}