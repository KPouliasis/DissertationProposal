\chapter{Extending JCalc to Dependent typing}
\newcommand{\Turn}[2]
{ {#1}\vdash_{\textbf{\sf IPC}}  {#2}}
\newcommand{\TurnTwo}[2]
{ {#1}\vdash_{\textbf{\sf J}}  {#2}}
\newcommand{\TurnT}[2]
{ \Delta_0;{#1}\vdash  {#2}}
\newcommand{\TurnTT}[2]
{ \Delta_0;{#1}\vdash_{\sf JC_1}  {#2}}
\newcommand{\Turnj}[1]
{ \Delta_0\vdash_{\sf J_0}  {#1}}
\newcommand{\Turnjc}[3]
{ {#1};{#2}\vdash_{\textbf{\sf JC}}  {#3}}
% A plausible reading of  G\"{o}del's incompleteness results (\cite{citeulike:713002})  is that the notion of ``validity" diverges from that of ``truth within a specific theory": given a theory that includes enough arithmetic, there are statements whose validity can only be established in a theory of larger proof-strength.  This phenomenon can be shown even with non-G\"{o}delian arguments in the relation e.g.  between ${\sf I\Delta_0}$ and ${\sf I\Sigma_1}$ arithmetic \cite{parikh:existence}, ${\sf I\Sigma_1}$ and {\sf PA}, {\sf PA} and {\sf ZF},  etc. \cite{Smith05anintroduction,Buss199879}. The very same issues arise in automated theorem proving. A good example is given by type systems and interactive theorem provers (e.g. Coq, Agda) of the typed functional paradigm. In such systems, when termination of functions has to be secured, one might need to invoke stronger proof principles. 
% The need for reasoning about two kinds of proof objects within a type system is apparent most of all when one wants to establish non-admissibility results for a theory $T$ that can, in contrast, be proved in some stronger $T'$. The type system, then, has to reconcile the existence of  a proof object of some type $\phi$ in some $T'$  and a proof object of type $\neg \exists s. Prov_{T}(s,\phi)$ that witnesses the non-provability of  $\phi$ (in $T$). 

% In this work, we argue that the explicit modality of Justification Logic \cite{DBLP:conf/jelia/Artemov08} can be used to axiomatize relations between objects of two different calculi such as those mentioned above.  It is well known that the provability predicate can be axiomatized using a modality \cite{citeulike:214701}, \cite{ArtBek05HPL}. The Logic of Proofs {\sf LP} \cite{Art94APAL} goes further and provides explicit proof terms (\textit{proof polynomials}) to inhabit judgments on validity. By translating reasoning in Intuitionistic Propositional Calculus ({\sf IPC}) to classical proofs, {\sf LP} obtains a classical semantics for {\sf IPC} through a modality (inducing a {\sf BHK} semantics). In this paper we axiomatize the relation between the two kinds of proof objects explicitly, by creating a modal type theory that reasons about bindings or linking of objects from two calculi: a lower-level theory $T$, formulated as ${\sf IPC}$ with Church-style $\lambda$-terms representing intuitionistic proof objects; and a higher-level, possibly stronger and classical (co-)theory $T^\prime$ fixed as foundational, with \textit{justifications} expressing its proof objects. The axiomatization of such a (co-)theory follows directly the proof system of Justification Logic (here restricted to its applicative $K$-fragment) and is used to interpret classically (meaning \textit{truth-functionally}) the constructions of the intuitionistic natural deduction. The underlying principle of our linking system is as follows: $$constructive\ \  necessity= admissible \ \ validity = truth\  (in\  T) + validity\  (in\  T')$$ 

% Necessity of a true (in $T$) proposition $P$ is, thus, sensitive to the existence of a proof (witnessed by a justification) of its intended interpretation within $T'$. We assume an interpretation function on types $Just$ that maps the type universe of $T$ into the type universe of $T'$.  We employ judgments of the kind $M:P$ (read as ``$M$ is a proof of type $P$ in $T$'') that represent truths in $T$ and  judgments of the kind $j:{\sf Just}\  P$ (to be read as "$j$ is a justification of the interpretation of $P$ in $T^\prime$") that represent truth in $T'$ (validity). Incorporating them, the principle can be rewritten in a judgmental fashion: $$M:P + j: {\sf Just}\  P \Rightarrow {\sf \Box^{j}}P\ true$$ Notice that the $\Box$-types are indexed by justifications (${\sf\Box^{j}}P$) being sensitive to the interpretation ($T^{\prime}$) chosen.   
% To complete the picture we need canonical elements of ${\sf \Box^{j}}$-types. Naturally, witnesses of this kind are \textit{links} between proof objects from $T$ and $T'$ with corresponding types ($P$ and ${\sf Just \ P}$). For that reason we  introduce a \textit{linking witness} constructor {\sf $Link$}. This is how necessity is introduced: by proof-checking deductions of $T$ with deductions of $T'$,  we reason constructively about admissibility of valid (via $T'$) statements in $T$.
% The principle thus becomes: 

% $$M:P + j: {\sf Just}\   P \Rightarrow Link(M,j):{\sf \Box^{j}}P$$
% We show how this principle is admissible in our system.

% A possible application of the presented type theory can be a refined type system for programming languages with modular programming constructs or external function calls as we show in section \ref{sec:module}. In these kinds of languages (e.g. of the {\sf ML} family) a program or module can call for external definitions that are implemented elsewhere (in another module or, even in another language)\footnote{See \cite{Harper98programmingin}.}. We can read functions within $\Box$-types indexed by justifications as  linking  processes for such languages that perform  the mapping of well--typed constructs importing and using module signatures into their residual programs. By residual programs we mean programs where all instances of module types and  function calls are replaced by (i.e. \textit{linked} to) their actual implementations, which remain hidden in the module. We show with a real example how, with slight modifications, our type system can find a natural application in this setting. Here we focus on the type system itself and not on its operational semantics. 


% The backbone of this work is the idea of representing the proof theoretic semantics for {\sf IPC} through modality that stems from \cite{Art01BSL},\cite{Art02CSLI}.  An operational approach to  modality related to this work can be found in \cite{Art95TR}. The modularity of {\sf LP}, i.e. its ability to realize other kinds of modal reasoning with proper changes in the axiomatization of proof polynomials, was shown with the development of the family of Justification Logics \cite{DBLP:conf/jelia/Artemov08}. This ability is easily seen to be preserved here. Our work incorporates the rich type system and modularity of Justification Logic within the proofs-as-programs doctrine. For that reason, we obtain an  extension of the Curry-Howard correspondence (\cite{Sorensen98lectureson}, \cite{citeulike:993095}) and adopt the judgmental approach of Intuitionistic Type Theory (\cite{inp:martin-loef79a}, \cite{martin-lof84:inttt}, \cite{citeulike:5251552}, \cite{citeulike:2310446}, \cite{awodey:kripke}). Our system borrows from other modal calculi developed  within the judgmental approach  (e.g. \cite{citeulike:5447115}, \cite{Goubault-Larrecq96oncomputational},\cite{Benaissa99logicalmodalities} and especially \cite{Bellin01extendedcurry-howard} for the modal logic {\sf K}). A main difference of our system with those systems, as well as with previous $\lambda$-calculi for {\sf LP} (\cite{AA00}, \cite{ArtBon07LFCS}) is that our type system  hosts a two-kinded typing relation for proof objects of corresponding formulae. It can be viewed as an attempt to add proof terms for validity judgments as presented in \cite{citeulike:5447115}.  The resulting type system adopts dependent typing (\cite{citeulike:4846}, \cite{Norell08dependentlytyped}) to relate the two kinds of proof objects with modality.  The construction of the type universe as well as of justificational terms draws a lot from ideas in \cite{Artemov:2012:OJL:2317882.2317912}  and from \cite{FittingManuscript-FITTLO}. Extending typed modal calculi with additional (contextual) terms of dependent typing can be also found in \cite{Nanevski:2008:CMT:1352582.1352591}. 

