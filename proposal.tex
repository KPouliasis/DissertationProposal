\documentclass[12pt]{report}

\usepackage{lmodern}
\usepackage{microtype}

\usepackage{fancyvrb}
\DefineVerbatimEnvironment{code}{Verbatim}{fontsize=\small}
% Package for customizing page layout

% \usepackage{fullpage}


\usepackage{amsmath,amsthm,amssymb}
\usepackage{mathtools}
\usepackage{etoolbox}
\usepackage{fancyhdr}
\usepackage{mathpartir}
\usepackage{xcolor}
\usepackage{hyperref}
\usepackage{xspace}
\usepackage{comment}
\usepackage{graphicx}
\usepackage{tikz-cd}
\usepackage{centernot}
\usepackage{listings}
\usetikzlibrary{shapes.geometric}
\usepackage[framemethod=TikZ]{mdframed}
\mdfdefinestyle{MyFrame}{%
    linecolor=blue,
    outerlinewidth=2pt,
    roundcorner=20pt,
    innertopmargin=\baselineskip,
    innerbottommargin=\baselineskip,
    innerrightmargin=20pt,
    innerleftmargin=20pt,
    backgroundcolor=gray!50!white}

\usepackage{url} % for url in bib entries
% for theorems, lemmas, etc
\newenvironment{theorem}[1][Theorem.]{\begin{trivlist}\item[\hskip \labelsep {\bfseries #1}]}{\end{trivlist}}
\newenvironment{lemma}[1][Lemma.]{\begin{trivlist}\item[\hskip \labelsep {\bfseries #1}]}{\end{trivlist}}
\newenvironment{corollary}[1][Corollary.]{\begin{trivlist}\item[\hskip \labelsep {\bfseries #1}]}{\end{trivlist}}
\newenvironment{definition}[1][Definition.]{\begin{trivlist}\item[\hskip \labelsep {\bfseries #1}]}{\end{trivlist}}
%%%%%%%%%%%%%%%%%%%%%%%%%%%%%%%%%%%%%%%%
% Acronyms
%%%%%%%%%%%%%%%%%%%%%%%%%%%%%%%%%%%%%%%%
\usepackage[acronym, shortcuts]{glossaries}

\newacronym{HoTT}{HoTT}{\emph{Homotopy Type Theory}}
\newacronym{IPL}{\sf{IPL}}{\emph{Intuitionistic Propositional Logic}}
\newacronym{LP}{\sf{LP}}{\emph{The Logic of Proofs}}
\newacronym{JL}{\sf{JL}}{\emph{Justification Logic}}
\newacronym{TT}{TT}{intuitionistic type theory}
\newacronym{LEM}{\sf{LEM}}{law of the excluded middle}
\newacronym{ITT}{ITT}{intensional type theory}
\newacronym{ETT}{ETT}{extensional type theory}
\newacronym{NNO}{NNO}{natural numbers object}
\newacronym{BHK}{\sf{BHK}}{\emph{Brower-Heyting-Kolmogorov}}
\newacronym{CHI}{\sf {CHI}}{\emph{Curry--Howard Isomorphism}}


% Make \ac robust.
\robustify{\ac}

%%%%%%%%%%%%%%%%%%%%%%%%%%%%%%%%%%%%%%%%
% Fancy page style
%%%%%%%%%%%%%%%%%%%%%%%%%%%%%%%%%%%%%%%%
\pagestyle{fancy}
\newcommand{\metadata}[2]{
  \lhead{}
  \chead{}
  \rhead{\bfseries Homotopy Type Theory}
  \lfoot{#1}
  \cfoot{#2}
  \rfoot{\thepage}
}
\renewcommand{\headrulewidth}{0.4pt}
\renewcommand{\footrulewidth}{0.4pt}


\newrobustcmd*{\vocab}[1]{\emph{#1}}
\newrobustcmd*{\latin}[1]{\textit{#1}}

%%%%%%%%%%%%%%%%%%%%%%%%%%%%%%%%%%%%%%%%
% Customize list enviroonments
%%%%%%%%%%%%%%%%%%%%%%%%%%%%%%%%%%%%%%%%
% package to customize three basic list environments: enumerate, itemize and description.
\usepackage{enumitem}
\setitemize{noitemsep, topsep=0pt, leftmargin=*}
\setenumerate{noitemsep, topsep=0pt, leftmargin=*}
\setdescription{noitemsep, topsep=0pt, leftmargin=*}

%%%%%%%%%%%%%%%%%%%%%%%%%%%%%%%%%%%%%%%%
% Some really basic macros.
% (Lots of them were stolen from HoTT/Book.)
% See macros.tex in HoTT/book.
%
% This is a mess.  Needs clean-ups.
%%%%%%%%%%%%%%%%%%%%%%%%%%%%%%%%%%%%%%%%
\newcommand{\Turnsi}[2]
	{ {#1}\vdash  {#2}}

\newcommand{\Turn}[2]
	{ {#1}\vdash_{\textbf{\sf IPC}}  {#2}}
\newcommand{\TurnTwo}[2]
	{ {#1}\vdash_{\textbf{\sf J}}  {#2}}
\newcommand{\TurnT}[2]
	{ \Delta_0;{#1}\vdash  {#2}}
\newcommand{\TurnTT}[2]
	{ \Delta_0;{#1}\vdash_{\sf JC_1}  {#2}}
\newcommand{\Turnj}[1]
	{ \Delta_0\vdash_{\sf J_0}  {#1}}
\newcommand{\Turnjc}[3]
    { {#1};{#2}\vdash_{\textbf{\sf JC}}  {#3}}
\newrobustcmd*{\ctx}{\Gamma}
\newrobustcmd*{\entails}{\vdash}

\newrobustcmd*{\judgmentfont}[1]{{\normalfont\sffamily #1}}
\newrobustcmd*{\postfixjudgment}[1]{%
  \relax\ifnum\lastnodetype>0\mskip\medmuskip\fi
  \text{\judgmentfont{#1}}%
}
\newrobustcmd*{\valid}{\postfixjudgment{Valid}}
\newrobustcmd*{\prop}{\postfixjudgment{Prop}}
\newrobustcmd*{\true}{\postfixjudgment{true}}
\newrobustcmd*{\type}{\postfixjudgment{type}}
\newrobustcmd*{\context}{\postfixjudgment{ctx}}
\newrobustcmd*{\nil}{\postfixjudgment{nil}}
\newrobustcmd*{\truth}{\top}
\newrobustcmd*{\conj}{\wedge}
\newrobustcmd*{\disj}{\vee}
\newrobustcmd*{\falsehood}{\bot}
\newrobustcmd*{\imp}{\supset}
\newrobustcmd*{\zero}{0}

%%% Judgmental equality
\newrobustcmd*{\jdeq}{\equiv}
%%% Definition
\newrobustcmd*{\defeq}{\vcentcolon\equiv}
%%% Binary sums
\newrobustcmd*{\inlsym}{{\mathsf{inl}}}
\newrobustcmd*{\inrsym}{{\mathsf{inr}}}
\newrobustcmd*{\inl}{\ensuremath\inlsym\xspace}
\newrobustcmd*{\inr}{\ensuremath\inrsym\xspace}
%%% Booleans
\newrobustcmd*{\ttsym}{{\mathsf{tt}}}
\newrobustcmd*{\ffsym}{{\mathsf{ff}}}
%%% Pairs
\newrobustcmd*{\pair}{\ensuremath{\mathsf{pair}}\xspace}
\newrobustcmd*{\tuple}[2]{(#1,#2)}
\newrobustcmd*{\proj}[1]{\ensuremath{\mathsf{pr}_{#1}}\xspace}
%% Empty type
\newrobustcmd*{\abort}[1]{\ensuremath{\mathsf{abort}_{#1}}}
%%% Path concatenation
\newrobustcmd*{\concat}{%
  \mathchoice{\mathbin{\raisebox{0.5ex}{$\displaystyle\centerdot$}}}%
  {\mathbin{\raisebox{0.5ex}{$\centerdot$}}}%
  {\mathbin{\raisebox{0.25ex}{$\scriptstyle\,\centerdot\,$}}}%
  {\mathbin{\raisebox{0.1ex}{$\scriptscriptstyle\,\centerdot\,$}}}
}
%%% Transport (covariant)
\newrobustcmd*{\trans}[2]{\ensuremath{{#1}_{*}\mathopen{}\left({#2}\right)\mathclose{}}\xspace}
% Natural numbers objects
\newrobustcmd*{\Nat}{\mathsf{Nat}}
\newrobustcmd*{\rec}{\ensuremath{\mathsf{rec}}\xspace}
% Sequence
\newrobustcmd*{\Seq}{\ensuremath{\mathsf{Seq}}\xspace}
% Identity type
\newrobustcmd*{\Id}[1]{\ensuremath{\mathsf{Id}_{#1}}\xspace}
% Reflection
\newrobustcmd*{\refl}[1]{\ensuremath{\mathsf{refl}_{#1}}\xspace}
\newrobustcmd*{\J}{\ensuremath{\mathsf{J}}\xspace}

% fst,snd,case,id
\newrobustcmd*{\fst}{\textsf{fst}}
\newrobustcmd*{\snd}{\textsf{snd}}
\DeclareMathOperator{\case}{\textsf{case}}
\DeclareMathOperator{\caseif}{\textsf{if}}
\DeclareMathOperator{\casesplit}{\textsf{split}}
\DeclareMathOperator{\ttrue}{\textsf{tt}\xspace}
\DeclareMathOperator{\ffalse}{\textsf{ff}\xspace}
\newrobustcmd*{\id}{\textsf{id}}

\newrobustcmd*{\op}[1]{\operatorname{#1}}

\newrobustcmd*{\universe}{\mathcal{U}}

%inductive types
\newrobustcmd*{\ind}{\ensuremath{\mathsf{ind}}\xspace}
%higher inductive types
%interval
\newrobustcmd*{\interval}{\ensuremath{I}\xspace}
\newrobustcmd*{\seg}{\ensuremath{\mathsf{seg}}\xspace}
%circle
\newrobustcmd*{\Sn}{\mathbb{S}}
\newrobustcmd*{\base}{\ensuremath{\mathsf{base}}\xspace}
\newrobustcmd*{\lloop}{\ensuremath{\mathsf{loop}}\xspace}
\tikzset{
	buffer/.style={
		draw,
		shape border rotate=180,
		regular polygon,
		regular polygon sides=3,
		fill=gray,
		node distance=2cm,
		minimum height=4em
	}
}
\tikzset{
	buffer2/.style={
		draw,
		shape border rotate=180,
		regular polygon,
		regular polygon sides=3,
		fill=gray,
		node distance=1cm,
		minimum height=4em
	}
}
\newcommand*{\PrTri}[1]{\begin{tikzpicture}
	\node[buffer]{#1};
	\end{tikzpicture}}
\tikzset{
	buffer2/.style={
		draw,
		shape border rotate=180,
		regular polygon,
		regular polygon sides=3,
		fill=gray,
		node distance=0.6cm
	}
}
\newcommand*{\PrTriSm}[1]{\begin{tikzpicture}
	\node[buffer2]{#1};
	\end{tikzpicture}}
\usepackage{stmaryrd}
\usepackage{proof-dashed}
\usepackage{tikz}
\usepackage{tikz-cd}
\usepackage[utf8]{inputenc} 
\usepackage{syntax}
\usepackage{amsfonts}
\usepackage{amssymb} 
\usepackage{amsmath}
%\usepackage{amsthm}
\usepackage{mathpartir}

\usepackage{setspace}
\doublespacing
\pagestyle{myheadings}
%opening
\title{Justifications, type theory and modal calculi: a proposal}
\author{Konstantinos Pouliasis}

\begin{document}

\maketitle


\begin{abstract}
This work  is structured in two parts: 
The first part is, essentially, my second examination paper with some revisions and additions. It  introduces basic elements of my research topic which rests in 
the intersection of Justification Logic, Constructive Modality and Type Theory. I will present the relevant systems syntactically and I will pause on the basic metatheoretic
proof techniques which will be useful in the rest of the text.
In the second part I will delineate the current state of my research in the area.
I will elaborate on a modal type system that enhances simple type theory with elements of justification logic. I will present its accompanying calculus 
 obtained  a la Curry-Howard and I will argue for its computational relevance. More specifically, I will show  
that the obtained calculus characterizes  certain computational phenomena that abound in modern programming language semantics. 
I will omit full metatheoretic
results here and  I will merely hint to  proof methods that can be adopted from the first part. Finally,  I will propose certain
directions for future work. Such developments together with the omitted metatheoretic results are expected to be the study of my dissertation thesis.
This is a  paper accompanying my dissertation proposal and a partial requirement  for my Phd candidacy under the supervision of Distinguished Professor Sergei Artemov at the Department of Computer Science of the Graduate Center, CUNY.
\end{abstract}
\tableofcontents
\chapter{Intuitionistic Logic}\label{intui}
\section{Intuitionism}\label{sec:intrui}
In this and the subsequent chapter, I will be presenting foundational work in the intersection of \emph{Intuitionistic Logic} and \emph{Type Theory}. 
The presentation is scaffolding following Robert Harper's lecture videos in \emph{Homotopy Type Theory}~\cite{HarperHOTT} and the accompanying notes by students of the class~\cite{HOTTNotes1}. I will often deviate to standard textbooks in the field ~\cite{Barendregt1984-BARTLC},~\cite{citeulike:993095},~\cite{Pierce:2002:TPL:509043} to present further important results. 
\subsection{A bird's eye view}  
In a nutshell, \emph{Intuitionistic mathematics}  is a program in foundations 
of mathematics  that extends \emph{Brouwer's program}~\cite{brouwer1975collected}.
Brouwer, in an almost Kantian fashion, viewed mathematical reasoning as a human faculty 
and mathematics as a language of the ``creative subject"
aiming to communicate mathematical concepts. 
The concept of \emph{algorithm} as a step--by--step constructive process is brought in the 
foreground in Brouwer's program. As a result, intuitionistic theories are amenable to 
computational interpretations. In the following I will be using the terms intutionistic 
and \emph{constructive} interchangeably.  

For the purposes of this paper, the main diverging point of Brouwer's program, 
later explicated by Heyting~\cite{heyting1966intuitionism} and Kolmogorov~\cite{kolmogorov1925principe}~\cite{artemov2004kolmogorov}, lies in the treatment of proofs. In contrast to classical approaches to foundations 
that treat proof objects as external to theories, the constructive approach treats proofs 
as the fundamental forms of construction and hence, as first class citizens. 
As a result, the constructive view of logic draws heavily from proof theory
and Gentzen's developments~\cite{gentzen1970collected}. 
For the reader interested also in the philosophical implications  
of constructive foundations and \emph{antirealism}, 
Dummet's treatment is a classic in the field~\cite{dummett2000elements}.    
 


It has to be emphasized that proofs in the intuitionistic approach 
 are treated as stand--alone and are not bound to formal systems 
(i.e the notion of proof \textit{precedes} that of a formal system). 
It is necessary, hence, to draw a distinction between the notion of 
 \emph{proof as construction} and the typical notion of \emph{proof in a formal system} 
 ~\cite{Harper2013,Harper2012}.

A \emph{formal proof}
is a proof given in a fixed formal system, such as Peano Arithmetic, and arises
from the application of the inductively defined rules in that system. Formal proofs can, thus, be viewed as strings or g\"{o}delizations of textual derivation in some fixed system. 

Although every formal proof (in a specific system)
is also a proof (assuming soundness of the system) the converse is not true.  
This conforms with G\"{o}del's Incompleteness Theorem, which precisely states that there
exist true propositions (with a proof in \emph{some} formal system), but for which 
there cannot be given a formal proof within the formal system in question. 
This \emph{openness} of the nature of proofs is necessary for a foundational 
treatment of proofs that respects  G\"{o}delian phenomena.

Following the same line of thought, and adopting the doctrine of \emph{proof relevance} 
for obtaining true judgments, leads to another main difference of the constructive 
approach and the classical one i.e the (default) absence of the 
\emph{law of excluded middle}. 


\section{IPL}
\ac{IPL} can be viewed as ``the logic of \emph{proof relevance}" conforming with the intuitionistic view described in \ref{sec:intrui}. To judge a fact as \emph{true} one may provide a \emph{proof}
appropriate of the fact. \emph{Proofs} can be synthesized to obtain 
proofs for more complex facts (\emph{introduction rules}) and consumed
 to provide proofs relevant for other  facts (\emph{elimination rules}). The importance of the interplay between introduction and elimination rules was developed by Gentzen. 
 A discussion on the meaning of the logical connectives that is prevalent in \emph{MLTT} can be found in \cite{martin1996meanings} 
Following the presentation style  by   Martin-L\"{o}f we split the notions of \emph{judgment} and \emph{proposition}. We have two main kinds of judgments:
\begin{itemize}
\item  \emph{Judgments} that are logical arguments about the truth(or, equivalently, proof) of a \emph{proposition}. They might, optionally, involve assumptions on the truth (or, equivalently, proof) of other propositions. We might call these \emph{logical judgments}. 

\item Judgments on \emph{propositionality} or typeability. \emph{Propositions} are the \emph{subjects}  of \emph{logical judgments}. If something is judged to be a proposition then it belongs to the universe of discourse and can be mentioned in \emph{logical judgments}. 
\end{itemize} 
In addition, since a \emph{logical judgment} might involve a set $\Gamma$ of assumptions (or a \emph{context}), it is convenient to add a third kind of judgment of the form $\Gamma \context$ 
Thus, in \ac{IPL}, we get the judgments $\phi\ \in \prop$, $\phi \true$ and $\Gamma \context$:
\begin{alignat*}{2}
  \phi \in \prop &&\quad& \text{$
  \phi$ is a (well-formed) proposition} \\
  \phi\  \true &&& \text{\begin{tabular}[t]{@{}l@{}}
                Proposition $\phi$ is true \\
                i.e., has a proof.
              \end{tabular}}\\
  \Gamma \context &&\quad& \text{$\Gamma$ is a (well-formed) context of assumptions} \\
\end{alignat*}

The natural deduction system of \ac{IPL} is given below:


\begin{mdframed}
\textbf{Prop Formation}
\begin{mathpar}
\inferrule*[right=Atom] { } {P_i \in {\sf Prop}}
\and
\inferrule*[right=Top] { } {\top \in {\sf Prop}}
\and
\inferrule*[right=Bottom] { } {\bot \in {\sf Prop}}
\and
\inferrule*[right=Arr] {{\phi_1 \in {\sf Prop }}\\ {\phi_2 \in {\sf Prop}}} {\phi_1\supset\phi_2\in {\sf Prop}}
\and
\inferrule*[right=Conj] {{\phi_1 \in {\sf Prop }}\\ {\phi_2 \in {\sf Prop}}} {\phi_1\wedge\phi_2\in {\sf Prop}}
\and
\inferrule*[right=Disj] {{\phi_1 \in {\sf Prop }}\\ {\phi_2 \in {\sf Prop}}} {\phi_1\vee\phi_2\in {\sf Prop}}

\end{mathpar}
\end{mdframed}

\begin{mdframed}
\textbf{Context Formation}
\begin{mathpar}
\inferrule*[right=Nil] { } {{\sf \circ}\  \context}
\and
\inferrule*[right=$\Gamma$-Ext] {{\Gamma\ } {\sf \context}  \\ {\phi \in {\sf Prop}}} {{\Gamma, \phi \true} \ \context}
\end{mathpar}
\end{mdframed}

\begin{mdframed}
\textbf{Context Reflection}
\begin{mathpar}
\inferrule*[right=$\Gamma$-Refl] { {\Gamma}\  {\sf \context}\\ {\phi \true \in \Gamma}}{\Turnsi {\Gamma} {\phi \true}}
\end{mathpar}
\end{mdframed}

\begin{mdframed}
\textbf{Top Introduction -- Bottom Elimination}
\begin{mathpar}
\inferrule*[right=$\top$I] { } {\Turnsi {\Gamma} { \top \true}}
\and
\inferrule*[right=$\bot$E] {\Turnsi {\Gamma} {\bot \true} } {\Turnsi {\Gamma} {  \phi \true}}
\end{mathpar}
\end{mdframed}

\begin{mdframed}
\textbf{Implication Introduction and Elimination}
\begin{mathpar}
\inferrule*[right=$\supset$I] {\Turnsi {\Gamma, \phi_1 \true} {\phi_2 \true}} {\Turnsi {\Gamma} { \phi_1\supset \phi_2 \true}}
\and
\inferrule*[right=$\supset$E] {\Turnsi {\Gamma} {\phi_1\supset\phi_2 \true}\\{\Turnsi {\Gamma} {\phi_1 \true}}} {\Turnsi {\Gamma} {  \phi_2 \true}}
\end{mathpar}
\end{mdframed}
\begin{mdframed}
\textbf{Conjunction Introduction and Elimination}
\begin{mathpar}
\inferrule*[right=$\wedge$I] {\Turnsi {\Gamma} {\phi_1\true}\\{\Turnsi {\Gamma} {\phi_2 \true}}} {\Turnsi {\Gamma} {  \phi_1 \wedge\phi_2 \true}}
\end{mathpar}
\begin{mathpar}
\inferrule*[right=$\wedge$El] {\Turnsi {\Gamma} {\phi_1\wedge\phi_2 \true}} {\Turnsi {\Gamma} {  \phi_1\true}}
\and
\inferrule*[right=$\wedge$Er] {\Turnsi {\Gamma} {\phi_1\wedge\phi_2 \true}} {\Turnsi {\Gamma} {  \phi_2\true}}
\end{mathpar}
\end{mdframed}
\begin{mdframed}
\textbf{Disjunction Introduction and Elimination}
\begin{mathpar}
\inferrule*[right=$\vee$Il] {\Turnsi {\Gamma} {\phi_1 \true}} {\Turnsi {\Gamma} {  \phi_1\vee\phi_2\true}}
\and
\inferrule*[right=$\vee$Ir] {\Turnsi {\Gamma} {\phi_2 \true}} {\Turnsi {\Gamma} {  \phi_1\vee\phi_2\true}}
\end{mathpar}


\begin{mathpar}
\inferrule*[right=$\vee$E] 
{ {\Turnsi {\Gamma} {  \phi_1\vee\phi_2\true}}\\
{\Turnsi {\Gamma,\phi_1 \true} {\phi \true}}\\
{\Turnsi {\Gamma,\phi_2 \true} {\phi \true}}
}
 {\Turnsi {\Gamma} {\phi \true}}
\end{mathpar}

\end{mdframed}


\subsection{Basic Properties of Intuitionistic Entailment}
\label{ssec:entail}
			\begin{mdframed}
			\textbf{Reflexivity}
			    
				\begin{mathpar}
			   \inferrule*[] 
			    { }
			    {\Turnsi {\Gamma,\phi\true} {\phi \true}} 
				\end{mathpar}
		  \end{mdframed}

		\begin{mdframed}
		\textbf{Transitivity}
			\begin{mathpar}
					   \inferrule*[] 
					    {\Turnsi {\Gamma} {\psi \true}\\ {\Turnsi {\Gamma,\psi\true}{\phi\true}}}
					    {\Turnsi {\Gamma,\phi\true} {\phi \true}} 
						\end{mathpar}
				\end{mdframed}
   			
   			\begin{mdframed}
   			\textbf{Contraction}
   						\begin{mathpar}
   								   \inferrule*[] 
   								    {\Turnsi {\Gamma,\phi\true,\phi \true
   								   } {\psi \true}} {\Turnsi {\Gamma,\phi \true}{\psi\true}}
   								    
   									\end{mathpar}
   			\end{mdframed}
		\begin{mdframed}{Exchange}
					\begin{mathpar}
							   \inferrule*[] 
							    {\Turnsi {\Gamma
							   } {\phi \true}} {\Turnsi {\operatorname{\pi}(\Gamma)}{\phi\true}}
							    \end{mathpar}
                  Where $\pi(\Gamma)$ is a meta-symbol standing for any permutation of $\Gamma$.
                \end{mdframed}

\section{Order Theoretic Semantics: \vocab{Heyting Algebras}}\label{ha:ax}
\vocab{IPL} viewed order theoretically gives rise to a \vocab{Heyting  Algebra(HA)}. 
To define \vocab{HA} we need the notion of a \emph{lattice}.
 For our purposes we define it as follows\footnote{One can take a lattice being a partial order. The same results hold with slight modifications.}: 
  

\begin{mdframed}
\textbf{Definition:}
A \textit{lattice} is a non-empty \emph{pre--order} with finite meets and joins.
\end{mdframed}
In addition, we define \emph{bounded lattice} as follows: 
\begin{mdframed}
\textbf{Definition:}
A \textit{bounded lattice} $(L,\le)$ is a lattice that additionally has a greatest element 1 and a least element 0, which satisfy

$0\le x \le 1$ for every $x$ in $L$
\end{mdframed}
Finally, we can define \emph{HA}:

\begin{mdframed}
\textbf{Definition:}
A \textit{HA} is a bounded lattice $(L,\le,0,1)$ s.t. for every $a,b\in L$ there exists an $x$ (we name it $a\rightarrow b$) with the properties: 
\begin{enumerate}
\item $a\wedge x\le b $
\item $x$ is the greatest such element
\end{enumerate}
\end{mdframed}
\subsubsection{Axiomatization of HAs}
We can axiomatize the meet (i.e. greatest lower bound)($\wedge$) of $\phi,\psi$ for any  lower bound $\chi$.
\begin{mdframed}
\begin{mathpar}
  \infer{\phi \conj \psi \leq \phi}{
    }
  \and
  \infer{\phi \conj \psi \leq \psi}{
    } 
\end{mathpar}
\begin{equation*}
  \infer{\chi \leq \phi \conj \psi}{
    \chi \leq \phi & \chi \leq \psi} 
\end{equation*}
\end{mdframed}

We can axiomatize the join ($\vee$)(i.e. the least upper bound) of $\phi,\psi$ for any upper bound $\chi$ as follows .
\begin{mdframed}
\begin{mathpar}
  \infer{\phi  \leq \phi\vee \psi}{
    }
  \and
  \infer{\psi \leq \phi \vee \psi}{
    } 
\end{mathpar}
\begin{equation*}
  \infer{\phi \vee \psi \leq \chi}{
    \phi \leq \chi & \psi \leq \chi} 
\end{equation*}
\end{mdframed}
We can axiomatize the existence of a greatest element as follows:
\begin{mdframed}
\begin{equation*}
  \infer{\chi \leq 1}{
    } 
\end{equation*}
which says that $1$ is the greatest element.
\end{mdframed}

We can axiomatize the existence of a least element as follows:
\begin{mdframed}
\begin{equation*}
  \infer{0 \leq \chi}{
    } 
\end{equation*}
which says that $0$ is the least element.
\end{mdframed}
Finally, to axiomatize \emph{HAs} we require the existence of exponentials for every $\phi$, $\psi$ as follows:

\begin{mdframed}
\begin{mathpar}
 

  \infer{\phi \wedge  (\phi\supset \psi)\leq\psi}{
    } 
    \and
    \infer{\chi\leq\phi\supset\psi}{\phi\wedge\chi\leq\psi}
\end{mathpar}
\end{mdframed}

\subsubsection{Soundness and Completeness}

\begin{mdframed}
\begin{theorem}\label{thm:cmpha}
$\Gamma\vdash_{IPL} \phi \true$ iff for any \vocab{Heyting Algebra} $H$ we have $\Gamma^+\leq\phi^{*}$ where $*$ is  defined as the lifting of any map of $\prop$s to elements of $H$ and $(+)$ is defined inductively on the length of $\Gamma$ as follows
\begin{alignat*}{2}
  {\sf \circ}^+  &&\quad = & \quad\top\\
  (\Gamma,\phi)^+&&\quad = &\quad
  \Gamma^+\wedge\phi* \
\end{alignat*}
\end{theorem}
\end{mdframed}

\chapter{Justification Logic}
In the second part of this paper I will give an overview of \ac {JL} highlighting the parts that are closely related to constructivity to remain coherent with \ref{intui}. I will emphasize LP, the very first logic of justification, and its deep relation with \ac{IPL}. My scaffolding will be based upon \cite{Art01BSL},~\cite{Art95TR}  that reflect this relation. Beforehand, I will allow for a more general discussion on \ac{JL} following \cite{sep-logic-justification} and other relevant papers.

%It is well known that the provability predicate can be axiomatized using a modality \cite{citeulike:214701}, \cite{ArtBek05HPL}. The Logic of Proofs {\sf LP} \cite{Art94APAL} goes further and provides explicit proof terms (\textit{proof polynomials}) to inhabit judgments on validity. By translating reasoning in Intuitionistic Propositional Calculus ({\sf IPC}) to classical proofs, {\sf LP} obtains classical semantics for {\sf IPC} through a modality (inducing a {\sf BHK} semantics).

\section{A bird's eye view}
According to \cite{sep-logic-justification}\begin{quotation} Justification logics are epistemic logics which allow knowledge and belief modalities to be ``unfolded'' into justification terms.
\end{quotation}
 More specifically, in \ac{JL} the modality in question is witnessed by a reason and propositions of the kind $\box\phi$ become $t:\phi$ that reads ``$\phi$ is justified by reason t". Witnesses in \ac{JL} have structure and operations. Different choices of operators result in logics that explicate different modalities ({\sf {$K$,$T$,$S4$,$S5$}}). For our purposes, and in addition to type theoretic approaches to logic, \ac{JL} reveals a computational content for \emph{validity} in classical terms. As we will see following~\cite{artemov97un}, \ac{JL} and especially its {\sf $S4$} counterpart \ac{LP}, can provide a unified classical \emph{semantics} for type theoretic formulations of intuitionistic logic. In addition, following \cite{Artemov2007a} and \cite{DBLP:journals/entcs/PouliasisP14}, \ac{JL} mechanics can be viewed type theoretically to provide for modal typed systems that enrich computational type theories with ``semantical" notions such as explicit reflection and modular binding. 
 

\section{Minimal Justification Logic $J_0$}~\label{min:jo}
To permit for an account of reasons, the logic is enriched with an extra sort for $j$ for justifications. The sort of propositions is then enriched with propositions of the kind $j:\phi$ with $\phi$ being a proposition. Here is the abstract syntax:

\begin{mdframed}
\begin{align*}
j := &s_i|\ C_i| j_1*j_2| j_2 + j_2\\
 \phi:=& P_i|\ \bot|\ \phi_1\wedge\phi_2|\ \phi_1\vee\phi_2| \ \phi_2\supset\phi_2|\ \neg\phi|\ j:\phi   
\end{align*}
\end{mdframed}
Constants $C_i$ are symbols that can be assigned to logic axioms that are assumed to be necessary. Weaker justifications logics exist without any assignment of constants (empty \emph{constant specifications}) or with partial constant specifications. Nevertheless, in order for the  \emph{rule of necessitation} to be admissible each axiom instance of the underlying propositional logic has to be assigned a constant. We will be coming back to this topic in later sections. Symbols $s_i$ stand for variables.

A Hilbert--style axiomatization of $J_0$ is given below. Its components are Hilbert's axioms for propositional logic together with two basic rules for justification: \emph{applicativity} and \emph{concatenation}. Concatenation internalizes weakening of proofs.
\begin{mdframed}
\sf{Propositional Axioms}
\begin{align*}
&\sf{P1}.  \vdash \phi\supset(\psi\supset\phi)\\
& \sf{P2}. \vdash (\phi\supset(\psi\supset\chi))\supset((\phi\supset\psi)\supset(\phi\supset\chi))\\
& \sf{P3}. \vdash \phi\supset\psi\supset\phi\wedge\psi\\
&\sf{P4}. \vdash \phi\supset\psi\supset\psi\wedge\phi\\
&\sf{P5}.  \vdash \phi\supset\phi\vee\psi\\
&\sf{P6}. \vdash \psi\supset\phi\vee\psi\\
&\sf{P7}. \vdash (\phi\supset\psi)\supset(\neg\psi\supset\neg\phi)\\
\end{align*}
\end{mdframed}


\begin{mdframed}
\sf{Justification Axioms}
\begin{align*}
& \sf{Times}. \vdash j:(\phi\supset\psi)\supset(j':\phi\supset j*j':\psi)\\
& \sf{PlusL}. \vdash j:\phi\supset(j+j':\phi)\\
& \sf{PlusR}. \vdash j:\phi\supset(j'+j:\phi)\\
\end{align*}
\end{mdframed}
The rule of the system is \emph{Modus Ponens}. 
\begin{mdframed}
\sf{Modus Ponens}
\begin{mathpar}
\inferrule*[right=\sf{MP}]{{\phi\supset\psi}\\{\phi}}{\psi}
\end{mathpar}
\end{mdframed}
For the rule of necessitation to be admissible, we need necessitation of axioms to be admissible. For that reason a constant specification is required. We focus here on axiomatically appropriate constant specification $\sf{CS}$ because of its relation to combinatorial calculi. An axiomatization of axiomatically appropriate $\sf{CS}$ given below. Elements of $\sf{CS}$ are pairs $(C,\phi)$ of constants and propositions:
\begin{mdframed} 
\sf{Axiomatic CS}
\begin{mathpar}
\inferrule*[right=$\sf{C_1}$]  { }   {\vdash({{\sf C_1}[\phi,\psi],\  \phi\rightarrow(\psi\rightarrow\phi))\in \sf{CS}}}
\and
\inferrule*[right=$\sf{C_2}$]  { }   {\vdash({{\sf C_2}[\phi,\psi,\chi],\ (\phi\supset(\psi\supset\chi))\supset((\phi\supset\psi)\supset(\phi\supset\chi))  ) )\in \sf{CS}}}
\and
\inferrule*[right=$\sf{C_3}$]  { }   {\vdash(  {{\sf C_3}[\phi,\psi],\  \phi\supset\psi\supset\phi\wedge\psi  )\in \sf{CS}}}
\and
\inferrule*[right=$\sf{C_4}$]  { }   {\vdash(  {{\sf C_4}[\phi,\psi],\  \phi\supset\psi\supset\psi\wedge\phi  )\in \sf{CS}}}
\and
\inferrule*[right=$\sf{C_5}$]  { }   {\vdash(  {{\sf C_5}[\phi,\psi],\  \phi\supset\phi\vee\psi  )\in \sf{CS}}}
\and
\inferrule*[right=$\sf{C_6}$]  { }   {\vdash(  {{\sf C_6}[\phi,\psi],\  \psi\supset\phi\vee\psi  )\in \sf{CS}}}
\and
\inferrule*[right=$\sf{C_7}$]  { }   {\vdash(  {{\sf C_7}[\phi,\psi],\  (\phi\supset\psi)\supset(\neg\psi\supset\neg\phi)\in \sf{CS}}}
\and
\inferrule*[right=$\sf{C_8}$] { } {\vdash(  {{\sf C_8}[\phi,\psi,j,j'],\  j:(\phi\supset\psi) \supset (j':\phi \supset j*j':\psi  ))\in \sf{CS}}}
\and
\inferrule*[right=$C!$]
{\vdash ( {\sf C},\phi)\in \sf{CS}} {\vdash(\sf{C!} ,\ \sf{C}:\phi )\in \sf{CS}}
\end{mathpar}
\end{mdframed}

Finally we require reflection on $\sf{CS}$: 
\begin{mdframed}
\sf{Specification Reflection}
\begin{mathpar}
\inferrule*[right=CSR]
{\vdash ( {\sf C},\phi)\in \sf{CS}} {\vdash\sf{C}:\phi}
\end{mathpar}

\end{mdframed}

The system can be given a Natural Deduction formulation a la \ac{IPL} since the following theorem holds:
\begin{mdframed}
\textbf{Deduction Theorem}
For any set of propositional assumptions $\Gamma$, \\ $\Gamma,\phi\vdash\psi$ implies $\Gamma\vdash\phi\supset\psi$ 
\end{mdframed}
\section{Epistemic motivation} 
 \ac{JL} as an epistemic logic departs from previous traditions of logic of knowledge based on  universality judgments. From \cite{sep-logic-justification}
\begin{quotation}
The modal approach to the logic of knowledge is, in a sense, built around the universal quantifier: X is known in a situation if X is true in all situations indistinguishable from that one. Justifications, on the other hand, bring an existential quantifier into the picture: X is known in a situation if there exists a justification for X in that situation
\end{quotation}

This fresh approach on epistemic tradition has been utilized to solve many problems in formal epistemology (see \cite{Artemov2014-ARTLOA}). We give here a solution to the famous 'Red barn problem' that is also a pedagogical example on how deduction in the system works.

The red barn problem can be stated as follows:
\begin{quote}
Suppose I am driving through a neighborhood in which, unbeknownst to me, papier-mâché barns are scattered, and I see that the object in front of me is a barn. Because I have barn-before-me percepts, I believe that the object in front of me is a barn. Our intuitions suggest that I fail to know barn. But now suppose that the neighborhood has no fake red barns, and I also notice that the object in front of me is red, so I know a red barn is there. This juxtaposition, being a red barn, which I know, entails there being a barn, which I do not, “is an embarrassment”
\end{quote}

The red barn example can be represented in a system of modal logic where $\Box \phi$ represents knowledge of $\phi$ that, in contrast to the the justified approach, is forgetful with respect to reasons. The formalization and the accompanying problem go as follows:

\begin{enumerate}
    \item $\Box B$, ‘I believe that the object in front of me is a red barn’.
    \item  $\Box(B \wedge R),$ ‘I believe that the object in front of me is a red barn’. 

At the metalevel, 2 is actually knowledge, whereas by the problem description, 1 is not knowledge.

   \item $\Box(B\wedge R\supset B)$, a knowledge assertion of a logical axiom.
	\end{enumerate}
\begin{quote}	
Within this formalization, it appears that epistemic closure in its modal form (2) is violated:line 2, $\Box(B \wedge R )$, and line 3, $(B \wedge R \supset B)$ are cases of knowledge whereas $\Box B$ (line 1) is not knowledge. The modal language here does not seem to help resolving this issue.
\end{quote}
Of course, one can resolve this by introducing a second modality(e.g. for 'I believe that'). But then similar problems can occur (e.g. by adding a third modality read as `it should be'). Indexing of modalities with reasons solves this problem in its generality: by permitting the applicative closure only on reasons of the same sort one can overcome this defect.
\begin{enumerate}
   \item $u:B$, ‘$u$ is a reason to believe that the object in front of me is a barn’;
   \item $v:(B \wedge R)$, ‘$v$ is a reason to believe that the object in front of me is a red barn’;
    \item $a:(B \wedge R \supset B)$, because of logical awareness.


 On the metalevel, the problem description states that 2 and 3 are cases of knowledge, and not merely belief, whereas 1 is belief which is not knowledge. Here is how the formal reasoning goes:

    \item $a:(B \wedge R \supset B)\supset(v:(B \wedge R) \supset  a*v:B)$, by {\sf Times} 
    \item $v:(B \wedge R) \supset a*v:B$, from 3 and 4, by propositional logic;
    $a*v:B$, from 2 and 5, by propositional logic.

\end{enumerate}
\section{Proof theoretic view}
	In ~\ref{intui} we gave an analytic account of the \ac{BHK} principles of  constructive proofs. In the paper ``Eine Interpretation des
	intuitionistischen Aussagenkalküls ", G\"oedel gave a classical provability interpretation of \ac{BHK} using the modal system ${\sf S4}$.
	
	The standard axiomatization of ${\sf S4}$ is given below: 
	\begin{mdframed}
	The system $\sf{S4}$ 
	\begin{align*}
	& \sf{P1-P7}&\\
	 \sf{K}.& \vdash \Box(\phi\supset\psi)\supset(\Box\phi\supset\Box\psi)\\
	 \sf{T}.& \vdash \Box\phi\supset\phi\\
	\sf{4}. &\vdash \Box\phi\supset\Box\Box\phi\\
	\end{align*}


	\sf{Modus Ponens}
	\begin{mathpar}
	\inferrule*[right=\sf{MP}]{{\phi\supset\psi}\\{\phi}}{\psi}
	\end{mathpar}
	\end{mdframed}
	
	G\"oedel's result can be summarized in the following theorem:

	\begin{mdframed}
		\textbf{G\"odel-Tarski Translation of Intuitionistic Logic}
	$$\Gamma\vdash_{\sf IPL}\phi \rightharpoondown \Gamma\vdash_{\sf S4}\operatorname{tr}(\phi)$$
		where $\operatorname{tr}(\phi)$ is obtained by $\phi$ by $\Box$-ing its subformulas. 
	\end{mdframed}

After this result the state of the project of a classical interpretation of \ac{BHK} semantics was as follows:
     IPC $\hookrightarrow$ S4 $\hookrightarrow$ ? $\hookrightarrow$ CLASSICAL PROOFS. Filling the missing part was 
     the motivation behind \ac{LP}, the first Justification Logic.

	
\section{The Logic of Proofs}
An axiomatization of \ac{LP} with axiomatically appropriate constant specification as defined in \ref{min:jo} can be given
as follows:
	\begin{mdframed}
	The system $\sf{LP}$ 
	\begin{align*}
	& \sf{P1-P7}\\
	 \sf{Times}.& \vdash j:(\phi\supset\psi)\supset(j':\phi\supset j*j':\psi)&\\
	  \sf{PlusL}.& \vdash j:\phi\supset(j+j':\phi)&\\
	  \sf{PlusR}.& \vdash j:\phi\supset(j'+j:\phi)&\\
	 \sf{T}.& \vdash j:\phi\supset\phi\\
	\sf{4}. &\vdash j:\phi\supset (j!:j:\phi)
		\end{align*}
	
   \end{mdframed}
\section{Metatheoretic Results}
The \emph{Deduction Theorem} holds for \ac{LP}
\begin{mdframed}
\textbf{Deduction Theorem}
Any deduction of the kind $\Gamma,\phi\vdash\psi$ implies $\Gamma\vdash \phi\supset\psi$.
\end{mdframed}
Also, the lifting property can be obtained:
\begin{mdframed}
\textbf{Lifting Lemma}

Any deduction of the kind $\vec{j}:\Gamma,\Delta\vdash\phi$ implies $\vec{j}:\Gamma,\vec{s}:\Delta\vdash j'(\vec{j},\vec{s}):\phi$ where $\vec{j}$ is a vector metavariables to be substituted for arbitrary polynomials and $\vec{s}$ is a vector of (object) variables. 

\end{mdframed}
In addition, \ac{LP} is the forgetful projection of {\sf $S4$}. More specifically, consider and formula of \ac{LP} $\phi$ and the transformation $F_\Box(\phi)$ that replaces all subformulae of $\phi$ of the kind $j:\phi'$ with $\Box\phi'$. The following theorem holds:
\begin{mdframed}
\textbf{Forgetful Projection Property}

$\Gamma\vdash_{\sf LP}\phi$ implies $\Gamma\vdash_{\sf{ S4}}F_{\Box}(\phi)$ 

\end{mdframed}
  The inverse also holds as the realization theorem says. Before introducing the realization procedure we give a motivating example.
  
  \begin{mdframed}
  \textbf{Example}: Realization of $\vdash_{\sf S4}\Box\phi\vee\Box\psi\supset\Box(\phi\vee\psi)$
  \begin{enumerate}
     \item $\phi\supset\phi\vee\psi$, $\psi\supset\phi\vee\psi$ Prop. Axioms;
     \item $C:(\phi\supset \phi\vee \psi)$, $C':(\psi\supset \phi\vee \psi)$ From {\sf CS} rules.
      \item $s:\phi\supset C*s:\phi\vee\psi$, From 1,2 and{\sf Times} and {\sf MP}
      \item $t:\psi\supset C'*t:\phi\vee\psi$, Similarly 
      \item $C*s:\phi\vee\psi\supset(C*s+C'*t):\phi\vee\psi$ and $C'*t:\phi\vee\psi\supset(C*s+C'*t):\phi\vee\psi$, From {\sf Rplus, Lplus} 
      \item $s:\phi\supset(C*s+C'*t):\phi\vee\psi$, 
      From 3,5 by Propositional Logic.
      \item $t:\psi\supset(C*s+C'*t):\phi\vee\psi$,
      From 4,5 by Propositional Logic.
      \item  $s:\phi\vee t:\psi\supset(C*s+C'*t):\phi\vee\psi$,
      From 6,7 and Propositional Logic.
      
  \end{enumerate}
  
  \end{mdframed}

\subsection{Realization}							
The realization gives an algorithmic process of transforming deductions in{\sf S4} to {\sf LP}.
By an {\sf LP}-realization of a modal formula $\phi$ we mean an assignment of proof polynomials to
all occurrences of the modality in$ \phi$. Let $\phi^{r}$  be the image of $\phi$ under a realization $r$. 

The polarity of $\Box$s in a formula is relevant in realizations.  
We define positive
and negative occurrences of modality in a formula and a sequent.
\begin{mdframed}

\textbf{$\Box$ Polarities}
\begin{enumerate}
\item The indicated occurrence of $\Box$ in $\Box\phi$ is of positive polarity; 
\item any occurrence of $\Box$ in the subformula $\phi$ of $\psi\supset\phi$,$\psi\wedge\phi$, $\phi\wedge\psi$, $\psi\vee\phi$, $\phi\vee\psi$, $\Box\phi$,$\Gamma\Rightarrow\Delta,\phi$ -- we will be defining $\Rightarrow$ momentarily -- has the same polarity as the same occurrence of $\Box$ in $\phi$.
\item any occurrence of $\Box$ in the subformula $\phi$ of $\neg\phi$,  $\phi\supset\psi$, $\Gamma,\phi\Rightarrow\Delta$, has polarity opposite to the polarity of the very same  occurrence of $\Box$ in $\phi$.

\end{enumerate}

\end{mdframed}


Next we give a a cut-free sequent formulation of ${\sf S4}$ (reference) with sequents $\Gamma\vdash\Delta$, where $\Gamma$ and $\Delta$ are finite multisets of modal formulas. The left hand multisets are to be read conjunctively and the right hand ones disjunctively. The rules are the rules given below together with the typical structural ones.
\begin{mdframed}

\begin{mathpar}

\inferrule*[right=Refl]  { }   {\Gamma,\phi\vdash\phi,\Delta}
\and
\inferrule*[right=$\neg$L]  {\Gamma\vdash\phi,\Delta}   {\Gamma,\neg\phi\vdash\Delta}
\and
\inferrule*[right=$\neg$R]  {\phi,\Gamma\vdash\Delta}   {\Gamma\vdash\neg\phi,\Delta}
\and
\inferrule*[right=$\wedge$L]  {\Gamma,\phi,\psi\vdash\Delta}    {\Gamma,\phi\wedge\psi\vdash\Delta}
\and
\inferrule*[right=$\wedge$L]  {{\Gamma\vdash\phi,\Delta}\\   {\Gamma\vdash\psi,\Delta}}{\Gamma\vdash\phi\wedge\psi,\Delta}
\and
\inferrule*[right=$\vee$L]  {{\Gamma,\phi\vdash\Delta}\\   {\Gamma,\psi\vdash\Delta}}{\Gamma,\phi\vee\psi\vdash\Delta}
\and
\inferrule*[right=$\vee$R]  {\Gamma\vdash\phi,\psi,\Delta}   {\Gamma\vdash\phi\vee\psi\vdash\Delta}
\and
\inferrule*[right=$\supset$L]  {{\Gamma\vdash\phi,\Delta}\\   {\Gamma\psi\vdash\Delta}}{\Gamma,\phi\supset\psi\vdash\Delta}
\and
\inferrule*[right=$\supset$R]  {{\Gamma\vdash\phi,\Delta}\\   {\Gamma\psi\vdash\Delta}}{\Gamma,\phi\supset\psi\vdash\Delta}
\and
\inferrule*[right=$\Box$L]  {\phi,\Gamma\vdash\Delta} {\Box\phi,\Gamma\vdash\Delta}
\and
\inferrule*[right=$\Box$R]  {\Box\Gamma\vdash\phi,\Delta}  {\Box\Gamma\vdash\Box\phi,\Delta}
\end{mathpar}

\end{mdframed}

Relevant in the realization proof is the sequent formulation of {\sf LP}, the system {\sf LPG} which enjoys the cut-elimination property resulting in the system ${\sf LPG^{-}}$. The rules relevant to justifications are given below.  

\begin{mdframed}

\begin{mathpar}

\inferrule*[right=$:$L]  {\Gamma,\phi\vdash\phi,\Delta}   {\Gamma,t:\phi\vdash\phi,\Delta}
\and
\inferrule*[right=$!$R]  {\Gamma\vdash t:phi,\Delta}   {\Gamma\vdash!t:t:\phi,\Delta}
\and
\inferrule*[right=$+$L]  {\Gamma\vdash t:\phi,\Delta}   {\Gamma\vdash(t+s):\phi,\Delta}
\and
\inferrule*[right=$+$R]  {\Gamma\vdash t:\phi,\Delta}    {\Gamma\vdash (s+t):\phi,\Delta}
\and
\inferrule*[right=$*$R]  {{\Gamma\vdash s:\phi\supset\psi,\Delta}\\   {\Gamma\vdash t:\phi,\Delta}}{\Gamma\vdash s*t:\psi,\Delta}
\and
\inferrule*[right=$c$R]  {\Gamma\vdash\phi,\Delta}   {\Gamma\vdash c:\phi,\Delta}

\end{mathpar}

\end{mdframed}
Utilizing the previous systems the realization theorem shows:
\begin{mdframed}
\textbf{Realization Theorem}
If $\Gamma\vdash_{\sf S4} \phi$ then there is a \emph{normal} realization s.t.  $\Gamma\vdash_{\sf LP} \phi^{r}$. By normal we mean a realization for which all occurrences of $\Box$ are realized by proof variables and the corresponding constant specification is injective.
   
\end{mdframed}
\subsection{Kripke -  Fitting Semantics}
In this section I will be discussing Kripke -- Fitting Semantics\cite{fitting2005logic} for Justification Logic $\sf{J_0} + CS$ very briefly.


  A  possible world justification logic model for the system ${\sf J_0 + CS}$  is a structure $M=\langle G, R, E, V\rangle$. $\langle G,R\rangle$ is a standard $K$ frame, where $G$ is a set of possible worlds and $R$ is a binary relation on it. $V$ is a mapping from propositional variables to subsets of $G$, specifying atomic truth at possible worlds.
$E$ is an evidence function that maps pairs of justification terms and formulas to sets of worlds.  

Given such a model, we define the $\models$ relation as follows:
\begin{mdframed}
$\forall \Gamma\in G$
\begin{itemize}
    

    \item[] $M, \Gamma \models P$ iff $\Gamma \in V(P)$ for $P$ a propositional letter
    \item  It is not the case that $M, \Gamma \models \bot$
     \item   $M, \Gamma \models \phi \supset \psi$ iff it is not the case that $M, \Gamma \models \phi$ or$ M, \Gamma\models Y$
     \item  $M,\Gamma \models (j:\phi)$ if and only if $\Gamma \in E(j,\phi)$ and, $\forall \Delta\in G$ with $\Gamma R \Delta$, we have that $M,\Delta\models \phi$.
     
\end{itemize}
\end{mdframed}


 The following conditions on evidence functions are assumed:
\begin{itemize}
\item[] $E(j,\phi\supset\psi)\cap E(j',\phi) \subseteq E(j*j',\psi)$
\item[]  $E(j,\phi) \cup E(j',\phi)\subseteq E(j + j',\phi) $
\end{itemize}
   


Finally, the Constant Specification CS should be taken into account. Recall that constants are intended to represent reasons for basic assumptions that are accepted outright. A model $M = \langle G,R,E,V\rangle$ meets Constant Specification CS provided: if $(C,\phi) \in CS$ then $E(c,\phi) = G$.

Typical, soundness and completeness results can be shown for such models. They can also be extended for all other justification logics.
\chapter{Modal Type Theory}
In this chapter I will give a short, example-driven introduction to typed modality and its applications. 
The discussion will remain informal when it comes to metatheory and will be focused on usage of modal typed calculi in applications. Apropos, I will discuss Operational Semantic as the essence of the computational approach to logic.  
\section{A bird's eye view}
Significant in the development of modal logic within a type-theoretic framework is the work of Moggi \cite{moggi1991notions}.  In his seminal work he mentioned the need of shifting from simply typed calculi to systems that can capture notions of computation such exceptions, partial functions,  binding constructs etc. Moggi's initiative stems from a categorical point of view. 

In the realm of deductive systems with explicit witnesses, Artemov's work on Operational Modal Logic\cite{artemovy1995operational}, and the ``Gang of four" \cite{benton1992term},\cite{Benton:1998:CTL:969611.96961} revived the interest in constructive modality and its computational view. Of course earlier work in the intersection of modal and constructive logic exists (cf. \cite{prawitz10natural},\cite{fitting1983proof} ) but its relation with programming languages is not yet explicit. It's worth mentioning that the work by DePaiva  initiates from a categorical view too. That is Categorical Semantics for Linear Logic. 

Since it is of great importance in my work, I will be presenting here the \emph{judgmental} approach followed by Pfenning's ``Judgmental Reconstruction of Modal Logic" \cite{pfenning2001judgmental} which is a foundational approach that captures previous work on $\Box$ Calculi for $S4$. Although the system presented is a judgmental reconstruction of the system {\sf $S4$} the approach can be used to host other modalities. 

\section{Judgmental Reconstruction of Modal Logic}
The $\Box$ fragment of Pfenning's Judgmental reconstruction, consists of the judgments of \ac{IPL} as developed in the first Chapter together with judgments of validity. The definition of validity is given in a proof theoretic manner. In a nutshell, judgments of validity internalize judgments of proof without assumptions. ``Evidence for validity of $\phi$, is simply unconditional evidence of truth of $\phi$".   

\begin{mdframed}
\begin{enumerate}
\item If ${\sf{nil}}\vdash \phi \true$ then $\phi\valid$
\item If $\phi\valid$  then $\Gamma\vdash \phi\true$

\end{enumerate}
\end{mdframed}

The logical judgments are now extended in the form:
\begin{mdframed}
$$\phi_1'\valid, \phi_2'\valid,\ldots,\phi_n'\valid;\phi_1\true\phi_2\true,\ldots\phi_m\true\vdash\phi\true $$
\end{mdframed}


In the rules, we restrict ourselves to proving judgments of the form $\phi \true$ (rather
than
$\phi \valid$), which is possible since the latter is directly defined  in terms of the former.
The meaning of hypothetical judgments yields the general substitution principle.

\begin{mdframed}
$\Delta\vdash\phi\valid$ and $\Delta,\phi\valid\vdash J$ then $\Delta\vdash J$ 
\end{mdframed}

This principle can be rewritten utilizing the definition of validity:
\begin{mdframed}
\textbf{Substitution Principle For Validity}
$\Delta;\nil\vdash\phi \true$ and $\Delta,\phi \valid;\Gamma\vdash J$ then $\Delta;\Gamma\vdash J$ 
\end{mdframed}


Additionally, we add the following hypothesis reflection rule for valid contexts:


\begin{mdframed}
\textbf{Validity Context Reflection}

\begin{mathpar}
\inferrule*[right=$\Delta$-Refl] { {\Delta \ {\sf \context}}\\{\Gamma\context} \\{\phi \valid \in \Delta}}{\Turnsi {\Delta;\Gamma} {\phi \true}}
\end{mathpar}
\end{mdframed}

In the typability rules we add the following:
\begin{mdframed}
\begin{mathpar}
\inferrule*[right=$\Box$F]{\phi\prop}{\Box\phi\prop}
\end{mathpar}
\end{mdframed}

The $\Box$ introduction rule just allows the internalization of the validity of
$\phi$ as truth of $\Box \phi$, according to the definition of validity.
\begin{mdframed}
\textbf{Necessity Introduction}
\begin{mathpar}
\inferrule*[right=$\Box$I]{\Delta;\nil\vdash\phi\true}{\Delta;\Gamma\vdash\Box\phi\true}
\end{mathpar}
\end{mdframed}
The elimination rule is harder. A simplified version like the one below is unsound  since the hypotheses in  $\Gamma$  are unjustified.:
\begin{mathpar}
\inferrule*[right=$\Box$E-Unsound]{\Delta;\Gamma\vdash\Box\phi\true}{\Delta;\vdash\Box\phi\true}
\end{mathpar}

Another approach  would be the rule below. Which is locally sound but not complete  Gentzen's inversion principle is not satisfied. After eliminating the $\Box$ we cannot re-introduce it.

\begin{mathpar}
\inferrule*[right=$\Box$E-Incomplete]{\Delta;\vdash\Box\phi\true}{\Delta;\Gamma\vdash\phi\true}
\end{mathpar}


The proposed rule that satisfies local reduction and local expansion is :
\begin{mdframed}
\begin{mathpar}
\inferrule*[right=$\Box$E]{{\Delta;\Gamma\vdash\Box\phi\true}\\{\Delta,\phi\valid;\Gamma\vdash\psi\true}}{\Delta;\Gamma\vdash\psi\true}
\end{mathpar}
\end{mdframed}

The negative fragment of the system is, thus, as follows:

\begin{mdframed}
\textbf{Prop Formation}
\begin{mathpar}
\inferrule*[right=Atom] { } {P_i \in {\sf Prop}}
\and
\inferrule*[right=Top] { } {\top \in {\sf Prop}}
\and
\inferrule*[right=Arr] {{\phi_1 \in {\sf Prop }}\\ {\phi_2 \in {\sf Prop}}} {\phi_1\supset\phi_2\in {\sf Prop}}


\end{mathpar}
\end{mdframed}

\begin{mdframed}
\textbf{Context $\Gamma$ Formation}
\begin{mathpar}
\inferrule*[right=Nil] { } {{\sf nil}\  \context}
\and
\inferrule*[right=$\Gamma$-Add] {{\Gamma\ } {\sf \context}  \\ {\phi \in {\sf Prop}}} {{\Gamma, \phi\true } \ \context}
\end{mathpar}
\end{mdframed}
\begin{mdframed}
\textbf{Context $\Delta$ Formation}
\begin{mathpar}
\inferrule*[right=Nil] { } {{\sf nil}\  \context}
\and
\inferrule*[right=$\Delta$-Add] {{\Delta\ } {\sf \context}  \\ {\phi \in {\sf Prop}}} {{\Delta, \phi\valid } \ \context}
\end{mathpar}
\end{mdframed}
\begin{mdframed}
\textbf{Compound $\Gamma;\Delta$ Context}
\begin{mathpar}
\inferrule*[right=$\Gamma;\Delta$-F] { {\Delta}\ {\sf \context}\\ {\Gamma \context}}
{\Turnsi {\Delta;\Gamma} {\context}}
\end{mathpar}
\end{mdframed}
\begin{mdframed}
\textbf{Context  $\Gamma$ Reflection}
\begin{mathpar}
\inferrule*[right=$\Gamma$-Refl] { {\Delta;\Gamma}\ {\sf \context}\\ {\phi \true \in \Gamma}}
{\Turnsi {\Delta;\Gamma} {\phi \true}}
\end{mathpar}
\end{mdframed}
\begin{mdframed}
\textbf{Context  $\Delta$ Reflection}
\begin{mathpar}
\inferrule*[right=$\Delta$-Refl] { {\Delta;\Gamma}\ {\sf \context}\\ {\phi \valid \in \Delta}}
{\Turnsi {\Delta;\Gamma} {\phi \true}}
\end{mathpar}
\end{mdframed}
\begin{mdframed}

\textbf{Top Introduction}
\begin{mathpar}
\inferrule*[right=$\top$I] { } {\Turnsi {\Delta;\Gamma} { \top \true}}
\end{mathpar}
\end{mdframed}
\begin{mdframed}

\textbf{Implication Introduction and Elimination}
\begin{mathpar}
\inferrule*[right=$\supset$I] {\Turnsi {\Delta;\Gamma, \phi_1 \true} {\phi_2 \true}} {\Turnsi {\Delta;\Gamma} { \phi_1\supset \phi_2 \true}}
\and
\inferrule*[right=$\supset$E] {\Turnsi {\Delta;\Gamma} {\phi_1\supset\phi_2 \true}\\{\Turnsi {\Delta;\Gamma} {\phi_1 \true}}} {\Turnsi {\Delta;\Gamma} {  \phi_2 \true}}
\end{mathpar}
\end{mdframed}
\begin{mdframed}
\textbf{Necessity Introduction and Elimination}
\begin{mathpar}
\inferrule*[right=$\Box$I]{\Delta;\nil\vdash\phi\true}{\Delta;\Gamma\vdash\Box\phi\true}
\and
\inferrule*[right=$\Box$E]{{\Delta;\Gamma\vdash\Box\phi\true}\\{\Delta,\phi\valid;\Gamma\vdash\psi\true}}{\Delta;\Gamma\vdash\psi\true}
\end{mathpar}
\end{mdframed}

\subsection{Properties of Entailment For the Judgmental Reconstruction}
The guiding transitivity principles, weakening, contraction, and exchange can be proved for the system:
\begin{mdframed}
\textbf{Transitivity}
The guiding substitution principle can be expressed as a property of this formal
system and also be proven by induction over the structure of derivations
\begin{itemize}
\item If $\Delta;\Gamma, \phi\true,\Gamma'\vdash\psi\true$ and $\Delta;\Gamma\vdash\phi\true$
then $\Delta;\Gamma,\Gamma'\vdash\psi \true$
\item If $\Delta,\phi \valid,\Delta';\Gamma\vdash\psi\true$ and $\Delta;\nil\vdash\phi \true$
then $\Delta\Delta';\Gamma\vdash\psi \true$
\end{itemize}
\textbf{Weakening, Contraction and Exchange properties can be also shown to hold.}
\end{mdframed}

\subsection{Adding proof terms}
The system can be assigned proof terms as follows:
\begin{mdframed}
\textbf{Context $\Gamma$ Formation}
\begin{mathpar}
\inferrule*[right=Nil] { } {{\sf nil}\  \context}
\and
\inferrule*[right=$\Gamma$-Add] {{\Gamma\ } {\sf \context}  \\ {\phi \in {\sf Prop}}\\x\not\in\Gamma} {{\Gamma, x:\phi} \ \context}
\end{mathpar}
\end{mdframed}
\begin{mdframed}
\textbf{Context $\Delta$ Formation}
\begin{mathpar}
\inferrule*[right=Nil] { } {{\sf nil}\  \context}
\and
\inferrule*[right=$\Delta$-Add] {{\Delta\ } {\sf \context}  \\ {\phi \in {\sf Prop}}\\s\not\in\Delta} {{\Delta, s::\phi} \ \context}
\end{mathpar}
\end{mdframed}
\begin{mdframed}
\textbf{Compound $\Gamma;\Delta$ Context}
\begin{mathpar}
\inferrule*[right=$\Gamma;\Delta$-F] { {\Delta}\ {\sf \context}\\ {\Gamma \context}}
{\Turnsi {\Delta;\Gamma} {\context}}
\end{mathpar}
\end{mdframed}
\begin{mdframed}
\textbf{Context  $\Gamma$ Reflection}
\begin{mathpar}
\inferrule*[right=$\Gamma$-Refl] {  }
{\Turnsi {\Delta;\Gamma,x:\phi,\Gamma'} {x:\phi }}
\end{mathpar}
\end{mdframed}
\begin{mdframed}
\textbf{Context  $\Delta$ Reflection}
\begin{mathpar}
\inferrule*[right=$\Delta$-Refl] {  }
{\Turnsi {\Delta,s::\phi,\Delta';\Gamma} {s:\phi }}
\end{mathpar}
\end{mdframed}
\begin{mdframed}

\textbf{Top Introduction}
\begin{mathpar}
\inferrule*[right=$\top$I] { } {\Turnsi {\Delta;\Gamma} { \langle\rangle:\top }}
\end{mathpar}
\end{mdframed}
\begin{mdframed}

\textbf{Implication Introduction and Elimination}
\begin{mathpar}
\inferrule*[right=$\supset$I] {\Turnsi {\Delta;\Gamma, x:\phi_1 } {M:\phi_2 }} {\Turnsi {\Delta;\Gamma} { \lambda x:\phi_1.M:\phi_1\supset \phi_2}}
\and
\inferrule*[right=$\supset$E] {\Turnsi {\Delta;\Gamma} {M:\phi_1\supset\phi_2 }\\{\Turnsi {\Delta;\Gamma} {N:\phi_1}}} {\Turnsi {\Delta;\Gamma} {(M N) : \phi_2 }}
\end{mathpar}
\end{mdframed}
\begin{mdframed}
\textbf{Necessity Introduction and Elimination}
\begin{mathpar}
\inferrule*[right=$\Box$I]{\Delta;\nil\vdash M:\phi}{\Delta;\Gamma\vdash\operatorname{box}(M) :\Box\phi\true}
\and
\inferrule*[right=$\Box$E]{{\Delta;\Gamma\vdash M:\Box\phi\true}\\{\Delta,s::\phi;\Gamma\vdash N:\psi}}{\Delta;\Gamma\vdash  \operatorname{let}\operatorname{box}(s)=M \operatorname{in} N: \psi}
\end{mathpar}
\end{mdframed}

\section{Computational Interpretation}
One of the possible ways to read $\Box\phi$ is as representing \emph{source} code of type $\phi$. This 
makes the $Box$ calculus given a framework for typing programs with explicit \emph{staged computation}.
Explicit staging exists in many languages. One of its most characteristic implementations is the \emph{quote} constructs in Lisp \cite{bawden1999quasiquotation}. We introduce the concept and the application of the calculus with a motivating example following \cite{PfenningCompMod}.

Consider the exponential function $\operatorname{exp}:nat\rightarrow nat\rightarrow nat$  and the two definitions
\begin{code}
exp(0) = lm x -> 1
exp(s(n)) = lm x -> x* exp n x
\end{code}
\begin{code}
exp'(0) = lm x -> 1
exp(s(n)) = let f = exp'n in lm x -> x* f x 
 \end{code}
The two functions although behaviorally equivalent have a completely different operational behavior.
For the first function applied to a $s(s(0))$ will unfold to 
\begin{code}
lm x-> x* exp(s(0)) x 
\end{code} 
the second though recurs completely on its argument unfolding  to:
\begin{code}
lm x-> x* (lm x-> x *( lm x->1) x) x
\end{code}
which after reduction under $\lambda$ can be reduced to:
\begin{code}
lm x-> x* x*1
\end{code}
We can see that the second version does a lot more computation than
the first. However, if the resulting function is applied many times, to many
different bases, then the second can be more efficient. We want to extend our language with types 
that discriminate between the two cases. 

We will try to explore this and show the how $\Box$ types can be useful to discriminate between the two.
Prior to this we have to speak about about operational semantics.
\subsection{Small Step Semantics}
Small steps semantics, is a transition system that describes how an abstract state machine would execute well typed programs expressed in a $\lambda$ calculus. Small steps semantics gives local reductions and follows a deterministic evaluation principle. Other kind of operational semantics exist (e.g big step or non-deterministic. The Church-Rosser theorem for a calculus can give a proof that all evaluation strategies are equivalent for a calculus). We also need a notion of value. That is a term that accepts no more reductions under our strategy. We work here with call-by-value semantics. That is functions in a $\lambda$ form are not further reduced and when the term is a function call the arguments are reduced to values before application.

 Here is an example of  transition system for small step semantics of the negative fragment of \ac{IPL}:
 \begin{mdframed}
 \begin{mathpar}
  \inferrule*{ } {\langle \rangle \ value }
  \and
    \inferrule*{ } {(\lambda x:\phi. M) \ value} 
    \and
 \inferrule*{ M	\ value} {(\lambda x:\phi. N) M \mapsto [M/x]N }
 
\and
 \inferrule*{{ M	\ value}\\{N \ value}} {\operatorname{fst}\langle M, N \rangle\mapsto M}
 \and
 \inferrule*{{ N	\ value}\\{N \ value}} {\operatorname{snd}\langle M, N \rangle\mapsto M}
\and
\inferrule*{M\mapsto M'} {\langle M, N\rangle\mapsto \langle M',M \rangle}
\and
  \inferrule*{{M \ value}\\{N\mapsto N'}} {\langle M, N\rangle\mapsto \langle M,N' \rangle}
 \and
 \inferrule*{M\mapsto M'} {\operatorname{fst}(M) \mapsto fst( M')}
 \and
 \inferrule*{M\mapsto M'} {\operatorname{snd}(M) \mapsto snd( M')}
 \and
 \inferrule*{M\mapsto M'} { (M N)  \mapsto ( M' N)}
 \and
  \inferrule*{ {M value} \\{N \mapsto N'}} {(M N)  \mapsto ( M N')}
 \end{mathpar}
 \end{mdframed}

Now we have \emph{preservation} and \emph{progress} property. Those are standard properties for any small step semantics transition system. They are formulated
only on closed terms because, unlike the process of proof reduction, we
only evaluate expressions that are closed.

\begin{mdframed}
\textbf{Preservation}
If $\nil\vdash M:\phi$ and $ M\mapsto M'$ then $\nil\vdash M':\phi$ \\
\textbf{Progress}
If $\nil\vdash M:\phi$ then either $\exists M'. M\mapsto M'$ or $M \ value$
\end{mdframed}

Finally we have the  \emph{weak normalization theorem} or \emph{termination theorem}. That is pertinent to the specific choice of semantics. In the simply typed lambda calculus the reduction strategy does not change the normalization property. That is \emph{strong normalization} can be shown. Moreover, from the Church--Rosser theorem and the normalization property one can deduce the existence and unicity of canonical forms.  For other systems this might not be the case.

\begin{mdframed}  
\textbf{Termination}
 If $\nil\vdash M:\phi$ then $\exists V. V \ value\  and \  M\mapsto ^{*} V$ . Where $\mapsto^{*}$ is the reflexive, transitive closure of $\mapsto$ 
 \end{mdframed}
 \subsection{Operational Semantics for Source Expressions}
 We extend the computational interpretation sketched above to encompass the necessity modality
 The interpretation  goes as follows:
 \begin{mdframed}
 \begin{tabular}{l l}
	 $x:\phi$ & $x$ stands for value of type $\phi$ \\
	$s::\phi$ & $s$ stands for a source expression $\phi$ \\
    $[\![ M/s ]\!]N$ & substitute the source expression $M$ for $s$ in $N$  \\
    $\operatorname{box} M$ & quote the closed term $M$ as a source expression \\
    $\operatorname{let}\operatorname{box} (s)=M \operatorname{in} N$ & evaluate $M$ up to the (quoted expression of the)\\ 
    & form $\operatorname{box}(M')$ and then evaluate   $[\![ M/s ]\!]N$\\ 
\end{tabular}
\end{mdframed}

We add the following values, a congruence rule and a reduction rule:
 \begin{mdframed}
 \begin{mathpar}
 \inferrule*{ } { \operatorname{box}{(M)\  value}}
 \and
 \inferrule*{M \mapsto M'}  { \operatorname{let}\operatorname{box}(s)=M \operatorname{in} N \mapsto  \operatorname{let}\operatorname{box}(u)=M' \operatorname{in} N }
 \and
  \inferrule*{  }  { \operatorname{let}\operatorname{box}(s)= \operatorname{box}(M)\operatorname{in} N \mapsto  [\![ M/s ]\!]N}
\end{mathpar}
 \end{mdframed}
 The importance of the typing is explained by Pfenning:
 \begin{quote}
 The crucial restriction of the typing rules ensures that in an expression $\operatorname{box}(M)$, the term
 $M$ does not refer to  any free variables $x$ that stand for values. It can, however, mention variables
 $s$ that stand for source expressions. So when we substitute
 $[\![N/s]\!] \operatorname{box}(M)$ then we are building a larger source expression from two smaller ones,
 $N$ and $M$.  Conversely, when we substitute a value
 $[V/x]\operatorname{box}M=\operatorname{box}(M)$ the source expression is not
 affected.
 \end{quote}
 
 Returning to our example, the system is able to discriminate between the two examples.  The first version of $\operatorname{exp}$ still has type $nat\rightarrow (nat \rightarrow nat)$ whereas the second can be rewritten in the new syntax and has type $nat\rightarrow \Box(nat\rightarrow nat)$.  It is crucial that the first version fails to be written as a source code generator since if we try to re-implement the definition as:
 \begin{code}
 exp(s(n))= box(lm x-> x* exp n x) 
 \end{code}   
 the expression is ill-typed due to the reference to the value variable $n$. 
 The second version can be written in our extension of the language as a code generator and it is well typed:
  \begin{code}
  exp'(0)= box(lm x-> 1): Box(nat -> nat)
  exp(s(n))= let box(f)= exp'(n) in box(lm x-> x* f  x): Box(nat -> nat) 
  \end{code}   
 
 The computational reading sheds new light to modal theorems as programming combinators.
 The canonical inhabitant of Axiom $4$ of modal logic (seen in the Curry--Howard fashion) is the type of the polymorphic metaprogram that quotes a quoted source code expression: 
 \begin{mathpar}
  \inferrule*{D}  { quote= \lambda x:\Box\phi. \operatorname{let}\operatorname{box}(s)= x  \operatorname{in} \operatorname{box}\operatorname{box}(x):\Box\phi\supset\Box\Box\phi }
 \end{mathpar}
 The $K$ axiom corresponds to the combinator that applies applicative source expressions and results to a larger source expression:
  \begin{mathpar}
   \inferrule*{}  {\lambda x:\Box(\phi\supset\psi).\lambda y:\Box\phi. \operatorname{let}\operatorname{box}(s)= x  \operatorname{in}  \operatorname{let}\operatorname{box}(t)=y \operatorname{in}\operatorname{box}(st):\Box(\phi\supset\psi)\supset\Box \phi\supset\Box \psi }
  \end{mathpar}
  Finally the factivity axiom corresponds to unquoting a quoted source code expression:
    \begin{mathpar}
     \inferrule*{D}  { unquote= \lambda x:\Box\phi. \operatorname{let}\operatorname{box}(s)= x  \operatorname{in} s: \Box\phi\supset\phi }
    \end{mathpar}
    
    These operations resemble monadic combinators in languages like Haskell or Scala (cf. \cite{wadler1992comprehending},\cite{Wadler:1992:EFP:143165.143169}). The connection of modality and monadic computation is thoroughly explored in \cite{kobayashi1997monad}. This discussion, pushes towards a judgmental reconstruction of the possibility modality which we will not be discussing here.
 







\nocite{Pfenning2009a, Pfenning2009b}


\bibliographystyle{plain}
\bibliography{secondexam}

\end{document}
